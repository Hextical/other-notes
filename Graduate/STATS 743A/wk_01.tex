\makeheading{Lecture 1}{\printdate{2022-09-07}}%chktex 8
\begin{itemize}
    \item Textbook: \textbf{Statistical Inference} by
          \emph{George Casella + Roger L.\ Berger}
    \item Office hours: Monday 1:30--2:30 in HH210.
\end{itemize}
\section*{Set Theory}
\begin{Definition}{Containment}{}
    \[ A\subset B\iff x\in A\implies x\in B. \]
\end{Definition}
\begin{Definition}{Equality}{}
    \[ A=B\iff A\subset B\text{ and }B\subset A. \]
\end{Definition}
\begin{Definition}{Union}{}
    The \textbf{union} of $ A $ and $ B $, written $ A\cup B $, is the set of
    elements that belong to either $ A $ or $ B $ or both:
    \[ A\cup B=\Set{x\given x\in A\text{ or } x\in B}. \]
    \tcblower{}
    For example, if $ A=\Set{0,2,4,6,8} $ and $ B=\Set{0,3,6,9} $, then
    \[ A\cup B=\Set{0,2,3,4,6,8,9}. \]
\end{Definition}
\begin{Definition}{Intersection}{}
    The \textbf{intersection} of $ A $ and $ B $, written $ A\cap B $, is the set of elements
    that belong to both $ A $ and $ B $:
    \[ A\cap B=\Set{x\given x\in A\text{ and } x\in B}. \]
\end{Definition}
\begin{Definition}{Complementation}{}
    The \textbf{complement} of $ A $, written $ A^c $, is the set of all elements that are not in $ A $:
    \[ A^c=\Set{x\given x\notin A}. \]
\end{Definition}
\begin{Definition}{Relative Complement}{}
    The \textbf{relative complement} of $ A $ in $ B $, written $ B\setminus A $, is the set of all elements that are in $ B $ and not in $ A $:
    \[ B\setminus A=\Set{x\given x\in B\text{ and } x\notin A}=B\cap A^c. \]
\end{Definition}
\begin{Theorem}{De Morgan's Laws}{}
    For any events $ A $ and $ B $ defined on a sample space $ S $,
    \begin{enumerate}[(i)]
        \item $ (A\cup B)^c=A^c\cap B^c $.
        \item $ (A\cap B)^c=A^c\cup B^c $.
    \end{enumerate}
    \tcblower{}
    \textbf{Proof}:
    \begin{enumerate}[(i)]
        \item Let $ x\in (A\cup B)^c $. We know that $ x\notin (A\cup B) $, so
              $ x\notin A $ and $ x\notin B $. Hence, $ x\in A^c $ and $ x\in B^c $, which means
              $ x\in (A^c\cap B^c) $. Therefore, $ (A\cup B)^c\subset (A^c\cap B^c) $.

              Let $ y\in (A^c\cap B^c) $. We know that $ y\in A^c $ and $ y\in B^c $, so
              $ y\notin A $ and $ y\notin B $. Hence, $ y\notin (A\cup B) $, which means
              $ y\in (A\cup B)^c $. Therefore, $ (A^c\cap B^c)\subset (A\cap B)^c $.
        \item Let $ x\in (A\cap B)^c $. We know that $ x\notin (A\cap B) $,
              so $ x\notin A $ or $ x\notin B $. Hence, $ x\in A^c $ or $ x\in B^c $, which means
              $ x\in (A^c\cup B^c) $. Therefore, $ (A\cap B)^c\subset (A^c\cup B^c) $.

              Let $ y\in (A^c\cup B^c) $. We know that $ y\in A^c $ or $ y\in B^c $,
              so $ y\notin A $ or $ y\notin B $. Hence, $ y\notin (A\cap B) $,
              which means $ y\in (A\cap B)^c $. Therefore, $ (A^c\cup B^c)\subset (A\cap B)^c $.
    \end{enumerate}
\end{Theorem}
\begin{Theorem}{Distributive Laws}{}
    \begin{enumerate}[(i)]
        \item $ A\cap (B\cup C)=(A\cap B)\cup (A\cap C) $.
        \item $ A\cup (B\cap C)=(A\cup B)\cap (A\cup C) $.
    \end{enumerate}
\end{Theorem}
\begin{Definition}{Injective and Surjective}{}
    Let $ A $ and $ B $ be sets and let $ f\colon A\to B $.
    \begin{itemize}
        \item We say $ f $ is \textbf{injective} (or \textbf{one-to-one}, written as $ 1\colon 1 $)
              when for all $ x,y\in A $, if $ f(x)=f(y) $, then $ x=y $.
        \item We say $ f $ is \textbf{surjective} (or \textbf{onto}) when for every $ y\in B $,
              there exists at least one $ x\in A $ such that $ f(x)=y $.
    \end{itemize}
\end{Definition}
\begin{Definition}{Countability}{}
    A set $ S $ is \textbf{countable} if there exists an injective function
    $ f\colon S\to\mathbf{N} $.
\end{Definition}
\begin{Example}{}{}
    The set $ \mathbf{Z} $ of all integers is countable.
    First, match $ 0 $ with $ 1 $. Then, for $ n>0 $, match $ n $ with
    $ 2n $ and match $ -n $ with $ 2n+1 $.
    \[ \begin{array}{c|c}
            1 & 0  \\
            \hline
            2 & 1  \\
            \hline
            3 & -1 \\
            \hline
            4 & 2  \\
            \hline
            5 & -2 \\
            \hline
            6 & 3  \\
            \hline
            7 & -3
        \end{array} \]
\end{Example}
\begin{Theorem}{}{}
    The unit interval $ [0,1] $ is not countable.
    \tcblower{}
    \textbf{Proof} (Cantor's diagonalization argument):
    Assume for a contradiction that there is some bijection
    $ f\colon \mathbf{N}\to [0,1] $.
    \[ \begin{array}{c|l}
            1 & f(1)=0.5000\cdots            \\
            \hline
            2 & f(2)=0.14152\cdots           \\
            \hline
            3 & f(3)=0.33333\cdots           \\
            \hline
            4 & f(4)=0.110100100010000\cdots \\
            \hline
            5 & f(5)=0.12345\cdots
        \end{array} \]
    Denote
    \begin{align*}
        f(1) & =0.a_{11}a_{12}a_{13}a_{14}\cdots \\
        f(2) & =0.a_{21}a_{22}a_{23}a_{24}\cdots \\
             & \vdotswithin{=}                   \\
        f(n) & =0.a_{n1}a_{n2}a_{n3}a_{n4}\cdots
    \end{align*}
    For example, $ a_{24}=5 $.
    Let
    \begin{align*}
        b_1 & =9-a_{11}\cdots \\
        b_2 & =9-a_{22}\cdots \\
        b_3 & =9-a_{33}\cdots \\
            & \vdotswithin{=} \\
        b_n & =9-a_{nn}\cdots
    \end{align*}
    Then, $ 0.b_1b_2b_3\cdots $ does not
    appear anywhere in my list, since for every $ n\ge 1 $, the
    $ n\textsuperscript{th} $ digit of this number is different
    from the $ n\textsuperscript{th} $ digit of the
    $ n\textsuperscript{th} $ number on my list. This contradicts
    my assumption that $ f $ is a bijection.
\end{Theorem}
\begin{Definition}{}{}
    A \textbf{probability space} is an ordered triple
    $ (\Omega,\mathcal{F},\mathbb{P}) $ where
    \begin{itemize}
        \item $ \Omega $ is a non-empty set, called the \emph{sample space}
              (where elements
              $ \omega\in\Omega $ are called ``events''),
        \item $ \mathcal{F} $ is a collection of subsets of $ \Omega $,
              called the \emph{$ \sigma $-algebra} (where elements
              $ A\in\mathcal{F} $ are called ``events'') with the following properties:
              \begin{enumerate}[S1]
                  \item $ \Omega\in\mathcal{F} $,
                  \item $ \forall A\in \mathcal{F} $, $ (\Omega\setminus A)=A^c\in\mathcal{F} $ (closed under complements),
                  \item For any sequence $ A_1,A_2,A_3,\ldots\in\mathcal{F} $, we get
                        $ \bigcup_i A_i\in \mathcal{F} $ (closed under countable unions),
              \end{enumerate}
        \item $ \mathbb{P}\colon \mathcal{F}\to[0,1] $ with
              \begin{enumerate}[P1]
                  \item $ \Prob{\Omega}=1 $,
                  \item $ \Prob{A}\ge 0 $ for all $ A $, and
                  \item if $ A_1,A_2,\ldots, $ are disjoint elements of $ \mathcal{F} $,
                        then
                        \[ \Prob*{\bigcup_i A_i}=\sum_i \Prob{A_i}\quad\text{(countable additivity)}. \]
              \end{enumerate}
    \end{itemize}
\end{Definition}
\makeheading{Lecture 2}{\printdate{2022-09-09}}%chktex 8
\begin{Example}{}{}
    Flip a fair coin.
    \begin{itemize}
        \item Sample space: $ \Omega=\Set{\text{H},\text{T}} $;
              that is, $ \abs{\Omega}=2 $.
        \item $ \mathcal{F}=\Set{\emptyset,
                      \Set{\text{H}},\Set{\text{T}},\Set{\text{H},\text{T}}} $;
              that is,
              $ \abs{\mathcal{F}}=2^{\abs{\Omega}}=4 $.
    \end{itemize}
    Whenever $ \Omega $ is countable, we define $ \mathcal{F} $ to be
    the set of \underline{all} subsets of $ \Omega $,
    $ \mathcal{F}=2^{\Omega} $ (we can always choose the power set of $ \Omega $
    as our discrete $ \sigma $-algebra).
    \begin{itemize}
        \item $ \text{H} $ is an outcome, $ \text{H}\in\Omega $.
        \item $ \emptyset $ is an event, $ \emptyset\in\mathcal{F} $.
        \item $ \Set{\text{H}} $ is an event,
              but $ \text{H} $ is not an event, and $ \Set{\text{H}} $
              is not an outcome.
        \item $ \Prob{\emptyset}=0 $.
        \item $ \Prob{\Set{\text{H}}}=1/2 $.
        \item $ \Prob{\Set{\text{T}}}=1/2 $.
        \item $ \Prob{\Set{\text{H},\text{T}}}=1 $.
        \item $ \Prob{\text{H}}=\text{undefined} $.
    \end{itemize}
\end{Example}
\begin{Example}{}{}
    Let $ \Omega=\Set{(x,y)\in\mathbf{R}^2\given x^2+y^2\le 25} $
    (disc of radius $5$). Suppose that we have a bullseye of
    radius $ 1 $, the probability of hitting the bullseye
    is $ 1/25 $.
    \begin{align*}
        \text{Bullseye} & =\Set{(x,y)\in\Omega\given x^2+y^2\le 1}.                 \\
        \Prob{\text{Bullseye}}
                        & =\frac{\text{Area}(\text{Bullseye})}{\text{Area}(\Omega)} \\
                        & =\frac{\pi\cdot 1^2}{\pi\cdot 5^2}                        \\
                        & =\frac{1}{25}.
    \end{align*}
    My $ \sigma $-algebra $ \mathcal{F} $ for dart-throwing will be
    the \underline{smallest $ \sigma $-algebra} that includes all sets of the form
    \[ \bigl(\interval[open left]{a}{b}\times \interval[open left]{c}{d}\bigr)\cap \Omega,\; a<b,\; c<d,\; a,b,c,d\in\mathbf{R}. \]
    \begin{itemize}
        \item $ \abs{\mathbf{N}}=\abs{\mathbf{Z}}=\abs{\mathbf{Q}}=\aleph_0 $.
        \item $ \abs{\mathbf{R}}=\abs[\big]{[0,1]}=\abs{\mathbf{R}^n}=2^{\mathbf{N}}
                      =2^{\aleph_0} $.
        \item $ \abs{2^{\mathbf{R}^2}}=2^{2^{\aleph_0}}\gg 2^{\aleph_0} $.
    \end{itemize}
\end{Example}
\begin{Proposition}{}{}
    Given a probability space $ (\Omega,\mathcal{F},\mathbb{P}) $,
    \begin{enumerate}[(i)]
        \item For all $ A\in\mathcal{F} $, $ \Prob{A^c}=1-\Prob{A} $.
        \item $ \Prob{\emptyset}=0 $.
        \item $ \forall A\in\mathcal{F} $, $ \Prob{A}\le 1 $.
        \item $ \forall A,B\in\mathcal{F} $,
              $ \Prob{B\cap A^c}=\Prob{B}-\Prob{A\cap B} $.
    \end{enumerate}
    \tcblower{}
    \textbf{Proof}:
    \begin{enumerate}[(i)]
        \item By (S2), $ A^c\in\mathcal{F} $. Since $ A^c\cap A=\emptyset $,
              \begin{align*}
                  \Prob{A^c}+\Prob{A}
                   & =\Prob{A^c\cup A} &  & \text{by (P3)}  \\
                   & =\Prob{\Omega}                         \\
                   & =1                &  & \text{by (P1)}.
              \end{align*}
        \item $ \Prob{\emptyset}=\Prob{\Omega^c}=1-\Prob{\Omega}=0 $ by (i).
        \item $ \Prob{A}=1-\Prob{A^c}\le 1 $ since $ \Prob{A^c}\ge 0 $.
        \item $ (A\cap B)\subseteq A $, and $ (A^c\cap B)\subseteq A^c $,
              so $ \bigl((A\cap B)\cap (A^c\cap B)\bigr)\subseteq (A\cap A^c)=\emptyset $.
              Thus,
              \begin{align*}
                  \Prob{A\cap B}+\Prob{A^c\cap B}
                   & =\Prob[\big]{(A\cap B)\cup (A^c\cap B)} \\
                   & =\Prob[\big]{B\cap (A\cup A^c)}         \\
                   & =\Prob{B\cap \Omega}                    \\
                   & =\Prob{B}.
              \end{align*}
    \end{enumerate}
\end{Proposition}
\begin{Theorem}{Inclusion-exclusion for two events}{}
    $ \Prob{A\cup B}=\Prob{A}+\Prob{B}-\Prob{A\cap B} $.
    \tcblower{}
    \textbf{Proof}:
    \begin{align*}
        A\cup B
         & =A\cup (B\cap \Omega)              \\
         & =A\cup (B\cap (A\cup A^c))         \\
         & =(A\cup (B\cap A))\cup (B\cap A^c) \\
         & =A\cup (B\cap A^c).
    \end{align*}
    Therefore, $ A $ is disjoint from $ B\cap A^c $. Thus,
    \begin{align*}
        \Prob{A\cup B}
         & =\Prob{A}+\Prob{B\cap A^c}           \\
         & =\Prob{A}+(\Prob{B}-\Prob{A\cap B}).
    \end{align*}
\end{Theorem}
\begin{Theorem}{Inclusion-exclusion principle for probabilities}{}
    For any $ A_1,A_2,\ldots,A_n\in\mathcal{F} $,
    \begin{align*}
        \Prob*{\bigcup_{i=1}^n A_i}
         & =
        \underbrace{\Prob{A_1}+\Prob{A_2}+\cdots+\Prob{A_n}}_{n\text{ terms}}                              \\
         & \quad\underbrace{-\Prob{A_1\cap A_2}-\cdots-\Prob{A_{n-1}\cap A_n}}_{\binom{n}{2}\text{ terms}} \\
         & \quad\underbrace{+\Prob{A_1\cap A_2\cap A_3}+\cdots}_{\binom{n}{3}\text{ terms}}                \\
         & \quad - \Prob{A_1\cap A_2\cap A_3\cap A_4}-\cdots                                               \\
         & \quad\vdots                                                                                     \\
         & =\sum_{J\subseteq \Set{1,2,\ldots,n},J\ne \emptyset}(-1)^{\abs{J}+1}
        \Prob*{\bigcap_{i\in J}A_i}.
    \end{align*}
\end{Theorem}
\begin{Proposition}{Bonferroni's Inequality}{}
    \[ \Prob{A\cap B}\ge \Prob{A}+\Prob{B}-1. \]
    \tcblower{}
    \textbf{Proof}: Using the inclusion-exclusion theorem,
    we have
    \begin{align*}
        \Prob{A\cap B}
         & =\Prob{A}+\Prob{B}-\Prob{A\cup B}                                             \\
         & \ge \Prob{A}+\Prob{B}-1           &  & \text{ since $ \Prob{A\cup B}\le 1 $}.
    \end{align*}
\end{Proposition}
