\makeheading{Lecture 1}{\printdate{2022-05-02}}%chktex 8
\section{Integers}
\begin{itemize}
    \item Natural numbers: $ \mathbf{N}=\Set{0,1,2,\ldots} $.
    \item Ring of integers: $ \mathbf{Z}=\Set{0,\pm 1,\pm 2,\ldots} $.
    \item Field of fractions $ \mathbf{Q}=\Set*{a/b\given a,b\in\mathbf{Z}\land b\ne 0} $.
    \item Field of real numbers: $ \mathbf{R} $.
    \item Field of complex numbers: $ \mathbf{C}=\Set{a+bi\given a,b\in\mathbf{R}\land i=\sqrt{-1}} $.
\end{itemize}
\begin{Definition}{Axioms}{}
    Set of integers as integral domains:
    \begin{enumerate}[label=V\arabic*]
        \item $ \mathbf{Z} $ has operations $ + $ (addition) and $ \cdot $ (multiplication). It is closed
              under these operations, in that if $ a,b\in\mathbf{Z} $, then $ a+b\in\mathbf{Z} $ and $ a\cdot b\in\mathbf{Z} $.
        \item Addition is associative: If $ a,b,c\in\mathbf{Z} $, then
              \[ a+(b+c)=(a+b)+c. \]
        \item There is an additive identity $ 0\in\mathbf{Z} $: For all $ a\in\mathbf{Z} $,
              \[ a+0=0+a=a. \]
        \item Every element has an additive inverse: If $ a\in\mathbf{Z} $, there is an element $ -a\in\mathbf{Z} $ such that
              \[ a+(-a)=0\text{ and }(-a)+a=0. \]
        \item Addition is commutative: If $ a,b\in\mathbf{Z} $, then
              \[ a+b=b+a. \]
        \item Multiplication is associative: If $ a,b,c\in\mathbf{Z} $, then
              \[ a\cdot(b\cdot c)=(a\cdot b)\cdot c. \]
        \item There is a multiplicative identity $ 1\in\mathbf{Z} $: For all $ a\in\mathbf{Z} $,
              \[ a\cdot 1=a=1\cdot a. \]
        \item Multiplication is commutative: If $ a,b\in\mathbf{Z} $, then
              \[ a\cdot b=b\cdot a. \]
        \item The Distributive Laws hold: If $ a,b,c\in\mathbf{Z} $, then
              \[ a\cdot (b+c)=a\cdot b+a\cdot c, \]
              \[ (a+b)\cdot c=a\cdot c+b\cdot c. \]
        \item There are no zero divisors: If $ a,b\in\mathbf{Z} $ and $ a\cdot b=0 $, then either
              $ a=0 $ or $ b=0 $.
    \end{enumerate}
\end{Definition}
\begin{Remark}{}{}
    \begin{enumerate}[(i)]
        \item As usual, we will often abbreviate $ m\cdot n $ to $ mn $.
        \item The last axiom is equivalent to the \textbf{cancellation law}.
              If $ a,b,c\in\mathbf{Z} $, $ a\ne 0 $, and $ab=ac$, then
              $ b=c $.
              \begin{align*}
                  ab     & =ac \\
                  ab-ac  & =0  \\
                  a(b-c) & =0.
              \end{align*}
              Since there are no zero divisors, either $a = 0$ or $b - c = 0$. Since $ a\ne 0 $ by assumption, we must have
              $b - c = 0$, so $b = c$.

              Notice that we did not divide both sides of the equation by $ a $; we cancelled $a$ from both sides. This shows
              that division and cancellation are not the ``the same thing.''
    \end{enumerate}
\end{Remark}
\begin{Exercise}{}{}
    If $ n\in\mathbf{Z} $, then prove that $ 0\cdot n=0 $.
    \tcblower{}
    \textbf{Solution}:
    \begin{align*}
        0\cdot n
         & =0\cdot n+0                    &  & \text{V3} \\
         & =0\cdot n+0\cdot n+(-0\cdot n) &  & \text{V4} \\
         & =(0+0)\cdot n+(-0\cdot n)      &  & \text{V9} \\
         & =0\cdot n+(-0\cdot n)          &  & \text{V3} \\
         & =0.                            &  & \text{V4}
    \end{align*}
\end{Exercise}
\begin{Exercise}{}{}
    If $ n\in\mathbf{Z} $, then prove that $ -n=(-1)\cdot n $.
    \tcblower{}
    \textbf{Solution}:
    \begin{align*}
        0\cdot n             & =0                  \\
        (-1+1)\cdot n        & =0   &  & \text{V4} \\
        (-1)\cdot n+1\cdot n & =0   &  & \text{V9} \\
        (-1)\cdot n+n        & =0   &  & \text{V7} \\
        (-1)\cdot n          & =-n.
    \end{align*}
\end{Exercise}
\begin{Theorem}{Well-Ordering Axiom (WOA)}{}
    Any non-empty subset of the positive integers ($ \mathbf{N} $) has a smallest element.
    \tcblower{}
    There are three main ways of using (WOA) in proofs.
    \begin{enumerate}[(1)]
        \item Pick the smallest element in a non-empty subset of $ \mathbf{N} $ and show that there
              is a smallest element.
        \item If a subset of natural numbers contains no smallest element, then the set
              is empty.
        \item Any statement that implies there is an infinite strictly decreasing sequence
              of natural numbers must be false. This is called the \emph{Principal of infinite
                  descent}.
    \end{enumerate}
\end{Theorem}
\begin{Theorem}{Principal of Mathematical Induction (POMI)}{}
    Let $ P(n) $ be a statement that depends on $ n\in\mathbf{N} $. If $ P(0) $ is true
    and for all $ k\in\mathbf{N} $, $ P(k) $ implies $ P(k+1) $, then $ P(n) $ is true for
    all $ n\in\mathbf{N} $.
\end{Theorem}
\begin{Exercise}{}{}
    State the Principal of Strong Induction (POSI).
    \tcblower{}
    Let $ P(n) $ be a statement that depends on $ n\in\mathbf{N} $. If $ P(0) $ is true
    and for all $ k\in\mathbf{N} $, $ P(0),\ldots,P(k) $ implies $ P(k+1) $, then $ P(n) $ is true for
    all $ n\in\mathbf{N} $.
\end{Exercise}
\makeheading{Lecture 2}{\printdate{2022-05-04}}%chktex 8
\section{Divisibility}
\begin{Definition}{Divides in $ \mathbf{Z} $}{}
    If $ a,b\in\mathbf{Z} $, we say $ a $ \textbf{divides} $ b $, or that $ a $
    is a factor of $ b $, when $ b=ak $ for some $ k\in\mathbf{Z} $. We also say at times
    that $ a $ is a divisor of $ b $. When this happens, we write $ a\mid b $, and
    when this does not happen, we write $ a\nmid b $.
\end{Definition}
\begin{Example}{}{}
    For example, $ -3\mid 12 $, but $ 6\nmid 9 $. Every $ a\mid 0 $ since $ 0=a\cdot 0 $,
    but $ 0\nmid a $ when $ a\ne 0 $. For otherwise, we would have some $ k $ such that
    \[ 0\ne a=0\cdot k=0. \]
    The integers $ \pm 1 $ divide every integer $ b $. Indeed, $ b=1\cdot b $ and $ b=(-1)(-b) $.
\end{Example}
\begin{Proposition}{}{LEC2_PROP1}
    Let $ a,b,c,x,y\in\mathbf{Z} $,
    \begin{enumerate}[(1)]
        \item $ a\mid b \land b\mid c \implies a\mid c $.
        \item $ c\mid a\land c\mid b\implies c\mid(ax+by) $.
        \item $ a\mid b \land b\ne 0 \implies \abs{a}\le\abs{b} $.
        \item $ a\mid b \land b\mid a \implies a=\pm b $.
        \item $ a\mid b \implies \pm a\mid \pm b $.
        \item $ a\mid b \land c\mid d \implies ac\mid bd $.
        \item $ \pm 1\mid a,\;\forall a\in\mathbf{Z} $.
        \item $ a\mid 0,\;\forall a\in\mathbf{Z} $.
        \item $ a\mid a,\;\forall a\in\mathbf{Z} $.
    \end{enumerate}
    \tcblower{}
    \textbf{Proof}:
    \begin{enumerate}[(1)]
        \item We have $ b=ak $ and $ c=b\ell $ for some $ k,\ell\in\mathbf{Z} $. Then, $ c=(ak)\ell=a(k\ell) $,
              and so $ a\mid c $ since $ k\ell\in\mathbf{Z} $.
        \item We have $ a=ck $ and $ b=c\ell $ for some $ k,\ell\in\mathbf{Z} $. Then,
              \[ ax+by=ckx+c\ell y=c(kx+\ell y). \]
              And so $ c\mid (ax+by) $ since $ kx+\ell y\in\mathbf{Z} $.
        \item We have $ b=ak $ for some $ k\in\mathbf{Z} $. Take absolute values to get $ \abs{b}=\abs{a}\abs{k} $.
              Since $ b\ne 0 $, we get $ \abs{k}>0 $, but $ k\in\mathbf{Z} $ so $ \abs{k}\ge 1 $. Hence, $ \abs{a}\le \abs{b} $.
        \item We have $ b=ak $ and $ a=b\ell $ for some $ k,\ell\in\mathbf{Z} $. So, $ b=(b\ell)k=b(\ell k) $.
              If $ b=0 $, then $ a=0 $ too, whereby $ a=\pm b $. If $ b\ne 0 $, cancel $ b $ to get $ 1=\ell k $. Thus,
              $ \ell=\pm 1 $, and so $ a=\pm b $.
        \item We have $ b=ak $ for some $ k\in\mathbf{Z} $. Then, $ -b=a(-k) $ and so $ a\mid(-b) $. Also,
              $ b=(-a)(-k) $ and so $ -a\mid b $. Continuing for the other two cases, we get that $ a\mid \pm b $.
    \end{enumerate}
\end{Proposition}
\begin{Corollary}{}{}
    Suppose $ a\mid b $ and $ a\mid c $, then
    \begin{enumerate}[(1)]
        \item $ a\mid b\pm c $.
        \item $ a\mid mb $ for all $ m\in\mathbf{Z} $.
    \end{enumerate}
    \tcblower{}
    \textbf{Proof}: Exercise.

    In words, (1) says that if a number divides two other numbers, then it also
    divides their sum and difference as well. And (2) says that if a number divides
    another number, then it divides the multiple of the other number.
\end{Corollary}
\begin{Example}{}{}
    Prove that if $ x $ is an even number, then $ x^2+2x+4 $ is divisible by $ 4 $.
    \tcblower{}
    If $ x $ is an even number, then $ x=2m $, where $ m\in\mathbf{Z} $.
    $ x^2+2x+4 $ is divisible by $ 4 $ implies $ \exists c\in\mathbf{Z} $ such that
    \[ (2m)^2+2(2m)+4=4c. \]
    Hence,
    \begin{align*}
        (2m)^2+2(2m)+4
         & =4m^2+4m+4  \\
         & =4(m^2+m+1) \\
         & =4c
        \implies c=m^2+m+1.
    \end{align*}
\end{Example}
\section{Quotients and Remainders}
When divisibility fails, we do look for remainders. Here is an important result about division of integers.
It will have a lot of uses; for example, it's the key
step in the Euclidean Algorithm, which is used to compute greatest common
divisors.
\begin{Theorem}{The Division Algorithm}{}
    Let $ a $ and $ b $ be integers with $ a>0 $, then there exists unique integers $ q,r $ such that
    \[ b=aq+r\text{ and }0\le r<a. \]
    \tcblower{}
    \textbf{Proof}: The idea is to find the remainder $r$ using Well-Ordering. What is
    division? Division is successive subtraction. You ought to be able to find $r$ by
    subtracting $a$'s from $b$ till you can't subtract without going negative. That idea
    motivates the construction which follows.

    Look at the set of integers
    \[ S=\Set{b-an\given n\in\mathbf{Z}\text{ and }b-an\ge 0}. \]
    In other words, we take $ b $ and subtract all possible multiples of $ a $. If we choose
    $ n<\frac{b}{a} $ (there always an integer less than any number), then $ an<b $, so $ b-an>0 $.
    This choice of $ n $ produces a positive integer $ a-bn $ in $ S $. Therefore, $ S $ is a non-empty
    set of non-negative integers and by WOA there is a smallest element $ r\in S $. Thus, $ r\ge 0 $ and $ r=b-aq $
    for $ q\in\mathbf{Z} $. Therefore,
    \[ b=aq+r. \]
    Moreover, if $ r\ge a $, then $ r-a\ge 0 $, so
    \[ b-aq-a\ge 0\text{ or }b-a(q+1)\ge 0, \]
    which implies $ b-a(q+1)\in S $, but $ r=b-aq>b-a(q+1) $. This contradicts
    our assumption that $r$ is the smallest element of $S$. Therefore,
    \[ b=aq+r\text{ and }0\le r<a. \]
    To show $ r $ and $ q $ are unique, suppose $ r^\prime $ and $ q^\prime $ also satisfy these conditions:
    \[ b=aq^\prime+r^\prime\text{ and }0\le r<a. \]
    Also, assume without loss of generality that $ r\le r^\prime $. Then,
    \begin{align*}
        aq+r          & =aq^\prime+r^\prime, \\
        a(q-q^\prime) & =r^\prime-r.
    \end{align*}
    Thus, $ r^\prime-r $ is a multiple of $ a $. Thus, $ \frac{r^\prime - r}{a} $ is an integer and hence $ r-r^\prime=0 $
    implies $ r=r^\prime$. Further, $ b(q-q^\prime)=0 $ showing that $ q=q^\prime $.
\end{Theorem}
\begin{Definition}{}{}
    Let $ a,b $ be integers and $ a>0 $. We write $ b=aq+r $, where $ 0\le r<a $. Then $ a $ is called modulus, $ b $ is
    called dividend, $ q $ is called quotient, and $ r $ is called remainder.
    \tcblower{}
    Note that for $ a>0 $, the expression $ a\mid b $ simply means that in $ b=aq+r $ with $ r=0 $.
\end{Definition}
\begin{Example}{}{}
    \begin{enumerate}[(a)]
        \item Apply the Division Algorithm to divide $59$ by $7$.
        \item Apply the Division Algorithm to divide $-59$ by $7$.
    \end{enumerate}
    \tcblower{}
    \begin{enumerate}
        \item $ 59=7(8)+3 $.
        \item $ -59=7(-8)+(-3) $.
    \end{enumerate}
\end{Example}
\makeheading{Lecture 3}{\printdate{2022-05-06}}%chktex 8
\begin{Example}{}{}
    Prove that if $ n\in\mathbf{Z} $, then $ n^2 $ does not leave a remainder of $ 2 $
    or $ 3 $ when it's divided by $ 5 $.
    \tcblower{}
    \textbf{Solution}: We will do this using the Division Algorithm as an illustration. If $ n $ is divided by $ 5 $,
    the remainder $ r $ satisfies $ 0\le r<5 $. Thus, $ r=0,1,2,3,4 $. Hence, $ n $ can have one of the following forms:
    \[ 5q+0,\, 5q+1,\, 5q+2,\, 5q+3,\, 5q+4. \]
    Check each case:
    \begin{align*}
        n^2 & =(5q)^2=25q^2=5(5q^2)+0                \\
        n^2 & =(5q+1)^2=25q^2+10q+1=5(5q^2+2q)+1     \\
        n^2 & =(5q+2)^2=25q^2+20q+4=5(5q^2+4q)+4     \\
        n^2 & =(5q+3)^2=25q^2+30q+9=5(5q^2+6q+1)+4   \\
        n^2 & =(5q+4)^2=25q^2+40q+16+5(5q^2+8q+3)+1.
    \end{align*}
    In all cases, dividing $ n^2 $ by $ 5 $ gave a remainder of $ 0 $, $ 1 $, or $ 4 $.
\end{Example}
As an illustration, $191273$ can't be perfect square because it leaves a remainder
of $3$ when it's divided by $5$.
\section{Greatest Common Divisor}
\begin{Definition}{}{}
    Let $ a,b\in\mathbf{Z} $ (not both zero). A number $ d\in\mathbf{Z}^+ $ is called the greatest common divisor (GCD) of $ a $ and $ b $ if
    \begin{enumerate}[(1)]
        \item $ d\mid a $ and $ d\mid b $,
        \item If $ c\mid a $ and $ c\mid b $, then $ c\mid d $.
    \end{enumerate}
    In other words, the greatest common divisor of two integers (not both zero) is
    the largest integer which divides both of them. If $a$ and $b$ are integers (not both
    $0$), the greatest common divisor of $a$ and $b$ is denoted $(a,b)$. The greatest common
    divisor is sometimes called the greatest common factor or highest common factor.
\end{Definition}
Here are some easy examples:
\begin{itemize}
    \item $ (8,6)=2 $;
    \item $ (15,15)=15 $;
    \item $ (60,0)=60 $;
    \item $ (18,-15)=3 $.
\end{itemize}
You were probably able to do the last examples by factoring the numbers in
your head. For instance, to find $(8,6)$, you see that $2$ is the only integer bigger
than $1$ which divides both $8$ and $6$.

In case $a = b = 0$, it might make sense to say there is no greatest common
divisor. Some say that $(0,0) = 0$, but in any case the issue for us will not arise.

When $a$ and $b$ are small integers, we can find $(a, b)$ by inspection. The problem
with this approach is that it requires that you factor the numbers. However, once
the numbers get too large --- currently, ``too large'' means ``on the order of several
hundred digits long'' --- this approach to finding the greatest common divisor won't
work. Fortunately, the Euclidean algorithm computes the greatest common divisor
of two numbers without factoring the numbers. We will discuss it after discussing
some elementary properties.
\begin{Proposition}{}{l3_prop1}
    Let $ a,b\in\mathbf{Z} $ (not both zero). Then,
    \begin{enumerate}[(1)]
        \item $ (a,b)\ge 1 $.
        \item $ (a,b)=(\abs{a},\abs{b}) $.
        \item $ (a,b)=(a+kb,b) $, for any integer $ k $.
    \end{enumerate}
    \tcblower{}
    \textbf{Proof}:
    \begin{enumerate}[(1)]
        \item Since $ 1\mid a $ and $ 1\mid b $, $ (a,b) $ must be at least as big as $ 1 $.
        \item $ x\mid a $ if and only if $ x\mid -a $; that is, $ a $ and $ -a $ have the same factors,
              but $ \abs{a} $ is either $ a $ or $ -a $, so $ a $ and $ \abs{a} $ have the same factors. Likewise,
              $ b $ and $ \abs{b} $ have the same factors. Therefore, $ x $ is a common factor of $ a $ and $ b $ if and only if it's
              a common factor of $ \abs{a} $ and $ \abs{b} $. Hence, $ (a,b)=(\abs{a},\abs{b}) $.
        \item First, if $ x $ is a common factor of $ a $ and $ b $, then $ x\mid a $ and $ x\mid b $. Then, $ x\mid kb $,
              so $ x\mid a+kb $. Thus, $ x $ is a common factor of $ a+kb $ and $ b $. Likewise, if $ x $ is a common factor of $ a+kb $
              and $ b $, then $ x\mid a+kb $ and $ x\mid b $ which implies
              \[ x\mid (a+kb)-kb=a. \]
              Thus, $ x $ is a common factor of $ a $ and $ b $. Therefore,
              \[ \Set{\text{common factors of $a$ and $b$}}=\Set{\text{common factors of $a+kb$ and $b$}}. \]
              So their largest element are same. The largest element of the first
              set is $(a, b)$, while the largest element of the second set is $(a + kb,b)$. Therefore,
              $(a, b) = (a + kb, b)$.
    \end{enumerate}
\end{Proposition}
\begin{Example}{}{}
    Use the property that $ (a,b)=(a+kb,b) $ to compute $ (998,996) $.
    \tcblower{}
    \textbf{Solution}: $ (998,996)=(2+996,996)=(2,996)=2 $.
\end{Example}
\begin{Example}{}{}
    Prove that if $ n\in\mathbf{Z} $, then $ (3n+4,n+1)=1 $.
    \tcblower{}
    $ (3n+4,n+1)=(3(n+1)+1,n+1)=(1,n+1) $. Now, $ (1,n+1)\mid 1 $, but the only positive integer that divides $ 1 $
    is $ 1 $. Hence, $ (1,n+1)=1 $, and so $ (3n+4,n+1)=1 $.
\end{Example}
\begin{Remark}{}{}
    In this course, we often use the special case where $ (a,b)=1 $.
\end{Remark}
\begin{Definition}{}{}
    Let $ a,b\in\mathbf{Z} $ (not both zero). If $ (a,b)=1 $, then we say that $ a $ and $ b $ are relatively prime or coprime.
    \tcblower{}
    For example, $ 49 $ and $ 54 $ are relatively prime, but $ 25 $ and $ 105 $ are not.
\end{Definition}
\begin{Proposition}{}{l3_prop2}
    If $ d=(m,n) $, then $ (\frac{m}{d},\frac{n}{d})=1 $.
    \tcblower{}
    \textbf{Proof}: Suppose $ m=da $ and $ n=db $. Then
    \[ \biggl(\frac{m}{d},\frac{n}{d}\biggr)=(a,b). \]
    Suppose that $ \ell\in\mathbf{Z}^+ $ be a common divisor of $ m $ and $ n $. Since $ d $ is the greatest common divisor,
    $ d\ge d\ell $. Therefore, $ 1\ge \ell $, so $ \ell=1 $ (since $ \ell $ is a positive integer). $ 1 $ is the only positive common divisor
    of $ a $ and $ b $. Therefore, $ 1 $ is the greatest common divisor of $ a $ and $ b $, that is,
    \[ \biggl(\frac{m}{d},\frac{n}{d}\biggr)=(a,b)=1. \]
\end{Proposition}
\section{The Euclidean Algorithm}
The main method for calculating the GCD of two integers is the Euclidean
Algorithm which is based on the Division Algorithm and~\Cref{prop:l3_prop1} (3).

\Cref{prop:l3_prop1} (2) shows that there's no harm in assuming
the integers are non-negative.
\begin{Theorem}{}{}
    Let $ a,b\in\mathbf{Z} $ with $ b>a>0 $. Use the Division Algorithm repeatedly as follows:
    \begin{align*}
        r_1     & =b\text{ and }r_2=a.                      \\
        r_1     & =b=r_2q_1+r_3,        &  & 0\le r_3<a=r_2 \\
        r_2     & =r_3q_2+r_4,          &  & 0\le r_4<r_3   \\
        r_3     & =r_4q_3+r_5,          &  & 0\le r_5<r_4   \\
                & \vdotswithin{=}                           \\
        r_{n-1} & =r_n q_{n-1}+r_{n+1},
    \end{align*}
    with $ r_{n+1}=0 $. Then $ (a,b)=(r_2,r_1)=(r_3,r_2)=\cdots=(r_n,0)=r_n $. We will show that this smallest
    positive integer $ r_n $ is $ (a,b) $.
    \tcblower{}
    \textbf{Proof}: From~\Cref{prop:l3_prop1} (3), we obtain
    \begin{align*}
        (a,b) & =(b,a)=(r_1,r_2)  \\
              & =(r_2q_1+r_3,r_2) \\
              & =(r_3,r_2)        \\
              & =(r_2,r_3)        \\
              & =(r_3q_2+r_4,r_3) \\
              & =(r_4,r_3)        \\
              & \vdotswithin{=}   \\
              & =(r_{n+1},r_n).
    \end{align*}
    One last step we see that $ (a,b)=(r_{n+1},r_n)=(0,r_n)=r_n $.
\end{Theorem}
\textbf{Note}: The Euclidean Algorithm always terminates as we have the decreasing
remainders $ r_1>r_2>r_3>\cdots\ge r_{n}>r_{n+1} $: By WOA sooner or later some remainder
becomes $0$ because the sequence is bounded below by $0$. After $n$ steps, this
sequence eventually reaches some smallest positive number $ r_n $.
