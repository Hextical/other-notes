\makeheading{Lecture 7}{\printdate{2022-05-16}}%chktex 8
\begin{Proposition}{}{}
    Let $ n,a_1,\ldots,a_n\in\mathbf{Z}^+ $.
    If $ p\mid a_1\cdots a_n $, then there exists some $ i\in[1,n] $ such that
    $ p\mid a_i $.
    \tcblower{}
    \textbf{Proof}: If $ n=1 $, then $ p\mid a_1 $ and we are done.
    Assume $ p\mid a_1\cdots a_k $ implies there exists some $ i\in[1,n] $
    such that $ p\mid a_i $. Then, by Euclid's Lemma, $ p\mid a_1\cdots a_{k+1} $
    implies $ p\mid a_{k+1} $ or $ p\mid a_1\cdots a_k $. If $ p\mid a_{k+1} $, then we are done.
    If $ p\mid a_1\cdots a_k $, then we are also done by the inductive hypothesis.
    Hence, the result is true for $ n=k+1 $ which implies the result is true for $ n\in\mathbf{Z}^+ $.
\end{Proposition}
\begin{Theorem}{Fundamental Theorem of Arithmetic (FTA)}{}
    Every integer greater than $1$ can be written uniquely (up to ordering) as the
    product of primes.
    \tcblower{}
    \textbf{Proof}: We will start by proving that every positive integer greater than $1$
    can be written as a product of primes. Let $S$ denote the collection of all positive
    integers greater than $1$ that cannot be written as a product of primes. Suppose
    that $S$ is non-empty. Since $ S\subset \mathbf{N} $ and by WOA, there exists a smallest element of
    $S$, say $n$. Then $n$ cannot be a prime as otherwise it would not be in $S$. Thus, $n$ is
    composite, say $n = ab$. Since $a$ and $b$ are both less than $n$, they can be written as
    products of prime numbers. Say
    \[ a=p_1^{\alpha_1}\cdots p_k^{\alpha_k},\qquad b=q_1^{\beta_1}\cdots q_{\ell}^{\beta_{\ell}}. \]
    But then,
    \[ n=p_1^{\alpha_1}\cdots p_k^{\alpha_k}q_1{\beta_1}\cdots q_1^{\beta_1}\cdots q_{\ell}^{\beta_{\ell}}, \]
    which is a contradiction. This means that $S$ is empty. So every integer greater that
    $1$ is a product of primes.

    To prove uniqueness, consider two distinct prime factorizations of $n$ as
    \[ p_1^{\alpha_1}\cdots p_k^{\alpha_k}=q_1{\beta_1}\cdots q_1^{\beta_1}\cdots q_{\ell}^{\beta_{\ell}}. \]
    Note here the $p$'s are distinct primes, the $q$'s are distinct primes, and all the exponents are greater than or equal to $1$.

    Consider $ p_1 $. It divides the left side, so it divides the right side. Using Proposition 1
    $ p_1\mid q_i^{\beta_i} $ for some $i$ which implies $ p_1=q_i $ since they are both prime numbers.
    To avoid a mess, renumber the $ q $'s so $ q_i $ becomes $ q_1 $ and vice versa. Thus,
    $ p_1=q_1 $, and the equation reads
    \[ p_1^{\alpha_1}\cdots p_k^{\alpha_k}=p_1{\beta_1}\cdots q_1^{\beta_1}\cdots q_{\ell}^{\beta_{\ell}}. \]
    If $ \alpha_1>\beta_1 $, then we have
    \[ p_1^{\alpha_1-\beta_1}p_2^{\alpha_2}\cdots p_k^{\alpha_k}=q_2^{\beta_2}\cdots q_{\ell}^{\beta_{\ell}}. \]
    This is impossible since now $ p_1 $ divides the left side, but not the right. For the same reason, $ \alpha<\beta_1 $ is impossible.
    It follows that $ \alpha_1=\beta_1 $, so we can cancel the $ p_1 $'s off both sides leaving
    \[ p_2^{\alpha_2}\cdots p_k^{\alpha_k}=q_2^{\beta_2}\cdots q_{\ell}^{\beta_{\ell}}. \]
    Keep going. At each stage, we pair up a power of $p$ with a power of $q$, and the
    preceding argument shows the powers are equal. We can't wind up with any primes
    left over at the end, or else I'd have a product of primes equals to $1$. So everything
    must have paired up, and the original factorizations were the same (except possibly
    for the order of the factors).
\end{Theorem}
\begin{Example}{}{}
    Consider the set $H$ of all numbers of the form $4n + 1$ where $n$ is
    a non-negative integer, that is,
    \[ H=\Set{1,5,9,13,17,21,25,29,\ldots}. \]
    These numbers are called \textbf{Hilbert} numbers. Observe that $H$ is closed under
    multiplication, that is, if we multiply any two Hilbert numbers, we get another
    Hilbert number. Indeed,
    \[ (4x+1)(4y+1)=4(4xy+x+y)+1. \]
    A number in $H$, other than $1$, that has no divisor in $H$ other than $1$ and itself
    is called Hilbert prime. The first few Hilbert Primes are $5$, $ 9 $, $ 13 $, $ 17 $, $ 21 $, and $ 29 $.
    Note $25$ is a Hilbert composite because $5$ is in the set.

    The set $H$ does not have unique prime factorization. Indeed, $693 = (9)(77) = (21)(33)$.
\end{Example}
\begin{Example}{Why is $1$ is not a prime in $ \mathbf{Z} $?}{}
    Some might argue that the integer $1$ deserve to be called a prime. After all,
    it cannot be factored down any further. However, if we allow $1$ to be prime then
    Unique factorization goes out the window. Indeed, we can factor the integer $1$ from
    any integer $a$ as much as we like:
    \[ a=(1)(1)(1)\cdots(1)(a). \]
    True prime don't do that. The number of times they appear in the Unique factorization of $a$ is unique.
    That's what allows the factorization to be called ``unique.''
    Better to leave $1$ out of the basket of integers known as primes.
\end{Example}
\section{Gaussian Integer}
The Gaussian integers were introduced by Gauss in 1832.
\begin{Definition}{Gaussian Integers}{}
    The set
    \[ \mathbf{Z}[i]=\Set{x+iy\given x,y\in\mathbf{Z}\land i^2=-1} \]
    is called the set of \textbf{Gaussian integers}.
\end{Definition}
Observe that $ \mathbf{Z}\subset\mathbf{Z}[i] $ since $ a+0i\in\mathbf{Z} $.
\begin{Definition}{Axioms in $ \mathbf{Z}[i] $}{}
    The Gaussian integers have all the same important properties as $ \mathbf{Z} $. This means
    that $ \in\mathbf{Z}[i] $ satisfies the following axioms:
    \begin{enumerate}[label=V\arabic*]
        \item $ \mathbf{Z}[i] $ has operations $ + $ (addition) and $ \cdot $ (multiplication). It is closed
              under these operations, in that if $ a,b\in\mathbf{Z}[i] $, then $ a+b\in\mathbf{Z}[i] $ and $ a\cdot b\in\mathbf{Z}[i] $.
        \item Addition is associative: If $ a,b,c\in\mathbf{Z}[i] $, then
              \[ a+(b+c)=(a+b)+c. \]
        \item There is an additive identity $ 0\in\mathbf{Z} $: For all $ a\in\mathbf{Z}[i] $,
              \[ a+0=0+a=a. \]
        \item Every element has an additive inverse: If $ a\in\mathbf{Z}[i] $, there is an element $ -a\in\mathbf{Z}[i] $ such that
              \[ a+(-a)=0\text{ and }(-a)+a=0. \]
        \item Addition is commutative: If $ a,b\in\mathbf{Z}[i] $, then
              \[ a+b=b+a. \]
        \item Multiplication is associative: If $ a,b,c\in\mathbf{Z}[i] $, then
              \[ a\cdot(b\cdot c)=(a\cdot b)\cdot c. \]
        \item There is a multiplicative identity $ 1\in\mathbf{Z}[i] $: For all $ a\in\mathbf{Z}[i] $,
              \[ a\cdot 1=a=1\cdot a. \]
        \item Multiplication is commutative: If $ a,b\in\mathbf{Z}[i] $, then
              \[ a\cdot b=b\cdot a. \]
        \item The Distributive Laws hold: If $ a,b,c\in\mathbf{Z}[i] $, then
              \[ a\cdot (b+c)=a\cdot b+a\cdot c, \]
              \[ (a+b)\cdot c=a\cdot c+b\cdot c. \]
        \item There are no zero divisors: If $ a,b\in\mathbf{Z}[i] $ and $ a\cdot b=0 $, then either
              $ a=0 $ or $ b=0 $.
    \end{enumerate}
    Clearly, $ \mathbf{Z}[i] $ is an integral domain.
\end{Definition}
Our goal is to determine ``whether the Gaussian integers have unique prime
factorization or not?'' To do this, we try to mimic what we did in the integers.

Notice that it is uncommon for the division of one Gaussian Integer by another
to yield a Gaussian Integer as its quotient (the analogous statement in $ \mathbf{Z} $ is also
seen to be true). For example, we can divide these two elements in $ \mathbf{C} $ to find:
\[ \frac{1+6i}{4+7i}=\frac{46}{75}+\frac{17}{75}i \]
is not a Gaussian integer. However, we find that some particular divisions do yield
a quotient in $ \mathbf{Z}[i] $:
\begin{align*}
    \frac{2+5i}{i}     & =5-2i \\
    \frac{-6+8i}{1+7i} & =1+i.
\end{align*}
To further understand this divisibility behaviour, we develop a tool to measure the
size of a Gaussian integer called the \textbf{norm} so that we have a meaning to be bigger
or smaller in the Gaussian integers.
\begin{Definition}{Norm}{}
    If $ z=x+iy\in\mathbf{Z}[i] $, then we define the norm of $ z $ by
    \[ N(z)=x^2+y^2=z\conj{z}. \]
\end{Definition}
\begin{Example}{}{}
    We have
    \begin{align*}
        N(1)     & =1^2+0^2=1      \\
        N(-2i)   & =0+(-2)^2=4     \\
        N(-3+2i) & =(-3)^2+2^2=13.
    \end{align*}
    So, by one Gaussian integer $z$ being smaller than another Gaussian integer $w$,
    we mean that $N (z) < N (w)$.
\end{Example}
However before we go back to trying to figure out the Division in Gaussian
integers, it makes sense to think about the properties of the norm.
\begin{Exercise}{}{}
    Create a bunch of your own examples with the purpose of trying
    to figure out what properties the norm might have.
\end{Exercise}
\begin{Theorem}{}{}
    If $ z,w\in\mathbf{Z}[i] $, then
    \begin{enumerate}[(1)]
        \item $ N(z)\in\mathbf{N}\cup\Set{0} $.
        \item $ N(z)=0\iff z=0 $.
        \item $ N(zw)=N(z)N(w) $.
    \end{enumerate}
\end{Theorem}
\section{Divisibility and units in \texorpdfstring{$ \mathbf{Z}[i] $}{Z[i]}}
We now define divisibility in $ \mathbf{Z}[i] $ analogously to divisibility in $ \mathbf{Z} $, using
the norm's multiplicativity to great effect.
\begin{Definition}{Divides in $ \mathbf{Z}[i] $}{}
    If $ \alpha,\beta\in\mathbf{Z}[i] $, we say that $ z $ divides $ w $, and write $ z\mid w $,
    provided that $ w=zX $ for some $ X\in\mathbf{Z}[i] $. In this case, $ w $ is a multiple of $ z $
    and $ z $ is a factor of $ w $.
\end{Definition}
\begin{Example}{}{}
    $ 1+2i $ divides $ 5+0i $ in $ \mathbf{Z}[i] $ because
    \[ 5+0i=(1+2i)(1-2i). \]
\end{Example}
Let us collect some facts and definitions regarding divisibility. The next theorem turns out to be very useful.
\begin{Theorem}{}{}
    If $ z\mid w $ in $ \mathbf{Z}[i] $, then $ N(z)\mid N(w) $ in $ \mathbf{Z} $.
\end{Theorem}
\begin{Exercise}{}{l7_ex2}
    Find $ q,r\in\mathbf{Z}[i] $ such that $ w=qz+r $ for each pair $ w,z\in\mathbf{Z}[i] $.
    \begin{enumerate}[(1)]
        \item $ w=3+7i $, $ z=4+5i $.
        \item $ w=7-3i $, $ z=2+7i $.
        \item $ w=1+2i $, $ z=3-i $.
    \end{enumerate}
    \tcblower{}
    \textbf{Solution}:
\end{Exercise}
\Cref{exercise:l7_ex2} shows that unlike in $ \mathbf{Z} $, the quotient and remainder are not unique
in the Gaussian integers.

\makeheading{Lecture 8}{\printdate{2022-05-18}}%chktex 8
\section{Primes in \texorpdfstring{$ \mathbf{Z}[i] $}{Z[i]}}
Recall that our goal here is to look at prime factorizations in $ \mathbf{Z}[i] $. So,
we are in need to define primes in $ \mathbf{Z}[i] $.

In $ \mathbf{Z}[i] $, we classified every number into one of four types: zero, unit,
prime, or composite. We do the same for $ \mathbf{Z}[i] $. The first two definitions
are the same: we have a number 0, and a number $ z\in\mathbf{Z}[i] $ is a unit,
if there exists a number $ w\in\mathbf{Z}[i] $, such that $ zw=1 $; that is, $ z $
must divide $ 1 $. However, in $ \mathbf{Z}[i] $, our definitions for primes and composite
numbers are not good. In $ \mathbf{Z} $, we talked about positive divisors, but we do not have a concept
of positive or negative numbers in $ \mathbf{Z}[i] $. Therefore, we need to
come up with better definitions for prime and composite numbers. To do this,
we must first make sure that we understand units.
\begin{Definition}{Unit in $ \mathbf{Z}[i] $}{}
    If $ z $ is a \textbf{unit} in $ \mathbf{Z}[i] $, then there exists $ w\in\mathbf{Z}[i] $
    such that $ zw=1 $.
\end{Definition}
\begin{Example}{}{}
    Find all units in $ \mathbf{Z} $.
\end{Example}
\begin{Theorem}{}{lec8_theorem3}
    A number $ z $ is a unit in $ \mathbf{Z}[i] $ if and only if $ N(z)=1 $.
    \tcblower{}
    \textbf{Proof}: If $ z $ is a unit in $ \mathbf{Z}[i] $, then by definition
    there exists $ w\in\mathbf{Z}[i] $ such that $ zw=1 $. Hence, we have
    \begin{align*}
        N(zw)    & =N(1) \\
        N(z)N(w) & =1.
    \end{align*}
    Thus, $ N(z)\mid 1 $ in $ \mathbf{Z} $. Therefore, $ N(z)=1 $ since $ N(z)\in\mathbf{N} $.

    On the other hand, if $ z=x+iy\in\mathbf{Z}[i] $ such that $ N(z)=1 $, then $ x^2+y^2=1 $.
    Since $ x,y\in\mathbf{Z} $, the only possibilities are $ z=1+0i $, $ z=-1+0i $, $ z=0+i $,
    and $ z=0-i $. It is easy to verify that each of these numbers are units in $ \mathbf{Z}[i] $.
\end{Theorem}
\begin{Definition}{Prime in $ \mathbf{Z}[i] $}{}
    Let $ z\in\mathbf{Z}[i] $. $ z $ is called a \textbf{prime} in $ \mathbf{Z}[i] $ if
    \begin{enumerate}[(i)]
        \item $ z $ is not a unit, and
        \item any factorization $ z=wu $ forces $ w $ or $ u $ to be a unit in $ \mathbf{Z}[i] $.
    \end{enumerate}
\end{Definition}
\begin{Exercise}{}{}
    Are all prime numbers in $ \mathbf{Z} $ also prime numbers in $ \mathbf{Z}[i] $?
\end{Exercise}
\begin{Example}{}{}
    The integer $ 2 $ is a prime in $ \mathbf{Z} $, but is not a prime in $ \mathbf{Z}[i] $ since
    $ 2=(1+i)(1-i) $, and neither $ 1+i $ nor $ 1-i $ is a unit. The number $ 3 $ is a prime
    in both $ \mathbf{Z} $ and $ \mathbf{Z}[i] $. Suppose that $ 3=zw $ for some $ z,w\in\mathbf{Z}[i] $.
    Then,
    \begin{align*}
        N(zw)    & =N(3)=9 \\
        N(z)N(w) & =9,
    \end{align*}
    which means that $ N(z)\mid 9 $ and $ N(w)\mid 9 $. Hence, $ N(z) $ equals one of $ 1 $, $ 3 $, or $ 9 $.
    \begin{itemize}
        \item If $ N(z)=1 $, then $ z $ must be a unit by~\Cref{thm:lec8_theorem3}.
        \item If $ N(z)=3 $, let $ z=x+iy $, then
              \[ 3=N(z)=x^2+y^2. \]
              If $ y\ne 0 $, then $ x^2+y^2\ne 3 $. If $ y=0 $, then $ x^2=3 $
              is a perfect square not equal to $ 3 $. Therefore, $ N(z)\ne 3 $. Analogously, $ N(w)\ne 3 $.
        \item If $ N(z)=9 $, then $ N(w)=1 $, and as seen already this forces $w$ to be unit.
    \end{itemize}
\end{Example}
\begin{Exercise}{}{}
    Prove that $ 7 $ is prime in $ \mathbf{Z}[i] $.
    \tcblower{}
    \textbf{Solution}:
    Note that
    \[ N(zw)=N(7)=49\iff N(z)N(w)=49, \]
    which means that $ N(z)\mid 49 $ and $ N(w)\mid 49 $. Hence, $ N(z) $ equals to one of $ 1 $, $ 7 $, or $ 49 $.
    \begin{itemize}
        \item If $ N(z)=1 $, then $ z $ must be a unit by Theorem.
        \item If $ N(z)=7 $, let $ z=x+iy $, then
              \[ 7=x^2+y^2. \]
              If $ y\ne 0 $, then $ x^2+y^2\ne 7 $. If $ y=0 $, then $ x^2=7 $
              is a perfect square not equal to $ 3 $. Therefore, $ N(z)\ne 3 $. Analogously, $ N(w)\ne 3 $.
        \item If $ N(z)=49 $, then $ N(w)=1 $, and as seen already this forces $ w $ to be a unit.
    \end{itemize}
\end{Exercise}
\begin{Exercise}{}{}
    Prove that $ 2+i $ is prime in $ \mathbf{Z}[i] $.
    \tcblower{}
    \textbf{Solution}: Note that
    \[ N(zw)=N(2+i)=3^2+1^2=10\iff N(z)N(w)=10, \]
    which means that $ N(z)\mid 10 $ and $ N(w)\mid 10 $. Hence, $ N(z) $ equals to one of $ 1 $, $ 2 $, $ 5 $, or $ 10 $.
    \begin{itemize}
        \item If $ N(z)=1 $, then $ z $ must be a unit by Theorem.
        \item If $ N(z)=2 $, let $ z=x+iy $, then
              \[ 2=x^2+y^2. \]
              If $ y\ne 0 $, then $ x^2+y^2\ne 2 $. If $ y=0 $, then $ x^2=2 $
              is a perfect square not equal to $ 2 $. Therefore, $ N(z)\ne 2 $. Analogously, $ N(w)\ne 2 $.
        \item If $ N(z)=5 $, let $ z=x+iy $, then
              \[ 5=x^2+y^2. \]
              If $ y\ne 0 $, then $ x^2+y^2\ne 5 $. If $ y=0 $, then $ x^2=5 $
              is a perfect square not equal to $ 5 $. Therefore, $ N(z)\ne 5 $. Analogously, $ N(w)\ne 5 $.
        \item If $ N(z)=10 $, then $ N(w)=1 $, and as seen already this forces $ w $ to be a unit.
    \end{itemize}
\end{Exercise}
\begin{Exercise}{}{}
    Make a conjecture to which primes in $ \mathbf{Z} $ are also prime in $ \mathbf{Z}[i] $.
\end{Exercise}
\makeheading{Lecture 9}{\printdate{2022-05-20}}%chktex 8
Next, we want to look at the GCD in $ \mathbf{Z}[i] $.
\begin{Definition}{Greatest Common divisor in $ \mathbf{Z}[i] $}{}
    Let $ z,w\in\mathbf{Z}[i] $ not both zero. We define the set of common divisors of $ z $ and $ w $ as
    \[ \Set{X\in\mathbf{Z}[i]\given X\mid z\land X\mid w}. \]
    There will be an element $ d\in\mathbf{Z}[i] $ with maximal norm in this set, and that we call the
    \textbf{greatest common divisor} of $z$ and $w$; that is, if $ c\mid z $ and $ c\mid w $, then $N (d) \ge N (c)$.
\end{Definition}
The Euclidean Algorithm also works for finding a GCD of two numbers in $ \mathbf{Z}[i] $.
\begin{Example}{}{}
    Use the Euclidean Algorithm to find a GCD of $ z=11+3i $ and $ w=1+8i $.
    \tcblower{}
    \textbf{Solution}: We have
    \begin{align*}
        11+3i & =(1-i)(1+8i)+2-4i  \\
        1+8i  & =(-2+i)(2-4i)+1-2i \\
        2-4i  & =2(1-2i)+0.
    \end{align*}
    Therefore, a GCD of $ z=11+3i $ and $ w=1+8i $ is $ 1-2i $.
\end{Example}
\begin{Exercise}{}{}
    Use the Euclidean Algorithm to find a GCD of $ z=3+10i $ and $ z=2+4i $.
    \tcblower{}
    \textbf{Solution}:
    \begin{align*}
        \frac{3+10i}{2+4i} & =\frac{23+4i}{10}\sim 2+0i \\
        \frac{2+4i}{-1+2i} & =1.2-1.6i\sim 1-2i         \\
        \frac{-1+2i}{-1}   & =1-2i.
    \end{align*}
    \begin{align*}
        3+10i & =(2+0i)(2+4i)+(-1+2i) \\
        2+4i  & =(1-2i)(-1+2i)+(-1)   \\
        -1+2i & =(1-2i)(-1)+0.
    \end{align*}
    So, $ (3+10i,2+4i)=1-2i $.
\end{Exercise}
\begin{Exercise}{}{}
    What are all GCD of $ w=11+3i $ and $ z=1+8i $?
\end{Exercise}
We now prove Bezout's Identity in $ \mathbf{Z}[i] $. You should be looking and thinking
about the difference between this proof and the proof of Bezout's Identity in $ \mathbf{Z} $.
\begin{Theorem}{Bezout's Identity in $ \mathbf{Z}[i] $}{}
    Let $ z,w\in\mathbf{Z}[i] $ not both zero. If $ d $ is a GCD of $ z $,
    then for any $ x,y\in\mathbf{Z}[i] $, $ d\mid (zx+wy) $. Moreover, there exists
    $ s,t\in\mathbf{Z}[i] $ such that
    \[ zs+wt=d. \]
    \tcblower{}
    \textbf{Proof}: Let $ c=zx+wy $ for some $ x,y\in\mathbf{Z}[i] $. By definition of GCD, we have
    $ z=da $ and $ w=db $ for some $ a,b\in\mathbf{Z}[i] $. Thus,
    \[ c=(da)x+(db)y=d(ax+by). \]
    Since $ ax+by\in\mathbf{Z}[i] $, we have $ d\mid c $ in $ \mathbf{Z}[i] $.

    Let $ S=\Set[\big]{N(zs+wt)\given s,t\in\mathbf{Z}[i]\land N(zs+wt)>0} $. Observe that $ S $
    is non-empty since $ z $ and $ w $ are both non-zero. Also, $ S\subset \mathbf{N} $ since $ N(zs+wt)\in\mathbf{N} $,
    so we can pick the smallest element in $ S $ by WOA\@. Let $ d=zs+wt $ be the number in $ \mathbf{Z}[i] $
    corresponding to the smallest element in $S$.

    By the Division Algorithm, there exists $ q,r\in\mathbf{Z}[i] $ such that $ z=qd+r $, where $ N(r)<N(d) $. Then,
    \[ r=z-qd=z-(zs+wt)q=z(1-sq)+w(-tq). \]
    If $ N(r)>0 $, then $ N(r)\in S $ with $ N(r)<N(d) $ which is a contradiction. So, we
    must have $ N(r)=0 $ which implies $ r=0 $. Thus, $ d\mid z $. Similarly, we can show that $ d\mid w $.

    Let $c$ be any other common divisor of $ z $ and $ w $. Then, there exists $ s^\prime, t^\prime\in\mathbf{Z}[i] $
    such that $ z=cs^\prime $ and $ w=ct^\prime $. Hence,
    \[ d=zs+wt=cs^\prime s+ct^\prime t=c(s^\prime s+t^\prime t). \]
    So, $c \mid d$ and hence $N (c)\le N (d)$.
\end{Theorem}
The Extended Euclidean Algorithm also works in $ \mathbf{Z}[i] $.
\begin{Example}{}{}
    Let $z = 3 + 10i$ and $w = 2 + 4i$. Find $ s,t\in\mathbf{Z}[i] $ such that $zs + wt$ equals to a GCD of $z$ and $w$.
\end{Example}
\begin{Example}{}{}
    Let $z = 32 + 9i$ and $w = 4 + 11i$. Find $ s,t\in\mathbf{Z}[i] $ such that $zs + wt$ equals to a GCD of $z$ and $w$.
\end{Example}
\begin{Exercise}{}{}
    Let $ z=11+3i $ and $ w=1+8i $. Find $ s,t\in\mathbf{Z}[i] $ such that $zs + wt$ equals to a GCD of $z$ and $w$.
    \tcblower{}
    \textbf{Solution}: We know that a GCD of $ z $ and $ w $ is $ 1-2i $. We want to find
    $ x,y\in\mathbf{Z}[i] $ such that
    \[ (11+3i)s+(1+8i)t=1-2i. \]
    EEA\@:
    \[ \begin{array}{lllll}
            r_i   & q_{i-1} & s_i & t_i  & \text{Check}                 \\
            \midrule
            11+3i &         & 1   & 0                                   \\
            1+8i  & 1-i     & 0   & 1                                   \\
            2-4i  & -2+i    & 1   & -1+i & (1)(11+3i)+(-1+i)(1+8i)=2-4i \\
            1-2i  & 2       & 2-i & 3i   & (2-i)(11+3i)+(3i)(1+8i)=1-2i
        \end{array} \]
    So, $ (2-i,3i) $ is a particular solution.
\end{Exercise}
Since GCDs are not unique, we define relatively prime in terms of the norm.
\begin{Definition}{}{}
    Two numbers $ z,w\in\mathbf{Z}[i] $ are relatively prime (coprime) if there exists $ x,y\in\mathbf{Z}[i] $
    such that
    \[ N(zx+wy)=1. \]
\end{Definition}
Finally, we prove the analogue of Euclid's Lemma for $\mathbf{Z}[i]$.
\begin{Proposition}{Euclid's Lemma in $\mathbf{Z}[i]$}{}
    If $ p $ is a prime in $ \mathbf{Z}[i] $ and $ p\mid zw $, then $ p\mid z $ or $ p\mid w $.
    \tcblower{}
    \textbf{Proof}: Suppose $p \mid zw$, so there exists
    $\ell \in \mathbf{Z}[i]$ such that $z w=\ell p$. Suppose $p \nmid z$. Let $d$ be the GCD of $p$ and $z$. So
    \[ d=z s+p t \text { for some } s, t \in \mathbf{Z}[i] \text { and } u \mid p \text { and } u \mid z \]
    Write $p=d k$ for some $k \in \mathbf{Z}[i]$. Since $p$ is a Gaussian prime, one of $u$ or $k$ is unit in $\mathbf{Z}[i]$.

    If $k$ is unit, then $d=p k^{-1}$, and we see $p \mid d$ and $d \mid z$ which implies $p \mid z$, a contradiction.
    Thus, $d$ is unit with $d^{-1} \in \mathbf{Z}[i]$. Now multiply $d=z s+p t$ by $w$ to get
    \begin{align*}
        d w & =w z s+w p t                  \\
        w   & =d^{-1} w z s+d^{-1} w p t    \\
        w   & =d^{-1} \ell p s+d^{-1} w p t \\
            & =d^{-1} p(\ell s+p t)
    \end{align*}
    Thus, $p \mid w $.
\end{Proposition}
We are now in a position to prove we have unique prime factorization in $ \mathbf{Z}[i] $.
\begin{Theorem}{}{}
    Every $z\in\mathbf{Z}[i]$, with $N (z) > 1$ has a unique factorization into
    primes (up to reordering and multiplication by units).
\end{Theorem}
\section{Congruences}
Throughout this section, we fix a positive integer $n$, and call it a modulus. This
comes from the Latin word for a ``measure,'' or as we might say, a ``yardstick.'' This
modulus is used to compare two integers.
\begin{Definition}{}{}
    Two integers $ a $ and $ b $ are said to be congruent modulo $ n $, and we write
    \[ a\equiv b\Mod{n}, \]
    provided $ a $ and $ b $ have equal remainders between $ 0 $ and $ n-1 $ when they are
    each divided by $ n $; that is, if
    \[ a=q_1n+r_1\text{ and }b=q_2n+r_2, \]
    where $ 0\le r_1,r_2<n $, then $ r_1=r_2 $.
    \tcblower{}
    When $n = 1$, we see that the only possible remainder upon division by $1$ is $0$.
    This is the trivial and interesting case. So, keep the story worthwhile, we typically
    assume that $ n\ge 2 $.
\end{Definition}
\begin{Example}{}{}
    For instance, when the modulus is $ n=5 $, here are all the integers congruent to each other with a remainder
    $ r $ ($ 0\le r<5 $) is of the form $ 5q+r $, where $ q $ is any integer.
    \begin{itemize}
        \item The integers with a remainder of $ 0 $:
              \[ \Set{5q\given q\in\mathbf{Z}}=\Set{0,\pm 5,\pm 10,\ldots}. \]
        \item The integers with a remainder of $ 1 $:
              \[ \Set{5q+1\given q\in\mathbf{Z}}=\Set{\ldots,-9,-4,1,6,11,\ldots}. \]
        \item The integers with a remainder of $ 2 $:
              \[ \Set{5q+2\given q\in\mathbf{Z}}=\Set{\ldots,-8,-3,2,7,12,\ldots}. \]
        \item The integers with a remainder of $ 3 $:
              \[ \Set{5q+3\given q\in\mathbf{Z}}=\Set{\ldots,-7,-2,3,8,13,\ldots}. \]
        \item The integers with a remainder of $ 4 $:
              \[ \Set{5q+4\given q\in\mathbf{Z}}=\Set{\ldots,-6,-1,4,9,14,\ldots}. \]
    \end{itemize}
    The same observation applies to any modulus $n$. Given a modulus $ n\ge 2 $, the
    set of integers $ \mathbf{Z} $ gets partitioned into $n$ disjoint pieces according to the $n$ possible
    remainders $ 0,1,2,\ldots,n-1 $.
\end{Example}
\begin{Example}{}{}
    \begin{itemize}
        \item We have $ 23\equiv 37\Mod{7} $ since $ 23 $ and $ 37 $ both have a remainder of $ 2 $ when divided by $ 7 $.
        \item We have $ 12\equiv 12\Mod{5} $ since $ 12 $ and $ 2 $ both have a remainder of $ 2 $ when divided by $ 5 $.
    \end{itemize}
\end{Example}
\begin{Exercise}{}{}
    Are the following statements true or false?
    \begin{enumerate}[(1)]
        \item $ 37\equiv 13\Mod{6} $.
        \item $ 15\equiv 15\Mod{5} $.
        \item $ 31\equiv -4\Mod{7} $.
    \end{enumerate}
    \textbf{Solution}:
    \begin{enumerate}[(1)]
        \item True.
        \item True.
        \item False.
    \end{enumerate}
\end{Exercise}
\begin{Exercise}{}{}
    Find the smallest non-negative integer satisfying the congruence.
    \begin{enumerate}[(1)]
        \item $ 101\equiv a\Mod{3} $.
        \item $ -7\equiv a\Mod{5} $.
        \item $ 45\equiv a\Mod{11} $.
    \end{enumerate}
    \tcblower{}
    \textbf{Solution}:
    \begin{enumerate}[(1)]
        \item $ a=2 $.
        \item $ a=3 $.
        \item $ a=1 $.
    \end{enumerate}
\end{Exercise}
A rather surprising fact is that the congruence relation ($ \equiv $) behaves much likely
the equality relation ($ = $).
\begin{Proposition}{}{}
    The congruence relation ($ \equiv $) is an equivalence relation ($ = $); that is, it satisfies the following axioms:
    \begin{enumerate}[(1)]
        \item Reflexivity: If $ a $ is any integer, then $ a\equiv a\Mod{n} $,
        \item Symmetry: If $ a\equiv b\Mod{n} $, then $ b\equiv a\Mod{n} $,
        \item Transitivity: If $ a\equiv b\Mod{n} $ and $ b\equiv c\Mod{n} $,
              then $ a\equiv c\Mod{n} $.
    \end{enumerate}
    \tcblower{}
    \textbf{Proof}: Exercise.
\end{Proposition}
Here is a quick and alternate way to tell if $ a\equiv b\Mod{n} $.
\begin{Proposition}{}{prop3_lec9}
    Two integers $ a $ and $ b $ are congruent modulo $ n $ if and only if $ n\mid(a-b) $.
    \tcblower{}
    \textbf{Proof}: If $ a\equiv b\Mod{n} $, then
    \[  a=q_{1} n+r \text { and } b=q_{2} n+r, \]
    for some integers $ q_1 $, $ q_2 $, and $ r $, where $0 \leq r<n$. Then,
    \[ a-b=(q_{1}-q_{2}) n, \]
    which clearly shows that $n \mid (a-b)$.

    On the other hand, suppose $n \mid (a-b)$ which implies $a-b=n k$ for some $k \in \mathbf{Z}$.
    According to Divison Algorithm, we have integers $q$, $ t $, $ r $, and $ s $ such that
    \[ a=n q+r,\; 0 \leq r<n \text { and } b=n t+s,\; 0 \leq s<n. \]
    Then,
    \begin{align*}
        r-s & =(a-n q)-(b-n t) \\
            & =a-b+n(t-q)      \\
            & =n k+n(t-q)      \\
            & =n(k+t-q)
    \end{align*}
    So, $n \mid (r-s)$. If $r-s \neq 0$, then by~\Cref{prop:LEC2_PROP1}(3) $ n \leq\abs{r-s} $. Since $0 \leq r<n$ where $0 \leq s<n$,
    we also have $\abs{r-s}$ which contradicts $n<n$. This forces us to conclude that $r-s=0$ implies $r=s$ and $a \equiv b\Mod{n} $.
\end{Proposition}