\makeheading{Week 4 | Monday}{\printdate{2022-05-23}}%chktex 8
Victoria Day.
\makeheading{Lecture 10}{\printdate{2022-05-25}}%chktex 8
\begin{Proposition}{}{}
    Let $ a,b,c,d\in\mathbf{Z} $, and suppose that
    \begin{align*}
        a & \equiv b\Mod{n}, \\
        c & \equiv d\Mod{n}.
    \end{align*}
    Then,
    \begin{align*}
        a\pm c & \equiv b\pm d\Mod{n}, \\
        ac     & \equiv bd\Mod{n}.
    \end{align*}
    \tcblower{}
    \textbf{Proof}: Use~\Cref{prop:LEC2_PROP1}.
\end{Proposition}
\begin{Corollary}{}{}
    Suppose $ a,b,c\in\mathbf{Z} $, $ n\ge 2 $, and $ a\equiv b\Mod{n} $, then
    \begin{align*}
        a\pm c & \equiv b\pm d\Mod{n}, \\
        ac     & \equiv bd\Mod{n}.
    \end{align*}
\end{Corollary}
\begin{Corollary}{}{}
    Let $ a,b\in\mathbf{Z} $, $ n\ge 2 $, and $ f(x) $ be a polynomial with integer coefficient. If $ a\equiv b\Mod{n} $,
    then
    \[ f(a)=f(b)\Mod{n}. \]
\end{Corollary}
\begin{Example}{}{}
    Simplify $ 994\cdot 996\cdot 997\cdot 998 \Mod{100} $ to a number in the range $ \Set{0,1,\ldots,999} $.
    \tcblower{}
    \textbf{Solution}: Rather than deal with large ``positive'' numbers, we'll convert them to small ``negative'' numbers:
    \begin{align*}
        994 & \equiv -6\Mod{1000}  \\
        996 & \equiv -4\Mod{1000}  \\
        997 & \equiv -3\Mod{1000}  \\
        998 & \equiv -2\Mod{1000}.
    \end{align*}
    Therefore, $ 994\cdot 996\cdot 997\cdot 998\equiv (-6)(-4)(-3)(-2)\Mod{1000}\equiv 144\Mod{1000} $.
\end{Example}
\begin{Example}{}{}
    Let $ f(x)=x^5-10x+7 $. Compute the remainder of $ f(27) $ divided by $ 5 $.
    \tcblower{}
    \textbf{Solution}: Note that $ 27\equiv 2\Mod{5} $, so
    \begin{align*}
        f(27)\equiv f(2)\Mod{5}\equiv 34\Mod{5}\equiv 4\Mod{5}.
    \end{align*}
    Therefore, $ 4 $ is the remainder of $ f(27) $ divided by $ 5 $.
\end{Example}
\begin{Proposition}{}{}
    Let $ a\in\mathbf{Z} $ and $ n\ge 2 $. If $ (a,n)=1 $, then there exists $ b\in\mathbf{Z} $ such that
    $ ab\equiv 1\Mod{n} $.
    \tcblower{}
    If $ (a,n)=1 $, then by Bezout's Identity, there exists $ b,c\in\mathbf{Z} $ such that
    \[ ab+cn=1. \]
    By~\Cref{prop:prop3_lec9}, $ ab\equiv 1\Mod{n} $.
\end{Proposition}
\begin{Definition}{}{}
    Let $ a\in\mathbf{Z} $ and $ n\in\mathbf{Z}^+ $ such that $ (a,n)=1 $. We call the integer $ b $
    such that $ ab\equiv 1\Mod{n} $ the inverse of $ a $ modulo $ n $ and write
    \[ b\equiv a^{-1}\Mod{n}. \]
\end{Definition}
\begin{Example}{}{}
    Find $ 47^{-1}\Mod{61} $.
    \tcblower{}
    \textbf{Solution}: Apply the Extended Euclidean Algorithm to 61 and 47:
    \[ \begin{array}{lllll}
            r_i & q_{i-1} & s_i & t_i & \text{Check}         \\
            \midrule
            61  &         & 1   & 0                          \\
            47  & 1       & 0   & 1                          \\
            14  & 3       & 1   & -1  & (1)(61)+(-1)(47)=14  \\
            5   & 2       & -3  & 4   & (-3)(61)+(4)(47)=5   \\
            4   & 1       & 7   & -9  & (7)(61)+(-9)(47)=4   \\
            1   & 4       & -10 & 13  & (-10)(61)+(13)(47)=1 \\
        \end{array} \]
    Write the linear combination, then reduce mod $61$:
    \begin{align*}
        (-10)\cdot 61+13\cdot 47 & =1 \\
        13\cdot 47\equiv 1\Mod{61}.
    \end{align*}
    Hence, $ 47^{-1}\equiv 13\Mod{61} $.
\end{Example}
\begin{Corollary}{}{cor3lec10}
    Let $ a,b\in\mathbf{Z} $ and $ n\in\mathbf{Z} $. If $ (a,n)=1 $ and $ ab\equiv ac\Mod{n} $,
    then $ b\equiv c\Mod{n} $.
\end{Corollary}
We can strengthen~\Cref{cor:cor3lec10} further.
\begin{Proposition}{}{}
    If $ (a,n)=d $ and $ ab\equiv ac\Mod{n} $, then $ b\equiv c\Mod{\frac{n}{d}} $.
    \tcblower{}
    \textbf{Proof}:
    Proof. Suppose, $a b \equiv a c\Mod{n}$, then $n \mid a b-a c$ which implies there exists $k \in \mathbb{Z}$ such that
    \[ a b-a c=k n \]
    Then,
    \[ (b-c) \frac{a}{d}=k \frac{n}{d} \]
    Notice both $\frac{a}{d}$ and $\frac{n}{d}$ are integers because $(a, n)=d$. Since $\frac{a}{d}$ divides the RHS,
    it must divide the LHS, that is, $\frac{a}{d} \mid k \frac{n}{d}$. Further, by~\Cref{prop:l3_prop2}, $ (\frac{a}{d}, \frac{n}{d})=1$, hence
    \begin{align*}
         & \frac{a}{d}\mid  k &  & \text{by \Cref{prop:l4_prop2}}       \\
         & k=\frac{a}{d} \ell &  & \text{for some } \ell \in \mathbb{Z}
    \end{align*}
    Hence,
    \[ (b-c) \frac{a}{d}=\frac{a}{d} \ell \frac{n}{d}, \]
    which implies
    \[ b-c=l \frac{n}{d} \]
    Thus,
    \[ b \equiv c\Mod{\frac{n}{d}}. \]
\end{Proposition}
\begin{Example}{}{}
    Reduce $ 5^{13}\Mod{17} $.
    \tcblower{}
    \textbf{Solution}: We have
    \begin{align*}
        5^1    & \equiv 5\Mod{17}                    \\
        5^2    & \equiv 25\equiv 8\Mod{17}           \\
        5^4    & \equiv 64\equiv -4\Mod{17}          \\
        5^8    & \equiv 16\equiv -1\Mod{17}          \\
        5^{13} & \equiv (-1)(-4)(5)\equiv 3\Mod{17}.
    \end{align*}
\end{Example}
\begin{Exercise}{}{}
    Find the remainder when $ 306^{100} $ is divided by $ 7 $.
    \tcblower{}
    \textbf{Solution}: TODO
\end{Exercise}
In practice, in order to compute $ a^k\Mod{n} $ for some large power $ n $,
we utilize the so-called Double-and-Add Algorithm. The algorithm is as follows: first write
the integer $ k $ in its binary expansion, that is,
\[ k=\sum_{i=0}^{t}c_i 2^i=c_t\cdot 2^t+c_{t-1}\cdot 2^{t-1}\cdots+c_1\cdot 2+c_0, \]
where $ c_i\in\Set{0,1} $. Then,
\begin{align*}
    a^k & \equiv a^{c_t\cdot 2^t+c_{t-1}\cdot 2^{t-1}\cdots+c_1\cdot 2+c_0}                   \\
        & \equiv (a^{2^t})^{c_t}(a^{2^{t-1}})^{c_{t-1}}\cdots(a^2)^{c_1}(a^0)^{c_{0}}\Mod{n}.
\end{align*}
But then note that for $ j $ such that $ 2\le j\le t $, we can deduce the value of $ a^{2^j} $
from $ a^{2^{j-1}}\Mod{n} $ as follows:
\[ a^{2^j}\equiv (a^{2^{j-1}})^2\Mod{n}. \]
Therefore, we can compute $ a^2,a^{2^2},\ldots a^{2^t} $ in $ t-1 $ steps.
\begin{Example}{}{}
    Let us compute $ n\equiv 7^{114}\Mod{23} $ such that $ 0\le n<23 $.
    \tcblower{}
    \textbf{Solution}: Note that
    \[ 114=2^6+2^5+2^4+2=64+32+16+2. \]
    Then,
    \begin{align*}
        7^2    & \equiv 49\equiv 3\Mod{23}                               \\
        7^4    & \equiv (7^2)^2\equiv 3^2\equiv 9\Mod{23}                \\
        7^8    & \equiv (7^4)^2\equiv 9^2\equiv 81\equiv 12\Mod{23}      \\
        7^{16} & \equiv (7^8)^2\equiv 12^2\equiv 144\equiv 6\Mod{23}     \\
        7^{32} & \equiv (7^{16})^2\equiv 6^2\equiv 36\equiv 13\Mod{23}   \\
        7^{64} & \equiv (7^{32})^2\equiv 13^2\equiv 169\equiv 8\Mod{23}.
    \end{align*}
    Thus,
    \begin{align*}
        7^{114} & \equiv 7^{64+32+16+2}\Mod{23}        \\
                & \equiv 7^{64}7^{32}7^{16}7^2\Mod{23} \\
                & \equiv (8)(13)(6)(3)\Mod{23}         \\
                & \equiv 1872\Mod{23}                  \\
                & \equiv 9\Mod{23}.
    \end{align*}
\end{Example}
\makeheading{Lecture 11}{\printdate{2022-05-27}}%chktex 8
We will now take a look at some interesting applications of modular arithmetic.
For example, it can be used to demonstrate that certain Diophantine equations have
no solutions.
\begin{Example}{}{}
    Show that the Diophantine equation
    \[ x^2+y^2=4z+3 \]
    has no integer solutions $ x,y,z $.
    \tcblower{}
    \textbf{Solution}: Since there are infinitely many possibilities for $ x,y,z $, it seems a
    bit daunting to show that none of them work. But a little trick with congruences
    and replacement makes this problem quite straightforward. This is the same as
    solving the congruence
    \[ x^2+y^2\equiv 3\Mod{4} \]
    in integers $ x $ and $ y $. Since every integer is congruent to either $ 0,1,2,3 $ modulo $ 4 $,
    there are essentially $ 16 $ possible combinations of $ x $ and $ y $ that we can check. However,
    the problem becomes even simpler if we note that
    \[ 0^2\equiv 0,\; 1^2\equiv 1,\; 2^2\equiv 0,\; 3^2\equiv 1\Mod{4}. \]
    Thus, every perfect square is congruent to either $ 0 $ or $ 1 $ modulo $ 4 $. Since we
    are dealing with the sum of two perfect squares, there are only three options left to check, namely
    \[ 0+0\equiv 0,\; 0+1\equiv 1,\; 1+1\equiv 2\Mod{4}. \]
    As we can see, none of them add up to $ 3 $, which implies that $ x^2+y^2\equiv 3\Mod{4} $
    has no solution in integer $ x,y $. Therefore, there are no solutions to the Diophantine equation
    $ x^2+y^2=4z+3 $.
\end{Example}
\begin{Example}{}{}
    Show that $ x^5\equiv x\Mod{5} $ for all $ x\in\mathbf{Z} $.
    \tcblower{}
    \textbf{Solution}: Every integer $ x $ is congruent mod $ 5 $ to one of its possible remainders
    $ 0,1,2,3,4 $. If the desired congruence holds for these remainders, then, by replacement,
    the congruence holds for any integer $x$. By routine calculation we see that
    \[ 0^5\equiv 0,\; 1^5\equiv 1,\; 2^5\equiv 2,\; 3^5\equiv 3,\; 4^5\equiv 4\Mod{5}. \]
    Having verified the result on the five possible remainders, replacement gives the
    result for all integers.
\end{Example}
\section{The Ring of Residue Classes \texorpdfstring{$ \mathbf{Z}_n $}{Zn}}
Assume that the modulus $ n $ is a positive integer ($ n\ge 2 $). By the Division
Algorithm, every integer $ b $ can be written as
\[ b=qn+a,\; 0\le a<n. \]
Reducing this equation mod $ n $, we have
\[ b\equiv a\Mod{n}. \]
Since $ 0\le a<n $, we have $ a\in\Set{0,1,2,\ldots,n-1} $. In other words,
mod $ n $ every integer can be reduced to a number in $ \Set{0,1,2,\ldots,n-1} $.
This set is called the standard residue system mod $ n $, and answers to modular arithmetic problems will
usually be simplified to a number in this range.
\begin{Definition}{}{}
    Let $ a\in\mathbf{Z} $. The set
    \[ [a]=\Set{qn+a\given q\in\mathbf{Z}}=\Set{b\in\mathbf{Z}\given b\equiv a\Mod{n}} \]
    is called the residue class (equivalence class of a modulo $ n $). The integer $ a $
    is called a representative of the residue class $ [a] $. The finite
    set of residues mod $ n $ will be denoted $ \mathbf{Z}_n $.
    \tcblower{}
    \underline{Remark}: $ [a]=[b]\iff b\equiv a\Mod{n} $.
\end{Definition}
\begin{Example}{}{}
    For Example 3 (Lecture 9), the five residue classes of $ \mathbf{Z}_n $ are:
    \begin{align*}
        [0] & =\Set{0+5q\given q\in\mathbf{Z}} \\
        [1] & =\Set{1+5q\given q\in\mathbf{Z}} \\
        [2] & =\Set{2+5q\given q\in\mathbf{Z}} \\
        [3] & =\Set{3+5q\given q\in\mathbf{Z}} \\
        [4] & =\Set{4+5q\given q\in\mathbf{Z}} \\
    \end{align*}
\end{Example}
\begin{Exercise}{}{}
    Let $ n\in\mathbf{Z}^+ $. Prove that the residue clases
    $ [0],[1],\ldots,[n-1] $ modulo $ n $ partition $ \mathbf{Z} $, that is,
    \[ [0]\cup [1]\cup\cdots\cup[n-1]=\mathbf{Z}, \]
    $ [a]\cap [b]\ne \emptyset \implies [a]=[b] $.
\end{Exercise}
\begin{Proposition}{}{}
    Let $ n\in\mathbf{Z}^+ $ and consider the collection $ \mathbf{Z}_n $
    of all residues modulo $ n $. Define the binary operation $ + $, $ - $, and $ \cdot $ as follows:
    \[ [a]\pm [b]=[a\pm b],\; [a]\cdot[b]=[a\cdot b]. \]
    Then, under these binary operations, $ \mathbf{Z}_n $ forms a commutative ring with identity $ [1] $.
    \tcblower{}
    \textbf{Proof}: Use Proposition 1 (Lecture 10).
\end{Proposition}
\begin{Example}{}{}
    \begin{enumerate}[(a)]
        \item What are the residue classes of modulo 6?
        \item Construct an addition table and multiplication table.
        \item Does the ring $ \mathbf{Z}_6 $ form an integral domain?
    \end{enumerate}
\end{Example}