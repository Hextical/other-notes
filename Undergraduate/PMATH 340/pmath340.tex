\documentclass[final]{article}
\usepackage[svgnames]{xcolor}
\usepackage[british]{babel}
\usepackage[protrusion,expansion,tracking,kerning,babel,final]{microtype}
\usepackage[margin=1in]{geometry}
\usepackage[pdfversion=1.7]{hyperref}
\usepackage[shortlabels]{enumitem}
\usepackage{graphicx}
\usepackage{mathtools}
\usepackage{cleveref}
\usepackage{booktabs}
\usepackage{nicematrix}
\usepackage{derivative}
\usepackage{etoolbox}
\usepackage{siunitx}
\usepackage{lmodern}
\usepackage[T1]{fontenc}
\usepackage[scaled=.98]{XCharter}
\usepackage[scaled=1.04,varqu,varl]{inconsolata}% inconsolata typewriter
\usepackage{amssymb}
\makeatletter
\@namedef{T1/zi4/m/it}{<->ssub*XCharterx/m/it}
\makeatother
\usepackage{bm}

\usepackage{pgfplots}
\pgfplotsset{compat=1.18}
\usepgfplotslibrary{fillbetween}

% just to make sure it exists
\providecommand\given{}%
% can be useful to refer to this outside \Set
\newcommand\SetSymbol[1][]{%
    \nonscript\:#1\vert{}%
    \allowbreak%
    \nonscript\:%
    \mathopen{}}%
\DeclarePairedDelimiterXPP{\E}[1]{\operatorname{\mathbb{E}}}[]{}{%
    \renewcommand\given{\SetSymbol[\delimsize]}%
    \ifblank{#1}{\:\cdot\:}%
    #1}%
\DeclarePairedDelimiterXPP\Esp[2]{\operatorname{\mathbb{E}}_{#2}}(){}{%
    \renewcommand\given{\SetSymbol[\delimsize]}%
    \ifblank{#1}{\:\cdot\:}%
    #1}%
\DeclarePairedDelimiterXPP{\Var}[1]{\operatorname{Var}}(){}{%
    \renewcommand\given{\SetSymbol[\delimsize]}%
    \ifblank{#1}{\:\cdot\:}%
    #1}%
\DeclarePairedDelimiterXPP{\Cov}[1]{\operatorname{Cov}}(){}{%
    \renewcommand\given{\SetSymbol[\delimsize]}%
    \ifblank{#1}{\:\cdot\:}%
    #1}%
\DeclarePairedDelimiterXPP\Prob[1]{\operatorname{\mathbb{P}}}(){}{%
    \renewcommand\given{\SetSymbol[\delimsize]}%
    \ifblank{#1}{\:\cdot\:}%
    #1}%
\DeclarePairedDelimiterXPP\Probsp[2]{\operatorname{\mathbb{P}}_{#2}}(){}{%
    \renewcommand\given{\SetSymbol[\delimsize]}%
    \ifblank{#1}{\:\cdot\:}%
    #1}%
\DeclarePairedDelimiterXPP\Ind[1]{\operatorname{\mathbb{I}}}\{\}{}{%
    \renewcommand\given{\SetSymbol[\delimsize]}%
    \ifblank{#1}{\:\cdot\:}%
    #1}%
\DeclarePairedDelimiterXPP\idx[2]{\operatorname{\mathcal{I}}_{#2}}(){}{\ifblank{#1}{\:\cdot\:}#1}%

\let\gcd\relax%
\let\exp\relax%
\let\log\relax%
\let\ln\relax%
\let\max\relax%
\let\min\relax%
\DeclarePairedDelimiterXPP{\gcd}[1]{\operatorname{gcd}}\{\}{}{%
    \ifblank{#1}{\:\cdot\:}%
    #1}%
\DeclarePairedDelimiterXPP{\exp}[1]{\operatorname{exp}}\{\}{}{%
    \ifblank{#1}{\:\cdot\:}%
    #1}%
\DeclarePairedDelimiterXPP{\log}[1]{\operatorname{log}}(){}{%
    \ifblank{#1}{\:\cdot\:}%
    #1}%
\DeclarePairedDelimiterXPP{\ln}[1]{\operatorname{ln}}(){}{%
    \ifblank{#1}{\:\cdot\:}%
    #1}%
\DeclarePairedDelimiterXPP{\min}[1]{\operatorname{min}}\{\}{}{%
    \ifblank{#1}{\:\cdot\:}%
    #1}%
\DeclarePairedDelimiterXPP{\max}[1]{\operatorname{max}}\{\}{}{%
    \ifblank{#1}{\:\cdot\:}%
    #1}%
\DeclarePairedDelimiterXPP{\diag}[1]{\operatorname{diag}}(){}{%
    \ifblank{#1}{\:\cdot\:}%
    #1}%
\DeclarePairedDelimiterXPP{\sign}[1]{\operatorname{sign}}(){}{%
    \ifblank{#1}{\:\cdot\:}%
    #1}%
\DeclarePairedDelimiterXPP{\expit}[1]{\operatorname{expit}}(){}{%
    \ifblank{#1}{\:\cdot\:}%
    #1}%
\DeclarePairedDelimiterXPP{\logit}[1]{\operatorname{logit}}(){}{%
    \ifblank{#1}{\:\cdot\:}%
    #1}%
\DeclarePairedDelimiterXPP{\order}[1]{\operatorname{o}}(){}{%
    \ifblank{#1}{\:\cdot\:}%
    #1}%
\DeclarePairedDelimiterXPP{\ord}[1]{\operatorname{ord}}(){}{%
    \ifblank{#1}{\:\cdot\:}%
    #1}%

\newcommand{\HN}{\text{H}_0}%
\newcommand{\HA}{\text{H}_{\text{A}}}%
\newcommand{\iid}{\overset{\text{iid}}{\sim}}%

\newcommand{\QR}{\text{QR}}%
\newcommand{\NR}{\text{NR}}%

% Discrete Distributions
\DeclarePairedDelimiterXPP{\BERN}[1]{\text{BERN}}(){}{#1}%
\DeclarePairedDelimiterXPP{\BIN}[1]{\text{BIN}}(){}{#1}%
\DeclarePairedDelimiterXPP{\NBt}[1]{\text{NB}_t}(){}{#1}%
\DeclarePairedDelimiterXPP{\NBf}[1]{\text{NB}_f}(){}{#1}%
\DeclarePairedDelimiterXPP{\GEOt}[1]{\text{GEO}_t}(){}{#1}%
\DeclarePairedDelimiterXPP{\GEOf}[1]{\text{GEO}_f}(){}{#1}%
\DeclarePairedDelimiterXPP{\DU}[1]{\text{DU}}(){}{#1}%
\DeclarePairedDelimiterXPP{\HG}[1]{\text{HG}}(){}{#1}%
\DeclarePairedDelimiterXPP{\POI}[1]{\text{POI}}(){}{#1}%
\DeclarePairedDelimiterXPP{\MN}[1]{\text{MN}}(){}{#1}%

% Continuous Distributions
\let\U\relax%
\DeclarePairedDelimiterXPP{\U}[1]{\text{U}}(){}{#1}%
\DeclarePairedDelimiterXPP{\N}[1]{\mathcal{N}}(){}{#1}%
\DeclarePairedDelimiterXPP{\BetaDist}[1]{\text{Beta}}(){}{#1}%
\DeclarePairedDelimiterXPP{\Erlang}[1]{\text{Erlang}}(){}{#1}%
\DeclarePairedDelimiterXPP{\EXP}[1]{\text{EXP}}(){}{#1}%

\DeclarePairedDelimiterXPP{\RSS}[1]{\text{RSS}}(){}{#1}%

\DeclarePairedDelimiterX\abs[1]\lvert\rvert{%
    \ifblank{#1}{\:\cdot\:}{#1}%
}%
\DeclarePairedDelimiterX\Set[1]\{\}{%
    \renewcommand\given{:}%
    #1%
}%
\DeclarePairedDelimiterX\conf[1]\lbrack\rbrack{#1}
\newcommand{\legendre}[2]{\biggl(\dfrac{#1}{#2}\biggr)}
\DeclareMathOperator*{\argmax}{arg\,max}
\DeclareMathOperator*{\argmin}{arg\,min}
\DeclareMathOperator*{\arginf}{arg\,inf}
\DeclareMathOperator*{\argsup}{arg\,sup}

\providecommand{\RandomVector}[1]{\bm{#1}}% general vectors in bold italic
\providecommand{\Vector}[1]{\bm{#1}}% general vectors in bold italic
\providecommand{\Matrix}[1]{\bm{#1}}
\providecommand{\MatrixCal}[1]{\bm{\mathcal{#1}}}
\providecommand{\Field}[1]{\bm{#1}}

\providecommand{\conj}[1]{\bar{#1}}
\newcommand{\Mod}[1]{\ (\mathrm{mod}\ #1)}
\newcommand{\df}[2]{[D(#1)](#2)}

\usepackage{stackengine}
\usepackage[british]{isodate}
\newcommand{\makeheading}[2]%
{%
\begin{center}%
    \makebox[\linewidth]{\raisebox{-.5ex}[0cm][0cm]{\stackanchor{\textcolor{Gray}{\textsc{#1}}}{\emph{\scriptsize\printyearoff#2}}\;}\color{Crimson!50}\hrulefill}%
\end{center}%
}%

\usepackage[theorems,breakable]{tcolorbox}
% Definitions
\definecolor{myyellow}{RGB}{255,255,168}
% Theorems
\definecolor{mypurple}{RGB}{216,216,255}
% Algorithms
\definecolor{mygray}{RGB}{232,232,232}
% Examples
\definecolor{mygreen}{RGB}{216,255,216}
% Exercises
\definecolor{myred}{RGB}{255,216,216}
% Remarks
\definecolor{mycyan}{RGB}{204,229,229}

\tcbset{
    common/.style={
            fonttitle=\bfseries,
            coltitle=black,
            boxrule=0pt,
            breakable
        },
    theorem/.style={
            common,
            colback=mypurple,
            colframe=mypurple!95!black,
            fontupper=\itshape{}
        },
}


\newtcbtheorem[number within=section, crefname={definition}{definitions}]
{Definition}{DEFINITION}{
    common,
    colback=myyellow,
    colframe=myyellow!95!black
}{def}

\newtcbtheorem[number within=section, crefname={example}{examples}]
{Example}{EXAMPLE}{
    common,
    colback=mygreen,
    colframe=mygreen!95!black,
}{ex}

\newtcbtheorem[number within=section, crefname={exercise}{exercises}]
{Exercise}{EXERCISE}{
    common,
    colback=myred,
    colframe=myred!95!black,
}{exercise}

\newtcbtheorem[number within=section, crefname={remark}{remarks}]
{Remark}{REMARK}{
    common,
    colback=mycyan,
    colframe=mycyan!95!black,
}{remark}

\newtcbtheorem[number within=section, crefname={theorem}{theorems}]
{Theorem}{THEOREM}{
    theorem
}{thm}

\newtcbtheorem[number within=section, crefname={proposition}{propositions}]
{Proposition}{PROPOSITION}{
    theorem
}{prop}

\newtcbtheorem[number within=section, crefname={corollary}{corollaries}]
{Corollary}{COROLLARY}{
    theorem
}{cor}

\newtcbtheorem[number within=section, crefname={lemma}{lemmas}]
{Lemma}{LEMMA}{
    theorem
}{lem}

\hypersetup{colorlinks=true,%
linkcolor=[rgb]{0,0.5,1},%
pdftitle={Elementary Number Theory (PMATH 340)},%
pdfauthor={Salma Shaheen},%
pdfsubject={Pure Mathematics},%
pdfkeywords={University of Waterloo, Fall 2021 (1219)}}%

\title{%
\LARGE Elementary Number Theory\\%
\large PMATH 340\\%
\normalsize Spring 2022 (1225)}%
\author{Cameron Roopnarine\thanks{\LaTeX{}er}\and Salma Shaheen\thanks{Instructor}}%
\date{\today}%

\usepackage{tikz}
\usetikzlibrary{petri,decorations.pathreplacing,calc}

\begin{document}
\maketitle
\tableofcontents
\newpage
\makeheading{Lecture 1}{\printdate{2022-01-11}}%chktex 8
\section*{Order Statistics}
Let $ X_1,X_2,\ldots,X_n $ be a random sample of size $ n $ from a population with CDF $ F(x) $
and PDF $ f(x) $. Let $ X_{1:n}\le \cdots \le X_{n:n} $ denote the corresponding order statistics obtained by arranging the $ X_i $'s
in increasing (non-decreasing) order of magnitude. Then, their distributions, dependence properties,
moments, characteristics, etc.\ can be made use of effectively to develop inferential methods.
\section*{Binomial Derivation}
The CDF of $ X_{r:n} $, for $ r=1,2,\ldots,n $, is
\begin{align*}
    F_{r:n}(x)
     & =\Prob{X_{r:n}\le x}                                                                                                    \\
     & =\Prob{\text{at least $ r $ of the $ X_i $'s are $ \le x $}}                                                            \\
     & =\sum_{i=r}^{n}\Prob{\text{exactly $ i $ of $ x_i $'s are $ \le x_i $}} &  & \text{because they are mutually exclusive} \\
     & =\sum_{i=r}^{n}\binom{n}{i}\bigl(F(x)\bigr)^i\bigl(1-F(x)\bigr)^{n-i}                                                   \\
     & =I_{F(x)}(r,n-r+1),
\end{align*}
where
\[ I_p(a,b)=\frac{1}{B(a,b)}\int_{0}^{p}t^{a-1}(1-t)^{b-1}\odif{t},\text{ for } 0<p<1, \]
is the Incomplete Beta Ratio function.

\section*{Pearson's Identity}
For $ 0<p<1 $,
\[ I_p(r,n-r+1)=\sum_{i=r}^{n}\binom{n}{i}p^i (1-p)^{n-i},\text{ for } r=1,2,\ldots,n. \]
It connects the survival function of binomial distribution
with the cumulative distribution function of beta distribution. (Proof by integration of parts)
\begin{Remark}{}{}
    The expression of the CDF of $ X_{r:n} $ in (1) holds whether the distribution $ F(x) $ is continuous or discrete.
    \tcblower{}
    If the distribution is continuous, then the PDF of $ X_{r:n} $, for $ r=1,\ldots,n $, can be obtained from (1) as
    \begin{align*}
        f_{r:n}(x)
         & =\odv*{F_{r:n}(x)}{x}                                                       \\
         & =\odv*{I_{F(x)}(r,n-r+1)}{x}                                                \\
         & =\odv*{\frac{1}{B(r,n-r+1)}\int_{0}^{F(x)}t^{r-1}(1-t)^{n-r}\odif{t}}{x}    \\
         & =\frac{1}{B(r,n-r+1)}\bigl(F(x)\bigr)^{r-1}\bigl(1-F(x)\bigr)^{n-r}f(x)     \\
         & =\frac{n!}{(r-1)!(n-r)!}\bigl(F(x)\bigr)^{r-1}\bigl(1-F(x)\bigr)^{n-r}f(x).
    \end{align*}
    If the population distribution $ F(x) $ is discrete, however, the above method can not be used. But,
    we can find the PDF of $ X_{r:n} $ (for $ r=1,\ldots,n $) as
    \begin{align*}
        f_{r:n}(x)
         & =\Prob{X_{r:n}=x}                                                   \\
         & =F_{r:n}(x)-F_{r:n}(x-)                                             \\
         & =\frac{1}{B(r,n-r+1)}\int_{F(x-)}^{F(x)}t^{r-1}(1-t)^{n-r}\odif{t}.
    \end{align*}
\end{Remark}
\section*{Derivation from Jacobian}
Now, let us focus on the case when the population distribution is continuous. In this case, with $ f(x) $
as the PDF, due to independence of $ X_i $'s, their joint density is
\[ f_{X_1,\ldots,X_n}(x_1,\ldots,x_n)=\prod_{i=1}^{n}f(x_i),\quad x_i\in S. \]
Now, let us introduce the transformation
\[ X_{1:n}=\min{X_1,\ldots,X_n},\ldots,X_{n:n}=\max{X_1,\ldots,X_n}. \]
Then, evidently, it is an $ n! $-to-$ 1 $ transformation, and so the joint density
function of $ (X_{1:n},\ldots,X_{n:n}) $ is
\[ f_{1,\ldots,n:n}(x_1,\ldots,x_n)=n!\prod_{i=1}^{n}f(x_i),\; x_1<x_2<\cdots<x_n. \]
From (4), we can obtain, by integrating out $ (x_{r+1},\ldots,x_n) $ and $ (x_1,\ldots,x_{r-1}) $,
the PDF of $ X_{r:n} $ ($ r=1,\ldots,n $) as follows:
\[ \underset{x_r<x_{r+1}<\cdots<x_n}{\int\int\cdots\int}f(x_{r+1})\cdots f(x_n)\odif{x_n}\odif{x_{n-1}}\cdots\odif{x_{r+1}}=\frac{[1-F(x_r)]^{n-r}}{(n-r)!} \]
and
\[ \underset{x_1<x_2<\cdots<x_r}{\int\int\cdots\int}f(x_1)\cdots f(x_{r-1})\odif{x_1}\odif{x_{2}}\cdots\odif{x_{r-1}}=\frac{[F(x_r)]^{r-1}}{(r-1)!} \]
so that we obtain the PDF of $ X_{r:n} $ as
\[ f_{r:n}(x_r)=\frac{n!}{(r-1)!(n-r)!}\bigl(F(x_r)\bigr)^{r-1}\bigl(1-F(x_r)\bigr)^{n-r}f(x_r). \]
Similarly, from (4), we can obtain, by integrating out
$ (x_{s+1},\ldots,x_n) $, $ (x_{r+1},\ldots,x_{s-1}) $
and $ (x_1,\ldots,x_{r-1}) $, the joint PDF
of $ (X_{r:n},X_{s:n}) $, for $ 1\le r<s\le n $, as follows:
\[ \underset{x_s<x_{s+1}<\cdots<x_n}{\int\int\cdots\int}f(x_{s+1})\cdots f(x_n)\odif{x_n}\odif{x_{n-1}}\odif{x_{s+1}}=\frac{[1-F(x_s)]^{n-s}}{(n-s)!}, \]
\[ \mathop{\int\int\cdots\int}\limits_{x_r<x_{r+1}<\cdots<x_{s-1}<x_s}f(x_{r+1})\cdots f(x_{s-1})\odif{x_{r+1}}\odif{x_{r+2}}\odif{x_{s-1}}=\frac{[F(x_s)-F(x_r)]^{s-r-1}}{(s-r-1)!}, \]
and
\[ \underset{x_1<x_2<\cdots<x_{r-1}<x_r}{\int\int\cdots\int}f(x_{1})\cdots f(x_{r-1})\odif{x_1}\odif{x_2}\odif{x_{r-1}}=\frac{[F(x_r)]^{r-1}}{(r-1)!}, \]
so that we can obtain the joint PDF of $ (X_{r:n},X_{s:n}) $,
for $ 1\le r<s\le n $ as
\begin{align*}
    f_{r,s:n}(x_r,x_s) & =\frac{n!}{(r-1)!(s-r-1)!(n-s)!}\bigl(F(x_r)\bigr)^{r-1}\bigl(F(x_s)-F(x_r)\bigr)^{s-r-1}\bigl(1-F(x_s)\bigr)^{n-s}f(x_r)f(x_s), \\
                       & \quad\text{ for } x_r<x_s.
\end{align*}
\section*{Multinomial Derivation}
For directly deriving the PDF of $ X_{r:n} $, let us consider
\[ \Prob{x\le X_{r:n}\le x+\Delta x}=
    \frac{n!}{(r-1)!1!(n-r)!}\bigl(F(x)\bigr)^{r-1}\bigl(F(x+\Delta x)-F(x)\bigr)^{1}\bigl(1-F(x+\Delta x)\bigr)^{n-r}
    +\mathcal{O}((\Delta x)^2), \]
where $ \mathcal{O}((\Delta x)^2) $ corresponds to more than
one $ x_i $ in the interval $ (x,x+\Delta x) $.
Then, we obtain the density of $ X_{r:n} $ as follows:
\begin{align*}
    f_{r:n}(x)
     & =\lim\limits_{{\Delta x} \to {0}}\frac{\Prob{x\le X_{r:n}\le x+\Delta x}}{\Delta x}                                       \\
     & =\frac{n!}{(r-1)!1!(n-r)!}\bigl(F(x)\bigr)^{r-1}\lim\limits_{{\Delta x} \to {0}}\biggl[\bigl(F(x+\Delta x)-F(x)\bigr)^{1} \\
     & \phantom{{}={}}\quad \bigl(1-F(x+\Delta x)\bigr)^{n-r}\biggr] +\mathcal{O}((\Delta x)^2)                                  \\
     & =\frac{n!}{(r-1)!(n-r)!}\bigl(F(x)\bigr)^{r-1}f(x)\bigl(1-F(x)\bigr)^{n-r},
\end{align*}
exactly as before.

Similarly, for deriving the joint PDF
of $ (X_{r:n},X_{s:n}) $, for $ 1\le r<s\le n $, let us consider the multinomial
probability
\begin{align*}
     & \Prob{x<X_{r:n}\le x+\Delta x,y< X_{s:n}\le y+\Delta y}                                                            \\
     & =\frac{n!}{(r-1)!1!(s-r-1)!1!(n-s)!}
    \bigl(F(x)\bigr)^{r-1}\bigl(F(x+\Delta x)-F(x)\bigr)^1                                                                \\
     & \phantom{=}\times\bigl(F(y)-F(x+\Delta x)\bigr)^{s-r-1}
    \bigl(F(y+\Delta y)-F(y)\bigr)^1\bigl(1-F(y+\Delta y)\bigr)^{n-s}                                                     \\
     & \phantom{=}+\mathcal{O}((\Delta x)^2 \Delta y)\to\text{corresponds to more than one $ X_i $ in $ (x,x+\Delta x) $} \\
     & \phantom{=}+\mathcal{O}(\Delta x (\Delta y)^2)\to\text{corresponds to more than one $ X_i $ in $ (y,y+\Delta y) $}
\end{align*}
Then, we obtain the joint density of $ (X_{r:n},X_{s:n}) $ as follows:
\begin{align*}
    f_{r,s:n}(x,y)
     & =\lim\limits_{\Delta {x} \to {0},\Delta y\to 0}
    \frac{\Prob{x<X_{r:n}\le x+\Delta x,y< X_{s:n}\le y+\Delta y}}{\Delta x \Delta y}                                                                                                                                 \\
     & =\frac{n!}{(r-1)!1!(s-r-1)!1!(n-s)!}\bigl(F(x)\bigr)^{r-1}                                                                                                                                                     \\
     & \times\lim\limits_{{\Delta x} \to {0}}\biggl[\frac{\bigl(F(x+\Delta x)-F(x)\bigr)^1}{\Delta x}\bigl(F(y)-F(x+\Delta x)\bigr)^{s-r-1}\biggr]                                                                    \\
     & \times \lim\limits_{{\Delta y } \to {0}}\biggl[\frac{\bigl(F(y+\Delta y)-F(y)\bigr)^1}{\Delta y}\bigl(1-F(y+\Delta y)\bigr)^{n-s}\biggr]+\mathcal{O}((\Delta x)^2 \Delta y)+\mathcal{O}(\Delta x (\Delta y)^2) \\
     & =\frac{n!}{(r-1)!(s-r-1)!(n-s)!}\bigl(F(x)\bigr)^{r-1}f(x)\bigl(F(y)-F(x)\bigr)^{s-r-1}f(y)\bigl(1-F(y)\bigr)^{n-s},\; x<y,
\end{align*}
exactly as before.
\begin{Example}{}{}
    Let us consider $ \text{Uniform}(0,1) $ distribution with
    \[ f(x)=1\text{ for }0<x<1,\qquad F(x)=x\text{ for }0<x<1. \]
    Then, from (2), we have the PDF of $ X_{r:n} $ (for $ 1\le r\le n $) to be
    \[ f_{r:n}(x)=\frac{1}{B(r,n-r+1)}x^{r-1}(1-x)^{n-r}\text{ for }0<x<1; \]
    that is,
    \[ X_{r:n}\dist \BetaDist{r,n-r+1}. \]
    So, we readily have
    \[ \E{X_{r:n}}=\frac{r}{n+1}=\pi_r,\qquad \Var{X_{r:n}}=\frac{r(n-r+1)}{(n+1)^2(n+2)}=\frac{\pi_r(1-\pi_r)}{n+2}. \]
    Similarly, from (6), we have the joint PDF of $ (X_{r:n},X_{s:n}) $
    for $ 1\le r<s\le n $, to be
    \[ f_{r,s:n}(x,y)=\frac{1}{B(r,s-r,n-s+1)}x^{r-1}(y-s)^{s-r-1}(1-y)^{n-s}\text{ for }0<x<y<1, \]
    which implies
    \[ (X_{r:n},X_{s:n})\dist \text{BivBeta}(r,s-r,n-s+1). \]
    From this, we can readily find, for $ 1\le r<s\le n $,
    \begin{align*}
        \Cov{X_{r:n},X_{s:n}}
         & =\E{X_{r:n}X_{s:n}}-\E{X_{r:n}}\E{X_{s:n}} \\
         & =\frac{\pi(n-s+1)}{(n+1)^2(n+2)}           \\
         & =\frac{\pi_r(1-\pi_s)}{n+2};
    \end{align*}
    observe that they are always positively correlated. Moreover,
    \[ \rho_{X_{r:n},X_{s:n}}=\frac{\Cov{X_{r:n},X_{s:n}}}{\sqrt{\Var{X_{r:n}}\Var{X_{s:n}}}}
        =\sqrt{\frac{\pi_r}{1-\pi_r}\times\frac{1-\pi_s}{\pi_s}}, \]
    free of $ n $ (just a function of proportions $ \frac{r}{n+1} $ and $ \frac{s}{n+1} $).
\end{Example}
\section*{Probability Integral Transform}
Suppose $ X $ is a continuous random variable with CDF $ F(x) $ and PDF
$ f(x) $. Then, the transformed variable $ U=F(x) $ is uniformly distributed
over $ (0,1) $.

\textbf{Proof}: For $ u\in (0,1) $, consider
\begin{align*}
    \Prob{U\le u}
     & =\Prob{F(x)\le u} \\
     & =\Prob{X\le Q(u)} \\
     & =F(Q(u))          \\
     & =u,
\end{align*}
where $ Q(u) $ is the quantile function (i.e., it is $ F^{-1} $ in the case
of absolute continuous function), which means
$ U=F(x) $ is $ \text{Uniform}(0,1) $.

\begin{Remark}{}{}
    The above result will hold even if the population distribution
    is not absolutely continuous, but has discontinuities.
    All we have to do is take $ Q $ as the generalized quantile
    function with right inverse.

    Since $ F(x) $ is a non-decreasing function, if we have
    $ (X_{1:n},X_{2:n},\ldots,X_{n:n}) $ as order statistics
    from a continuous distribution with CDF $ F(x) $, then the
    transformed variables $ \bigl(F(X_{1:n}),F(X_{2:n}),\ldots,F(X_{n:n})\bigr) $
    will be distributed as $ \text{Uniform}(0,1) $ order statistics,
    $ (U_{1:n},U_{2:n},\ldots,U_{n:n}) $ no matter what the distribution
    of $ F(\:\cdot\:) $ is!
\end{Remark}

\section*{Probability-Probability Plot}
One important application of the previous result is in model validation methods.
Because
\[ \bigl(F(X_{1:n},\ldots,F(X_{n:n}))\bigr)\text{ and }
    (U_{1:n},\ldots,U_{n:n}) \]
have identical distributions no matter what the population
distribution $  F(\:\cdot\:) $ is, we can use it to examine whether an
assumed $  F(\:\cdot\:) $ is reasonable for the data at hand. This is done by a
P-P plot as follows:
\begin{itemize}
    \item Step 1: From the given data $ (x_{1:n},\ldots,x_{n:n}) $,
          estimate the parameters of the assumed model $  F(\:\cdot\:) $;
    \item Step 2: With the estimated parameter values,
          say $ \hat{\theta} $, find the values of
          \[ \bigl(F(x_{1:n};\hat{\theta}),F(x_{2:n};\hat{\theta}),\ldots,
              F(x_{n:n};\hat{\theta})\bigr). \]
          These are the ``empirical'' (or observed) probabilities;
    \item Step 3: Plot these against the ``theoretical''
          probabilities
          \[ \biggl(\E{U_{1:n}}=\frac{1}{n+1},\E{U_{2:n}}=\frac{2}{n+1},\ldots,
              \E{U_{n:n}}=\frac{n}{n+1}\biggr). \]
          A near straight line fit would provide support for the assumed model
          $  F(\:\cdot\:) $.
\end{itemize}
\begin{Remark}{}{}
    One can also indicate variability at each point by estimating
    $ \Var{F(x_{i:n})} $ (using delta method).
\end{Remark}
\section*{Quantile-Quantile Plot}
Another important and related application is the Q-Q plot.
In it, we invert the distributional relationship to use
\[ (X_{1:n},X_{2:n},\ldots,X_{n:n})\text{ and }
    (F^{-1}(U_{1:n}),F^{-1}(U_{2:n}),\ldots,
    F^{-1}(U_{n:n})) \]
have identical distribution, where $ Q\equiv F^{-1}(\:\cdot\:) $
is the quantile function of the assumed model. Then,
the $ Q-Q $ plot proceeds as follows:
\begin{itemize}
    \item Step 1: Determine the order statistics from the given data,
          $ x_{1:n},x_{2:n},\ldots,x_{n:n} $, which are the empirical quantiles;
    \item Plot them against the theoretical quantiles
          \[ \biggl(F^{-1}\biggl(\frac{1}{n+1}\biggr),F^{-1}\biggl(\frac{2}{n+1}\biggr),\ldots,
              F^{-1}\biggl(\frac{n}{n+1}\biggr)\biggr). \]
          Once again, a near straight line fit would provide support for the assumed
          model $ F(\:\cdot\:) $.
\end{itemize}
\begin{Remark}{}{}
    Once again, we can indicate the variability at each point by estimating
    $ \Var{X_{i:n}} $.
\end{Remark}
\begin{Remark}{}{}
    In both cases, rather than making a qualitative assessment on
    ``near straight line fit'', can can make it quantitatively
    by using the correlation coefficient or any other
    ``measure of fit'' (correlation-type goodness-of-fit test).
\end{Remark}
\begin{Remark}{}{}
    Note that estimation of the model parameters is avoided in Q-Q plot,
    but is necessary in a P-P plot!
\end{Remark}
\section*{Pivot 1}
A pivot is a random variable, which is a function of the data
and the parameter of the model, whose distribution is free of the parameter(s).

Pivots are essential quantities for developing inferential methods such as
interval estimation, hypothesis tests, etc.

\begin{Example}{}{}
    Let $ X_1,\ldots,X_n $ be a random sample from $ \EXP{\theta} $
    distribution with PDF
    \[ f(x;\theta)=\frac{1}{\theta}e^{-x/\theta},\; x>0,\, \theta>0. \]
    Then, it is well-known that
    \[ \sum_{i=1}^{n}X_i \sim \GAM{n,\theta}, \]
    where $ n $ is the shape parameter and $ \theta $ is the scale parameter
    of the Gamma distribution. Hence,
    \[ Y=\frac{\sum_{i=1}^{n}X_i}{\theta} \]
    is a pivot (for $ \theta $) since its distribution is $ \GAM{n,1} $
    which is free of the parameter $ \theta $.
\end{Example}
\begin{Example}{}{}
    Suppose $ X_1,\ldots,X_n $ is a random sample from
    $ \N{\mu,\sigma^2} $ distribution.
    Let $ \bar{X} $ and $ S^2 $ denote the sample mean
    and sample variance, respectively. Then, it
    is known that
    \[ \bar{X} \sim \N*{\mu,\frac{\sigma^2}{n}}\text{ and }\frac{(n-1)S^2}{\sigma^2}\sim \chi^2_{n-1}. \]
    So, if we consider
    \[ \bar{X}-\mu \sim \N*{0,\frac{\sigma^2}{n}}, \]
    it will not be a pivot if $ \sigma^2 $ is unknown, and it will be a pivot
    for $ \mu $ only if $ \sigma^2 $ is known. However,
    if we consider
    \[ \frac{\bar{X}-\mu}{S/\sqrt{n}}, \]
    it will be a pivot (for $ \mu $) since its distribution will be
    Student's $ t $ distribution with $ n-1 $ degrees of freedom,
    as it is free of both $ \mu $ and $ \sigma^2 $. Similarly,
    \[ \frac{(n-1)S^2}{\sigma^2} \]
    will be a pivot (for $ \sigma^2 $) as its distribution is central $ \chi^2 $
    distribution with $ n-1 $ degrees of freedom, and is free of $ \mu $
    and $ \sigma^2 $.
\end{Example}
\begin{Example}{}{}
    Suppose $ X_1,X_2,\ldots,X_n $ is a random sample from
    $ \U{0,\theta} $ distribution. Let $ X_{1:n},\dots,X_{n:n} $
    be the corresponding order statistics. Then,
    \[ T=\frac{X_{n:n}}{\theta} \]
    is a pivot (for $ \theta $) since its density function is
    \[ f_T(t)=nt^{n-1},\; 0<t<1, \]
    which is free of $ \theta $.
\end{Example}
\begin{Remark}{}{}
    All the pivotal quantities discussed above are all
    ``parametric pivots'' as they are pivots for
    parameters of a specific parametric model assumed.
\end{Remark}
\begin{Example}{}{}
    Let $ X_1,\ldots,X_n $ be a random sample from
    $ \BERN{\pi} $ distribution, where $ \pi\in(0,1) $ is the probability
    of success. Then, it is well-known that
    \[ Y=\sum_{i=1}^{n}X_i \sim \BIN{n,\pi}. \]
    As the distribution of $ Y $ depends on the parameter $ \pi $,
    it is not a pivot. In fact, no exact pivot exists in this case.

    However, by the use of Central Limit Theorem, it is known that
    \[ Z=\frac{Y-n\pi}{\sqrt{n\pi(1-\pi)}}\stackrel{\text{asymp.}}{\sim}\N{0,1}. \]
    As the distribution of $ Z $ is, asymptotically, standard normal,
    which is free of $ \pi $, it can serve as a pivot. But, note that
    it is only an approximate pivot for $ \pi $.
\end{Example}
\section*{Pivot for Population Quantile}
Suppose we have a random sample from $ \N{\mu,\sigma^2} $
distribution, and that we are interested in inferring about $ p $-th
quantile $ \xi_p=\mu+\sigma z_p $, where $ z_p $ is the $ p $-th quantile
of the standard normal distribution. Then, it is evident that, with
$ \bar{X} $ and $ S $ as estimates of $ \mu $ and $ \sigma $
respectively, then an estimate of $ \xi_p $ is
\[ \hat{\xi}_p=\hat{\mu}+\hat{\sigma}z_p=\bar{X}+z_p S. \]
Then, the variable
\begin{align*}
    T
     & =\frac{\hat{\xi}_p-\xi_p}{S}                                 \\
     & =\frac{(\bar{X}+z_p S)-(\mu+z_p\sigma)}{S}                   \\
     & =\frac{(\bar{X}-\mu)+z_p(S-\sigma)}{S}                       \\
     & =\frac{\bar{X}-\mu}{S}+z_p\biggl(1-\frac{1}{S/\sigma}\biggr)
\end{align*}
is a pivot as its distribution is free of parameters $ \mu $
and $ \sigma $. Hence, this pivot could be used for developing inference for
the $ p $-th quantile $ \xi_p $ of the normal distribution.
\begin{Exercise}{}{}
    Can you think of a way to find its percentage points?
\end{Exercise}
\begin{Remark}{}{}
    Though the above derivation is shown for normal distribution,
    it can be done similarly for any member of location-scale family
    of distributions like Logistic, Laplace, Gumbel distributions.
\end{Remark}
\section*{Non-parametric Confidence Interval for Quantile}
Now, let us assume we have a random sample
$ X_1,X_2,\ldots,X_n $ from a distribution function $ F(x) $
that is continuous. Let $ \xi_p $ be the $ p $-th quantile
of $ F $. Then,
$ F(\xi_p)=\Prob{X\le \xi_p}=p $. We are interested in a confidence interval
for $ \xi_p $, but without assuming a specific form of the
distribution $ F $, like normal!

Let $ X_{1:n},X_{2:n},\ldots,X_{n:n} $ denote the order statistics obtained
from the sample. Let $ X_{r:n} $ and $ X_{s:n} $
be two selected order statistics, for $ 1\le r<s\le n $. Then, we have:
\begin{align*}
    \Prob{X_{r:n}\le \xi_p\le X_{s:n}}
     & =\Prob{F(X_{r:n})\le F(\xi_p)\le F(X_{s:n})}                                         \\
     & =\Prob{U_{r:n}\le p\le U_{s:n}}                                                      \\
     & =\Prob{p\le U_{s:n}}-\Prob{p<U_{r:n}}                                                \\
     & =1-\Prob{U_{s:n}<p}-\bigl(1-\Prob{U_{r:n}<p}\bigr)                                   \\
     & =\Prob{U_{r:n}<p}-\Prob{U_{s:n}<p}                                                   \\
     & =\sum_{i=r}^{p}\binom{n}{i}p^i (1-p)^{n-i}-\sum_{i=s}^{n}\binom{n}{i}p^i (1-p)^{n-i} \\
     & =\sum_{i=r}^{s-1}\binom{n}{i}p^i (1-p)^{n-i},
\end{align*}
where $ U_{r:n} $ and $ U_{s:n} $ are order statistics from $ \text{Uniform}(0,1) $ distribution.
Thus, $ (X_{r:n},X_{s:n}) $ is a non-parametric confidence interval for the $ p $-th
population quantile $ \xi_p $, with its coverage probability not
depending on $ F $, but only on $ p $ and $ n $.

So, for a given confidence level $ 1-\alpha $,
all we need to do is, for a given sample size $ n $ and the quantile
level $ p $, we need to determine integers $ r $ and $ s $ such that
\[ 1\le r<s\le n \]
and
\[ \sum_{i=r}^{s-1}\binom{n}{i}p^i (1-p)^{n-i}\approx 1-\alpha. \]
Note that $ 1-\alpha $ may not be achievable exactly as the binomial
distribution is discrete and so has jumps.
\begin{Remark}{}{}
    The choice of $ r $ and $ s $ may not be unique. So,
    if there is more than one choice of $ (r,s) $ satisfying
    the above conditions, then it would be meaningful to choose
    that pair $ (r,s) $ for which
    \[ s-r\text{is the smallest} \]
    among all these choices satisfying the conditions. This would then correspond
    to the ``narrowest non-parametric confidence interval for population quantile.''
\end{Remark}
\makeheading{Lecture 2}{\printdate{2022-01-10}}%chktex 8

\textbf{Survey samples}:
A survey sample, denoted as $S$, is a subset of the population
$ U=\Set{1,2,\ldots,N} $.

Sample size $ n=\abs{S} $ is the number
of units in the sample:
\[ S=\Set{i_1,i_2,\ldots,i_n}\text{a set of $n$ ``unordered'' units}. \]
We could simply use $ S=\Set{1,2,\ldots,n} $.

$ N=10 $, $ n=3 $:
\begin{align*}
      U & =\Set{1,2,3,4,5,6,7,8,9,10} \\
      S & =\Set{7,4,9}                \\
        & =\Set{i_1,i_2,i_3}.         \\
      S & =\Set{1,2,3}.
\end{align*}

\textbf{Non-probability samples versus probability samples}:\\
\textbf{Non-probability samples} are selected by subjective or any
convenient methods. Examples include
\begin{itemize}
      \item \emph{Quota sampling}: The sample is obtained by a number of
            interviewers, each of whom is required to sample certain
            numbers of units with certain types or characteristics. How to
            select the units is completely left in the hands of the interviewers.
      \item \emph{Judgement or purposive sampling}: The sample is selected based
            on what the sampler believes to be ``typical'' or ``most
            representative'' of the population.
      \item \emph{Restricted sampling}: The sample is restricted to certain parts of
            the population which are readily accessible.
      \item \emph{Sample of convenience}: The sample is taken from those who are
            easy to reach.
      \item \emph{Sample of volunteers}: The sample consists of those who
            volunteer to participate.
      \item \emph{Web panels}: The sample is selected from a panel of people who
            signed up to do surveys in order to receive cash or other
            incentives.
\end{itemize}

The most serious issue with non-probability survey samples:\\
Biased sample with unknown inclusion probabilities.

Non-probability survey samples are not the focus of this course. But
the topic is becoming important in recent years, since data from
non-probability survey samples become useful sources.

Yilin Chen's PhD thesis research is on statistical analysis with
non-probability survey samples, to be introduced in the last lecture.

\textbf{Probability samples}, theoretically speaking, are selected through a
probability measure over a pool of candidate samples.
Let
\[ \Omega=\Set{S\given S\subseteq U} \]
be the set of all possible subsets of the survey population $U$. Let $ \mathcal{P} $ be
a probability measure over $ \Omega $ such that
\[ \mathcal{P}(S)\ge 0\text{ for any $ S\in \Omega $ and} \sum_{S:S\in\Omega}\mathcal{P}(S)=1. \]
A probability sample $ S $ is selected based on the \textbf{probability sampling
      design}, $ \mathcal{P} $.

$ \mathcal{P}(\:\cdot\:) $ is a discrete probability measure.

\textbf{Example 1.4}. $ N=3 $; $ U=\Set{1,2,3} $, $ n=1 $ or $ 2 $.
\begin{itemize}
      \item $ n=1 $: $ S_1=\Set{1} $, $ S_2=\Set{2} $, $ S_3=\Set{3} $.
      \item $ n=2 $: $ S_4=\Set{1,2} $, $ S_5=\Set{1,3} $, $ S_6=\Set{2,3} $.
      \item $ n=3 $: $ S_7=\Set{1,2,3} $ (census).
\end{itemize}
\[ \begin{NiceArray}{c|ccccccc}
            S              & S_1 & S_2 & S_3 & S_4 & S_5 & S_6 & S_7 \\
            \midrule
            \mathcal{P}(S) & 1/6 & 1/6 & 1/6 & 1/6 & 1/6 & 1/6 & 0   \\
            \midrule
            \mathcal{P}(S) & 0   & 0   & 0   & 1/3 & 1/3 & 1/3 & 0   \\
            \midrule
            \mathcal{P}(S) & 0   & 0   & 0   & 1/2 & 1/4 & 1/4 & 0
      \end{NiceArray} \]
Note that we have $ \mathcal{P}(S)\ge 0 $ for any $ S\in \Omega $ and
$ \sum_{\Set{S\given S\in\Omega}}\mathcal{P}(S)=1 $.

\textbf{Sampling design $ \mathcal{P} $ with fixed sample size}: $ \mathcal{P}(S)=0 $
if $ \abs{S}\ne n $. The probability measure is defined over
\[ \Omega_n=\Set{S\given S\subseteq U\text{ and }\abs{S}=n}. \]

\textbf{The cumulative sum method for generating a discrete random
      variable}:
\[ X \sim f(x):\quad p_i=f(x_i)=\Prob{X=x_i},\; i=1,2,\ldots. \]
\begin{itemize}
      \item \emph{Step 1}. Probability cumulation.
            \begin{align*}
                  b_0 & =0                 \\
                  b_1 & =p_1               \\
                  b_2 & =p_1+p_2           \\
                  b_3 & =p_1+p_2+p_3       \\
                      & \vdotswithin{=}    \\
                  b_j & =\sum_{i=1}^{j}p_i \\
                      & \vdotswithin{=}
            \end{align*}
      \item \emph{Step 2}. Generate $ r\sim U(0,1) $.
      \item \emph{Step 3}. Let $ X^\star=x_j $ if $ b_{j-1}<r\le b_j $.
\end{itemize}
Can show $ X \sim f(x) $.

\textbf{Survey variables and population parameters}:
\begin{itemize}
      \item $ y $: the response variable; $ \Vector{x} $
            the vector of auxiliary variables.
      \item $ (y_i;\Vector{x}_i) $: the values of $ (y,\Vector{x}) $
            associated with unit $ i $, $ i=1,2,\ldots,N $.
      \item A common assumption in survey sampling: the values $ (y_i,\Vector{x}_i) $
            can be measured without error if $ i $ is selected in the sample.
      \item Population totals:
            \[ T_y=\sum_{i=1}^{N}y_i\text{ and }T_{\Vector{x}}=\sum_{i=1}^{N}\Vector{x}_i. \]
      \item Population means:
            \[ \mu_y=\frac{1}{N}\sum_{i=1}^{N}y_i\text{ and }\mu_{\Vector{x}}=\frac{1}{N}\sum_{i=1}^{N}\Vector{x}_i. \]
      \item Population variance of $ y $:
            \[ \sigma_y^2=\frac{1}{N-1}\sum_{i=1}^{N}(y_i-\mu_y)^2=\frac{1}{N-1}\biggl(\sum_{i=1}^{N}y_i^2-N\mu_y^2\biggr). \]
\end{itemize}

\textbf{An important special case: $y$ is a binary variable}:
\[ y_i=\begin{cases}
            1, & \text{if unit $ i $ has attribute ``$A$''}, \\
            0, & \text{otherwise}.
      \end{cases} \]
$ N $: the total number of units in the population (population size).
$ M $: the total number of units in the population having attribute ``$A$.''
\begin{itemize}
      \item Population total:
            \[ T_y=\sum_{i=1}^{N}y_i=M. \]
      \item Population mean:
            \[ \mu_y=\frac{T_y}{N}=\frac{M}{N}=P, \]
            where $ P $ is the population proportion of units with attribute ``A.''
      \item Population variance:
            \begin{align*}
                  \sigma_y^2
                   & =\frac{1}{N-1}\biggl(\sum_{i=1}^{N}y_i^2-N\mu_y^2\biggr)                                 \\
                   & =\frac{1}{N-1}(M-NP^2)                                                                   \\
                   & =\frac{N}{N-1}P(1-P)                                                                     \\
                   & \approx P(1-P)                                           &  & \text{ if $ N $ is large}.
            \end{align*}
\end{itemize}

\textbf{Probability sampling and design-based inference}:
\begin{itemize}
      \item The survey population $ U=\Set{1,2,\ldots,N} $ is viewed as fixed.
      \item The values $ y_i $ and $ \Vector{x}_i $ attached to unit $ i $
            and the population parameters such as $ T_y $ and $ \mu_y $
            are also viewed as fixed.
      \item The values of the population parameters can be determined without error by conducting
            a census.
      \item The sample $ S $ is selected according to a probability sampling design $ \mathcal{P} $.
      \item The sample $ S $ is a random set under $ \mathcal{P} $.
      \item Each unit in the population has a probability to be included in the
            sample.
      \item Randomization is induced by the probability sampling design for
            the selection of the survey sample.
\end{itemize}

\textbf{Basic sampling techniques and advanced topics}:
\begin{itemize}
      \item Basic sampling techniques and theory are developed for the
            estimation of the population total $T_y$ and the population mean $ \mu_y $.\\
            (Chapters 1--5 in the textbook)
      \item The basic methods and theory can be extended to handle more
            advanced topics, such as design-based regression analysis using
            survey data.\\
            (Chapters 6--11 in the textbook)
\end{itemize}

\chapter{Review of Simple Random Sampling}
\section{Simple Random Sampling Without Replacement (SRSWOR)}
The task: Select a sample of size $n$ from a population of size $N$ with
equal probability among all candidate samples.

The total number of candidate samples: $ \binom{N}{n}=\frac{N(N-1)\cdots(N-n+1)}{n!} $.

The probability measure for the sampling design:
\[ \mathcal{P}(S)
      =\begin{cases}
            \frac{1}{\binom{N}{n}}, & \text{if $\abs{S}=n$}     \\
            0,                      & \text{if $\abs{S}\ne n$}.
      \end{cases} \]
$ \mathcal{P}(S) $ cannot be used to select a sample in practice. $ N=1000 $, $ n=3 $:
\[ \binom{N}{n}=\frac{1000\times 999\times 998}{3}. \]
$ \mathcal{P}(S) $ is a theoretical tool.

\textbf{Sampling scheme or sampling procedure}:
Select the survey sample through a sequential draw-by-draw method;
select units from the sampling frame, one-at-a-time, until the final
sample is chosen.

\textbf{SRSWOR} is a sampling procedure to select a sample of size $n$ with
equal probability among all candidate samples.

\textbf{The sampling frame for SRSWOR}:
A complete list of $N$ units in the population.

\textbf{The SRSWOR sampling procedure}:
\begin{enumerate}
      \item Select the first unit from the $N$ units on the sampling frame with
            equal probabilities $1/N$; denote the selected unit as $i_1$;
      \item Select the second unit from the remaining $ N-1 $ units on the
            sampling frame with equal probabilities $ 1/(N-1) $; denote the
            selected unit as $i_2$;
      \item Continue the process and select the $n\textsuperscript{th}$ unit from the remaining
            $ N-n+1 $ units on the sampling frame with equal probabilities
            $ 1/(N-n+1) $; denote the selected unit as $ i_n $.
\end{enumerate}

\textbf{Theorem 2.1}. Under simple random sampling without replacement,
the selected sample satisfies the probability measure $ \mathcal{P} $ specified as
\[ \mathcal{P}(S)=\begin{cases}
            1/\binom{N}{n}, & \text{if $ \abs{S}=n $}, \\
            0,              & \text{otherwise}.
      \end{cases} \]

Let $ S=\Set{i_1,i_2,\ldots,i_n} $ be the final sample.
\[ \mathcal{P}(S)=\frac{n(n-1)\cdots(2)(1)}{N(N-1)\cdots(N-n+1)}=\frac{1}{\binom{N}{n}}. \]
\begin{itemize}
      \item Survey sample selection always focuses on units, that is, the labels.
      \item Survey sample data: $ \Set[\big]{(y_i,x_i),i\in S} $.
\end{itemize}

\makeheading{Lecture 3}{\printdate{2022-10-12}}%chktex 8
\textbf{Sample mean and sample variance}:
\[ \bar{y}=\frac{1}{n}\sum_{i\in S}y_i. \]
\[ s_y^2=\frac{1}{n-1}\sum_{i\in S}(y_i-\bar{y})^2=\frac{1}{n-1}\biggl(\sum_{i\in S}y_i^2-n\bar{y}^2\biggr). \]
\textbf{Remarks}:
\begin{itemize}
      \item The sample mean $ \bar{y} $ and $ s_y^2 $ are useful
            statistics under simple random sampling, but not necessarily under
            other sampling methods.
      \item The notation $ \sum_{i\in S} $ is preferred over
            $ \sum_{i=1}^{n} $.
      \item The form of estimators for population parameters depends on the
            sampling methods.
      \item The combination of ``sampling design'' and ``estimation method''
            is called a ``sampling strategy'' (Thompson, 1997; Rao, 2005).
\end{itemize}

\textbf{Expectation and variance under design-based inferences}:

In classic statistics: $ X_1,X_2,\ldots,X_n $ are iid with $ \E{X_i}=\mu $,
$ \V{X_i}=\sigma^2 $. Let $ \bar{X}=\frac{1}{n}\sum_{i=1}^{n}X_i $.
\begin{itemize}
      \item \textbf{Sample mean}:
            \[ \E{\bar{X}}=\frac{1}{n}\sum_{i=1}^{n}\E{X_i}=\frac{1}{n}\sum_{i=1}^{n}\mu=\mu. \]
      \item \textbf{Sample variance}:
            \[ \V{\bar{X}}=\frac{1}{n^2}\sum_{i=1}^{n}\V{X_i}=\frac{1}{n^2}\sum_{i=1}^{n}\sigma^2=\frac{\sigma^2}{n}. \]
\end{itemize}
Under SRSWOR\@:
\[ \E{\bar{y}}=\E*{\frac{1}{n}\sum_{i\in S}y_i}\ne \frac{1}{n}\sum_{i\in S}\E{y_i}. \]
\begin{itemize}
      \item $ S $: a random set.
      \item $ \sum_{i\in S} $: a random ``sum.''
      \item $ y_i $: a fixed quantity for the given $ i $.
\end{itemize}

\textbf{Three fundamental results in survey sampling under SRSWOR}:

\textbf{(a)} The sample mean $ \bar{y}=n^{-1}\sum_{i\in S}y_i $
is a design-unbiased estimator for the population
mean $ \mu_y=N^{-1}\sum_{i=1}^{N}y_i $: $ \boxed{\E{\bar{y}}=\mu_y} $.

There are three possible ways to prove (a), depending on how
the randomization under SRSWOR is handled.

Method 1. Use the probability measure $ \mathcal{P}(S) $
for the survey design, that is, $ \mathcal{P}(S)=\frac{1}{\binom{N}{n}} $
for $ \abs{S}=n $.

Also, $ \bar{y} $ depends only on $ S $.
\[ \bar{y}=\frac{1}{n}\sum_{i\in S}y_i=\bar{y}(S), \]
that is, $ \bar{y} $ is a function of $ S $.
\begin{align*}
      \E{\bar{y}}
       & =\sum(\text{value})(\text{prob})                                                  \\
       & =\sum_S \bar{y}(S)\mathcal{P}(S)                                                  \\
       & =\sum_{S:\abs{S}=n}\frac{1}{n}\sum_{i\in S}y_i \frac{1}{\binom{N}{n}}             \\
       & =\frac{1}{n}\frac{1}{\binom{N}{n}}\sum_{\Set{S\given \abs{S}=n}} \sum_{i\in S}y_i \\
       & =\frac{1}{n}\frac{1}{\binom{N}{n}}\sum_{i=1}^{N}t_i y_i                           \\
       & =\frac{1}{N}\sum_{i=1}^{N}y_i                                                     \\
       & =\mu_y,
\end{align*}
where $ t_i = $ number of $S$ which includes the unit $ i $:
\[ t_i=\binom{N-1}{n-1}. \]
$ N=3 $, $ n=2 $: $ S_1=\Set{1,2} $, $ S_2=\Set{1,3} $, $ S_3=\Set{2,3} $.
\begin{align*}
      \sum_{\Set{S\given\abs{S}=2}} \sum_{i\in S}y_i
       & =(y_1+y_2)+(y_1+y_3)+(y_2+y_3) \\
       & =2 y_1+2y_2+2y_3.
\end{align*}

Method 2. Use the sampling scheme, the sequential
draw-by-draw procedure. Let $ Z_k $ be the $ y $-value from the
$ k\textsuperscript{th} $ draw:
\begin{itemize}
      \item $ S=\Set{i_1,i_2,\ldots,i_n} $.
      \item $ Z_k=y_{ik} $ for $k=1,2,\ldots,n$.
      \item $ \bar{y}=\frac{1}{n}\sum_{i\in S}y_i=\frac{1}{n}\sum_{k=1}^{n}Z_k $.
\end{itemize}
Hence,
\[ \E{\bar{y}}=\E*{\frac{1}{n}\sum_{k=1}^n Z_k}
      =\frac{1}{n}\sum_{k=1}^{n}\E{Z_k}. \]
What's the probability function of $ Z_k $?
\[ \begin{NiceArray}{c|cccc}
            Z_k          & y_1 & y_2 & \cdots & y_N \\
            \midrule
            f(\:\cdot\:) & 1/N & 1/N & \cdots & 1/N
      \end{NiceArray} \]
Therefore,
\[ \E{Z_k}=\sum_{i=1}^{N}y_i \frac{1}{N}=\mu_y. \]

Method 3. Use the sample inclusion indicator variables.
\[ A_i=\begin{cases}
            1, & \text{if $ i\in S $},    \\
            0, & \text{if $ i\notin S $}.
      \end{cases}\qquad i=1,2,\ldots,N. \]
The $ A_i $'s are random variables.
\begin{align*}
      \Prob{A_i=1} & =p=\Prob{i\in S}=\frac{1\times \binom{N-1}{n-1}}{\binom{N}{n}}=\frac{n}{N}. \\
      \Prob{A_i=0} & =1-p.                                                                       \\
      \E{A_i}      & =p=\frac{n}{N}.                                                             \\
      \V{A_i}      & =p(1-p)=\frac{n}{N}\biggl(1-\frac{n}{N}\biggr).                             \\
      \E{\bar{y}}  & =\E*{\frac{1}{n}\sum_{i\in S}y_i}                                           \\
                   & =\E*{\frac{1}{n}\sum_{i=1}^{N}A_i y_i}                                      \\
                   & =\frac{1}{n}\sum_{i=1}^{n}y_i\E{A_i}                                        \\
                   & =\frac{1}{N}\sum_{i=1}^{N}y_i                                               \\
                   & =\mu_y.
\end{align*}

\textbf{(b)} The design-based variance of $ \bar{y} $
under SRSWOR is given by
\[ \V{\bar{y}}=\biggl(1-\frac{n}{N}\biggr)\frac{\sigma_y^2}{n}, \]
where $ \sigma_y^2 $ is the population variance. The term
$ (1-n/N) $ is called the \emph{finite population correction} (fpc)
factor; The ratio $ n/N $ is called the \emph{sampling fraction}.

This result can be proved using different methods. Use the indicator variables:
\begin{align*}
      \V{\bar{y}}
       & =\V*{\frac{1}{n}\sum_{i=1}^N A_i y_i}                                                                   \\
       & =\frac{1}{n^2}\biggl(\sum_{i=1}^{N}y_i^2\V{A_i}+ \mathop{\sum\sum}_{i\ne j}y_i y_j\Cov{A_i,A_j}\biggr).
\end{align*}
\[ \V{A_i}=\frac{n}{N}\biggl(1-\frac{n}{N}\biggr). \]
\[ \Cov{A_i,A_j} =\E{A_i A_j}-\underbrace{\E{A_i}}_{n/N}\underbrace{\E{A_j}}_{n/N}. \]
\begin{align*}
      \E{A_i A_j} & =\sum_i\sum_j a_i a_j \Prob{A_i=a_i}\Prob{A_j=a_j}     \\
                  & =\Prob{A_i=1,A_j=1}                                    \\
                  & =\Prob{i\in S,j\in S}                                  \\
                  & =\frac{1\times 1\times \binom{N-2}{n-2}}{\binom{N}{n}} \\
                  & =\frac{n(n-1)}{N(N-1)}.
\end{align*}
\begin{align*}
      \mu_y^2
       & =\frac{1}{N^2}\biggl(\sum_{i=1}^{N}y_i\biggr)^{\!2}                                \\
       & =\frac{1}{N^2}\sum_{i=1}^N \sum_{j=1}^{N}y_i y_j                                   \\
       & =\frac{1}{N^2}\biggl(\sum_{i=1}^{N}y_i^2+\mathop{\sum\sum}_{i\ne j}y_i y_j\biggr).
\end{align*}

\textbf{(c)} The sample variance $ s_y^2 $ is an unbiased estimator for the
population variance $ \sigma_y^2 $ under SRSWOR, i.e.,
$ \boxed{\E{s_y^2}=\sigma_y^2} $.

\textbf{(c)$^*$} An unbiased variance estimator for $ \bar{y} $ is given by
\[ \v{\bar{y}}=\biggl(1-\frac{n}{N}\biggr)\frac{s_y^2}{n}, \]
which satisfies
\[ \E{\v{\bar{y}}}=\V{\bar{y}}, \]
where
\[ \V{\bar{y}}=\biggl(1-\frac{n}{N}\biggr)\frac{\sigma_y^2}{n}. \]

\begin{align*}
      s_y^2 & =\frac{1}{n-1}\biggl(\sum_{i\in S}y_i^2-n\bar{y}^2\biggr)                         \\
            & =\frac{n}{n-1}\biggl(\frac{1}{n}\sum_{i\in S}y_i^2\biggr)-\frac{n}{n-1}\bar{y}^2.
\end{align*}
$ \E{\bar{y}}=\mu_y $ implies
\[ \E*{\frac{1}{n}\sum_{i\in S}y_i^2}=\frac{1}{N}\sum_{i=1}^N y_i^2. \]
\begin{align*}
      \E{\bar{y}^2}
       & =\V{\bar{y}}+\bigl(\E{\bar{y}}\bigr)^2                    \\
       & =\biggl(1-\frac{n}{N}\biggr)\frac{\sigma_y^2}{n}+\mu_y^2.
\end{align*}
Homework: Show that
\[ \E{s_y^2}=\sigma_y^2. \]

\textbf{Summary of the main theoretical results under SRSWOR}:
\begin{itemize}
      \item The population mean $ \mu_y $ and the population variance
            $ \sigma_y^2 $ are fixed (but unknown) population parameters.
      \item The sample mean $ \bar{y} $ and the sample variance $ s_y^2 $
            are random variables under the survey design.
      \item The $ \bar{y} $ is an unbiased estimator $ \mu_y $: $ \E{\bar{y}}=\mu_y $.
      \item $ \V{\bar{y}}=\bigl(1-\frac{n}{N}\bigr)\frac{\sigma_y^2}{n} $
            is the theoretical variance of $ \bar{y} $ and is a fixed, but unknown
            quantity depending on the population variance $ \sigma_y^2 $.
      \item $ \v{\bar{y}}=\bigl(1-\frac{n}{N}\bigr)\frac{s_y^2}{n} $
            is unbiased estimator for $ \bar{y} $ (computable with the given sample data).
      \item The population size $ N $ is known under SRSWOR\@. (As part of the sampling
            frame information).
\end{itemize}

The R function for SRSWOR and SRSWR (next section) with
specified $N$ and $n$: \texttt{sample(N,n)}
%\begin{noindent}
\begin{minted}{R}
N=10
n=4
sam=sample(N,n)
> sam
[1] 7 1 4 2
sam=sample(N,n,replace=T)
> sam
[1] 6 6 6 1
N=100
n=4
sam=sample(N,n)
> sam
[1] 57 67 62 91
sam=sample(N,n,replace=T)
> sam
[1] 88 73 9 63
\end{minted}
%\end{noindent}

\makeheading{Lecture 4}{\printdate{2022-01-17}}%chktex 8
\textbf{Summary of theoretical results under SRSWOR}:
\begin{enumerate}[1]
      \item $ \E{\bar{y}}=\mu_y $.
      \item $ \V{\bar{y}}=(1-\frac{n}{N})\frac{\sigma_y^2}{n}=(\frac{1}{n}-\frac{1}{N})\sigma_y^2 $.
      \item $ \E{s_y^2}=\sigma_y^2 $.
      \item $ \E{\v{\bar{y}}}=\V{\bar{y}} $, where $ \v{\bar{y}}=(1-\frac{n}{N})\frac{s_y^2}{n} $.
\end{enumerate}
\textbf{Special cases where the response variable $y$ is binary}:
\begin{itemize}
      \item $ \mu_y=\frac{M}{N}=P $; $ \sigma_y^2=\frac{N}{N-1}P(1-P)\approx P(1-P) $ if $ N $ is large.
      \item $ \bar{y}=\frac{1}{n}\sum_{i\in S}y_i=\frac{m}{n}=p $.
            \begin{itemize}
                  \item $ m= $ \# units in $ S $ with attribute $ A $.
                  \item $ p=\frac{m}{n}= $ sample proportion.
            \end{itemize}
      \item $ s_y^2=\frac{n}{n-1}p(1-p)\approx p(1-p) $ if $ n $ is large.
      \item $ \E{p}=P $; $ \v{p}=(1-\frac{n}{N})\frac{1}{n}\frac{n}{n-1}p(1-p)=(1-\frac{n}{N})\frac{1}{n-1}p(1-p) $.
\end{itemize}

\section{Simple Random Sampling With Replacement (SRSWR)}
\textbf{The required sampling frame}:

A complete list of all $ N $ units in the population.

\textbf{The sampling procedure}:
\begin{enumerate}
      \item Select the first unit from the $N$ units on the sampling frame with
            equal probabilities $1/N$; denote the selected unit as $ i_1 $;
      \item Select the second unit from the $N$ units on the sampling frame
            with equal probabilities $1/N$; denote the selected unit as $ i_2 $;
      \item Continue the process and select the $n$th unit from the $N$ units on
            the sampling frame with equal probabilities $1/N$; denote the
            selected unit as $ i_n $.
\end{enumerate}
\textbf{Note}: SRSWR is not very useful in survey practice but has theoretical
values due to its connection to iid samples.

\textbf{Two possible treatments for SRSWR}:

\textbf{(1) Keep duplicated units}

Let $ S^*=\Set{i_1,i_2,\ldots,i_n} $. Under SRSWR, certain units might be included in
$ S^* $ more than once ($ S^* $ may include duplicated units).

Let $ Z_k=y_{ik} $ be the $ y $ value from the $ k $th selection, $ k=1,2,\ldots,n $. Let
the sample mean be computed as
\[ \bar{Z}=\frac{1}{n}\sum_{k=1}^{n}Z_k. \]
We have
\[ \E{\bar{Z}}=\mu_y\quad\text{ and }\quad \V{\bar{Z}}=\biggl(1-\frac{1}{N}\biggr)\frac{\sigma_y^2}{n}. \]
\begin{enumerate}[(i)]
      \item The $ Z_1,Z_2,\ldots,Z_n $ are iid random variables.
      \item The common probability function for $ Z_1,Z_2,\ldots,Z_n $:
            \[ \begin{NiceArray}{c|cccc}
                        Z_k          & y_1 & y_2 & \cdots & y_N \\
                        \midrule
                        f(\:\cdot\:) & 1/N & 1/N & \cdots & 1/N
                  \end{NiceArray}\iid \text{Discrete Uniform} \]
      \item The mean and variance of $ Z_k $:
            \begin{align*}
                  \E{Z_k}
                   & =\sum_{i=1}^{N}y_i\times \frac{1}{N}=\mu_y. \\
                  \V{Z_k}
                   & =\E*{(Z_k-\E{Z_k})^2}                       \\
                   & =\frac{1}{N}\sum_{i=1}^{N}(y_i-\mu_y)^2     \\
                   & =\frac{N-1}{N}\sigma_y^2                    \\
                   & =\biggl(1-\frac{1}{N}\biggr)\sigma_y^2.
            \end{align*}
            \[ \E{\bar{Z}}=\mu_y,\qquad \V{\bar{Z}}=\frac{1}{n}\V{Z_1}=\biggl(1-\frac{1}{N}\biggr)\frac{\sigma_y^2}{n}. \]
\end{enumerate}
\textbf{(2) Remove duplicated units}

Let $ S $ be the set of distinct units from SRSWR\@; let $ m=\abs{S} $ be the number of distinct units.

\textbf{Note}: $ m $ is a random number under SRSWR\@.

The sample mean based on the $ m $ distinct units is computed as
\[ \bar{y}_m=\frac{1}{m}\sum_{i\in S}y_i. \]
It can be shown that (Problem 2.2 of Chapter 2)
\[ \E{\bar{y}_m}=\mu_y\quad\text{ and }\quad \V{\bar{y}_m}=\biggl[\E*{\frac{1}{m}}-\frac{1}{N}\biggr]\sigma_y^2. \]
(Proof is required for Stat 854).

\textbf{Efficiency comparisons between SRSWOR and SRSWR}:

Three estimators of the population mean $ \mu_y $ (assume $ n\ge 2 $):
\begin{enumerate}
      \item $ \bar{y} $ under SRSWOR\@.
      \item $ \bar{Z} $ under SRSWR\@.
      \item $ \bar{y}_m $ under SRSWR\@.
\end{enumerate}
\begin{itemize}
      \item All three estimators are unbiased (first-order equivalence).
      \item $ \bar{y} $ is more efficient than the other two in terms of variance:
            \begin{align*}
                  \V{\bar{y}}=\biggl(1-\frac{n}{N}\biggr)\frac{\sigma_y^2}{n} & <\biggl(1-\frac{1}{N}\biggr)\frac{\sigma_y^2}{n}=\V{\bar{Z}}.        \\
                  \V{\bar{y}}=\biggl(\frac{1}{n}-\frac{1}{N}\biggr)\sigma_y^2 & <\biggl[\E*{\frac{1}{m}}-\frac{1}{N}\biggr]\sigma_y^2=\V{\bar{y}_m}.
            \end{align*}
\end{itemize}

\textbf{Efficiency comparisons through Monte Carlo simulation studies}:
\begin{enumerate}
      \item Generate a finite population of size $ N $, $ \Set{y_1,y_2,\ldots,y_N} $ (from any distribution),
            and compute $ \mu_y $: This is the ``unknown'' population mean.
      \item Take a sample $ S $ of size $ n $, and obtain the sample data $ \Set{y_i,i\in S} $;
            compute the estimate $ \hat{\mu}_1 $ for the estimator $ \hat{\mu}_y $.
      \item Repeat (2) a large number $ K $ ($ \ge 1000 $) times, independently, to obtain $ \hat{\mu}_1,\hat{\mu}_2,\ldots,\hat{\mu}_K $.
      \item Evaluate the performance of the estimator $ \hat{\mu}_y $ using the relative bias (RB, in \%) and the mean squared
            error (MSE) from the simulation:
            \begin{itemize}
                  \item RB (in \%):
                        \[ RB=\frac{1}{K}\sum_{k=1}^{K}\frac{\hat{\mu}_k-\mu_y}{\mu_y}\times 100, \]
                        if $ \mu_y=0 $ then we use the regular bias.
                  \item MSE\@:
                        \[ MSE=\frac{1}{K}\sum_{k=1}^{K}(\hat{\mu}_k-\mu_y)^2=\text{Var}+\text{Bias}^2. \]
            \end{itemize}
            (The MSE $ \approx $ the variance if the RB is very small (i.e., $ <1\% $)).
\end{enumerate}

\textbf{A simulation example in R comparing $ \bar{y} $ and $ \bar{Z} $}:

\begin{minted}{R}
set.seed(1234567,kind=NULL) #Results duplicable!
N=1000
n=200
Y=rexp(N)
muy=mean(Y)
RB=c(0,0)
MSE=c(0,0)
K=1000
for(k in 1:K){
sam1=sample(N,n)
ysam1=Y[sam1]
sam2=sample(N,n,replace=T)
ysam2=Y[sam2]
mu1=mean(ysam1)
mu2=mean(ysam2)
RB[1]=RB[1]+mu1-muy
RB[2]=RB[2]+mu2-muy
MSE[1]=MSE[1]+(mu1-muy)^2
MSE[2]=MSE[2]+(mu2-muy)^2
}
RB=(RB/(K*muy))*100
MSE=MSE/K
> RB
[1] 0.09909944 -0.45677068
> MSE
[1] 0.003621282 0.004583952
    \end{minted}
\textbf{Re-do the simulation with $N = 20000$ and $n = 200$}:
\begin{minted}{R}
> RB
[1] 0.2398581 0.1099942
> MSE
[1] 0.005152047 0.005227150
\end{minted}

\textbf{Note}: The rule of thumb on how many decimal points to be reported
\begin{itemize}
      \item For RB in percentages, two decimal points, i.e., $0.10\%$ and
            $-0.46\%$ from the 1st example.
      \item For MSE, use two or three nearest decimal points to reflect the
            difference, i.e., $0.0036$ and $0.0046$ from the 1st example.
\end{itemize}
\textbf{Homework}:
\begin{itemize}
      \item Install the R package on your laptop \href{https://www.r-project.org}{https://www.r-project.org}.
      \item Re-run the simulation study with a different seed for the random
            number generator and compare the results.
      \item Re-run the simulation study with fixed $n = 200$ and different
            sampling fractions $n/N = 1\%, 2\%, 5\%, 10\%$ and compare the
            results.
      \item \textbf{Challenge part}: Include $ \bar{y}_m $ in the simulation and compare the
            results.
\end{itemize}

\makeheading{Lecture 5}{\printdate{2022-01-19}}%chktex 8
\section{Central Limit Theorem and Confidence Intervals}
\textbf{Asymptotic framework for finite populations}\\
(A frame to allow $ n\to\infty $):

We assume there is a sequence of finite populations (indexed by $ \nu $)
and an associated sequence of survey samples. Both the population size $ N_\nu $ and the sample
size $ n_\nu $ go to infinite as $ \nu\to\infty $. The particular finite population and the survey
sample are part of the sequence.

We use $ n\to\infty $ or $ N\to\infty $, but the limiting process is under $ \nu\to\infty $.

For stratified populations, there are two versions of the asymptotic framework:
\begin{itemize}
      \item The total number of strata is bounded, but the stratum population sizes grow to infinity
            for the sequence of populations.
      \item The stratum population sizes are bounded, but the total number of
            strata goes to infinity for the sequence of populations.
\end{itemize}

\textbf{The Hájek Theorem (1960)}

Suppose that the sampling fraction $ n/N\to f\in(0,1) $ as $ n\to\infty $.

Suppose also that the population values of the response variable $ y $ satisfy
\[ \lim\limits_{{N} \to {\infty}}\frac{\max_{1\le i\le N}{(y_i-\mu_y)^2}}{\sum_{i=1}^{N}(y_i-\mu_y)^2}=0. \]
Then under SRSWOR, the Wald-type statistic
\[ \frac{\bar{y}-\mu_y}{\sqrt{\v{\bar{y}}}}\tod \N{0,1}, \]
as $ n\to\infty $, where $ \v{\bar{y}}=\bigl(1-\frac{n}{N}\bigr)\frac{s_y^2}{n} $ is the estimated variance of $ \bar{y} $.

We also have
\[ \frac{\bar{y}-\mu_y}{\sqrt{\V{\bar{y}}}}\tod \N{0,1}, \]
as $ n\to\infty $, where $ \V{\bar{y}}=\bigl(1-\frac{n}{N}\bigr)\frac{\sigma_y^2}{n} $ is the theoretical variance of $ \bar{y} $.

Note that
\[ \frac{\bar{y}-\mu_y}{\sqrt{\V{\bar{y}}}}=\frac{\bar{y}-\mu_y}{\sqrt{\v{\bar{y}}}}\cdot \frac{\sqrt{\v{\bar{y}}}}{\sqrt{\V{\bar{y}}}}\quad\text{and}\quad \frac{\sqrt{\v{\bar{y}}}}{\sqrt{\V{\bar{y}}}}\inp 1. \]

\section{Sample Size Calculation}

One of the major questions for survey design and planning: How large
should the sample size $n$ be? The answer depends on three factors:
\begin{itemize}
      \item The total budget for the survey.
      \item The cost for surveying one unit and taking all required
            measurements.
      \item The accuracy required for the main statistical inference problem
            from the survey data.
\end{itemize}
The answer also depends on the sampling methods: More efficient
sampling methods require a smaller sample size to achieve the same
goal. We discuss sample size calculation under the simple scenario
where
\begin{itemize}
      \item The sampling method is SRSWOR\@.
      \item The accuracy requirements are for estimating the population
            mean.
\end{itemize}


\textbf{(1) Accuracy specified by the absolute tolerable error}

We want the estimator $ \bar{y} $ for estimating the parameter $ \mu_y $ to satisfy
\[ \Prob[\big]{\abs{\bar{y}-\mu_y}\ge e}\le \alpha, \]
or equivalently,
\[ \Prob[\big]{\abs{\bar{y}-\mu_y}< e}\le 1-\alpha, \]
for a given $ \alpha\in(0,1) $ and a pre-specified error margin $ e $. What is the required $ n $?

We assume that
\[ \frac{\bar{y}-\mu_y}{\sqrt{\V{\bar{y}}}}\tod \N{0,1} \]
can be used as an approximation to derive the required sample size.

We compare
\[ \Prob*{\frac{\abs{\bar{y}-\mu_y}}{\sqrt{\V{\bar{y}}}}<\frac{e}{\V{\bar{y}}}}\ge 1-\alpha \]
with
\[ \Prob[\big]{\abs{Z}<Z_{\alpha/2}}=1-\alpha, \]
where $ Z \sim \N{0,1} $ and $ Z_{\alpha/2} $ is the upper $ \alpha/2 $ quantile of $ \N{0,1} $.

\begin{align*}
      \frac{e}{\sqrt{\V{\bar{y}}}}=Z_{\alpha/2}\implies\V{\bar{y}} & =\frac{e^2}{Z_{\alpha/2}^2}  \\
      \biggl(\frac{1}{n}-\frac{1}{N}\biggr)\sigma_y^2              & =\frac{e^2}{Z_{\alpha/2}^2}.
\end{align*}
Doing some algebra,
\begin{align*}
      n   & =\frac{Z_{\alpha/2}^2\sigma_y^2/e^2}{1+(Z_{\alpha/2}^2\sigma_y^2/e^2)/N}=\frac{n_0}{1+n_0/N}<n_0 \\
      n_0 & =Z_{\alpha/2}^2\sigma_y^2/e^2,
\end{align*}
where $n\approx n_0 $ for large $ N $ ($ N=+\infty $).

\textbf{(2) Accuracy specified by the relative tolerable error}

Suppose that $ \mu_y\ne 0 $. We want the estimator $ \bar{y} $ satisfies
\[ \Prob*{\frac{\abs{\bar{y}-\mu_y}}{\abs{\mu_y}}\ge e}\le \alpha. \]
What is the required $ n $?

Why is sometimes relative tolerable error preferred?

The absolute tolerable error $ e $ specified in $ \abs{\bar{y}-\mu_y}<e $ is scale-dependent.
The choice of $ e $ in the relative tolerable error is scale-free, and can easily be decided as, for instance,
0.01--0.03 (that is, 1\%--3\%).

The accuracy requirement can be re-written as
\[ \Prob[\big]{\abs{\bar{y}-\mu_y}\ge e^*}\le \alpha, \]
where $ e^*=e\abs{\mu_y} $. The required sample size $ n $ is given by
\[ n=\frac{n_0}{1+n_0/N}. \]
\begin{align*}
      n_0
       & =Z_{\alpha/2}^2\sigma_y^2/(e^*)^2                           \\
       & =Z_{\alpha/2}^2\biggl(\frac{\sigma_y^2}{\mu_y^2}\biggr)/e^2 \\
       & =Z_{\alpha/2}^2\bigl[\CV{y}\bigr]^2/e^2,
\end{align*}
where $ \CV{y}=\frac{\sigma_y}{\mu_y} $.

A useful result: if $ y_i=a x_i $ for all $ i $, then $ \CV{y}=\CV{x} $.

\textbf{Notes on sample size calculations}:
\begin{itemize}
      \item The question of sample size calculation or sample size
            determination is part of the survey planning; the actual survey
            sample data are not available at this stage.
      \item Formulas for sample size calculations typically involve unknown
            population quantities such as $ \mu_y $ and $ \sigma_y^2 $.
      \item How to obtain the required population information to calculate
            $n$?
            \begin{itemize}
                  \item Existing data sources:
                        \begin{itemize}
                              \item Other similar surveys.
                              \item Census data.
                        \end{itemize}
                  \item Pilot surveys
                        \begin{itemize}
                              \item Do a small survey first ($ n=50 $?)
                        \end{itemize}
            \end{itemize}
      \item The population information for sample size calculations does not
            need to be very accurate, because the calculated n is used for
            survey planning, which needs to be further adjusted by cost and
            other factors.
\end{itemize}

\textbf{Example 2.1. Sample size calculation for estimating a population proportion}

Suppose that the goal is to estimate the population proportion $ P=M/N $
using a survey sample to be selected by SRSWOR\@. Using the sample proportion $ p=m/n $
to estimate $ P $, a common absolute tolerable error is 3\% and the $ \alpha $ is set to $0.05$. In other words,
the estimation accuracy is specified as
\[ \Prob[\big]{\abs{p-P}\le 0.03}\ge 0.95. \]
Noting that $ 0.95=19/20 $, the probability statement is often quoted in media reports as
``\emph{The result is accurate within three percentage points, 19 times out of 20}.''

What is the required sample size $ n $?

\[ n=\frac{n_0}{1+n_0/N}<n_0. \]
$ n_0 $ is a conservative choice for any $ N $.
\[ \sigma_y^2\approx P(1-P)\le \frac{1}{4}. \]
\begin{align*}
      n_0
       & =Z_{\alpha/2}^2\sigma_y^2/e^2       \\
       & =1.96^2\sigma_y^2/0.03^2            \\
       & \le 1.96^2\times \frac{1}{4}/0.03^2 \\
       & \approx 1067.
\end{align*}

\makeheading{Week 4 | Monday}{\printdate{2022-05-23}}%chktex 8
Victoria Day.
\makeheading{Lecture 10}{\printdate{2022-05-25}}%chktex 8
\begin{Proposition}{}{}
    Let $ a,b,c,d\in\mathbf{Z} $, and suppose that
    \begin{align*}
        a & \equiv b\Mod{n}, \\
        c & \equiv d\Mod{n}.
    \end{align*}
    Then,
    \begin{align*}
        a\pm c & \equiv b\pm d\Mod{n}, \\
        ac     & \equiv bd\Mod{n}.
    \end{align*}
    \tcblower{}
    \textbf{Proof}: Use~\Cref{prop:LEC2_PROP1}.
\end{Proposition}
\begin{Corollary}{}{}
    Suppose $ a,b,c\in\mathbf{Z} $, $ n\ge 2 $, and $ a\equiv b\Mod{n} $, then
    \begin{align*}
        a\pm c & \equiv b\pm d\Mod{n}, \\
        ac     & \equiv bd\Mod{n}.
    \end{align*}
\end{Corollary}
\begin{Corollary}{}{}
    Let $ a,b\in\mathbf{Z} $, $ n\ge 2 $, and $ f(x) $ be a polynomial with integer coefficient. If $ a\equiv b\Mod{n} $,
    then
    \[ f(a)=f(b)\Mod{n}. \]
\end{Corollary}
\begin{Example}{}{}
    Simplify $ 994\cdot 996\cdot 997\cdot 998 \Mod{100} $ to a number in the range $ \Set{0,1,\ldots,999} $.
    \tcblower{}
    \textbf{Solution}: Rather than deal with large ``positive'' numbers, we'll convert them to small ``negative'' numbers:
    \begin{align*}
        994 & \equiv -6\Mod{1000}  \\
        996 & \equiv -4\Mod{1000}  \\
        997 & \equiv -3\Mod{1000}  \\
        998 & \equiv -2\Mod{1000}.
    \end{align*}
    Therefore, $ 994\cdot 996\cdot 997\cdot 998\equiv (-6)(-4)(-3)(-2)\Mod{1000}\equiv 144\Mod{1000} $.
\end{Example}
\begin{Example}{}{}
    Let $ f(x)=x^5-10x+7 $. Compute the remainder of $ f(27) $ divided by $ 5 $.
    \tcblower{}
    \textbf{Solution}: Note that $ 27\equiv 2\Mod{5} $, so
    \begin{align*}
        f(27)\equiv f(2)\Mod{5}\equiv 34\Mod{5}\equiv 4\Mod{5}.
    \end{align*}
    Therefore, $ 4 $ is the remainder of $ f(27) $ divided by $ 5 $.
\end{Example}
\begin{Proposition}{}{}
    Let $ a\in\mathbf{Z} $ and $ n\ge 2 $. If $ (a,n)=1 $, then there exists $ b\in\mathbf{Z} $ such that
    $ ab\equiv 1\Mod{n} $.
    \tcblower{}
    If $ (a,n)=1 $, then by Bezout's Identity, there exists $ b,c\in\mathbf{Z} $ such that
    \[ ab+cn=1. \]
    By~\Cref{prop:prop3_lec9}, $ ab\equiv 1\Mod{n} $.
\end{Proposition}
\begin{Definition}{}{}
    Let $ a\in\mathbf{Z} $ and $ n\in\mathbf{Z}^+ $ such that $ (a,n)=1 $. We call the integer $ b $
    such that $ ab\equiv 1\Mod{n} $ the inverse of $ a $ modulo $ n $ and write
    \[ b\equiv a^{-1}\Mod{n}. \]
\end{Definition}
\begin{Example}{}{}
    Find $ 47^{-1}\Mod{61} $.
    \tcblower{}
    \textbf{Solution}: Apply the Extended Euclidean Algorithm to 61 and 47:
    \[ \begin{array}{lllll}
            r_i & q_{i-1} & s_i & t_i & \text{Check}         \\
            \midrule
            61  &         & 1   & 0                          \\
            47  & 1       & 0   & 1                          \\
            14  & 3       & 1   & -1  & (1)(61)+(-1)(47)=14  \\
            5   & 2       & -3  & 4   & (-3)(61)+(4)(47)=5   \\
            4   & 1       & 7   & -9  & (7)(61)+(-9)(47)=4   \\
            1   & 4       & -10 & 13  & (-10)(61)+(13)(47)=1 \\
        \end{array} \]
    Write the linear combination, then reduce mod $61$:
    \begin{align*}
        (-10)\cdot 61+13\cdot 47 & =1 \\
        13\cdot 47\equiv 1\Mod{61}.
    \end{align*}
    Hence, $ 47^{-1}\equiv 13\Mod{61} $.
\end{Example}
\begin{Corollary}{}{cor3lec10}
    Let $ a,b\in\mathbf{Z} $ and $ n\in\mathbf{Z} $. If $ (a,n)=1 $ and $ ab\equiv ac\Mod{n} $,
    then $ b\equiv c\Mod{n} $.
\end{Corollary}
We can strengthen~\Cref{cor:cor3lec10} further.
\begin{Proposition}{}{}
    If $ (a,n)=d $ and $ ab\equiv ac\Mod{n} $, then $ b\equiv c\Mod{\frac{n}{d}} $.
    \tcblower{}
    \textbf{Proof}:
    Proof. Suppose, $a b \equiv a c\Mod{n}$, then $n \mid a b-a c$ which implies there exists $k \in \mathbb{Z}$ such that
    \[ a b-a c=k n \]
    Then,
    \[ (b-c) \frac{a}{d}=k \frac{n}{d} \]
    Notice both $\frac{a}{d}$ and $\frac{n}{d}$ are integers because $(a, n)=d$. Since $\frac{a}{d}$ divides the RHS,
    it must divide the LHS, that is, $\frac{a}{d} \mid k \frac{n}{d}$. Further, by~\Cref{prop:l3_prop2}, $ (\frac{a}{d}, \frac{n}{d})=1$, hence
    \begin{align*}
         & \frac{a}{d}\mid  k &  & \text{by \Cref{prop:l4_prop2}}       \\
         & k=\frac{a}{d} \ell &  & \text{for some } \ell \in \mathbb{Z}
    \end{align*}
    Hence,
    \[ (b-c) \frac{a}{d}=\frac{a}{d} \ell \frac{n}{d}, \]
    which implies
    \[ b-c=l \frac{n}{d} \]
    Thus,
    \[ b \equiv c\Mod{\frac{n}{d}}. \]
\end{Proposition}
\begin{Example}{}{}
    Reduce $ 5^{13}\Mod{17} $.
    \tcblower{}
    \textbf{Solution}: We have
    \begin{align*}
        5^1    & \equiv 5\Mod{17}                    \\
        5^2    & \equiv 25\equiv 8\Mod{17}           \\
        5^4    & \equiv 64\equiv -4\Mod{17}          \\
        5^8    & \equiv 16\equiv -1\Mod{17}          \\
        5^{13} & \equiv (-1)(-4)(5)\equiv 3\Mod{17}.
    \end{align*}
\end{Example}
\begin{Exercise}{}{}
    Find the remainder when $ 306^{100} $ is divided by $ 7 $.
    \tcblower{}
    \textbf{Solution}: TODO
\end{Exercise}
In practice, in order to compute $ a^k\Mod{n} $ for some large power $ n $,
we utilize the so-called Double-and-Add Algorithm. The algorithm is as follows: first write
the integer $ k $ in its binary expansion, that is,
\[ k=\sum_{i=0}^{t}c_i 2^i=c_t\cdot 2^t+c_{t-1}\cdot 2^{t-1}\cdots+c_1\cdot 2+c_0, \]
where $ c_i\in\Set{0,1} $. Then,
\begin{align*}
    a^k & \equiv a^{c_t\cdot 2^t+c_{t-1}\cdot 2^{t-1}\cdots+c_1\cdot 2+c_0}                   \\
        & \equiv (a^{2^t})^{c_t}(a^{2^{t-1}})^{c_{t-1}}\cdots(a^2)^{c_1}(a^0)^{c_{0}}\Mod{n}.
\end{align*}
But then note that for $ j $ such that $ 2\le j\le t $, we can deduce the value of $ a^{2^j} $
from $ a^{2^{j-1}}\Mod{n} $ as follows:
\[ a^{2^j}\equiv (a^{2^{j-1}})^2\Mod{n}. \]
Therefore, we can compute $ a^2,a^{2^2},\ldots a^{2^t} $ in $ t-1 $ steps.
\begin{Example}{}{}
    Let us compute $ n\equiv 7^{114}\Mod{23} $ such that $ 0\le n<23 $.
    \tcblower{}
    \textbf{Solution}: Note that
    \[ 114=2^6+2^5+2^4+2=64+32+16+2. \]
    Then,
    \begin{align*}
        7^2    & \equiv 49\equiv 3\Mod{23}                               \\
        7^4    & \equiv (7^2)^2\equiv 3^2\equiv 9\Mod{23}                \\
        7^8    & \equiv (7^4)^2\equiv 9^2\equiv 81\equiv 12\Mod{23}      \\
        7^{16} & \equiv (7^8)^2\equiv 12^2\equiv 144\equiv 6\Mod{23}     \\
        7^{32} & \equiv (7^{16})^2\equiv 6^2\equiv 36\equiv 13\Mod{23}   \\
        7^{64} & \equiv (7^{32})^2\equiv 13^2\equiv 169\equiv 8\Mod{23}.
    \end{align*}
    Thus,
    \begin{align*}
        7^{114} & \equiv 7^{64+32+16+2}\Mod{23}        \\
                & \equiv 7^{64}7^{32}7^{16}7^2\Mod{23} \\
                & \equiv (8)(13)(6)(3)\Mod{23}         \\
                & \equiv 1872\Mod{23}                  \\
                & \equiv 9\Mod{23}.
    \end{align*}
\end{Example}
\makeheading{Lecture 11}{\printdate{2022-05-27}}%chktex 8
We will now take a look at some interesting applications of modular arithmetic.
For example, it can be used to demonstrate that certain Diophantine equations have
no solutions.
\begin{Example}{}{}
    Show that the Diophantine equation
    \[ x^2+y^2=4z+3 \]
    has no integer solutions $ x,y,z $.
    \tcblower{}
    \textbf{Solution}: Since there are infinitely many possibilities for $ x,y,z $, it seems a
    bit daunting to show that none of them work. But a little trick with congruences
    and replacement makes this problem quite straightforward. This is the same as
    solving the congruence
    \[ x^2+y^2\equiv 3\Mod{4} \]
    in integers $ x $ and $ y $. Since every integer is congruent to either $ 0,1,2,3 $ modulo $ 4 $,
    there are essentially $ 16 $ possible combinations of $ x $ and $ y $ that we can check. However,
    the problem becomes even simpler if we note that
    \[ 0^2\equiv 0,\; 1^2\equiv 1,\; 2^2\equiv 0,\; 3^2\equiv 1\Mod{4}. \]
    Thus, every perfect square is congruent to either $ 0 $ or $ 1 $ modulo $ 4 $. Since we
    are dealing with the sum of two perfect squares, there are only three options left to check, namely
    \[ 0+0\equiv 0,\; 0+1\equiv 1,\; 1+1\equiv 2\Mod{4}. \]
    As we can see, none of them add up to $ 3 $, which implies that $ x^2+y^2\equiv 3\Mod{4} $
    has no solution in integer $ x,y $. Therefore, there are no solutions to the Diophantine equation
    $ x^2+y^2=4z+3 $.
\end{Example}
\begin{Example}{}{}
    Show that $ x^5\equiv x\Mod{5} $ for all $ x\in\mathbf{Z} $.
    \tcblower{}
    \textbf{Solution}: Every integer $ x $ is congruent mod $ 5 $ to one of its possible remainders
    $ 0,1,2,3,4 $. If the desired congruence holds for these remainders, then, by replacement,
    the congruence holds for any integer $x$. By routine calculation we see that
    \[ 0^5\equiv 0,\; 1^5\equiv 1,\; 2^5\equiv 2,\; 3^5\equiv 3,\; 4^5\equiv 4\Mod{5}. \]
    Having verified the result on the five possible remainders, replacement gives the
    result for all integers.
\end{Example}
\section{The Ring of Residue Classes \texorpdfstring{$ \mathbf{Z}_n $}{Zn}}
Assume that the modulus $ n $ is a positive integer ($ n\ge 2 $). By the Division
Algorithm, every integer $ b $ can be written as
\[ b=qn+a,\; 0\le a<n. \]
Reducing this equation mod $ n $, we have
\[ b\equiv a\Mod{n}. \]
Since $ 0\le a<n $, we have $ a\in\Set{0,1,2,\ldots,n-1} $. In other words,
mod $ n $ every integer can be reduced to a number in $ \Set{0,1,2,\ldots,n-1} $.
This set is called the standard residue system mod $ n $, and answers to modular arithmetic problems will
usually be simplified to a number in this range.
\begin{Definition}{}{}
    Let $ a\in\mathbf{Z} $. The set
    \[ [a]=\Set{qn+a\given q\in\mathbf{Z}}=\Set{b\in\mathbf{Z}\given b\equiv a\Mod{n}} \]
    is called the residue class (equivalence class of a modulo $ n $). The integer $ a $
    is called a representative of the residue class $ [a] $. The finite
    set of residues mod $ n $ will be denoted $ \mathbf{Z}_n $.
    \tcblower{}
    \underline{Remark}: $ [a]=[b]\iff b\equiv a\Mod{n} $.
\end{Definition}
\begin{Example}{}{}
    For Example 3 (Lecture 9), the five residue classes of $ \mathbf{Z}_n $ are:
    \begin{align*}
        [0] & =\Set{0+5q\given q\in\mathbf{Z}} \\
        [1] & =\Set{1+5q\given q\in\mathbf{Z}} \\
        [2] & =\Set{2+5q\given q\in\mathbf{Z}} \\
        [3] & =\Set{3+5q\given q\in\mathbf{Z}} \\
        [4] & =\Set{4+5q\given q\in\mathbf{Z}} \\
    \end{align*}
\end{Example}
\begin{Exercise}{}{}
    Let $ n\in\mathbf{Z}^+ $. Prove that the residue clases
    $ [0],[1],\ldots,[n-1] $ modulo $ n $ partition $ \mathbf{Z} $, that is,
    \[ [0]\cup [1]\cup\cdots\cup[n-1]=\mathbf{Z}, \]
    $ [a]\cap [b]\ne \emptyset \implies [a]=[b] $.
\end{Exercise}
\begin{Proposition}{}{}
    Let $ n\in\mathbf{Z}^+ $ and consider the collection $ \mathbf{Z}_n $
    of all residues modulo $ n $. Define the binary operation $ + $, $ - $, and $ \cdot $ as follows:
    \[ [a]\pm [b]=[a\pm b],\; [a]\cdot[b]=[a\cdot b]. \]
    Then, under these binary operations, $ \mathbf{Z}_n $ forms a commutative ring with identity $ [1] $.
    \tcblower{}
    \textbf{Proof}: Use Proposition 1 (Lecture 10).
\end{Proposition}
\begin{Example}{}{}
    \begin{enumerate}[(a)]
        \item What are the residue classes of modulo 6?
        \item Construct an addition table and multiplication table.
        \item Does the ring $ \mathbf{Z}_6 $ form an integral domain?
    \end{enumerate}
\end{Example}
\makeheading{Lecture 8}{\printdate{2022-01-31}}%chktex 8
\textbf{Basic Concepts of Cluster Sampling}:
\begin{itemize}
      \item The population consists of $K$ clusters (groups).
      \item \emph{Single-stage cluster sampling}: A subset of the clusters is
            selected, and all units in the selected cluster are observed for the
            final sample.
      \item \emph{Two-stage cluster sampling}: A subset of the clusters is selected,
            and within each selected cluster, a subset of units is selected for
            the final sample.
      \item Sampling frames for single-stage and two-stage cluster sampling:\\
            (1) First stage sampling frame: A complete list of clusters in the
            population\\
            (2) Second stage sampling frames: A complete list of units for
            each selected cluster
      \item More complex sampling designs: Stratified \emph{multi-stage cluster
                  sampling} with unequal selection probabilities at each stage.
\end{itemize}

\section{Single-stage Cluster Sampling}
\subsection{Notation}
\begin{itemize}
      \item $ K $: The total number of clusters in the population.
      \item $ M_i $: The total number of units in cluster $ i $.
      \item $ y_{ij} $: The value of $ y $ for unit $ j $ in cluster $ i $.
      \item $ N=\sum_{i=1}^{K}M_i $: The overall population size.
\end{itemize}
The mean and the total for the $ i\textsuperscript{th} $ cluster are given by
\[ \mu_i=\frac{1}{M_i}\sum_{j=1}^{M_i}y_{ij},\qquad T_i=\sum_{j=1}^{M_i}y_{ij}=M_i \mu_i,\; i=1,2,\ldots,K. \]
The population total is given by
\[ T_y=\sum_{i=1}^{K}\sum_{j=1}^{M_i}y_{ij}=\sum_{i=1}^{K}T_i=\sum_{i=1}^{K}M_i \mu_i, \]
and the population mean is given by $ \mu_y=T_y/N $.

\subsection{Single-stage cluster sampling with clusters selected by SRSWOR}
The sampling procedure:
\begin{enumerate}
      \item Select $ k $ clusters from the list of $ K $ clusters using SRSWOR, with a pre-specified
            $ k $. Let $ S_c $ be the set of labels for the $ k $ selected clusters.
      \item For $ i\in S_c $, select all $ M_i $ units for the final sample.
\end{enumerate}
The total number of units in the final sample (overall sample size):
\[ n=\sum_{i\in S_c}M_i. \]
The sample data on the $ y $-variable:
\[ \Set{y_{ij}\given j=1,2,\ldots,M_i,\, i\in S_c}. \]
The cluster total $ T_i=\sum_{j=1}^{M_i}y_{ij} $ is known for $ i\in S_c $. The
``condensed'' sample data set:
\[ \Set{T_i,\, i\in S_c}. \]

Other information available from the sampling frames and the design:
\begin{itemize}
      \item The total number of clusters, $ K $.
      \item The number of clusters selected, $ k $.
      \item The cluster size $ M_i $ for $ i\in S_c $ (selected clusters). $ M_i $
            may not be known if $ i\notin S_c $.
\end{itemize}
Other notes:
\begin{itemize}
      \item The overall sample size $ n=\sum_{i\in S_c} M_i $ is typically a random number and is not controlled
            at the design stage except for the special case where the cluster sizes $ M_i=M $ are all equal. In this case,
            $ n=kM $.
      \item The overall population size $ N=\sum_{i=1}^{K}M_i $ is often \textbf{unknown}.
      \item Estimation of $ T_i=\sum_{i=1}^{K}T_i $ does not lead to estimation of $ \mu_y=T_y/N $ and vice versa.
      \item Need to have an estimator for $ N $.
\end{itemize}

\subsection{Estimation of the population total \texorpdfstring{$ T_y $}{Ty}}

Re-write the population total as

\[ T_y=\sum_{i=1}^{K}T_i=K\biggl(\frac{1}{K}\sum_{i=1}^{K}T_i\biggr)=K\mu_T. \]
\textbf{The question}: Why do we introduce
\[ \mu_T=\frac{1}{K}\sum_{i=1}^{K}T_i?\qquad \mu_y=\frac{1}{N}\sum_{i=1}^{N}y_i. \]
\textbf{The answer}: The $ \mu_T $ is not a parameter of interest, but it is
a ``population mean'' and can be estimated by the corresponding ``sample mean''
under SRSWOR,
\[ \hat{\mu}_T=\frac{1}{k}\sum_{i\in S_c}T_i.\qquad \bar{y}=\frac{1}{n}\sum_{i\in S}y_i. \]
This leads to $ \hat{T}_y=K\hat{\mu}_T $, where $ K $ is known.

\textbf{Main results on estimating $ T_y $}: Under single-stage cluster sampling
with clusters selected by SRSWOR,
\begin{enumerate}[(a)]
      \item An unbiased estimator for the population total $ T_y $ is given by
            \[ \hat{T}_y=K\biggl(\frac{1}{k}\sum_{i\in S_c}T_i\biggr)=K\hat{\mu}_T, \]
            where $ \hat{\mu}_T=k^{-1}\sum_{i\in S_c}T_i $ is the sample mean of cluster totals.
      \item The design-based variance of $ \hat{T}_y $ is given by
            \[ \V{\hat{T}_y}=K^2\biggl(1-\frac{k}{K}\biggr)\frac{\sigma_T^2}{k}, \]
            where $ \sigma_T^2=(K-1)^{-1}\sum_{i=1}^{K}(T_i-\mu_T)^2 $, and $ \mu_T=K^{-1}\sum_{i=1}^{K}T_i $
            is the population mean of cluster totals.
      \item An unbiased variance estimator for $ \hat{T}_y $ is given by
            \[ \v{\hat{T}_y}=K^2\biggl(1-\frac{k}{K}\biggr)\frac{s_T^2}{k}, \]
            where $ s_T^2=(k-1)^{-1}\sum_{i\in S_c}(T_i-\hat{\mu}_T)^2 $ and
            $ \hat{\mu}_T=k^{-1}\sum_{i\in S_c}T_i $.
\end{enumerate}

\subsection{Estimation of the population mean \texorpdfstring{$\mu_y$}{μy}}

If $ N $ is known, we can simply use $ \hat{\mu}_y=\hat{T}_y/N $.

If $ N=\sum_{i=1}^{K}M_i $ is unknown, we re-write the population mean as
\[ \mu_y=\frac{1}{N}\sum_{i=1}^{K}T_i=\frac{\sum_{i=1}^{K}T_i}{\sum_{i=1}^{K}M_i}=\frac{K^{-1}\sum_{i=1}^{K}T_i}{K^{-1}\sum_{i=1}^{K}M_i}=\frac{\mu_T}{\mu_M}, \]
where
\[ \mu_M=\frac{1}{K}\sum_{i=1}^{K}M_i \]
is the ``population mean'' for the variable $ M_i $ (average cluster size), and can be
estimated by the corresponding ``sample mean''
\[ \hat{\mu}_M=\frac{1}{k}\sum_{i\in S_c}M_i. \]

The population mean $ \mu_y $ can be estimated by
\[ \hat{\mu}_y=\frac{\hat{\mu}_T}{\hat{\mu}_M}=\frac{k^{-1}\sum_{i\in S_c}T_i}{k^{-1}\sum_{i\in S_c}M_i}
      =\frac{\sum_{i\in S_c}T_i}{\sum_{i\in S_c}M_i}=\frac{1}{n}\sum_{i\in S_c}\sum_{j=1}^{M_i}y_{ij}, \]
where $ n=\sum_{i\in S_c}M_i $ is the overall sample size.

\textbf{Notes}:
\begin{itemize}
      \item The overall sample size $ n $ is usually a random number.
      \item The $ \hat{\mu}_y $ looks like a sample mean, but its theoretical properties need to
            be derived using a ``ratio estimator.''
      \item Ratio estimators will be discussed in Chapter 5.
\end{itemize}

\subsection{A comparison between SRSWOR and Single-stage cluster sampling}

(This is technically a challenge topic under general scenarios with
unequal $ M_i $)

Consider a simple scenario where
\begin{itemize}
      \item All clusters have the same size: $ M_i=M $ ($ M\ge 2 $).
      \item The overall population size is $ N=KM $ (and is known).
      \item The overall sample size is $ n=kM $ (and is a fixed number).
      \item The sampling fraction $ n/N=(kM)/(KM)=k/K $.
      \item The estimators $ \hat{\mu}_M $ and $ \hat{\mu}_y $ reduce to
            \[ \hat{\mu}_M=k^{-1}\sum_{i\in S_c}M_i=M,\qquad \hat{\mu}_y=\frac{\hat{\mu}_T}{\hat{\mu}_M}=M^{-1}\hat{\mu}_T, \]
            and $ \hat{\mu}_T=\frac{1}{k}\sum_{i\in S_c}T_i $ is a ``sample mean.''
      \item The $ \hat{\mu}_y $ is an unbiased estimator of $ \mu_y $.
\end{itemize}

Under single-stage cluster sampling with clusters selected by
SRSWOR,
\[ \V{\hat{\mu}_y}=\frac{1}{M^2}\biggl(1-\frac{k}{K}\biggr)\frac{\sigma_T^2}{k}=\biggl(1-\frac{n}{N}\biggr)\frac{M^{-1}\sigma_T^2}{n}. \]
It can be shown (Problem 3.8 for STAT 854)
\[ M^{-1}\sigma_T^2\approx \sigma_y^2\bigl(1+(M-1)\rho\bigr), \]
where $ \rho $ is the \emph{intra-cluster correlation coefficient}
and is defined as follows: Randomly select a cluster, and then randomly select two units
from the cluster without replacement; let $ Z_1 $ and $ Z_2 $ be the
values of $ y $ for the two selected units,
\[ \rho=\frac{\Cov{Z_1,Z_2}}{\sqrt{\V{Z_1}\V{Z_2}}}. \]
We have
\[ \V{\hat{\mu}_y}\approx \biggl(1-\frac{n}{N}\biggr)\frac{\sigma_y^2}{n}\bigl(1+(M-1)\rho\bigr). \]

\textbf{Key results for the comparison of two sampling strategies}:
\begin{itemize}
      \item Under the simple scenario with single stage cluster sampling,
            \[ \V{\hat{\mu}_y}\approx \biggl(1-\frac{n}{N}\biggr)\frac{\sigma_y^2}{n}\bigl(1+(M-1)\rho\bigr). \]
      \item If we take a sample of the same overall size $ n $ by SRSWOR and use the sample mean $ \bar{y} $
            to estimate $ \mu_y $, we have
            \[ \V{\bar{y}}=\biggl(1-\frac{n}{N}\biggr)\frac{\sigma_y^2}{n}. \]
\end{itemize}
\begin{enumerate}[(i)]
      \item It is very common in survey practice that units within the same
            cluster are positively correlated, i.e., $ \rho>0 $ and consequently
            single-stage cluster sampling is less efficient than SRSWOR\@.
      \item For situations where $ \rho<0 $, cluster sampling can be more
            efficient.
      \item When $\rho = 0$, the clusters behave like random groups of units
            from the population. Under such scenarios single-stage cluster
            sampling will result in a final sample which is similar to the one
            selected by non-cluster sampling methods.
\end{enumerate}
\begin{itemize}
      \item \textbf{Homework}: Find examples of the three scenarios listed above.
\end{itemize}
\makeheading{Lecture 9}{\printdate{2022-02-02}}%chktex 8
\section{Two-stage Cluster Sampling}

\emph{Primary sampling unit (PSU)}: clusters.\\
(The first-stage sample selects $k$ clusters from the population of $K$
clusters)

\emph{Secondary sampling unit (SSU)}: units within clusters.\\
(The second-stage sample selects $ m_i $ units from the list of $ M_i $ units if
cluster $i$ is selected in the first stage)

\subsection{Two-stage cluster sampling with SRSWOR at both stages}
The sampling procedures:
\begin{enumerate}
      \item Select $k$ clusters from the list of $K$ clusters using SRSWOR, with
            a pre-specified $k$. Let $ S_c $ be the set of labels for the $k$ selected
            clusters.
      \item For $ i\in S_c $ and a pre-specified $ m_i $, select a second-stage sample $ S_i $
            of $ m_i $ units from the list of $ M_i $ units in cluster $i$ using SRSWOR\@;
            the processes are carried out independently for different clusters.
\end{enumerate}

The overall sample size is
\[ n=\sum_{i\in S_c}m_i. \]
The choice of $ m_i $ (as part of the survey design):
\begin{itemize}
      \item A constant $ m_i=m $ is used across all clusters; $ n=mk $ is fixed.
      \item A fixed second-stage sampling fraction, i.e., choose $ m_i $ such that
            $ m_i/M_i=c $ for a pre-specified proportion $ c $ across all clusters;
            $ n=c\sum_{i\in S_c}M_i $ is a random number (e.g., $ c=5\%,10\%,\ldots$).
\end{itemize}
The sample data on the $ y $-variable:
\[ \Set{y_{ij}:j\in S_i,i\in S_c}. \]
Other information available:
\begin{itemize}
      \item The total number of clusters, $K$, and the number of clusters
            sampled, $k$.
      \item The cluster size $ M_i $ and the second-stage sample size $ m_i $ for
            $ i\in S_c $.
\end{itemize}

The cluster mean and the cluster variance (cluster level population
parameters):
\[ \mu_i=\frac{1}{M_i}\sum_{j=1}^{M_i}y_{ij},\qquad \sigma_i^2=\frac{1}{M_i-1}\sum_{j=1}^{M_i}(y_{ij}-\mu_i)^2. \]
\textbf{Note}:
\begin{itemize}
      \item Under stage-stage cluster sampling, both $ \mu_i $ and $ \sigma_i^2 $ can be computed from the sample data for cluster
            $ i\in S_c $.
      \item Under two-stage cluster sampling, both $ \mu_i $ and $ \sigma_i^2 $ are \textbf{unknown} even if cluster
            $ i $ is selected in the first stage.
      \item Under two-stage cluster sampling, $ T_i=\sum_{j=1}^{M_i}y_{ij}=M_i\mu_i $ are unknown.
\end{itemize}

\subsection{Estimation of the population total \texorpdfstring{$ T_y $}{Ty}}

The second-stage cluster sample mean and sample variance:
\[ \bar{y}_i=\frac{1}{m_i}\sum_{j\in S_i}y_{ij},\qquad s_i^2=\frac{1}{m_i-1}\sum_{j\in S_i}(y_{ij}-\bar{y}_i)^2. \]
The second-stage sample $ S_i $ of size $ m_i $ is selected by SRSWOR from the cluster of $ M_i $ units:
\[ \E{\bar{y}_i}=\mu_i,\qquad \V{\bar{y}_i}=\biggl(1-\frac{m_i}{M_i}\biggr)\frac{\sigma_i^2}{m_i}. \]
The cluster total $ T_i=M_i\mu_i $ can be estimated by
\[ \hat{T}_i=M_i\bar{y}_i. \]
\textbf{Note}: $ M_i $ is known for $ i\in S_c $. (Part of second stage frame info).

The ``introduced bridge parameter'' $ \mu_T=K^{-1}\sum_{i=1}^{K}T_i $ can be estimated by
\[ \tilde{\mu}_T=\frac{1}{k}\sum_{i\in S_c}\hat{T}_i=\frac{1}{k}\sum_{i\in S_c}M_i\bar{y}_i. \]
\textbf{Comparison to single-stage cluster sampling}:
\[ \hat{\mu}_T=\frac{1}{k}\sum_{i\in S_c}T_i. \]
(Difference in notation: tilde vs hat).

The population total $ T_y=\sum_{i=1}^{K}T_i=K\mu_T $ can be estimated by
\[ \tilde{T}_y=K\tilde{\mu}_T. \]
$ K $ is available from the first-stage sampling frame information.
\begin{enumerate}
      \item $ \E{\tilde{T}_y}=K\E{\tilde{\mu}_T} $.
      \item $ \V{\tilde{T}_y}=K^2\V{\tilde{\mu}_T} $.
      \item $ \v{\tilde{T}_y}=K^2\v{\tilde{\mu}_T} $.
\end{enumerate}

\textbf{Main theoretical results on $ \tilde{\mu}_T $}: Under two stage-cluster sampling
with SRSWOR at both stages,
\begin{enumerate}[(a)]
      \item The estimator $ \tilde{\mu}_T $ is unbiased for $ \mu_T $.
      \item The design-based variance of $ \tilde{\mu}_T $ is given by
            \[ \V{\tilde{\mu}_T}=\biggl(1-\frac{k}{K}\biggr)\frac{\sigma_T^2}{k}+\frac{1}{k}\frac{1}{K}\sum_{i=1}^{K}M_i^2\biggl(1-\frac{m_i}{M_i}\biggr)\frac{\sigma_i^2}{m_i}, \]
            where $ \sigma_T^2=(K-1)^{-1}\sum_{i=1}^{K}(T_i-\mu_T)^2 $ and $ \sigma_i^2 $ is the cluster variance.
      \item An unbiased variance estimator for $ \tilde{\mu}_T $ is given by
            \[ \v{\tilde{\mu}_T}=\biggl(1-\frac{k}{K}\biggr)\frac{\hat{\sigma}_T^2}{k}+\frac{1}{K}\frac{1}{k}\sum_{i\in S_c}M_i^2\biggl(1-\frac{m_i}{M_i}\biggr)\frac{s_i^2}{m_i}, \]
            where $ \hat{\sigma}_T^2=(k-1)^{-1}\sum_{i\in S_c}(\hat{T}_i-\tilde{\mu}_T)^2 $ and $ s_i^2 $ is the cluster sample variance.
\end{enumerate}

\textbf{Two technical arguments for the proofs of (a), (b) and (c)}:
\begin{enumerate}
      \item For any random variables $ X $ and $ Y $, we have
            \[ \E{X}=\E[\big]{\E{X\given Y}}, \]
            and
            \[ \V{X}=\E[\big]{\V{X\given Y}}+\V[\big]{\E{X\given Y}}. \]
\end{enumerate}
The proofs involve
\begin{itemize}
      \item $ \Esp{}{1} $ and $ \Vsp{}{1} $: the expectation and the variance with respect to the first stage sampling design.
      \item $ \Esp{}{2} $ and $ \Vsp{}{2} $: the conditional expectation and the conditional
            variance with respect to the second stage sampling design given
            the first stage sample.
\end{itemize}
Proof of (a):
\[ \tilde{\mu}_T=\frac{1}{k}\sum_{i\in S_c}M_i\bar{y}_i. \]
\begin{align*}
      \E{\tilde{\mu}_T}
       & =\Esp[\big]{ \Esp{\tilde{\mu}_T}{2} }{1}                    \\
       & =\Esp*{\frac{1}{k}\sum_{i\in S_c}M_i\Esp{\bar{y}_i}{2} }{1} \\
       & =\Esp*{\frac{1}{k}\sum_{i\in S_c}M_i\mu_i}{1}               \\
       & =\Esp*{\frac{1}{k}\sum_{i\in S_c}T_i}{1}                    \\
       & =\frac{1}{K}\sum_{i=1}^{K}T_i                               \\
       & =\mu_T.
\end{align*}
Proof of (b):
\[ \tilde{\mu}_T=\frac{1}{k}\sum_{i\in S_c}M_i\bar{y}_i. \]
\begin{align*}
      \V{\tilde{\mu}_T}
       & =\Vsp[\big]{\Esp{\tilde{\mu}_T}{2}}{1}+\Esp[\big]{\Vsp{\tilde{\mu}_T}{2}}{1}                                                                                                                                   \\
       & =\Vsp*{\frac{1}{k}\sum_{i\in S_c}T_i}{1}+\Esp*{\frac{1}{k^2}\sum_{i\in S_c}M_i^2\biggl(1-\frac{m_i}{M_i}\biggr)\frac{\sigma_i^2}{m_i}}{1}                                                                      \\
       & =\biggl(1-\frac{k}{K}\biggr)\frac{\sigma_T^2}{k}+\frac{1}{k}\Esp[\Bigg]{\underbrace{\frac{1}{k}\sum_{i\in S_c}M_i^2\biggl(1-\frac{m_i}{M_i}\biggr)\frac{\sigma_i^2}{m_i}}_{\text{first stage sample mean}}}{1} \\
       & =\biggl(1-\frac{k}{K}\biggr)\frac{\sigma_T^2}{k}+\frac{1}{k}
      \underbrace{\frac{1}{K}\sum_{i=1}^{K}M_i^2\biggl(1-\frac{m_i}{M_i}\biggr)\frac{\sigma_i^2}{m_i}}_{\text{first stage population mean}}.
\end{align*}
\begin{enumerate}[(2)]
      \item Re-write $ \V{\tilde{\mu}_T} $ as
            \[ \V{\tilde{\mu}_T}=\biggl(\frac{1}{k}-\frac{1}{K}\biggr)\sigma_T^2+\frac{1}{k}W, \]
            where
            \[ \sigma_T^2=\frac{1}{K-1}\sum_{i=1}^{K}(T_i-\mu_T)^2,\qquad W=\frac{1}{K}\sum_{i=1}^{K}M_i^2\biggl(1-\frac{m_i}{M_i}\biggr)\frac{\sigma_i^2}{m_i}. \]
            The ``plug-in'' estimator
            \[ \hat{\sigma}_T^2=\frac{1}{k-1}\sum_{i\in S_c}(\hat{T}_i-\tilde{\mu}_T)^2 \]
            is not unbiased for $ \sigma_T^2 $, and instead satisfies (\textbf{homework for STAT 854},
            hints from Problem 3.11 in the textbook)
            \[ \E{\hat{\sigma}_T^2}=\sigma_T^2+W. \]
\end{enumerate}
Proof of (c):
\[ \V{\tilde{\mu}_T}=\biggl(\frac{1}{k}-\frac{1}{K}\biggr)\sigma_T^2+\frac{1}{k}W, \]
\[ W=\frac{1}{K}\sum_{i=1}^{K}M_i^2\biggl(1-\frac{m_i}{M_i}\biggr)\frac{\sigma_i^2}{m_i}. \]
\begin{enumerate}[(i)]
      \item Homework: Show that $ \E{\hat{W}}=W $ (using the same argument from (a)), where
            \[ \hat{W}=\frac{1}{k}\sum_{i\in S_c}M_i^2\biggl(1-\frac{m_i}{M_i}\biggr)\frac{s_i^2}{m_i}. \]
      \item $ \E{\hat{\sigma}_T^2}=\sigma_T^2+W $.
      \item Homework: Show that $ \E[\big]{\v{\tilde{\mu}_T}}=\V{\tilde{\mu}_T} $, where
            \[ \v{\tilde{\mu}_T}=\biggl(\frac{1}{k}-\frac{1}{K}\biggr)\hat{\sigma}_T^2+{\textcolor{red}{\frac{1}{K}}}\hat{W}. \]
\end{enumerate}

\makeheading{Lecture 11}{\printdate{2022-10-26}}%chktex 8
\begin{Definition}{}{}
    The \textbf{joint probability mass function}
    (joint pmf) of a sequence $ X_1,\ldots,X_n $
    of discrete random variables is a function
    $ p\colon\mathbf{R}^n\to[0,1] $
    with
    \[ p(a_1,\ldots,a_n)=\Prob[\big]{\Set{X_1=a_1}\cap\cdots\cap \Set{X_n=a_n}}. \]
\end{Definition}
\begin{Example}{}{}
    Suppose we are rolling two 4-sided die independently. The joint pmf is
    \[ p(a,b)=\begin{cases}
            \frac{1}{16}, & a,b\in\Set{1,2,3,4}, \\
            \text{0},     & \text{otherwise}.
        \end{cases} \]
\end{Example}
\begin{Example}{}{}
    Suppose we roll a die and flip a coin. Let $ X $ be a die roll and
    \[ Y=\begin{cases}
            X,   & \text{if H}, \\
            5-X, & \text{if T}.
        \end{cases} \]
    \[ \begin{array}{cc|c|c|c|c}
            \multicolumn{2}{c}{} & \multicolumn{4}{c}{a}                         \\
                                 &                       & 1   & 2   & 3   & 4   \\
            \cline{2-6}
            \multirow{4}{*}{$b$} & 1                     & 1/8 & 0   & 0   & 1/8 \\
            \cline{2-6}
                                 & 2                     & 0   & 1/8 & 1/8 & 0   \\
            \cline{2-6}
                                 & 3                     & 0   & 1/8 & 1/8 & 0   \\
            \cline{2-6}
                                 & 4                     & 1/8 & 0   & 0   & 1/8
        \end{array} \]
    Note that
    \[ \Prob{\Set{Y=3}}=\frac{1}{8}+\frac{1}{8}=\frac{1}{4}. \]
\end{Example}
\begin{Remark}{}{}
    If $ p $ is the joint pmf of $ (X,Y) $, then
    \begin{align*}
        \Prob{\Set{X=k}} & =\sum_{j}\underbrace{p(k,j)}_{\Prob{\Set{X=k,Y=j}}}. \\
        \Prob{\Set{Y=k}} & =\sum_{j}p(j,k).
    \end{align*}
    In this context of starting with a joint distribution,
    distribution of components are called ``marginal distributions.''
\end{Remark}
\begin{Definition}{}{}
    If $ p $ is the joint pmf of
    $ X_1,\ldots,X_n $, then the marginal distribution
    of $ X_k $ for any $ k\in\Set{1,2,\ldots,n} $ is
    \[ \Prob{\Set{X_k=a}}=\sum_{b_1,\ldots,b_{k-1},b_{k+1},\ldots,b_n}
        p(b_1,\ldots,b_{k-1},a,b_{k+1},\ldots,b_n). \]
\end{Definition}
\begin{Theorem}{}{}
    $ X_1,\ldots,X_n $ (discrete) are jointly independent
    if and only if their joint pmf is the product of their individual pmfs;
    that is,
    \[ p_{X_1,\ldots,X_n}(b_1,\ldots,b_n)=
        p_{X_1}(b_1)\cdots p_{X_n}(b_n). \]
\end{Theorem}
\begin{Example}{}{}
    Let $ X $ and $ Y $ be independent with pmfs
    \begin{align*}
        p_X(-1) & =\frac{1}{2}, \\
        p_X(0)  & =\frac{1}{4}, \\
        p_X(1)  & =\frac{1}{4}, \\
        p_Y(0)  & =\frac{1}{3}, \\
        p_Y(1)  & =\frac{2}{3}.
    \end{align*}
    They have joint pmf
    \[ \begin{array}{cc|c|c|c}
            \multicolumn{2}{c}{} & \multicolumn{3}{c}{X}                     \\
                                 &                       & -1  & 0    & 1    \\
            \cline{2-5}
            \multirow{2}{*}{$Y$} & 0                     & 1/6 & 1/12 & 1/12 \\
            \cline{2-5}
                                 & 1                     & 1/3 & 1/6  & 1/6
        \end{array} \]
\end{Example}
\begin{Definition}{}{}
    If $ X_1,\ldots,X_n $ are continuous random variables and
    $ f\colon\mathbf{R}^n\to\interval[open right]{0}{\infty} $ ($ A\subseteq\mathbf{R}^n $)
    that satisfies
    \[ \int\cdots\int\limits_{A}f(x_1,\ldots,x_n)\odif{x_1}\cdots\odif{x_n} \]
    then $ f $ is a joint pdf for these variables, and they are said to be
    jointly continuous.
\end{Definition}
\begin{Example}{}{}
    Suppose we have two continuous random variables $ X $ and $ Y $.
    \[ \Prob{\Set{(X,Y)\in A}}=\iint\limits_{A}f(x,y)\odif{x}\odif{y}. \]
    If $ A  $ is a rectangle, then $ A=[a,b]\times[c,d] $,
    which implies
    \[ \Prob{\Set{a\le X\le b, c\le Y\le d}}
        =\int_{c}^{d}\int_{a}^{b}f(x,y)\odif{x}\odif{y}. \]
\end{Example}
\begin{Theorem}{}{}
    $ X_1,\ldots,X_n $ (continuous) are jointly independent
    if and only if they are jointly continuous with joint pdf
    \[ f_{X_1,\ldots,X_n}(a_1,\ldots,a_n)=f_{X_1}(a_1)\cdots f_{X_n}(a_n). \]
\end{Theorem}
\begin{Example}{}{}
    \[ f(x,y)=\begin{cases}
            2x^2, & x\in[0,1],\; \abs{y}\le x, \\
            0,    & \text{otherwise}.
        \end{cases} \]
    \underline{Verifying we have a probability density function}:
    \begin{align*}
        \int_{0}^{1}\int_{-x}^{x}2x^2\odif{y}\odif{x}
         & =\int_{0}^{1}\biggl[2x^2y\biggr]_{y=-x}^{y=x}\odif{x} \\
         & =\int_{0}^{1}2x^2(x-(-x))\odif{x}                     \\
         & =\int_{0}^{1}4x^3\odif{x}                             \\
         & =\biggl[x^4\biggr]_{x=0}^{x=1}                        \\
         & =1.
    \end{align*}
    \underline{Calculating Probabilities}:
    To calculate $ \Prob{\Set{Y\ge 1/2}} $,
    we could work out
    the system of inequalities: $ 0\le x\le 1 $, $ -x\le y\le x $, and $ 1/2\le y $ yields
    \[ 1/2\le y\le x\le 1. \]
    Or we can work it out graphically.
    \begin{align*}
        \Prob*{\Set*{Y\ge \frac{1}{2}}}
         & =\int_{1/2}^{1}\int_{1/2}^{x}2x^2\odif{y}\odif{x}        \\
         & =\int_{1/2}^{1}\biggl[2x^2y\biggr]_{y=1/2}^{y=x}\odif{x} \\
         & =\int_{1/2}^{1}(2x^3-x^2)\odif{x}                        \\
         & =\biggl[\frac{x^4}{2}-\frac{x^3}{3}\biggr]_{x=1/2}^{x=1} \\
         & =\frac{1}{2}-\frac{1}{32}-\frac{1}{3}+\frac{1}{24}.
    \end{align*}
\end{Example}
\begin{Definition}{}{}
    The marginal density of $ X $ is
    \[ f_X(t)=\int_{-\infty}^{\infty}f_{X,Y}(t,u)\odif{u}. \]
\end{Definition}
\begin{Definition}{Expectation (Continuous)}{}
    \[ \E[\big]{g(X,Y)}
        =\int_{-\infty}^{\infty}\int_{-\infty}^{\infty}g(x,y)f_{X,Y}(x,y)\odif{x}\odif{y}. \]
    \tcblower{}
    For example, to calculate $ \E{XY} $ we use $ g(x,y)=xy $.
\end{Definition}
\begin{Example}{Polya Urn}{}
    \[ \Prob{\Set{3\textsuperscript{rd}\text{ pick R}}\given \Set{\text{BB}}}
        =\frac{1}{4}. \]
    If $ Y $ is the limiting percentage of blue, then
    \begin{align*}
        \Prob*{\Set*{Y\le \frac{1}{2}}\given \Set{\text{BB}}}
         & =\Prob*{X_1,X_2,Y\le \frac{1}{2}}                                          \\
         & =\int_{0}^{1/2}\int_{0}^{1/2}\int_{0}^{1/2}1\odif{x_1}\odif{x_2}\odif{x_3} \\
         & =\frac{1}{2}\cdot \frac{1}{2}\cdot \frac{1}{2}                             \\
         & =\frac{1}{8}.
    \end{align*}
    \[ \Prob{\Set{Y\le t}}=t^3. \]
    \[ f_Y(t)=\begin{cases}
            3t^2, & t\in[0,1]         \\
            0,    & \text{otherwise},
        \end{cases} \]
    which is a $ \BetaDist{3,1} $ distribution.
    \[ \frac{1}{B(\alpha,\beta)}x^{\alpha-1}(1-x)^{\beta-1},\; x\in[0,1]. \]
\end{Example}
\makeheading{Lecture 12}{\printdate{2022-10-28}}%chktex 8
Discussion on gamma function when $ \alpha=0 $.
\begin{Example}{}{}
    Suppose $ X \sim \GAM{\alpha,\lambda} $. Find
    $ M_X(t) $.
    \tcblower{}
    \textbf{Solution}:
    \begin{align*}
        M_X(t)
         & =\E{e^{tX}}                                                                                                                                                                               \\
         & =\int_{0}^{\infty}e^{tx}\frac{\lambda^\alpha}{\Gamma(\alpha)}x^{\alpha-1}e^{-\lambda x}\odif{x}                                                                                           \\
         & =\frac{\lambda^\alpha}{\Gamma(\alpha)}\int_{0}^{\infty}x^{\alpha-1}e^{-\lambda x(1-t/\lambda)}\odif{x}                                                                                    \\
         & =\frac{\lambda^\alpha}{\Gamma(\alpha)}\int_{0}^{\infty}\frac{u^{\alpha -1}}{(\lambda-t)^{\alpha-1}}e^{-u}\frac{1}{\lambda-t}\odif{u} &  & u=x(\lambda-t)\iff \odif{u}=(\lambda-t)\odif{x} \\
         & =\frac{\lambda^\alpha}{\Gamma(\alpha)(\lambda-t)^{\alpha}}\int_{0}^{\infty}u^{\alpha-1}e^{-u}\odif{u}                                                                                     \\
         & =\frac{\lambda^\alpha}{\Gamma(\alpha)(\lambda-t)^{\alpha}}\Gamma(\alpha)                                                                                                                  \\
         & =\biggl(\frac{\lambda}{\lambda-t}\biggr)^{\! \alpha}.
    \end{align*}
\end{Example}
\begin{Example}{}{}
    Suppose $ X_1 \sim \GAM{1/2,2} $ and $ X_2 \sim \GAM{3,2} $ are independent.
    Find $ M_Y(t) $ where $ Y=X_1+X_2 $.
    \tcblower{}
    \textbf{Solution}: Since $ X_1 $ and $ X_2 $ are independent,
    \begin{align*}
        M_Y(t)
         & =M_{X_1}(t)M_{X_2}(t)                                             \\
         & =\biggl(\frac{2}{2-t}\biggr)^{1/2}\biggl(\frac{2}{2-t}\biggr)^{3} \\
         & =\biggl(\frac{2}{2-t}\biggr)^{7/2}.
    \end{align*}
    Therefore, $ Y \sim \GAM{3.5,2} $.
\end{Example}
\begin{Example}{}{}
    The pdf for $ \GAM{1,\lambda} $ is
    \[ f_X(t)=\frac{\lambda^1}{\Gamma(1)}t^0 e^{-\lambda t}=\lambda e^{-\lambda t}, \]
    which is $ \EXP{1} $.
\end{Example}
\begin{Remark}{}{}
    \[ \BIN*{n,\frac{\lambda}{n}}\xrightarrow{n\to\infty} \POI{\lambda}. \]
\end{Remark}
\begin{Example}{}{}
    Suppose Chocolat gets 1 customer every 10 minutes, on average (discrete time).
    \begin{enumerate}[(i)]
        \item Model level 1:
              \begin{itemize}
                  \item Every minute there is an independent $ 1/10 $ chance for a customer to enter
                        ($0$ chance for multiple customers in the same minute).
                  \item Let $ T_1 $ be the waiting time for the first customer in minutes,
                        \[ T_1=\text{waiting time for the first customer in minutes}\sim \GEO*{\frac{1}{10}}, \]
                        and $ \E{T_1}=10 $.
                        \[ N_{60}=\text{number of customers in the first hour}\sim \BIN*{60,\frac{1}{10}}. \]
              \end{itemize}
        \item Model level 2:
              \begin{itemize}
                  \item Every second there is a $ 1/600 $ chance for a customer to enter, independently.
                        \[ T_1=\text{waiting time in minutes}=\frac{\tilde{T}_1}{60},\; \text{where }\tilde{T}_1 \sim \GEO*{\frac{1}{600}}, \]
                        and $ \E{T_1}=600/60=10 $.
                        \[ N_{60}=\text{number of customers in the first hour}\sim \BIN*{3600,\frac{1}{600}}. \]
              \end{itemize}
    \end{enumerate}
    As we approach continuity,
    \[ N_{60}\tod\POI*{\frac{60}{10}},\;
        T_1\tod\EXP*{\frac{1}{10}}. \]
    For $ t\ge 0 $,
    \[ N(t)=\text{number of arrivals in the first $t$ minutes}\sim \POI*{\frac{1}{10}t}. \]
\end{Example}
\begin{Definition}{}{}
    A \textbf{Poisson process} ($ N(t) $ for $ t\ge 0 $) with rate $ \lambda $ is a stochastic process
    with the properties:
    \begin{enumerate}[(1)]
        \item For $ 0\le t_1<t_2 $,
              \[ (N(t_2)-N(t_1))\sim \POI{\lambda(t_2-t_1)}. \]
        \item For $ 0\le t_1<t_2<\cdots<t_n $, the variables
              \[ (N(t_2)-N(t_1)),(N(t_3)-N(t_2)),\ldots (N(t_n)-N(t_{n-1})) \]
              are jointly independent.
    \end{enumerate}
\end{Definition}
\begin{Definition}{}{}
    \[ T_n=\inf{\Set{t\ge 0\given N(t)\ge n}} \]
    is the arrival time of the $ n\textsuperscript{th} $ customer.
\end{Definition}
\begin{Theorem}{Interarrival Times}{}
    $ \Delta_1=T_1 $, and $ \Delta_n=T_n-T_{n-1} $ for $ n\ge 2 $ are known as
    \textbf{interarrival times}. Then,
    $ \Delta_1,\ldots,\Delta_n \iid \EXP{\lambda} $ variables.
\end{Theorem}
\begin{Corollary}{}{}
    For $ 0\le n_1<n_2<\cdots<n_k $,
    \[ T_{n_2}-T_{n_1},T_{n_3}-T_{n_2},\ldots,T_{n_k}-T_{n_{k-1}} \]
    are jointly independent with respective probability
    distributions
    \[ (T_{n_{j+1}}-T_{n_j})\sim \GAM{n_{j+1}-n_j,\lambda}. \]
\end{Corollary}
\begin{Example}{}{}
    $ T_3 \sim \GAM{3,\lambda} $ and
    $ T_5-T_3 \sim \GAM{2,\lambda} $ are independent.
\end{Example}
\begin{Example}{}{}
    Suppose $ X \sim \GAM{\alpha,1} $ and
    $ Y \sim \GAM{\beta,1} $
    are independent (rate doesn't matter, set it equal to $1$
    for simplicity).
    \begin{Example}{}{}
        $ \alpha=3 $, $ \beta=5 $, $ X=T_3 $, $ Y=T_8-T_3 $.
        What is the distribution of $ T_3/T_8 $?
        If it took two hours for 8 people to arrive, what is the
        conditional distribution of how long it took for three people to arrive?
    \end{Example}
    That is, find the distribution of
    \[ Z=\frac{X}{X+Y},\; 0\le Z\le 1. \]
    For $ t\in[0,1] $,
    \[ \frac{x}{x+y}\le t\implies x\le \frac{ty}{1-t}. \]
    Thus, noting that $ X $ and $ Y $ are independent,
    \begin{align*}
        \Prob{\Set{Z\le t}}
         & =\int_{0}^{\infty}\int_{0}^{ty/(1-t)}
        \frac{1}{\Gamma(\alpha)}x^{\alpha-1}e^{-x}\frac{1}{\Gamma(\beta)} y^{\beta-1}e^{-y}\odif{x}\odif{y} \\
         & =\frac{1}{\Gamma(\alpha)\Gamma(\beta)}\int_{0}^{\infty}\int_{0}^{ty/(1-t)}
        x^{\alpha-1}y^{\beta-1}e^{-(x+y)}\odif{x}\odif{y}.
    \end{align*}
    Multivariable substitution:
    \[ u=\frac{x}{x+y},\; v=x+y\implies x=uv,\; y=v-uv=v(1-u). \]
    \[ J=\pdv{(x,y)}{(u{,}v)}=\begin{vmatrix}
            \pdv{x}{u} & \pdv{x}{v} \\
            \pdv{y}{u} & \pdv{y}{v}
        \end{vmatrix}=\begin{vmatrix}
            v  & u   \\
            -v & 1-u
        \end{vmatrix}=\abs{(v)(1-u)-(u)(-v)}=v. \]
    Note that $ u\le t $ and $ 0\le v<\infty $, which implies
    \begin{align*}
         & =\frac{1}{\Gamma(\alpha)\Gamma(\beta)}\int_{0}^{\infty}\int_{0}^{1}
        (uv)^{\alpha-1}\bigl(v(1-u)\bigr)^{\beta-1}e^{-v}v\odif{u}\odif{v}                                                                       \\
         & =\frac{1}{\Gamma(\alpha)\Gamma(\beta)}
        \int_{0}^{\infty}\underbrace{v^{\alpha-1}v^{\beta-1}}_{v^{\alpha+\beta-1}} e^{-v}\odif{v}\int_{0}^{1}u^{\alpha-1}(1-u)^{\beta-1}\odif{u} \\
         & =\frac{\Gamma(\alpha+\beta)}{\Gamma(\alpha)\Gamma(\beta)}\int_{0}^{1}u^{\alpha-1}(1-u)^{\beta-1}\odif{u}.
    \end{align*}
\end{Example}
\begin{Definition}{Beta Distribution}{}
    We say $ X \sim \BetaDist{\alpha,\beta} $ with shape parameters $ 0<\alpha\in\mathbf{R} $ and $ 0<\beta\in\mathbf{R} $ if it has pdf
    \[ f_X(t\mid \alpha,\beta)=\frac{1}{B(\alpha,\beta)}t^{\alpha-1}(1-t)^{\beta-1},\; t\in[0,1] \]
    where $ B(\alpha,\beta) $ denotes the beta function,
    \[ B(\alpha,\beta)=\int_{0}^{1}x^{\alpha-1}(1-x)^{\beta-1}\odif{x}=
        \frac{\Gamma(\alpha)\Gamma(\beta)}{\Gamma(\alpha+\beta)}. \]
\end{Definition}
\begin{Theorem}{}{}
    If $ X \sim \GAM{\alpha,1} $ and $ Y \sim \GAM{\beta,1} $, then
    \[ Z=\frac{X}{X+Y}\sim \BetaDist{\alpha,\beta} \]
    and is independent of
    \[ X+Y \sim \GAM{\alpha+\beta,1}. \]
\end{Theorem}
\makeheading{Lecture 7}{\printdate{2022-03-01}}%chktex 8
\begin{Example}{Uniform Distribution}{}
    Let $ X_1,\ldots,X_n $ be a random sample from $ \text{Uniform}(0,\theta) $
    distribution. Then, we have seen before that $ X_{n:n} $ is a sufficient statistic for $ \theta $.

    Further, the PDF of $ T=X_{n:n} $ is
    \[ f_T(t\mid \theta)=n\bigl(F(t)\bigr)^{n-1}f(t)=\frac{nt^{n-1}}{\theta^n},\; 0<t<\theta. \]
    Now, let $ g(\:\cdot\:) $ be a measurable function of $ t $ such that $ \Esp{g(T)}{\theta}=0 $
    for all $ \theta>0 $. Then,
    \[ \Esp{g(T)}{\theta}=\frac{n}{\theta^n}G(\theta),\; \text{where }G(\theta)=\int_{0}^\theta g(t)t^{n-1}\odif{t}. \]
    Note $ G(\theta) $ is differentiable almost everywhere, and further
    \[ G'(\theta)=g(\theta)\theta^{n-1}. \]
    If $ G(\theta)=0 $ for all $ \theta>0 $, then $ G'(\theta)=0 $
    for all $ \theta>0 $ and so $ g(\theta)=0 $ for all $ \theta>0 $.

    This shows that $ T=X_{n:n} $ is a complete sufficient statistic for
    $ \theta $. Since $ \E{T}=\E{X_{n:n}}=\frac{n}{n+1}\theta $,
    we find readily $ \frac{n+1}{n}T=\frac{n+1}{n}X_{n:n} $
    is an unbiased estimator of $ \theta $. Hence,
    it is an Uniformly Minimum Variance Unbiased Estimator of $ \theta $.
\end{Example}
\subsection*{Optimal Linear Estimation}
The celebrated Gauss-Markov theorem states that the
OLS (Ordinary Least Squares) estimator possesses
the smallest sampling variance, within the class of all \underline{linear}
unbiased estimators, if the error variables in the linear regression
model are uncorrelated and have zero means and equal variances.

\underline{Note}: The errors need not be normally distributed, nor be independent
and identically distributed. They only need to be uncorrelated and with zero mean
and homoscedastic (all equal) finite variance.

\underline{Note}: The result, under the assumptions
of normality and independence, was
first established by Gauss.
Much later, the result was proved by Markov under the weaker conditions
of uncorrelated errors (and also without the assumption of normality).

\underline{Gauss-Markov Theorem}. Consider
the linear model
\[ \Vector{y}=\Matrix{X}\Vector{\beta}+\Vector{\varepsilon}, \]
with $ \Vector{y}\in\R^n $, $ \Matrix{X}\in\R^{n\times k} $,
$ \Vector{\beta}\in\R^k $, $ \Vector{\varepsilon}\in\R^n $,
or equivalently
\[ y_i=\sum_{j=1}^{k}\beta_j X_{ij}+\varepsilon_i,\;\text{for }i=1,2,\ldots,n, \]
where $ \beta_{ij} $ are non-random parameters that are unobservable.
$ X_{ij} $ are non-random explanatory variables that
are observable, the errors $ \varepsilon_i $ are random,
and consequently the dependent variables $ y_i $ ($ i=1,2,\ldots,n $)
are random. $ \varepsilon_i $'s are commonly referred
to as ``noise'' or ``disturbance,'' but
most commonly as ``error.''

\underline{Note}: To include a constant in the above model,
one can simply take $ X_{i0}=1 $ for all $ i=1,\ldots,n $,
in order to introduce $ \beta_0 $ as the unobservable intercept term.
Then, the vector $ \Vector{\beta} $ will be
$ (\beta_0,\beta_1,\ldots,\beta_k)^\top_{(k+1)\times 1} $, and
$ \Matrix{X}\in\R^{n\times(k+1)} $.

\underline{Assumptions}:
\begin{enumerate}[(1)]
    \item Errors are ``centred,'' i.e., $ \E{\varepsilon_i}=0 $ for $ i=1,\ldots,n $;
    \item Errors are ``homoscedastic,'' i.e., $ \Var{\varepsilon_i}=\sigma^2<\infty $
          for $ i=1,\ldots,n $;
    \item Errors are ``uncorrelated,'' i.e., $ \Cov{\varepsilon_i,\varepsilon_j}=0 $
          for $ i\ne j $. In other words, the variance-covariance matrix of
          $ \Vector{\varepsilon}=(\varepsilon_1,\varepsilon_2,\ldots,\varepsilon_n)^\top $
          as $ \sigma^2 \Matrix{I} $, where $ \Matrix{I} $ is an identity matrix of order $ n $.
\end{enumerate}
Then, the OLS (Ordinary Least Squares) estimator of $ \Vector{\beta} $
is given by
\[ \hat{\Vector{\beta}}=(\Matrix{X}^\top \Matrix{X})^{-1}\Matrix{X}^\top \Vector{y}, \]
provided $ \Matrix{X}^\top \Matrix{X} $ is of full rank, and that this estimator
is BLUE (Best Linear Unbiased Estimator).

\underline{Preamble}: A linear estimator of the parameter $ \beta_j $
is of the form
\[ \hat{\beta}_j=c_{1j}y_1+\cdots+c_{nj}y_n \]
in which the coefficients $ c_{1j},\ldots,c_{nj} $ are
not allowed to depend on the coefficients $ \beta_j $
as they are unobservable, but are allowed to depend on $ X_{ij} $
as they are observable.

The dependence of the coefficients on $ X_{ij} $ would typically be non-linear;
but the estimator is linear in each $ y_i $ and hence in each random
error $ \varepsilon_i $. This is why we call it ``linear regression.''

The estimator $ \hat{\beta}_j $ will be unbiased if and only if
\[ \E{\hat{\beta}_j}=\beta_j \]
regardless of $ X_{ij} $. Now, let $ L=\sum_{j=0}^{k}\ell_j\beta_j $
be some linear combination of the coefficients $ \beta_0,\ldots,\beta_k $.
Then, the MSE (mean squared error) of the corresponding estimator $ \hat{L} $
is
\[ \E*{\biggl(\sum_{j=0}^{k}\ell_j\hat{\beta}_j-\sum_{j=0}^{k}\ell_j\beta_j\biggr)^{\!2}}
    =\E*{\biggl(\sum_{j=0}^{k}\ell_j(\hat{\beta}_j-\beta_j)\biggr)^{\!2}}. \]
It is the expectation of the square of the weighted sum (across the parameters)
of the differences between the estimators and the corresponding parameters
to be estimated.

Observe that as we are considering the case in which
all the parameter estimates are unbiased, the MSE is the same
as the variance of the estimator of the linear combination
$ L $.

The BLUE of $ \Vector{\beta}=(\beta_0,\ldots,\beta_k)^\top $
is the one in which the smallest MSE for every vector $ L $
of the linear combination of parameters. This is equivalent to saying
\[ \Var{\tilde{\Vector{\beta}}}-\Var{\hat{\Vector{\beta}}} \]
is a positive semi-definite matrix for any other linear unbiased estimator
$ \tilde{\Vector{\beta}} $.

\underline{Derivation of OLS estimator}:
The MSE function that we need to minimize is
\begin{align*}
    Q(\beta_0,\beta_1,\ldots,\beta_k)
     & =\sum_{i=1}^{n}(y_i-\beta_0-\beta_1x_{i1}-\beta_k x_{ik})^2                                               \\
     & =(\Vector{y}-\Matrix{X}\Vector{\beta})^\top_{1\times n}(\Vector{y}-\Matrix{X}\Vector{\beta})_{n\times 1}.
\end{align*}
The first derivative is
\begin{align*}
    \odv{Q}{\Vector{\beta}}
     & =-2 \Matrix{X}^\top (\Vector{y}-\Matrix{X}\Vector{\beta})                 \\
     & =-2 \begin{bmatrix}
               \sum_{i=1}^{n}(y_i-\beta_0-\beta_1x_{i1}-\cdots-\beta_k x_{ik})       \\
               \sum_{i=1}^{n}x_{i1}(y_i-\beta_0-\beta_1x_{i1}-\cdots-\beta_k x_{ik}) \\
               \vdots                                                                \\
               \sum_{i=1}^{n}x_{ik}(y_i-\beta_0-\beta_1x_{i1}-\cdots-\beta_k x_{ik})
           \end{bmatrix} \\
     & =\Vector{0}_{(k+1)\times 1},
\end{align*}
where $ \Matrix{X} $ is the design matrix
\[ \Matrix{X}=\begin{bmatrix}
        1      & x_{11} & \cdots & x_{1k} \\
        1      & x_{21} & \cdots & x_{2k} \\
        \vdots & \vdots & \ddots & \vdots \\
        1      & x_{n1} & \cdots & x_{nk}
    \end{bmatrix}_{n\times(k+1)}\in\R^{n\times(k+1)} \]
and $ n\ge k+1 $ (so that $ \Matrix{X}^\top \Matrix{X} $ of
order $ (k+1)\times(k+1) $ can be of full rank).

The solution from the equation
\[ -2 \Matrix{X}^\top(\Vector{y}-\Matrix{X}\Vector{\beta})=\Vector{0}\iff
    (\Matrix{X}^\top \Matrix{X})\Vector{\beta}=\Matrix{X}^\top \Vector{y} \]
is clearly
\[ \hat{\Vector{\beta}}=(\Matrix{X}^\top \Matrix{X})^{-1}\Matrix{X}^\top \Vector{y}, \]
where $ \Matrix{X} $ is the design matrix of as presented above.

\underline{Checking for minimum}:
The Hessian matrix of second derivatives is readily obtained to be
\begin{align*}
    \mathcal{H}
     & =\odv[order=2]{Q}{\Vector{\beta}}=\begin{pmatrix}
                                             n                    & \sum_{i=1}^{n}x_{i1}       & \cdots & \sum_{i=1}^{n}x_{ik}       \\
                                             \sum_{i=1}^{n}x_{i1} & \sum_{i=1}^{n}x_{i1}^2     & \cdots & \sum_{i=1}^{n}x_{i1}x_{ik} \\
                                             \vdots               & \vdots                     & \ddots & \vdots                     \\
                                             \sum_{i=1}^{n}x_{ik} & \sum_{i=1}^{n}x_{i1}x_{ik} & \cdots & \sum_{i=1}^{n}x_{ik}^2
                                         \end{pmatrix} \\
     & =2 \Matrix{X}^\top \Matrix{X}.
\end{align*}
Assume that the columns of the design matrix $ \Matrix{X} $
are linearly independent, so that $ \Matrix{X}^\top \Matrix{X} $
is invertible. Let
\[ \Matrix{X}=\begin{bmatrix}
        \Vector{c}_0 & \Vector{c}_1 & \cdots & \Vector{c}_k
    \end{bmatrix}. \]
Then,
\[ \lambda_0 \Vector{c}_0+\lambda_1 \Vector{c}_1+\cdots+\lambda_k \Vector{c}_k=0
    \iff \lambda_0=\lambda_1=\cdots=\lambda_k=0. \]
Now, let
$ (\lambda_0,\lambda_1,\ldots,\lambda_k)\in\R^{(k+1)\times 1} $
be an eigenvector of the Hessian matrix $ \mathcal{H} $. Then,
\[ \Vector{\lambda}\ne \Vector{0}\implies
    (\lambda_0 \Vector{c}_0+\lambda_1 \Vector{c}_1+\cdots+\lambda_k \Vector{c}_k)^2>0. \]
So, we find
\begin{align*}
    \begin{bmatrix}
        \lambda_0 & \lambda_1 & \cdots & \lambda_k
    \end{bmatrix}\begin{bmatrix}
                     \Vector{c}_0 \\
                     \Vector{c}_1 \\
                     \vdots       \\
                     \Vector{c}_k
                 \end{bmatrix}\begin{bmatrix}
                                  \Vector{c}_0 &
                                  \Vector{c}_1 &
                                  \cdots       &
                                  \Vector{c}_k
                              \end{bmatrix}\begin{bmatrix}
                                               \lambda_0 \\ \lambda_1 \\ \vdots \\ \lambda_k
                                           \end{bmatrix}
     & =\Vector{\lambda}^\top(\tfrac{1}{2}\mathcal{H})\Vector{\lambda} &  & \text{since $ \mathcal{H}=2 \Matrix{X}^\top \Matrix{X} $}    \\
     & =\tfrac{1}{2}\Vector{\lambda}^\top \mathcal{H}\Vector{\lambda}                                                                    \\
     & =\tfrac{1}{2}(\Vector{\lambda}^\top \Vector{\lambda})a          &  & \text{since $ \Vector{\lambda} $ is a vector of eigenvalues} \\
     & >0,
\end{align*}
where $ a $ is the eigenvalue corresponding to the eigenvector $ \Vector{\lambda} $.
Further,
\[ \Vector{\lambda}^\top \Vector{\lambda}=\sum_{i=1}^{k}\lambda_i^2>0\implies a>0. \]
Finally, as eigenvalue $ \Vector{\lambda} $ is arbitrary, all eigenvalues
of $ \mathcal{H} $ are positive, and so $ \mathcal{H} $ is positive definite.
Hence, the OLS estimator
\[ \hat{\Vector{\beta}}=(\Matrix{X}^\top \Matrix{X})^{-1}\Matrix{X}^\top \Vector{y} \]
does indeed correspond to a global minimum.

\underline{Uniqueness of the OLS estimator}: Assume
\[ \tilde{\Vector{\beta}}=\Matrix{A}\Vector{y} \]
be another linear estimator of $ \Vector{\beta} $ with
\[ \Matrix{A}=(\Matrix{X}^\top \Matrix{X})^{-1}\Matrix{X}^\top + \Matrix{B}, \]
where $ \Matrix{B} $ is a $ (k+1)\times n $ non-zero matrix. Then, we find
\begin{align*}
    \E{\tilde{\Vector{\beta}}}
     & =\E{\Matrix{A}\Vector{y}}                                                                                                                                                                                                                            \\
     & =\E*{\Set*{(\Matrix{X}^\top \Matrix{X})^{-1}\Matrix{X}^\top + \Matrix{B}}(\Matrix{X}\Vector{\beta}+\Vector{\varepsilon})}                                                                                                                            \\
     & =\Set*{(\Matrix{X}^\top \Matrix{X})^{-1}\Matrix{X}^\top + \Matrix{B}}\Matrix{X}\Vector{\beta}+\Set*{(\Matrix{X}^\top \Matrix{X})^{-1}\Matrix{X}^\top + \Matrix{B}}\E{\Vector{\varepsilon}}                                                           \\
     & =\Set*{(\Matrix{X}^\top \Matrix{X})^{-1}\Matrix{X}^\top + \Matrix{B}}\Matrix{X}\Vector{\beta}                                                                                              &  & \text{since $ \E{\Vector{\varepsilon}}=\Vector{0} $} \\
     & =(\Matrix{X}^\top \Matrix{X})^{-1}\Matrix{X}^\top \Matrix{X}\Vector{\beta}+\Matrix{B}\Matrix{X}\Vector{\beta}                                                                                                                                        \\
     & =(\Matrix{I}_{(k+1)\times(k+1)}+\Matrix{B}\Matrix{X})\Vector{\beta},
\end{align*}
where $ \Matrix{I}_{(k+1)\times(k+1)} $ is an identity matrix of
order $ (k+1)\times(k+1) $. Thus, $ \tilde{\Vector{\beta}} $
is an unbiased estimator of $ \Vector{\beta} $ if and only if
\[ \Matrix{B}\Matrix{X}=\Matrix{O}. \]
Now, let us consider
\begin{align*}
    \Var{\tilde{\Vector{\beta}}}
     & =\Var{\Matrix{A}\Vector{y}}                                                                                                                                                                                        \\
     & =\Matrix{A}\Var{\Vector{y}}\Matrix{A}^\top                                                                                                                                                                         \\
     & =\Matrix{A}(\sigma^2 \Matrix{I})\Matrix{A}^\top                                                                                                                                                                    \\
     & =\sigma^2 \Matrix{A}\Matrix{A}^\top                                                                                                                                                                                \\
     & =\sigma^2\Set*{(\Matrix{X}^\top \Matrix{X})^{-1}\Matrix{X}^\top + \Matrix{B}}\Set*{(\Matrix{X}^\top \Matrix{X})^{-1}\Matrix{X}^\top + \Matrix{B}}^\top                                                             \\
     & =\sigma^2\Set*{(\Matrix{X}^\top \Matrix{X})^{-1}\Matrix{X}^\top + \Matrix{B}}\Set*{\Matrix{X}(\Matrix{X}^\top \Matrix{X})^{-1} + \Matrix{B}^\top}                                                                  \\
     & =\sigma^2\Set*{(\Matrix{X}^\top \Matrix{X})^{-1}\Matrix{X}^\top \Matrix{X}(\Matrix{X}^\top \Matrix{X})^{-1}+(\Matrix{X}^\top \Matrix{X})^{-1}\Matrix{X}^\top \Matrix{B}^\top+
    \Matrix{B}\Matrix{X}(\Matrix{X}^\top \Matrix{X})^{-1}+\Matrix{B}\Matrix{B}^\top}                                                                                                                                      \\
     & =\sigma^2(\Matrix{X}^\top \Matrix{X})^{-1}+\sigma^2(\Matrix{X}^\top \Matrix{X})^{-1}(\Matrix{B}\Matrix{X})^\top+\sigma^2(\Matrix{B}\Matrix{X})(\Matrix{X}^\top \Matrix{X})^{-1}+\sigma^2 \Matrix{B}\Matrix{B}^\top \\
     & =\sigma^2(\Matrix{X}^\top \Matrix{X})^{-1}+\sigma^2 \Matrix{B}\Matrix{B}^\top,\;\text{due to unbiasedness condition}                                                                                               \\
     & =\Var{\hat{\Vector{\beta}}}+\sigma^2 \Matrix{B}\Matrix{B}^\top,\;\text{since $ \Var{\hat{\Vector{\beta}}}=\sigma^2(\Matrix{X}^\top \Matrix{X})^{-1} $}.
\end{align*}
As $ \Matrix{B}\Matrix{B}^\top $ is a positive semidefinite matrix,
$ \Var{\tilde{\Vector{\beta}}} $ exceeds $ \Var{\hat{\Vector{\beta}}} $
by a positive semidefinite matrix.

\underline{Another interpretation for this property}:
Let $ \Vector{\ell}^\top \hat{\Vector{\beta}} $ and $ \Vector{\ell}^\top \tilde{\Vector{\beta}} $
be both linear unbiased estimators of $ \Vector{\ell}^\top \Vector{\beta} $. Then,
\begin{align*}
    \Var{\Vector{\ell}^\top \tilde{\Vector{\beta}}}
     & =\Vector{\ell}^\top \Var{\tilde{\Vector{\beta}}}\Vector{\ell}                                                                                  \\
     & =\sigma^2 \Vector{\ell}^\top\Set*{(\Matrix{X}^\top \Matrix{X})^{-1}+\Matrix{B}\Matrix{B}^\top}\Vector{\ell}                                    \\
     & =\sigma^2 \Vector{\ell}^\top(\Matrix{X}^\top \Matrix{X})^{-1}\Vector{\ell}+\sigma^2 \Vector{\ell}^\top \Matrix{B}\Matrix{B}^\top \Vector{\ell} \\
     & =\Var{\Vector{\ell}^\top\hat{\Vector{\beta}}}+\sigma^2(\Matrix{B}^\top \Vector{\ell})^\top(\Matrix{B}^\top \Vector{\ell})                      \\
     & =\Var{\Vector{\ell}^\top\hat{\Vector{\beta}}}+\sigma^2\lVert \Matrix{B}^\top \Vector{\ell}\rVert                                               \\
     & \ge \Var{\Vector{\ell}^\top\hat{\Vector{\beta}}}.
\end{align*}
Furthermore, equality holds if and only if $ \Matrix{B}^\top \Vector{\ell}=\Vector{0} $.
Then, in this case, we readily find
\begin{align*}
    \Vector{\ell}^\top \tilde{\Vector{\beta}}
     & =\Vector{\ell}^\top\Set*{(\Matrix{X}^\top \Matrix{X})^{-1}\Matrix{X}+\Matrix{B}}\Vector{y}                                                                                                   \\
     & =\Vector{\ell}^\top(\Matrix{X}^\top \Matrix{X})^{-1}\Matrix{X}^\top\Vector{y}+\Vector{\ell}^\top\Matrix{B}\Vector{y}                                                                         \\
     & =\Vector{\ell}^\top \hat{\Vector{\beta}}+(\Matrix{B}^\top \Vector{\ell})^\top \Vector{y},\;\text{since $ \hat{\Vector{\beta}}=(\Matrix{X}^\top \Matrix{X})^{-1}\Matrix{X}^\top \Vector{y} $} \\
     & =\Vector{\ell}^\top \hat{\Vector{\beta}},\;\text{since $ \Matrix{B}^\top \Vector{\ell}=\Vector{0} $}.
\end{align*}
This establishes that the equality holds if and only if
$ \Vector{\ell}^\top \tilde{\Vector{\beta}}=\Vector{\ell}^\top \hat{\Vector{\beta}} $,
which shows the uniqueness of the OLS estimator as a
Best Linear Unbiased Estimator.

\underline{Warning}: The condition that the estimator be unbiased
cannot be dropped, since biased estimators may exist
with lower variance (and even with lower MSE).

To see this, consider the following example: Let
$ X_1,X_2,\ldots,X_n $ be a random sample from $ \EXP{\theta} $
distribution. Then, $ \bar{X} $ is an unbiased estimator of $ \theta $
with
\[ \E{\bar{X}}=\theta\quad\text{and}\quad\Var{\bar{X}}=\frac{\theta^2}{n}. \]
Now, consider another linear estimator of $ \theta $ to be
\[ \tilde{\theta}=\sum_{i=1}^{n}X_i a_i. \]
Then, evidently, we have:
\begin{align*}
    \E{\tilde{\theta}}          & =\sum_{i=1}^{n}a_i\E{X_i}=\theta\sum_{i=1}^{n}a_i,                           \\
    \text{Bias}(\tilde{\theta}) & =\E{\tilde{\theta}}-\theta=\theta\Set[\bigg]{\sum_{i=1}^{n}a_i-1},           \\
    \Var{\tilde{\theta}}        & =\sum_{i=1}^{n}a_i^2\Var{X_i}=\theta^2 \sum_{i=1}^{n}a_i^2,
    \text{MSE}(\tilde{\theta})  & =\Var{\tilde{\theta}}+\bigl(\text{Bias}(\tilde{\theta})\bigr)^2              \\
                                & =\theta^2\Set*{\sum_{i=1}^{n}a_i^2+\biggl(\sum_{i=1}^{n}a_i-1\biggr)^{\!2}}.
\end{align*}
To minimize $ \text{MSE}(\tilde{\theta}) $, we find the partial derivatives to be
\begin{align*}
    \pdv{\text{MSE}(\tilde{\theta})}{a_i}
                 & =\theta^2\Set*{2 a_i+2\biggl(\sum_{j=1}^{n}a_j-1\biggr)}=0 \\
    \implies a_i & =-\biggl(\sum_{j=1}^{n}a_j-1\biggr),\; i=1,\ldots,n.
\end{align*}
Adding these $ n $ equations,
\[ \sum_{j=1}^{n}a_j=-n\biggl(\sum_{j=1}^{n}a_j-1\biggr)=-n \sum_{j=1}^{n}a_j+n \]
and so
\[ \sum_{j=1}^{n}a_j=\frac{n}{n+1}. \]
This immediately yields
\[ a_i=-\biggl(\sum_{j=1}^{n}a_j-1\biggr)=-\biggl(\frac{n}{n+1}-1\biggr)=\frac{1}{n+1},\;\text{for }i=1,\ldots,n. \]
Hence, the linear estimator $ \tilde{\theta} $ becomes
\[ \tilde{\theta}=\sum_{i=1}^{n}a_i X_i=\frac{1}{n+1}\sum_{i=1}^{n}X_i=\frac{n}{n+1}\bar{X}. \]
Observe that
\begin{align*}
    \text{Bias}(\tilde{\theta}) & =\frac{n}{n+1}\theta-\theta=-\frac{1}{n+1}\theta,                                                 \\
    \Var{\tilde{\theta}}        & =\frac{n^2}{(n+1)^2}\Var{\bar{X}}=\frac{n^2}{(n+1)^2}\frac{\theta^2}{n}=\frac{n}{(n+1)^2}\theta^2 \\
                                & <\frac{\theta^2}{n},
\end{align*}
which is the variance of $ \bar{X} $, the OLS estimator of $ \theta $. Furthermore,
\begin{align*}
    \text{MSE}(\tilde{\theta})
     & =\Var{\tilde{\theta}}+\bigl(\text{Bias}(\tilde{\theta})\bigr)^2          \\
     & =\frac{n}{(n+1)^2}\theta^2+\frac{\theta^2}{(n+1)^2}=\frac{\theta^2}{n+1} \\
     & <\frac{\theta^2}{n},
\end{align*}
the variance of $ \bar{X} $ (the OLS estimator). This is an example
to demonstrate why unbiasedness is needed in the Gauss-Markov theorem.

\underline{Things to keep in mind}:
\begin{enumerate}[(1)]
    \item In most applications of OLS, the regressors (parameters of interest)
          in the design matrix $ \Matrix{X} $ are assumed to be fixed in
          repeated samples. This assumption may not be reasonable in some cases (like Econometrics).
          Instead, there are assumptions of Gauss-Markov theorem are stated conditional on $ \Matrix{X} $;
    \item $ \Vector{y} $ is assumed to be a linear function of the variables
          specified in the model. But, the specification must be linear in its parameters;
          it does not mean that there must be a linear relationship between $ \Vector{y} $
          and the independent variables $ \Matrix{X} $. The independent variables
          can take on non-linear forms as long as the parameters are linear;
    \item Data transformations can be made use to convert an equation into a linear form.
          For example, ``the power law model'' of the form
          \[ y=\alpha X^\beta e^{\varepsilon} \]
          can be transformed, by natural logarithms, to
          \[ \ln{y}=\ln{\alpha}+\beta\ln{X}+\varepsilon. \]
          Similarly, the ``Cobb-Douglas model'' of the form
          \[ y=A L^{\alpha}K^{\alpha} \]
          can be transformed, by natural logarithms, to
          \[ \ln{y}=\ln{A}+\alpha\ln{L}+(1-\alpha)\ln{K}+\varepsilon; \]
          remember the parameters that minimize the residuals of the transformed
          model would not minimize the residuals of the original model;
    \item Recall we assumed that $ \Matrix{X} $ must have full column rank, i.e.,
          \[ \text{rank}(\Matrix{X})=k+1; \]
          otherwise, $ \Matrix{X}^\top \Matrix{X} $ is not invertible and consequently
          the OLS estimator cannot be computed.

          This gets violated in case of ``perfect multicollinearity,''
          a case when some explanatory variables are linearly dependent. This
          can happen, for example, in a ``dummy variable trap,''
          when a base dummy variable is not omitted resulting in perfect correlation
          between the dummy variable and the constant term.

          Interestingly, ``near multicollinearity''
          would result in unbiased estimation, even though with much lesser precision.
          The estimators would be less efficient and very sensitive to particular sets of data.

          Multicollinearity can be detected from ``variance inflation factor''
          (which is the variance of estimating a parameter in a model that includes
          multiple other parameters divided by the variance of a model using only that variable),
          or ``conditional number'' (which measures how much the output value can change for a small change
          in the input argument);
    \item ``Spherical errors model'' means that the outer product of the error
          vector must be spherical, i.e.,
          \[ \E{\Vector{\varepsilon}^\top \Vector{\varepsilon}\given \Matrix{X}}
              =\Var{\Vector{\varepsilon}\given \Matrix{X}}=
              \begin{bmatrix}
                  \sigma^2 & \cdots & 0        \\
                  \vdots        & \ddots & \vdots        \\
                  0        & \cdots & \sigma^2
              \end{bmatrix}_{n\times n}=\sigma^2 \Matrix{I},\;\text{with } \sigma^2>0. \]
          This means that the error term has uniform variance (i.e.,
          homoscedasticity) and no serial dependence. Recall
          that in the case of multivariate normal distribution
          for $ \Vector{\varepsilon}\sim\mathcal{N}_n(\Vector{0},\sigma^2 \Matrix{I}) $,
          the equation $ f(\Vector{\varepsilon})=c $ will correspond to a ball
          centred at $ \Vector{0} $ with radius $ \sigma $ in $ \R^n $.

          Homoscedasticity will get violated when there is ``autocorrelation.''
          In this case, the OLS estimator is still unbiased, but is inefficient.

          When there is ``spherical errors,'' the OLS estimator remains BLUE;
    \item Finally, it may be observed that the expectation of errors,
          conditioned on the regressors, is
          \[ \E{\Vector{\varepsilon}_i\given \Vector{X}}=\E{\Vector{\varepsilon}_i\given \Vector{x}_1,\ldots,\Vector{x}_n}=\Vector{0}, \]
          where $ \Vector{x}_i=(1,x_{i1},\ldots,x_{ik})^\top $ is the data vector of
          regressors from the $ i\textsuperscript{th} $ observation, and then
          \[ \Matrix{X}=\begin{pmatrix}
                  \Vector{x}_1^\top & \Vector{x}_2^\top & \cdots & \Vector{x}_n^\top
              \end{pmatrix}^\top \]
          is the design matrix (on the data matrix). So, the above statement means that
          $ \Vector{x}_i $ and $ \Vector{\varepsilon}_i $ are orthogonal to each other, i.e., their inner product
          \[ \E{\Vector{x}_j\cdot \varepsilon_i}
              =\begin{pmatrix}
                  \E{1\cdot \varepsilon_i}      \\
                  \E{x_{j1}\cdot \varepsilon_i} \\
                  \vdots                        \\
                  \E{x_{jk}\cdot \varepsilon_i}
              \end{pmatrix}=\Vector{0},\;\text{for all }i,j. \]
          This will get violated if the explanatory variables are stochastic, for example,
          when they are measured with error, or when they are ``endogenous'' (i.e.,
          when an explanatory variable is correlated with error variable).

          Instrument variable methods become useful in this case!
\end{enumerate}
\makeheading{Lecture 19}{\printdate{2022-06-20}}%chktex 8


\makeheading{Lecture 20}{\printdate{2022-06-22}}%chktex 8
\section{Quadratic Congruences}
Let $ n\ge 3 $ be a modulus and $ a,b,c\in\mathbf{Z} $. We will now turn our attention
to the quadratic congruence
\[ ax^2+bx+c\equiv 0\Mod{n}, \]
where $ a,b,c\in\mathbf{Z} $, $ n\nmid a $, and $ x $ is unknown modulo $ n $.

In terms of equations in $ \mathbf{Z}_n $,
\[ [a]x^2+[b]x+[c]=[0], \]
where $ [a],[b],[c]\in\mathbf{Z} $, and $ x $ is an unknown residue class in $ \mathbf{Z}_n $.

Note:
\begin{enumerate}[(1)]
    \item For a quadratic congruence $ ax^2+bx+c\equiv 0\Mod{n} $, we require $ n\nmid a $. Otherwise,
          the quadratic congruence collapse to the linear congruence $ bx+c\equiv 0\Mod{n} $.
    \item For $ n=2 $, the quadratic congruence $ ax^2+bx+c\equiv 0\Mod{n} $ collapses into a linear
          congruence. Indeed, by Fermat's Invariant (Corollary 1 Lecture 14), $ x^2\equiv x\Mod{2} $
          regardless of $ x $ and so
          \[ ax^2+bx+c\equiv (a+b)x+c\Mod{2}. \]
    \item For simplicity of interpretation, we will assume the $ n $ to be prime and
          denote it by $ p $. If $ p $ is an odd prime (i.e., $ p\ne 2 $), then $ \frac{p-1}{2}\in\mathbf{Z} $.
\end{enumerate}
\begin{Proposition}{}{}
    Let $ p $ be an odd prime, and $ a,b,c,\in\mathbf{Z} $ with $ p\nmid a $. The quadratic congruence
    \[ ax^2+bx+c\equiv 0\Mod{p} \]
    has a solution if and only if the congruence
    \[ y^2\equiv b^2-4ac\Mod{p} \]
    has a solution. In that case $ y\equiv 2ax+b\Mod{p} $.
    \tcblower{}
    \textbf{Proof}:
\end{Proposition}
Proposition 1 tells us that solving the quadratic congruence
\[ ax^2+bx+c\equiv 0\Mod{p} \]
is equivalent to solving a simplifier quadratic congruence
\[ y^2\equiv d\Mod{p}, \]
where $ d=b^2-4ac $. The integer $ d $ is called the discriminant of the polynomial
$ ax^2+bx+c $.
\begin{Example}{}{}
    Solve $ 2x^2+18x+3\equiv 0\Mod{23} $.
    \tcblower{}
    \textbf{Solution}: The discriminant of this quadratic polynomial modulo $ 23 $ is
    \[ 18^2-4(2)(3)\equiv 300\equiv 1\Mod{23}. \]
    According to Proposition 1, we should first solve
    \[ y^2\equiv 1\Mod{23}. \]
    By inspection, we see that $ y=1 $ and $ y=22 $ are the solutions. According to Proposition 1, we need to solve
    \[ 4x+18\equiv 1\Mod{23},\text{ and }4x+18\equiv 22\Mod{23}. \]
    Solving them gives
    \[ x\equiv 1\Mod{23},\text{ and }x\equiv 13\Mod{23}, \]
    which are the solutions of the given congruence.
\end{Example}
In order to find solutions of $ y^2\equiv d\Mod{p} $, we need to understand which residue classes
of $ \mathbf{Z}_p $ are squares.
\begin{Definition}{}{}
    Let $ p $ be an odd prime. A residue $ [a] $ in $ \mathbf{Z}_p $ is called a quadratic residue when
    \[ [a]\in\mathbf{Z}_p^*\text{ and $ [a]=[b]^2 $ for some other residue $ [b]\in\mathbf{Z}_p^* $.} \]
    If no such $ [b] $ exists, then $ [a] $ is called a quadratic non-residue.

    In terms of congruences, an integer $ a $ is a quadratic residue modulo $ p $ if
    \[ (p,a)=1\land \exists b\in\mathbf{Z}\; a\equiv b^2\Mod{p}\land (b,p)=1. \]
\end{Definition}
\begin{Example}{}{}
    Find quadratic residues in $ \mathbf{Z}_7^* $.
    \tcblower{}
    \textbf{Solution}: Note that
    \begin{align*}
        1^2 & \equiv 1\Mod{7} \\
        2^2 & \equiv 4\Mod{7} \\
        3^2 & \equiv 2\Mod{7} \\
        4^2 & \equiv 2\Mod{7} \\
        5^2 & \equiv 4\Mod{7} \\
        6^2 & \equiv 1\Mod{7}
    \end{align*}
    Thus, we can say that integers having quadratic residues modulo $ 11 $ are those that
    are congruent to $ 1,2,4 $.
\end{Example}
\begin{Exercise}{}{}
    Determine all quadratic non-residues modulo $ 17 $.
\end{Exercise}
\begin{Proposition}{}{}
    Let $ p $ be an odd prime. Then, there are exactly $ \frac{p-1}{2} $ quadratic residues
    modulo $ p $ and exactly $ \frac{p-1}{2} $ quadratic non-residues modulo $ p $.
\end{Proposition}
\begin{Exercise}{}{}
    Prove Proposition 1.
\end{Exercise}
\textbf{Detecting Quadratic Residues with Primitive Roots}:
Since $ p $ is an odd prime, by the Primitive Root Theorem (Theorem 2 Lecture 17),
there exists a primitive root modulo $ p $, say $ a $, and by Proposition 2 (Lecture 17),
the number of such primitive roots is $ \varphi(p-1) $. Also, the powers
\[ a,a^2,\ldots,a^{p-1} \]
are all distinct modulo $ p $ and exhausts $ \mathbf{Z}_p^* $. By looking at $ k $, we can
decide whether $ a^k $ is a quadratic residue or not.
\begin{Proposition}{}{}
    If $ a $ is a primitive root modulo $ p $ with $ (a,p)=1 $ and $ a^i\equiv a^j\Mod{p} $,
    then these exponents will be both even or both odd.
    \tcblower{}
    \textbf{Proof}:
\end{Proposition}
\makeheading{Lecture 21}{\printdate{2022-06-24}}%chktex 8
For a simpler notation, let's write $\QR$ for a quadratic residue modulo $p$ and $\NR$
for a quadratic non residue modulo $p$: From the Example 1 (Lecture 20), we see that we have
\begin{align*}
    \QR\cdot \QR & =\QR  \\
    \QR\cdot \NR & =\NR  \\
    \NR\cdot \NR & =\QR.
\end{align*}
Note that quadratic residues are the perfect squares of $ \mathbf{Z}_p^* $ and you can easily
get quadratic residues by squaring all the elements of $ \mathbf{Z}_p^* $.
\begin{Proposition}{}{}
    If $ p $ is an odd prime, then
    \begin{enumerate}[(1)]
        \item The product of two quadratic residues modulo $p$ is a quadratic residue
              modulo $p$.
        \item The product of a quadratic residue modulo $p$ and a quadratic non-residue
              modulo $p$ is quadratic non-residue modulo $p$.
        \item The product of two quadratic non-residues modulo $p$ is a quadratic residue.
    \end{enumerate}
    \tcblower{}
    \textbf{Proof}:
\end{Proposition}
What does the multiplication rule of quadratic residues and quadratic non
residues remind of you?
\begin{align*}
    1\cdot 1       & =1  \\
    1\cdot (-1)    & =-1 \\
    (-1)\cdot (-1) & =1.
\end{align*}
In 1798, the French Mathematician Adrien-Marie Legendre introduced a handy
symbol to mark this distinction.

For an odd prime $ p $, and $ a\in\mathbf{Z} $ with $ p\nmid a $, the \textbf{Legendre symbol},
$ \legendre{a}{p} $ is defined by
\[ \legendre{a}{p}=\begin{cases}
        +1, & \text{$a$ is a $\QR$ modulo $ p $}, \\
        -1, & \text{$a$ is a $\NR$ modulo $ p $}.
    \end{cases} \]
Note:
\begin{enumerate}[(1)]
    \item Keep in mind that the Legendre symbol is not a fraction, even though it
          sort of look like one.
    \item $ 1 $ is a quadratic residue modulo $ p $ for any odd prime $ p $; that is, $ \legendre{1}{p}=1 $.
    \item $ -1 $ might or might not be a quadratic residue, depending on the $ p $. For example,
          $ \legendre{-1}{19}=-1 $ and $ \legendre{-1}{5}=1 $.
    \item We can rewrite Proposition 1 using the Legendre symbol as
          \[ \legendre{a}{p}\legendre{b}{p}=\legendre{ab}{p}. \]
    \item If $ a\equiv b\Mod{p} $, then $ \legendre{a}{p} $ because the quadratic residue is the same for all congruent integers.
\end{enumerate}
\begin{Exercise}{}{}
    Calculate $ \legendre{3}{13} $, $\legendre{11}{13}$, and $ \legendre{6}{17} $.
\end{Exercise}
The Proposition 1 suggest an Algorithm for calculating the Legendre polynomial $ \legendre{a}{p} $.
First, we need to find the primitive root $b$ modulo $p$ and then determine
the parity of $x$ in $ b^x\equiv a\Mod{p} $. Euler came up with a much simpler procedure.
\begin{Proposition}{Euler's Test}{}
    If $ p $ is an odd prime and $ a\in\mathbf{Z} $ with $ p\nmid a $, then
    \[ a^{\frac{p-1}{2}}=\legendre{a}{p}\Mod{p}. \]
    \tcblower{}
    \textbf{Proof}:
\end{Proposition}
\begin{Example}{}{}
    Does $ 79 $ have a quadratic residue modulo $ 31 $?
    \tcblower{}
    \textbf{Solution}: Note that $ \frac{31-1}{2}=15 $ and by Euler's Test we reduce $ 79^{15}\Mod{31} $ (using the double-and-add algorithm).
    Note that $ 15=1+2+4+8 $, and so $ 17^{15}=17^1\cdot 17^2\cdot 17^4+17^8 $. We have
    \[ 17\equiv 17,\; 17^2\equiv 10,\; 17^4\equiv 7,\; 17^8\equiv 18\Mod{31}. \]
    Thus,
    \[ 17^{15}=17\cdot 10\cdot 7\cdot 18\equiv 30\equiv -1\Mod{31}. \]
    According to Euler's test, $79$ does not have a quadratic residue modulo $31$. In the
    Language of the Legendre symbol we have found that
    \[ \legendre{79}{31}=-1. \]
\end{Example}
\makeheading{Lecture 22}{\printdate{2022-10-14}}%chktex 8
\section{Introduction to Bayesian Inference}
\subsection*{Introduction}
\begin{itemize}
      \item I would like to discuss some Bayesian topics in multivariate statistics.
      \item But I am aware that many of you will not yet have encountered the
            Bayesian paradigm.
      \item Therefore, I am going to devote a lecture or two to introducing the
            Bayesian paradigm.
      \item This introduction will assume $n$ realizations of a univariate random
            variable $ X $, which we denote $ x_1,\ldots,x_n $ or $ \Vector{x} $.
\end{itemize}
\subsection*{Probability \& the Bayesian Approach}
\begin{itemize}
      \item First, we are going to look a little more closely at what a probability
            means.
      \item We will also look at some aspects of the Bayesian approach.
      \item I am going to present this material from what may be considered a
            reasonably pro-Bayesian viewpoint.
      \item This is only fair really\textellipsis{}
\end{itemize}
\subsection*{What is Probability?}
\begin{itemize}
      \item You will have seen this standard definition before.
      \item We talk about the probability of an event. If $ E $ is an event, then we denote the probability of $ E $
            occurring by $ \Prob{E} $.
      \item For an experiment with possible outcomes $ E_1,\ldots,E_n $, the probability $ \Prob{E_i} $ must obey the following rules:
            \begin{itemize}
                  \item $ 0\le \Prob{E_i}\le 1 $ for all $ E_i $.
                  \item $ \Prob{E_1}+\cdots+\Prob{E_n}=1 $.
                  \item $ \Prob{\emptyset}=0 $.
                  \item The OR law for mutually exclusive events.
            \end{itemize}
\end{itemize}
\subsection*{The Rules of Probability}
\begin{itemize}
      \item A probability is a number that measures our uncertainty in a random
            variable $X$.
      \item  A popular way to view a probability is to consider that it must obey the
            following three laws, known as the \textbf{calculus of probability}:
            \begin{itemize}
                  \item Convexity: $ 0\le \Prob{X}\le 1 $.
                  \item Additivity: If both $X_1$ and $X_2$ are mutually exclusive then
                        \[ \Prob{X_1\cup X_2}=\Prob{X_1}+\Prob{X_2}. \]
                  \item Multiplicativity:
                        \[ \Prob{X_1\cap X_2}=\Prob{X_1}\Prob{X_2\given X_1}. \]
            \end{itemize}
      \item We must also consider that a certain event that is assigned a probability
            of $1$ and an impossible event is assigned a probability of $0$.
\end{itemize}
\subsection*{Alternatives to Probability}
\begin{itemize}
      \item There are several competing theories to probability for quantifying
            uncertainty.
      \item The most well known is fuzzy logic and its counterpart fuzzy set theory.
      \item Another is possibility theory.
      \item These competitors have been used successfully in some applications.
      \item However, they all lack a justification from more fundamental ideas about
            uncertainty that probability possesses.
\end{itemize}
\subsection*{The Meaning of a Probability}
\begin{itemize}
      \item What precisely is the meaning of a probability?
      \item We can easily define a probability very precisely, but that will not tell us
            it means.
      \item The actually meaning of a probability is a topic that continues to be the
            subject of some debate.
      \item We are going to look at two popular concepts.
      \item Namely, \textbf{physical probability} and \textbf{psychological probability}.
\end{itemize}
\subsection*{Physical Probability}
\begin{itemize}
      \item Also known as material probability, intrinsic probability, objective
            probability or propensity.
      \item Probability is viewed as a property of the material world, like mass or
            volume, which exists irrespective of minds and logic.
      \item A physical probability is typically defined in one of two ways: first
            principles and relative frequency.
      \item \textbf{First Principles}:
            There are $n$ possible outcomes to some random quantity. For an event
            $E$ to occur, one of $r$ specific outcomes must occur. By making an
            assumption that each outcome is equally likely, $ \Prob{E}=r/n $.
      \item \textbf{Relative Frequency}:
            The proportion of times that $X$ has been observed to occur in a long
            sequence of essentially identical experiments.
      \item The relative frequency notion is usually adopted and people who adopt
            this interpretation of probability are called frequentists.
\end{itemize}
\subsection*{Psychological Probability}
\begin{itemize}
      \item A degree of belief, or intensity of conviction, used for betting or making
            decisions, not necessarily after mature consideration.
      \item It does not have to be consistent with one's other opinions.
      \item When a psychological probability is coherent (see De Finetti) and obeys
            the laws of probability then it is called \textbf{subjective probability}.
      \item The subjective probability notion is usually adopted and people who
            adopt this notion are called \textbf{subjectivists} or, some would say, \textbf{Bayesians}.
\end{itemize}
\subsection*{Frequentist vs Subjectivist}
\begin{itemize}
      \item Frequentist
            \begin{enumerate}
                  \item Probably still the most widely used.
                  \item Consistent (everyone should get the same probability for a given
                        event).
                  \item Assumes that an experiment can be repeated indefinitely under
                        almost identical conditions. This excludes one-off events.
                  \item What are `almost identical conditions'?
            \end{enumerate}
      \item Subjectivist
            \begin{enumerate}
                  \item Concerned with individual behaviour.
                  \item Varies from individual to individual --- no ``correct'' probability for a
                        given event.
                  \item Applies to a wider range of situations, including one-off situations.
            \end{enumerate}
\end{itemize}
\subsection*{The Bayesian Paradigm}
\begin{itemize}
      \item For our purposes today, we are going to assume that all Bayesians are
            subjectivists.
      \item Bayesians obey the laws of probability strictly and conduct statistical
            inference accordingly.
      \item One can think of Bayesians as using observed data to update their
            existing, or prior, knowledge.
      \item In a moment, an example, first some revision.
\end{itemize}
\subsection*{Conditional Probability}
\begin{itemize}
      \item Recall the \textbf{AND} Rule: if $E$ and $F$ are two events then
            \[ \Prob{E\cap F}=\Prob{E}\Prob{F\given E}, \]
            where $F \mid E$ means the occurrence of an event $F$ given that an event $E$
            has already occurred.
      \item Now, dividing both sides of the above equation by $ \Prob{E} $ gives us the
            definition of a conditional probability.
      \item The probability that an event $F$ occurs given that an event $E$ has
            already occurred is given by
            \[ \Prob{F\given E}=\frac{\Prob{E\cap F}}{\Prob{E}}. \]
      \item Using the fact that $ \Prob{E\cap F}=\Prob{F\cap E} $ and noting that
            $ \Prob{F\cap E}=\Prob{E\given F}\Prob{F} $, we can write the expression for $ \Prob{F\given E} $
            as follows.
            \begin{equation}
                  \Prob{F\given E}=\frac{\Prob{E\given F}\Prob{F}}{\Prob{E}}\label{eqbayes}.
            \end{equation}
      \item Now, we can also rewrite the term $ \Prob{E} $ in this equation.
\end{itemize}
\subsection*{The Partition Law}
\begin{itemize}
      \item This approach can be generalized to get a general formula for $ \Prob{E} $ in
            terms of conditional probabilities.
      \item \textbf{The Partition Theorem}: Suppose the outcome of an event $ E $ depends on an event
            $ F $ which has possible outcomes $ F_1,\ldots,F_n $, then
            \[ \Prob{E}=\sum_{i=1}^{n}\Prob{E\given F_i}\Prob{F_i}. \]
\end{itemize}
\subsection*{Bayes' Theorem}
\begin{itemize}
      \item The Partition Theorem replaces $ \Prob{E} $ in the denominator of~\Cref{eqbayes} to give Bayes' Theorem.
      \item \textbf{Bayes' Theorem}: Suppose the outcome of an event $ E $ depends on an event
            $ F $ which has possible outcomes $ F_1,\ldots,F_n $, then
            \[ \Prob{F_j\given E}=\frac{\Prob{E\given F_j}\Prob{F_j}}{\sum_{i=1}^{n}\Prob{E\given F_i}\Prob{F_i}}, \]
            for $ j=1,2,\ldots,n $.
      \item However, the result is not actually due to Bayes.
\end{itemize}
\subsection*{Thomas Bayes}
\begin{itemize}
      \item Fisher wrote that:

            ``For the first serious attempt known to us to give a rational
            account of the process of scientific inference as a means of
            understanding the real world, in the sense in which this term is
            understood by experimental investigators, we must look back over
            two hundred years to an English clergyman, the Reverend Thomas
            Bayes, whose life spanned the first half of the eighteenth century.''
      \item What Bayes actually showed was for one case of continuous $X$, and it is
            not clear that he would agree entirely with all aspects of what has
            become known as \textbf{Bayesian inference}.
\end{itemize}
\subsection*{A Bayesian Example}
\begin{itemize}
      \item Suppose I am Bayesian.
      \item I want to estimate the height of men in Ireland.
      \item A reasonable description of my (prior) belief is that the height of men in
            Ireland is $ \N{1.70,0.15^2} $.
      \item Data are then collected, and my views are `updated'.
      \item But how?
\end{itemize}
\subsection*{The Bayesian Approach I}
\begin{itemize}
      \item Denote the probability density that describes my prior belief by $ h(\theta) $.
      \item Then, $ h(\theta) $ is called the \textbf{prior distribution}.
      \item Once we observe our data, we write down the likelihood
            $ \mathcal{L}(\theta\mid \Vector{x})=p(\Vector{x}\mid \theta) $.
      \item We then compute the \textbf{posterior distribution} $ h(\theta\mid \Vector{x}) $.
      \item We can find $ h(\theta\mid \Vector{x}) $ using Bayes theorem.
\end{itemize}
\subsection*{The Bayesian Approach II}
\begin{itemize}
      \item From Bayes theorem,
            \[ h(\theta\mid \Vector{x})=\frac{p(\Vector{x}\mid \theta)h(\theta)}{\int_{\Theta}p(\Vector{x}\mid \theta)h(\theta)\odif{\theta}}. \]
      \item Noting that the denominator is a constant with respect to $ \theta $, we can write
            \[ h(\theta\mid \Vector{x})\propto p(\Vector{x}\mid \theta)h(\theta). \]
      \item We also know that
            \[ \int_{\Theta}h(\theta\mid \Vector{x})\odif{\theta}=1. \]
\end{itemize}
\subsection*{Example 1}
\begin{itemize}
      \item Suppose we have $ x_1,x_2,\ldots,x_n $ each from a $ \BERN{\theta} $.
      \item Suppose that our prior is a $ \BetaDist{\alpha,\beta} $, so that
            \[ h(\theta)=\frac{\Gamma(\alpha,\beta)}{\Gamma(\alpha)\Gamma(\beta)}\theta^{\alpha-1}(1-\theta)^{\beta-1},\; 0<\theta<1. \]
      \item What is the posterior distribution?
            \begin{framed}
                  First,
                  \[ p(\Vector{x}\mid \theta)=\prod_{i=1}^n \theta^{x_i}(1-\theta)^{1-x_i}=\theta^{\sum_{i=1}^{n}x_i}(1-\theta)^{n-\sum_{i=1}^{n}x_i}. \]
                  Therefore,
                  \begin{align*}
                        h(\theta\mid \Vector{x})
                         & \propto p(\Vector{x}\mid \theta)h(\theta)                                                                                                                \\
                         & =\theta^{\sum_{i=1}^{n}x_i}(1-\theta)^{n-\sum_{i=1}^{n}x_i}\frac{\Gamma(\alpha,\beta)}{\Gamma(\alpha)\Gamma(\beta)}\theta^{\alpha-1}(1-\theta)^{\beta-1} \\
                         & \propto \theta^{\sum_{i=1}^{n}x_i}(1-\theta)^{n-\sum_{i=1}^{n}x_i}\theta^{\alpha-1}(1-\theta)^{\beta-1}                                                  \\
                         & = \theta^{\sum_{i=1}^{n}x_i+\alpha-1}(1-\theta)^{n-\sum_{i=1}^{n}x_i+\beta-1},
                  \end{align*}
                  which is the functional form of $ \BetaDist{\sum_{i=1}^{n}x_i+\alpha,n-\sum_{i=1}^{n}x_i+\beta} $.
            \end{framed}
      \item What is the posterior mean?
            \begin{framed}
                  From our distribution above,
                  \begin{align*}
                        \E{\theta\mid \Vector{x}}=\frac{\sum_{i=1}^{n}x_i+\alpha}{\alpha+n+\beta}
                         & =\frac{\sum_{i=1}^{n}x_i+\alpha}{\alpha+n+\beta}                                                                                                              \\
                         & =\biggl(\frac{n}{\alpha+\beta+n}\biggr)\frac{1}{n}\sum_{i=1}^{n}x_i+\biggl(\frac{\alpha+\beta}{\alpha+\beta+n}\biggr)\frac{\alpha}{\alpha+\beta}              \\
                         & =\biggl(1-\frac{\alpha+\beta}{\alpha+\beta+n}\biggr)\frac{1}{n}\sum_{i=1}^{n}x_i+\biggl(\frac{\alpha+\beta}{\alpha+\beta+n}\biggr)\frac{\alpha}{\alpha+\beta} \\
                         & =(1-\gamma_n)\hat{\theta}+\gamma_n\E{\theta},
                  \end{align*}
                  where $ \E{\theta}=\frac{\alpha}{\alpha+\beta} $ is the prior mean, $ \hat{\theta}=\frac{1}{n}\sum_{i=1}^{n}x_i $ is the MLE
                  of the data, and $ \gamma_n=\frac{\alpha+\beta}{\alpha+\beta+n} $. We observe that as $ n\to\infty $,
                  $ \gamma_n\to 0 $, so that $ \E{\theta\mid \Vector{x}}\to \hat{\theta} $, which is obvious. If we collect an infinite amount of
                  data, we do not care about our prior knowledge.
            \end{framed}
\end{itemize}
\subsection*{Notes}
\begin{itemize}
      \item Note that in using this beta prior, we have distilled our prior beliefs to
            two numbers: $ \alpha $ and $ \beta $.
      \item Note also that some people use a $ \BetaDist{1,1} $, which is a uniform, to
            indicate `no' prior knowledge.
      \item There are problems using uniform priors in this way, which we shall see
            later on.
      \item When the prior and the posterior for $ \theta $ are from the same distribution,
            this is called \textbf{conjugacy}.
      \item The definition of a conjugate prior is important, so let's be careful.
      \item A \textbf{conjugate prior} $ h(\theta) $ is a distribution such that the posterior
            $ h(\theta\mid \Vector{x}) $ is of the same distribution type.
      \item If $ X \sim \BetaDist{\alpha,\beta} $, then
            \[ \E{X}=\frac{\alpha}{\alpha+\beta},\quad \Var{X}=\frac{\alpha\beta}{(\alpha+\beta)^2(\alpha+\beta+1)}. \]
\end{itemize}
\subsection*{Example 2}
\begin{itemize}
      \item Show that the conjugate prior distribution for the $ \POI{\lambda} $ is the Gamma distribution.
            \begin{framed}
                  First,
                  \[ p(\Vector{x}\mid \lambda)=\prod_{i=1}^n p(x_i\mid \lambda)=\lambda^{x_i}\frac{e^{-\lambda}}{x_i!}
                        =\lambda^{\sum_{i=1}^{n}x_i}e^{-n\lambda}\prod_{i=1}^n \frac{1}{x_i}\propto \lambda^{\sum_{i=1}^{n}x_i}e^{-n\lambda}. \]
                  Assuming that $ \lambda \sim \GAM{\alpha,\beta} $, we have
                  \[ h(\lambda)=\frac{\beta^\alpha}{\Gamma(\alpha)}\lambda^{\alpha-1}e^{-\beta\lambda}\propto \lambda^{\alpha-1}e^{-\beta\lambda}. \]
                  Therefore,
                  \begin{align*}
                        h(\lambda\mid \Vector{x})
                         & \propto p(\Vector{x}\mid \lambda)h(\lambda)                                         \\
                         & \propto \lambda^{\sum_{i=1}^{n}x_i}e^{-n\lambda}\lambda^{\alpha-1}e^{-\beta\lambda} \\
                         & =\lambda^{\alpha+\sum_{i=1}^{n}x_i-1}e^{-(\beta+n)\lambda},
                  \end{align*}
                  which is the functional form of $ \GAM{\alpha+\sum_{i=1}^{n}x_i,\beta+n} $.
            \end{framed}
\end{itemize}
\subsection*{Example 3}
\begin{itemize}
      \item We want to assess the probability $ \theta $ that a quarter, when tossed, will
            land heads-up.
      \item Your prior belief is that $ \theta $ is probably $0.5$ but may be anywhere from $0.3$
            to $0.7$.
      \item Construct a beta distribution for $ \theta $ that reflects your prior beliefs.
      \item The coin is tossed $10$ times and lands heads-up $7$ times.
      \item What is the posterior mean and variance?
\end{itemize}
\begin{framed}
      Since prior belief is that $ \theta $ is probably $0.5$, but may be anywhere from $0.3$
      to $0.7$, we have
      \[ \E{\theta}\pm 2\sqrt{\Var{\theta}}=(0.3,0.7), \]
      where $ \E{\theta}=0.5 $. Hence,
      \[ 0.5\pm 2\sqrt{\Var{\theta}}=(0.3,0.7)\implies \Var{\theta}=0.01. \]
      Hence,
      \[ \frac{\alpha}{\alpha+\beta}=0.5\implies \alpha=\beta. \]
      \[ \frac{\alpha\beta}{(\alpha+\beta)^2(\alpha+\beta+1)}=0.01\implies \alpha=12.  \]
      Therefore, $ h(\theta) $, our prior distribution, is
      \[ h(\theta)\propto \theta^{11}(1-\theta)^{11}, \]
      which is $ \BetaDist{12,12} $.

      From earlier, we see that $ h(\Vector{x}\mid \theta)\propto \BetaDist{7+12,10-7+12}=\BetaDist{19,15} $.
\end{framed}
\makeheading{Lecture 17}{\printdate{2022-11-30}}%chktex 8
\begin{Definition}{}{}
    We say a sequence of events $ (A_n)_{n\ge 1} $
    happens \textbf{infinitely often}
    on an outcome $ \omega $ if
    for all $ N $, there exists $ n>N $ such that
    $ \omega\in A_n $ where
    \[ \Set[\big]{(A_n)_{n\ge 1}\text{ i.o}}=\bigcap_{N\ge 1}\bigcup_{n\ge N}. \]
\end{Definition}
\begin{Theorem}{Borel-Cantelli Lemma}{borel}
    If $ \sum_{n=1}^{\infty}\Prob{A_n}<\infty $, then
    \[ \ProbB{(A_n)_{n\ge 1}\text{ i.o}}=0. \]
    \tcblower{}
    \textbf{Proof}:
    Let $ Y=\sum_{n=1}^{\infty}\Ind{A_n} $, so $ Y\in\mathbf{N}\cup \Set{\infty} $.
    \begin{align*}
        \E{Y}
         & =\E*{\sum_{n=1}^{\infty}\Ind{A_n}}               \\
         & =\sum_{n=1}^{\infty}\E{\Ind{A_n}}                \\
         & =\sum_{n=1}^{\infty}(1 \Prob{A_n}+0\Prob{A_n^c}) \\
         & =\sum_{n=1}^{\infty}\Prob{A_n}                   \\
         & <\infty.
    \end{align*}
    Thus, $ \ProbB{Y=\infty}=0 $. Alternatively,
    \[ \ProbB{Y>n}\le \frac{\E{Y}}{n}\xrightarrow[]{n\to\infty}0, \]
    so $ \ProbB{Y=\infty}=0 $.
\end{Theorem}
\begin{Corollary}{}{}
    If the events $ A_n $ are independent, then the converse of~\Cref{thm:borel} is also true.
    \tcblower{}
    \textbf{Proof}: Suppose the $ (A_n)_{n\ge 1} $ are independent
    and that $ \sum_{n=1}^{\infty}\Prob{A_n}=\infty $ and we will show
    $ \ProbB{(A_n)\text{ i.o.}}=1 $.
    For all $ N\in\mathbf{N} $,
    \begin{align*}
        \Prob*{\bigcup_{n\ge N}A_n}
         & =1-\Prob*{\bigcup_{n\ge N}^{\infty}A_n^c} \\
         & =1-\prod_{n\ge N}(1-\Prob{A_n})           \\
         & \ge 1-\prod_{n\ge N}e^{-\Prob{A_n}}       \\
         & =1-e^{-\sum_{n\ge N}\Prob{A_n}}           \\
         & =1-e^{-\infty}                            \\
         & =1.
    \end{align*}
\end{Corollary}
\begin{Example}{Convergence in Probability $\not\implies$ Almost Sure Convergence}{}
    For all $ n\ge 1 $, $ X_n\iid \BERN{1/n} $.
    Then,
    \[ \ProbB{\abs{X_n-0}>\varepsilon}=\frac{1}{n}\to 0. \]
    But,
    \[ \sum_{n=1}^{\infty}\ProbB{X_n=1}=\sum_{n=1}^{\infty}\frac{1}{n}=\infty. \]
    Let $ Y_n=n X_n $, so
    \[ Y_n=\begin{cases}
            n, & \text{w.p. }\frac{1}{n},   \\
            0, & \text{w.p. }1-\frac{1}{n}.
        \end{cases} \]
    $ \E{Y_n}=1 $ for every $ n $,
    \[ Y_n\inp 0, \]
    but $ \E{Y_n}\to 1 $, so $ Y_n\not\as 0 $.
\end{Example}
\begin{Lemma}{Kronecker's Lemma}{}
    For a sequence $ (X_n)_{n\ge 1}\in (0,\infty)^{\mathbf{N}} $,
    if
    $ \sum_{n=1}^{\infty}\frac{X_n}{n}<\infty $,
    then
    $ \lim\limits_{{N} \to {\infty}}\frac{1}{N}\sum_{n=1}^{N}X_n=0 $.
    \tcblower{}
    \textbf{Proof}: Suppose
    $ S=\sum_{n=1}^{\infty}\frac{X_n}{n}<\infty $, then
    \[ \sum_{n=1}^{N}\frac{X_n}{n}-\sum_{n=1}^{N}\frac{X_n}{N}
        =\sum_{n=1}^{N}\frac{X_n}{n}\biggl(1-\frac{n}{N}\biggr)\xrightarrow[]{n\to\infty}S. \]

    Fix $ \varepsilon>0 $. Let $ N_1 $ be sufficiently large such that
    \[ \sum_{n=N_1}^{\infty}\frac{X_n}{n}<\frac{\varepsilon}{2}, \]
    and let $ N_2>N_1 $ be sufficiently large so that
    \[ \frac{N_1}{N_2}<\frac{\varepsilon}{2S}, \]
    that is, $ N_2=\lceil \frac{2N_1S}{\varepsilon}\rceil $.
    Then,
    \begin{align*}
        \sum_{n=1}^{N_2}\frac{X_n}{n}\biggl(1-\frac{n}{N_2}\biggr)
         & \ge \sum_{n=1}^{N_1}\frac{X_n}{n}\biggl(1-\frac{N_1}{N_2}\biggr)                \\
         & \ge \biggl(1-\frac{\varepsilon}{2S}\biggr)\sum_{n=1}^{N_1}\frac{X_n}{n}         \\
         & \ge \biggl(1-\frac{\varepsilon}{2S}\biggr)\biggl(S-\frac{\varepsilon}{2}\biggr) \\
         & =S-S \frac{\varepsilon}{2S}-\frac{\varepsilon}{2}+\frac{\varepsilon^2}{4S}      \\
         & \ge S-\frac{\varepsilon}{2}-\frac{\varepsilon}{2}                               \\
         & =S-\varepsilon.
    \end{align*}
    We conclude that
    \[ \sum_{n=1}^{N}\frac{X_n}{n}-\sum_{n=1}^{N}\frac{X_n}{N}\xrightarrow[]{n\to\infty}S. \]
    Therefore,
    \[ \sum_{n=1}^{N}\frac{X_n}{N}\xrightarrow[]{N\to\infty} S-S=0. \]
\end{Lemma}
\begin{Theorem}{Strong Law of Large Numbers}{}
    If $ X_1,X_2,\ldots $ is a sequence of IID random variables with
    $ \E{\abs{X_i}}<\infty $, then
    \[ \frac{1}{n}\sum_{j=1}^{n}X_j\as \E{X_1} \]
    as $ n\to\infty $.
    \tcblower{}
    \textbf{First Step of Proof}:
    Let $ Y_n=X_n\Ind{\abs{X_n}\le n} $.
\end{Theorem}
\makeheading{Lecture 18}{\printdate{2022-12-02}}%chktex 8
\begin{Theorem}{Kolmogorov's Inequality}{}
    Suppose $ X_1,\ldots,X_n $ are independent random variables
    with finite expectation. For $ 1\le j\le n $,
    let $ S_j=X_1+\cdots+X_j $. Then for any $ \varepsilon>0 $,
    \[ \Prob*{\operatorname*{max}_{1\le j\le n}\abs[\big]{S_j-\E{S_j}}\ge \varepsilon}\le
        \frac{\Var{S_n}}{\varepsilon^2}. \]
    \tcblower{}
    \textbf{Proof}: Assume WLOG, $ \E{X_j}=0 $ for $ j=1,\ldots,n $,
    so $ \E{S_j}=0 $ as well. Let
    \[ A_j=\begin{cases}
            \abs{S_j}<\varepsilon,    & 1\le j<k,         \\
            \abs{S_k}\ge \varepsilon, & \text{otherwise}.
        \end{cases} \]
    \[ A=\bigcup_{k=1}^n A_k=\Set{\operatorname*{max}_{1\le j\le n}\abs{S_j}\ge \varepsilon}. \]
    Let $ \Ind{A}=1 $ if $ A $ happens, and $\Ind{A}=0$ otherwise.
    Now,
    \[ \Var{S_n}=\E{S_n^2}\ge \E{S_n^2\Ind{A}}=\E*{S_n^2\biggl(\sum_{k=1}^{n}\Ind{A_k}\biggr)}=\sum_{k=1}^{n}\E{S_n^2\Ind{A_k}}. \]
    For $ 1\le k\le n $, define $ Y_k=X_{k+1}+X_{k+2}+\cdots+X_n $ so that
    \[ S_n=S_k+Y_k. \]
    \begin{align*}
        \E{S_n^2\Ind{A_k}}
         & =\E*{(S_k+Y_k)^2\Ind{A_k}}                                                       \\
         & =\E{S_k^2\Ind{A_k}}+2\E{S_k Y_k\Ind{A_k}}+\E{Y_k^2\Ind{A_k}}                     \\
         & =\E{S_k^2\Ind{A_k}}+2\E{S_k\Ind{A_k}}\underbrace{\E{Y_k}}_{0}+\E{Y_k^2\Ind{A_k}} \\
         & =\E{S_k^2\Ind{A_k}}+\underbrace{\E{Y_k^2\Ind{A_k}}}_{\ge 0}                      \\
         & \ge \E{S_k^2\Ind{A_k}}                                                           \\
         & \ge \E{\varepsilon^2\Ind{A_k}}                                                   \\
         & =\varepsilon^2\Prob{A_k}.
    \end{align*}
    Plugging this back in,
    \[ \Var{S_n}\ge \sum_{k=1}^{n}\E{S_n^2\Ind{A_k}}
        \ge \sum_{k=1}^{n}\varepsilon^2\Prob{A_k}
        =\varepsilon^2\Prob{A_k}. \]
    Thus,
    \[ \Prob{A}\le \frac{\Var{S_n}}{\varepsilon^2}. \]
\end{Theorem}
\begin{Theorem}{Kolmogorov's Criterion}{}
    Suppose $ X_1,X_2,\ldots $ are independent
    random variables with
    \[ \sum_{k=1}^{\infty}\frac{\Var{X_n}}{k^2}<\infty. \]
    Then,
    \[ \frac{S_n-\E{S_n}}{n}\as 0\text{ as $ n\to\infty $}, \]
    where $ S_n=\sum_{k=1}^{n}X_k $ for $ n\ge 1 $.
    \tcblower{}
    \textbf{Proof}: Assume WLOG that
    $ \E{S_k}=0 $ for $ k\ge 1 $. Fix $ \varepsilon>0 $.
    Let
    \[ A_k=\frac{\abs{S_n}}{n}\ge \varepsilon,\; \text{ for some } n\in\interval[open left]{2^{k-1}}{2^k}. \]
    We want to show
    \[ \ProbB{(A_k)_{k\ge 1}\text{ i.o.}}=0. \]
    Using the Borel-Cantelli lemma, we want to show
    $ \sum_{k=1}^{\infty}\Prob{A_k}<\infty $.
    \begin{align*}
        \Prob{A_k}
         & \le\ProbB{\abs{S_n}\ge 2^{k-1}\varepsilon}            &  & \text{for some }n\le 2^k          \\
         & \le \frac{\Var{S_{2^k}}}{(2^{k-1}\varepsilon)^2}      &  & \text{by Kolmogorov's Inequality} \\
         & =\frac{4}{\varepsilon^2}\frac{\Var{S_{2^k}}}{2^{2k}}.
    \end{align*}
    Therefore,
    \begin{align*}
        \sum_{k=1}^{\infty}\Prob{A_k}
         & \le \frac{4}{\varepsilon^2}\sum_{k=1}^{\infty}\frac{\Var{S_{2^k}}}{2^{2k}}                                     \\
         & =\frac{4}{\varepsilon^2}\sum_{k=1}^{\infty}2^{-2k}\sum_{j=1}^{2^k}\Var{X_j}                                    \\
         & =\frac{4}{\varepsilon^2}\sum_{\substack{1\le k<\infty                                                          \\1\le j\le 2^k}}2^{-2k}\Var{X_j}                         \\
         & =\frac{4}{\varepsilon^2}\sum_{j=1}^{\infty}\Var{X_j}\sum_{k=\lceil\mathrm{log}_2(j)\rceil }^{\infty}(2^{-2})^k \\
         & =\frac{4}{\varepsilon^2}\sum_{j=1}^{\infty}\Var{X_j}\frac{(2^{-2})^{\lceil \mathrm{log}_2(j)\rceil}}{1-2^{-2}} \\
         & \le \frac{4}{\varepsilon^2}\frac{4}{3}\sum_{j=1}^{\infty}\Var{X_k}(2^{-2})^{\mathrm{log}_2(j)}                 \\
         & =\frac{16}{3\varepsilon^2}\sum_{j=1}^{\infty}\Var{X_j}j^{-2}                                                   \\
         & <\infty
    \end{align*}
    by our hypothesis. It's worth noting that to change the sums we have
    $ j\le 2^k $, $ 2^k\ge j $, $ k\ge \mathrm{log}_2(j) $ so $ k\ge \lceil \mathrm{log}_2(j)\rceil $.
    Therefore,
    by Borel-Cantelli lemma,
    \[ \ProbB{(A_k)_{k\ge 1}\text{ i.o.}}=0. \]
    Since this holds for every $ \varepsilon>0 $,
    \[ \ProbB*{\lim\limits_{{n} \to {\infty}}\frac{\abs{S_n}}{n}=0}=1. \]
\end{Theorem}
\begin{Theorem}{Strong Law of Large Numbers (IID)}{}
    If $ X_1,X_2,\ldots $ are IID variables with finite expectation
    and $ S_n=\sum_{j=1}^{n}X_j $ for $ n\ge 1 $, then
    \[ \frac{S_n}{n}\as \E{X_1}\text{ as }n\to\infty. \]
    \tcblower{}
    \textbf{Proof}: For $ n\ge 1 $, let
    $ Y_n=X_n\Ind{\abs{X_n}\le n} $.
    \begin{align*}
        \sum_{n=1}^{\infty}\ProbB{\abs{X_n}>n}
         & =\sum_{n=1}^{\infty}\sum_{k=n}^{\infty}\ProbB{k<\abs{X_1}\le k+1} \\
         & =\sum_{k=1}^{\infty}\sum_{n=1}^{k}\ProbB{k<\abs{X_1}\le k+1}      \\
         & =\sum_{k=1}^{\infty}k\ProbB{k\le \abs{X_1}\le k+1}                \\
         & \le \E[\big]{\abs{X_1}}                                           \\
         & <\infty.
    \end{align*}
    Thus, by Borel-Cantelli,
    \[ \ProbB{X_n\ne Y_n\text{ i.o.}}=0. \]
    Hence, it suffices to prove
    \[ \frac{S_n'}{n}\as \E{X_1}\text{ as }n\to\infty, \]
    where $ S_n'=\sum_{j=1}^{n}Y_j $. We can also assume WLOG
    $ \E{X_1}=0 $.
    \[ \E{Y_n}=\E{X_1\Ind{\abs{X_1}\le n}}\to \E{X_1}\text{ as }n\to\infty \]
    (Application of the Dominated Convergence Theorem). Therefore,
    \[ \frac{1}{n}\sum_{j=1}^{n}\E{Y_j}\to 0\text{ as }n\to\infty. \]
    It would suffice to prove
    \[ \frac{1}{n}\sum_{j=1}^{n}(\underbrace{Y_j-\E{Y_j}}_{Z_j})\as 0\text{ as }n\to\infty. \]
    Note that $ \E{Z_j}=0\implies \Var{Z_j}=\Var{Y_j} $.
    By Kolmogorov's Criterion, it would be sufficient to show
    \[ \sum_{k=1}^{\infty}\frac{\Var{Z_j}}{j^2}<\infty. \]
    \begin{align*}
        \sum_{k=1}^{\infty}\frac{\Var{Y_k}}{k^2}
         & \le \sum_{k=1}^{\infty}\frac{\E{Y_k^2}}{k^2}                                                                   \\
         & =\sum_{k=1}^{\infty}\frac{\E{X_k^2\Ind{X_k}<k}}{k^2}                                                           \\
         & =\sum_{k=1}^{\infty}\frac{1}{k^2}\sum_{j=1}^{k}\E{X_1^2\Ind{j-1<\abs{X_1}<j}}                                  \\
         & =\sum_{j=1}^{\infty}\E{X_j^2\Ind{j-1<\abs{X_j}\le j}}\sum_{k=j}^{\infty}\frac{1}{k^2}                          \\
         & \le \sum_{j=1}^{\infty}\E{X_j^2\Ind{j-1<\abs{X_j}\le j}} \frac{C}{j}                  &  & \text{for some }C>0 \\
         & \le \sum_{j=1}^{\infty}\E{j\abs{X_1}\Ind{j-1<\abs{X_1}\le j}}\frac{C}{j}                                       \\
         & =C\sum_{j=1}^{\infty}\E{\abs{X_1}\Ind{j-1<\abs{X_1}\le j}}                                                     \\
         & =C\E*{\abs{X_1}\sum_{j=1}^{\infty}\Ind{j-1<\abs{X_1}\le j}}                                                    \\
         & =C\E{\abs{X_1}}                                                                                                \\
         & <\infty.
    \end{align*}
\end{Theorem}
\makeheading{Lecture 27}{\printdate{2022-07-11}}%chktex 8
\section{Continued Fraction}
\subsection*{Finite Continued Fractions}
In particular, we begin by look at how we can use the Euclidean algorithm
to write rational numbers in an alternate way.
\begin{Example}{}{}
    Using the Euclidean Algorithm to find $ (87,32) $
    gives
    \begin{align*}
        87 & =2\cdot 32+23 \\
        32 & =1\cdot 23+9  \\
        23 & =2\cdot 9+5   \\
        9  & =1\cdot 5+4   \\
        5  & =1\cdot 4+1.
    \end{align*}
    We can rewrite each of these equations by dividing through
    the divisor as:
    \begin{align*}
        \frac{87}{23} & =2+\frac{23}{32} \\
        \frac{32}{23} & =1+\frac{9}{23}  \\
        \frac{23}{9}  & =2+\frac{5}{9}   \\
        \frac{9}{5}   & =1+\frac{4}{5}   \\
        \frac{5}{4}   & =1+\frac{1}{4}.
    \end{align*}
    Observe that the last number in each line is the
    reciprocal of the first number in the line below it.
    This allows us to write $ \frac{87}{32} $ as:
    \begin{align*}
        \frac{87}{23}
         & =2+\frac{1}{\frac{32}{23}}                                     \\
         & =2+\frac{1}{1+\frac{9}{23}}                                    \\
         & =2+\frac{1}{1+\frac{1}{\frac{23}{9}}}                          \\
         & =2+\frac{1}{1+\frac{1}{2+\frac{5}{9}}}                         \\
         & =2+\frac{1}{1+\frac{1}{2+\frac{1}{\frac{9}{5}}}}               \\
         & =2+\frac{1}{1+\frac{1}{2+\frac{1}{1+\frac{4}{5}}}}             \\
         & =2+\frac{1}{1+\frac{1}{2+\frac{1}{1+\frac{1}{\frac{5}{4}}}}}   \\
         & =2+\frac{1}{1+\frac{1}{2+\frac{1}{1+\frac{1}{1+\frac{1}{4}}}}} \\
    \end{align*}
\end{Example}
\begin{Definition}{}{}
    An expression of the form
    \[ a_0+\frac{1}{a_1+\frac{1}{a_2+\frac{1}{a_3+\cdots}}}, \]
    where $ a_0\in\mathbf{Z} $, and $ a_i\in\mathbf{Z}^+ $ for all $ i\ge 1 $
    is called a \textbf{continued fraction}.
    The number $ a_i $'s are called the partial quotient of the continued fraction.
\end{Definition}
\begin{Remark}{}{}
    In some cases, people have considered continued fractions where the
    numerators don't have to be 1. For example,
    \[ \frac{4}{\pi}=1+\frac{1^2}{2+\frac{3^2}{2+\frac{5^2}{2+\cdots}}}. \]
    In this case, we refer to it as a \textbf{simple continued fraction}.
\end{Remark}
Throughout this section, we will only
be considering continued fractions with numerator $ 1 $
and use the notation $ \conf{a_0;a_1,a_2,\ldots} $
for continued fractions
introduced by Dirichlet in 1854.

The useful formulas are:
\[ \conf{a_0;a_1,a_2,\ldots,a_n}=a_0+\frac{1}{\conf{a_1;a_2,a_3,\ldots,a_n}}
    =\conf*{a_0;a_1,a_2,\ldots,a_{n-1}+\frac{1}{a_n}}. \]
\begin{Example}{}{}
    Find a continued fraction expansion for $ x=\frac{47}{17} $.
    \tcblower{}
    \textbf{Solution}:
    We have:
    \begin{align*}
        \frac{47}{17}
         & =2+\frac{13}{17}                        \\
         & =2+\frac{1}{\frac{17}{13}}              \\
         & =2+\frac{1}{1+\frac{4}{13}}             \\
         & =2+\frac{1}{1+\frac{1}{\frac{13}{4}}}   \\
         & =2+\frac{1}{1+\frac{1}{3+\frac{1}{4}}}.
    \end{align*}
    Thus,
    \[ \frac{47}{17}=\conf{2;1,3,4}. \]
\end{Example}
\underline{Remark}: The continued fraction
of a rational number is not unique.
In particular, for Example 2, we could have written
\[ \frac{47}{17}=2+\frac{1}{1+\frac{1}{3+\frac{1}{4}}}=2+\frac{1}{1+\frac{1}{3+\frac{1}{3+\frac{1}{1}}}}. \]
And in general,
\[ \conf{a_0;a_1,a_2,\ldots,a_{n-1},a_n}=
    \conf{a_0;a_1,a_2,\ldots,a_n-1,1}. \]
\begin{Exercise}{}{}
    Calculate a continued fraction expansion
    for $ x=\frac{354}{49} $.
    \tcblower{}
    \textbf{Answer}: $ \conf{7;4,2,5} $ or $ \conf{7;4,2,4,1} $.
\end{Exercise}
\begin{Exercise}{}{}
    Calculate a continued fraction expansion
    for $ x=\frac{72}{19} $.
    \tcblower{}
    \textbf{Answer}: $ \conf{3;1,3,1,3} $.
\end{Exercise}
We can also evaluate a given continued fraction.
\begin{Example}{}{}
    Find the rational number $ x=[-2;5,2,4] $.
    \tcblower{}
    \textbf{Answer}: $ -\frac{89}{49} $.
\end{Example}
\begin{Exercise}{}{}
    Find the rational number $ x=[0;1,8,2] $.
    \tcblower{}
    \textbf{Answer}: $ \frac{17}{19} $.
\end{Exercise}
\begin{Proposition}{}{}
    Every finute continued fraction with integer terms represent a rational
    number and vice versa; that is,
    every rational number can be expressed as a simple continued fraction.
\end{Proposition}
\subsection*{Infinite Continued Fractions}
Our goal is to approximate
irrational numbers. So, we now look at how to find
a simple continued fraction expansion
of an irrational number. Continued fractions can be traced back to Euclid's time.
However, until the 1500s, they were just used to solve linear equations.
In the 1500s, Cataldi and Bombelli used continued fractions to approximate square roots.

\begin{Example}{}{}
    Calculate a continued fraction expansion for $ \sqrt{2} $.
    \tcblower{}
    \textbf{Solution 1}: First, we know the integer part of $ \sqrt{2} $
    is $ 1 $, so we can write
    \[ \sqrt{2}=1+x. \]
    Rather than trying to work with decimals, we can simply take
    $ x=\sqrt{2}-1 $ and the difference of squares formula tells us that
    \begin{align*}
        (\sqrt{2}-1)(\sqrt{2}+1) & =2-1=1 \\
        \sqrt{2}-1=\frac{1}{1+\sqrt{2}}.
    \end{align*}
    Thus, we have
    \[ \sqrt{2}=1+\frac{1}{1+\sqrt{2}}. \]
    We can simply substitute what we have in for $ \sqrt{2} $ to get
    \begin{align*}
        \sqrt{2}
         & =1+\frac{1}{1+1+\frac{1}{1+\sqrt{2}}}            \\
         & =1+\frac{1}{2+\frac{1}{1+\sqrt{2}}}              \\
         & =1+\frac{1}{2+\frac{1}{2+\frac{1}{1+\sqrt{2}}}}.
    \end{align*}
    Thus, $ \sqrt{2}=\conf{1,\overline{2}} $.

    \textbf{Solution 2}: We can also verify that this answer by a similar idea used in Example 3.
\end{Example}
\begin{Exercise}{}{}
    Calculate the continued fraction expansion for $ \sqrt{3} $
    and $ \sqrt{6} $ and verify your answer.
    \tcblower{}
    \textbf{Answer}: $ \sqrt{3}=\conf{1;1,2,1,2,\ldots} $
    and $ \sqrt{6}=\conf{2;2,4,2,4,\ldots} $.
\end{Exercise}
\makeheading{Lecture 28}{\printdate{2022-07-13}}%chktex 8
\end{document}