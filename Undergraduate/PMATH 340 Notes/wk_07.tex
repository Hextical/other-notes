\makeheading{Lecture 18 (Part I)}{\printdate{2022-06-13}}%chktex 8
\section{Costas Array}
Here is a challenge. Given an $ n\times n $ array, put dots into the centre of
boxes such that
\begin{enumerate}[(1)]
    \item Every row has exactly one dot.
    \item Every column has exactly one dot.
    \item If you draw all $\frac{n(n-1)}{2}$ lines segments, then any two lines that have the same slope, must have different length.
\end{enumerate}
A grid satisfying all three conditions is called a \textbf{Costas array}.

The third condition is equivalent to: when a Costas array and a replica of itself
are overlaid with an offset of an integer number of row and columns shifts such that
$1$ overlays another $1$, then that will be the only 1s that overlay.
In Costas array, we represent each entry either by the $1$ for the dot or by $0$ for
the absence of dot.

\begin{Example}{}{}
    A Costas array for $ n=5 $ is:
    \[ \begin{array}{|c|c|c|c|c|}%chktex 44
            \hline%chktex 44
            1 & 0 & 0 & 0 & 0 \\
            \hline%chktex 44
            0 & 0 & 0 & 1 & 0 \\
            \hline%chktex 44
            0 & 1 & 0 & 0 & 0 \\
            \hline%chktex 44
            0 & 0 & 1 & 0 & 0 \\
            \hline%chktex 44
            0 & 0 & 0 & 0 & 1 \\
            \hline%chktex 44
        \end{array} \]
\end{Example}
\begin{Exercise}{}{}
    Draw a Costas array for $ n=3,4,6 $.
\end{Exercise}
John Costas and Edgar Gilbert independently introduced Costas arrays in 1965.
To get a Costas array for $n = p- 1$, where $p$ is prime, we will use the following
algorithm. Gilbert at that time had discovered the Welch algorithm which was
rediscovered by Lloyd Welch in 1982.

\textbf{Welch Algorithm}: Let $a$ be a primitive root of $p$. Define the array $ A_{i,j} $ by
\[ A_{i,j}=\begin{cases}
        1 & a^i\equiv j\Mod{p} \\
        0 & \text{otherwise}.
    \end{cases} \]
\begin{Example}{}{}
    Draw a Costas array for $ n=4 $.
    \tcblower{}
    \textbf{Solution}: We have $ p=5 $ and a primitive root of $ 5 $ is $ a=3 $. So, we have
    \begin{align*}
        3^1 & \equiv 3\Mod{5}\implies A_{1,3}=1  \\
        3^2 & \equiv 4\Mod{5}\implies A_{2,4}=1  \\
        3^3 & \equiv 2\Mod{5}\implies A_{3,2}=1  \\
        3^4 & \equiv 1\Mod{5}\implies A_{4,1}=1.
    \end{align*}
    Therefore, the Costas array for $ n=4 $ is:
    \[ \begin{array}{|c|c|c|c|}%chktex 44
            \hline%chktex 44
            0 & 0 & 1 & 0 \\
            \hline%chktex 44
            0 & 0 & 0 & 1 \\
            \hline%chktex 44
            0 & 1 & 0 & 0 \\
            \hline%chktex 44
            1 & 0 & 0 & 0 \\
            \hline%chktex 44
        \end{array} \]
\end{Example}
\begin{Exercise}{}{}
    Draw a Costas array for $ n=10 $.
\end{Exercise}

\section{Indices}
The facts that every prime $ p $ has a primitive root combined with the fact that for any primitive root $ a $ of $ p $, we have that
\[ a,a^2,\ldots,a^{p-1}=1 \]
gives every number $ 1,2,\ldots,p-1 $ exactly once turns out to be rather useful.

Consider the powers of the primitive root $ 2 $ modulo $ 11 $:
\[ \begin{array}{cccccccccc}
        2^1 & 2^2 & 2^3 & 2^4 & 2^5 & 2^6 & 2^7 & 2^8 & 2^9 & 2^{10} \\
        \midrule
        2   & 4   & 8   & 5   & 10  & 9   & 7   & 3   & 6   & 1
    \end{array} \]
Now, rewrite this by ordering the second row as:
\[ \begin{array}{cccccccccc}
        2^{10} & 2^1 & 2^8 & 2^2 & 2^4 & 2^9 & 2^7 & 2^3 & 2^6 & 2^5 \\
        \midrule
        1      & 2   & 3   & 4   & 5   & 6   & 7   & 8   & 9   & 10
    \end{array} \]
We can rewrite the first rows by just indicating the powers as:
\[ \begin{array}{cccccccccc}
        10 & 1 & 8 & 2 & 4 & 9 & 7 & 3 & 6 & 5  \\
        \midrule
        1  & 2 & 3 & 4 & 5 & 6 & 7 & 8 & 9 & 10
    \end{array} \]
If we now think about exponent laws, we get that addition of numbers in the first row,
say $ 2+4=6 $, corresponds to multiplication modulo $ 11 $ in the bottom row $ 4\cdot 5=9 $.

Gauss defined `index' in 1801 to solve polynomial congruences. Jacobi published a table of indices for all primes powers less than 1000 in 1839.

\begin{Definition}{}{}
    Let $ a $ be a primitive root of $ p $, where $ p $ is prime. If $ g\equiv a^{\ell}\Mod{p} $,
    then we say that $ \ell $ is the index of $ g $ modulo $ p $ to the base $ a $, and we write it as:
    \[ \ell=\idx{g}{a}. \]
\end{Definition}
From the above example,
\[ \begin{array}{ccccccccccc}
        g          & 1  & 2 & 3 & 4 & 5 & 6 & 7 & 8 & 9 & 10 \\
        \midrule
        \idx{g}{2} & 10 & 1 & 8 & 2 & 4 & 9 & 7 & 3 & 6 & 5  \\
    \end{array} \]
\begin{Exercise}{}{}
    Write an index table of $ g $ modulo $ p=5 $ to the base $ a=3 $, and an index table of $ g $ modulo $ p $ to the base $ a=2 $.
\end{Exercise}
\begin{Lemma}{}{}
    Let $ a $ be a primitive root of $ p $, where $ p $ is prime.
    \[ a^b\equiv a^c\Mod{p}\iff b-c\equiv 0\Mod{p-1}. \]
\end{Lemma}
\begin{Exercise}{}{}
    Consider the following table of indices of $ g $ modulo $ 37 $ to the base $ 2 $:
    \[ \begin{array}{ccccccccccccccccccc}
            g          & 1  & 2  & 3  & 4  & 5  & 6  & 7  & 8  & 9  & 10 & 11 & 12 & 13 & 14 & 15 & 16 & 17 & 18 \\
            \midrule
            \idx{g}{2} & 36 & 1  & 26 & 2  & 23 & 27 & 32 & 3  & 16 & 24 & 30 & 28 & 11 & 33 & 13 & 4  & 7  & 17 \\
            \midrule\midrule
            g          & 19 & 20 & 21 & 22 & 23 & 24 & 25 & 26 & 27 & 28 & 29 & 30 & 31 & 32 & 33 & 34 & 35 & 36 \\
            \midrule
            \idx{g}{2} & 35 & 25 & 22 & 31 & 15 & 29 & 10 & 12 & 6  & 34 & 21 & 14 & 9  & 5  & 20 & 8  & 19 & 18 \\
        \end{array} \]
    Compute $ \idx{8}{2}+\idx{9}{2} $, $ \idx{8\cdot 9}{2} $, $ 2\idx{3}{2} $, and $ \idx{9}{2} $.
\end{Exercise}

\textbf{NOTE}: When calculating $ \idx{8}{2}+\idx{9}{2} $, make sure that you are reducing modulo $ p-1 $.

Thus, we can see that indices behave a lot like logarithms (we must be very
careful about the modulus though!). We have the following properties.

\begin{Proposition}{}{}
    If $ a $ is a primitive root of $ p $, where $ p $ is prime, then we have
    \begin{enumerate}[(1)]
        \item $ x\equiv y\Mod{p}\iff \idx{x}{a}\equiv \idx{y}{a}\Mod{p-1} $.
        \item $ \idx{a^r}{a}\equiv r\Mod{p-1} $.
        \item $ \idx{a}{a}=1 $.
        \item $ \idx{x\cdot y}{a}\equiv \idx{x}{a}+\idx{y}{a}\Mod{p-1} $.
        \item $ \idx{x^k}{a}\equiv k\idx{x}{a}\Mod{p-1} $.
    \end{enumerate}
\end{Proposition}

\begin{Example}{}{}
    Use the table of indices of $ g $ modulo $ 37 $ to the base $ 2 $ to solve the following:
    \begin{enumerate}[(1)]
        \item $ x\equiv 3\cdot 5\Mod{37} $.
        \item $ x\equiv 33\cdot 29\Mod{37} $.
        \item $ x\equiv 17^{12}\Mod{37} $.
        \item $ 19x\equiv 23\Mod{37} $.
    \end{enumerate}
    \tcblower{}
    \textbf{Solution}:
    \begin{enumerate}[(1)]
        \item $ \idx{3}{2}=26 $ and $ \idx{5}{2}=23 $, so we get
              \[ \idx{3\cdot 5}{2}\equiv \idx{3}{2}+\idx{5}{2}\equiv 26+23\equiv 13\Mod{36}. \]
              $ \idx{x}{2}=13 \implies x=15 $.
        \item $ \idx{17^{12}}{2}\equiv 12\idx{17}{2}\equiv 12(7)\equiv 12\Mod{36} $. Now,
              $ \idx{x}{2}=12\implies x=26 $
        \item Note that
              \begin{align*}
                  19x                    & \equiv 23\Mod{37}         \\
                  \idx{19x}{2}           & \equiv\idx{23}{2}\Mod{36} \\
                  \idx{19}{2}+\idx{x}{2} & \equiv 15\Mod{36}         \\
                  35+\idx{x}{2}          & \equiv 15\Mod{36}         \\
                  \idx{x}{2}             & \equiv -20\Mod{36}        \\
                                         & \equiv 16\Mod{36}
              \end{align*}
              $ \idx{x}{2}=16\implies x=9 $.
    \end{enumerate}
\end{Example}
\textbf{Note}: The Index is also known as discrete logarithm because the properties
of indices and logarithm are same. Originally tables of indices, just like logarithm tables,
were used to make numerical calculations much faster.

However, in recent years, indices have been revived for use in cryptography.
In particular, if we are given a large prime $p$ and two numbers $a$ and $g$ modulo $p$,
then it is very difficult to find the exponent $k$ such that
\[ a^k\equiv g\Mod{p}. \]
This is called the discrete logarithm problem. One example of such a system is
called the EIGamal cryptosystem.

\makeheading{Week 7}{\printdate{2022-06-15}}%chktex 8
Midterm.

\makeheading{Lecture 18 (Part II)}{\printdate{2022-06-17}}%chktex 8
\section{An Application to Communications Security}

