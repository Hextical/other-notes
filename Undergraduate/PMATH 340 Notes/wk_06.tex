\makeheading{Lecture 15}{\printdate{2022-06-06}}%chktex 8
\begin{Theorem}{}{}
    If $ f(x) $ is a polynomial with integer coefficients and $ f(a)\equiv 0\Mod{n} $,
    then there exists a polynomial $ g(x) $ with integer coefficients such that $ f(x)\equiv (x-a)g(x)\Mod{n} $.
    \tcblower{}
    \textbf{Proof}: Using Polynomial division, we can divide $ f(x) $ by $ (x-a) $
    to get
    \[ f(x)=(x-a)g(x)+b,\; b\in\mathbf{Z}. \]
    Substitute $ a $ to get $ f(a)=b $. Thus,
    \[ b\equiv f(a)\equiv 0\Mod{n}. \]
    Hence, $ f(x)\equiv (x-a)g(x)\Mod{n} $.
\end{Theorem}
\begin{Example}{}{}
    Factor $ f(x)=x^2+1\Mod{5} $.
    \tcblower{}
    \textbf{Solution}:
    We saw that $ x=2 $ is a root of $ f(x) $ modulo $ 5 $. Using long division, we have
    $ x^2+1=(x-2)(x+2)\Mod{5} $. Alternatively, observe that
    \[ x^2+1\equiv x^2-4\equiv (x-2)(x+2)\Mod{5}. \]
\end{Example}
\begin{Proposition}{}{}
    Let $ f(x) $ be a polynomial with integer coefficients. Let $ m $ and $ n $
    be coprime moduli.
    \[ f(x)\equiv 0\Mod{mn}\iff f(x)\equiv 0\Mod{m}\land f(x)\equiv 0\Mod{m}. \]
    \tcblower{}
    \textbf{Proof}: Similar to the proof of Proposition 3 (Lecture 12).
\end{Proposition}
If $ n=p_1^{e_1}\cdots p_k^{e_k} $ is the prime factorization of $ n $, and
$ x_1,\ldots,x_k\in\mathbf{Z} $ satisfy
\[ f(x_i)\equiv 0\Mod{p_i^{e_i}},\; i=1,\ldots,k, \]
then we can find $ x $ such that $ x\equiv x_i\Mod{p_i^{e_i}} $ for all $ i $ using
the GCRT, but then such an $ x $ would satisfy $ f(x)\equiv 0\Mod{p_i^{e_i}} $ for all $ i $,
and so $ f(x)\equiv 0\Mod{n} $. It follows that if each congruence
$ f(x)\equiv 0\Mod{p_i^{s_i}} $ has a solution $ s_i $, then $ f(x)\equiv 0\Mod{n} $
has $ s_1\cdots s_k $ solutions by the GCRT\@.

Now, we would show that a polynomial congruence $ f(x)\equiv 0\Mod{p} $
has at most $ d $ solutions, where $ d $ is the degree of $ f(x) $.

\begin{Proposition}{Lagrange's Theorem}{}
    If $ p $ is a prime and $ f(x) $ is a non-zero polynomial of degree $ d $ modulo $ p $,
    then $ f(x)\equiv 0\Mod{p} $ has at most $ d $ distinct roots modulo $ p $.
    \tcblower{}
    \textbf{Proof}: Omitted.
\end{Proposition}
\begin{Example}{}{}
    Solve the polynomial congruence:
    \[ x^{49}+2x^{33}+24\equiv 0\Mod{119}. \]
    \tcblower{}
    \textbf{Solution}: Note that $ 119=7\times 17 $. So by Proposition 1, there is a one-to-one correspondence
    between the roots of the above congruence and the roots to the
    system of congruences
    \begin{align*}
        x^{49}+2x^{33}+24 & \equiv 0\Mod{7}   \\
        x^{49}+2x^{33}+24 & \equiv 0\Mod{17}.
    \end{align*}
    Consider $ n=7 $ with $ \varphi(7)=6 $. Note that $ x\equiv 0\Mod{7} $ is not a solution. This means that $ (x,7)=1 $,
    so by Euler's Theorem
    \begin{align*}
        x^{49}+2x^{33}+24 & \equiv x^{8\cdot 6+1}+2x^{5\cdot 6+3}+24 \\
                          & \equiv x+2x^3+24                         \\
                          & \equiv 2x^3+x+24\Mod{7}.
    \end{align*}
    After evaluating the LHS at $ x=1,\ldots,6 $, we see that the only solutions are
    \[ x\equiv 2\Mod{7},\; x\equiv 6\Mod{7}. \]
    Consider $ n=17 $ with $ \varphi(17)=16 $. Note that $ x\equiv 0\Mod{17} $ is not a solution.
    This means that $ (x,17)=1 $, so by Euler's Theorem
    \begin{align*}
        x^{49}+2x^{33}+24 & \equiv x^{3\cdot 16+1}+2x^{2\cdot 16+1}+24 \\
                          & \equiv x+2x+24                             \\
                          & \equiv 3x+24\Mod{17}.
    \end{align*}
    Thus, we need to solve the congruence
    \begin{align*}
        3x+24     & \equiv 0\Mod{17}  \\
        3x        & \equiv 10\Mod{17} \\
        6\cdot 3x & \equiv 6\cdot 10  \\
        x         & \equiv 9\Mod{17}.
    \end{align*}
    By Theorem 1 (Lecture 12), this is the only solution. Since there are two solutions modulo $ 7 $ and
    only one solution modulo $ 17 $, we conclude that there are $ 2\cdot 1 $ solutions modulo $ 119 $.
    These solutions correspond to two system of equations
    \[ \begin{cases}
            x\equiv 2\Mod{7} \\
            x\equiv 9\Mod{17},
        \end{cases}\qquad \begin{cases}
            x\equiv 6\Mod{7} \\
            x\equiv 9\Mod{17}.
        \end{cases} \]
    \begin{align*}
        x\equiv 2\Mod{17}\iff x=7j+2 & \equiv 9\Mod{17}              \\
        7j                           & \equiv 7\Mod{17}              \\
        j                            & \equiv 1\Mod{17}\iff j=17k+1.
    \end{align*}
    So $ x=7(17k+1)+2=119k+9 $. Therefore, $ x\equiv 9\Mod{119} $ is a solution to the first system.
    The second system of congruences can be solved analogously and gives us a
    solution $ x\equiv 111\Mod{119} $.
\end{Example}
\begin{Exercise}{}{}
    Find all the roots of $ f(x)=x^3+3x^2+31x+23 $ modulo $ 35 $.
\end{Exercise}
\begin{Exercise}{}{}
    Find all solutions of $ f(x)=x^2+x $ modulo $ 6 $.
\end{Exercise}
\begin{Example}{}{}
    Note that $ 2x-4\equiv 0\Mod{6} $ has two roots, namely
    \[ x\equiv 2\Mod{6},\quad x\equiv 5\Mod{6}. \]
    But the degree of the polynomial is $ 1 $.
\end{Example}
\makeheading{Lecture 16}{\printdate{2022-06-08}}%chktex 8
\section{The Order of Elements in \texorpdfstring{$ \mathbf{Z}_n^* $}{Zn*}}
Let $ n $ be a modulus. We already looked at certain kinds of equations in
$ \mathbf{Z}_n $. For example, in Lecture 11, we learned that neither
$ [x]^2+[y]^2=3 $ in $ \mathbf{Z}_4 $ nor $ [x]^2+[y]^2+[z]^2=7 $
in $ \mathbf{Z}_8 $ have solutions. In Lecture 12, we studied the equation
$ [a][x]=[b] $ in $ \mathbf{Z}_n $ and
saw that the usual application of the Extended Euclidean Algorithm allows us to produce all of its solutions.

Now, we want to understand how to handle \emph{exponential} equations in $ \mathbf{Z}_n^* $.
In these kinds of equations, we are given residue classes $ [a] $ and $ [b] $
from $ \mathbf{Z}_n^* $, and we want to determine all integer solutions
$ x $ to the equation $ [a]^x=[b] $. This is essentially the same as solving
the congruence
\[ a^x\equiv b\Mod{n}. \]
The problem of finding solutions to these exponential equations is known as the
\emph{discrete logarithm problem}, or DLP\@.
\begin{Example}{}{}
    We already saw an example of an exponential equation in $ \mathbf{Z}_n^* $, namely
    \[ a^x\equiv 1\Mod{n}. \]
    According to Euler's Theorem, this equation always has a non-zero solution whenever $ a $ and $ n $ are coprime.
    In particular, any $ x\equiv 0\Mod{\varphi(n)} $ satisfies the above congruence, for if $ x\equiv \varphi(n)k $ for some integer
    $ k $, then
    \[ a^x\equiv a^{\varphi(n)k}\equiv (a^{\varphi(n)})^k\equiv 1^k\equiv 1\Mod{n}. \]
    However, we do not know whether there are no other solutions to this equation.
    Depending on the choice of $ a $, there might exist other solutions as well.
\end{Example}
In order to understand how solutions to $ a^x\equiv b\Mod{n} $ look like,
we need to understand certain fundamental properties of group units in $ \mathbf{Z}_n^* $.
\begin{Definition}{}{}
    If $ a\in\mathbf{Z}_n^* $, the \textbf{order} of $ a $ is the smallest exponent $ k\ge 1 $
    such that $ [a]^k=1 $. The order is denoted by $ k=\ord{a} $.
\end{Definition}
\begin{Example}{}{}
    The order of $ [5] $ in $ \mathbf{Z}_{13}^* $ is $ 4 $. Indeed, $ [5]^4=[1] $.
\end{Example}
From Euler's Theorem, it follows that for all $ a\in\mathbf{Z}_n^* $, it is the case that
$ \ord{a}\le \varphi(n) $. In fact, a much stronger result holds.
\begin{Proposition}{}{}
    Let $ a\in\mathbf{Z} $, $ n\ge 2 $, and $ (a,n)=1 $. A positive integer $ m $ satisfies $ a^m\equiv 1\Mod{n} $
    if and only if $ \ord{a}\mid m $.
    \tcblower{}
    \textbf{Proof}: By the Division Algorithm, we have
    \[ m=kq+r\text{ where }0\le r<k. \]
    Then, since $ a^k\equiv 1\Mod{n} $, we obtain
    \[ 1\equiv a^m\equiv a^{kq+r}\equiv (a^k)^q a^r\equiv 1^q a^r\equiv a^r\Mod{n}. \]
    Since $ k $ is the order of $ a $ congruence modulo $ n $, it must be the case $ r=0 $. Hence, $ k\mid m $.

    On the other hand, let $ m=kq $ for some $ q $. Then,
    \[ a^m\equiv a^{kq}\equiv (a^k)^q\equiv 1^q\equiv 1\Mod{n}. \]
\end{Proposition}
\begin{Corollary}{}{}
    If $ a\in\mathbf{Z} $, $ n\ge 2 $, and $ (a,n)=1 $, then $ \ord{a}\mid \varphi(n) $.
    \tcblower{}
    \textbf{Proof}: By Euler's Theorem, $ a^{\varphi(n)}\equiv 1\Mod{n} $. So by Proposition 1, we have
    $ \ord{a}\mid \varphi(n) $.
\end{Corollary}
Let $ D $ be the set of positive divisors of $ \varphi(n) $. By Corollary 1, to find order of
an $ a\in\mathbf{Z} $ modulo $ n $, we just need to find the smallest element of $ D $
such that $ a^d\equiv 1\Mod{n} $. Thus, the result greatly narrows down which powers we have to check.
\begin{Example}{}{}
    Find the order of 3 and 9 modulo $ 17 $.
    \tcblower{}
    \textbf{Solution}: For $ n=17 $, $ \varphi(17)=16 $. The complete list of positive divisors of $ 16 $
    are $ D=\Set{1,2,4,8,16} $. The smallest $ d $ satisfying $ 3^d\equiv 1\Mod{17} $ is
    the order of $ 3 $ modulo $ 17 $. Thus,
    \begin{align*}
        3^1    & \equiv 3\Mod{17}                        \\
        3^2    & \equiv 9\Mod{17}                        \\
        3^4    & \equiv 9^2\equiv 81\equiv -4\Mod{17}    \\
        3^8    & \equiv (-4)^2\equiv 16\equiv -1\Mod{17} \\
        3^{16} & \equiv (-1)^2\equiv 1\Mod{17}.
    \end{align*}
    Thus, $ \ord{3}=16 $.

    For order of $ 9 $,
    \begin{align*}
        9^1 & \equiv 9\Mod{17}                        \\
        9^2 & \equiv 81\equiv -4\Mod{17}              \\
        9^4 & \equiv (-4)^2\equiv 16\equiv -1\Mod{17}
        9^8 & \equiv (-1)^2\equiv 1\Mod{17}.
    \end{align*}
    Thus, $ \ord{9}=8 $.
\end{Example}
\begin{Exercise}{}{}
    Can we find the order of $ 4 $ in modulo $ 6 $? Find the order of $ 5 $ in modulo $ 6 $.
\end{Exercise}
Proposition 1 allows us to clarify all solutions to the exponential congruence
\[ a^x\equiv b\Mod{n}\text{ where }(a,n)=1=(b,n). \]
\begin{Proposition}{}{}
    Let $ a,b\in\mathbf{Z} $, $ n\in\mathbf{Z}^+ $, and $ (a,n)=1=(b,n) $. If $ x $
    is a solution to the congruence $ a^x\equiv b\Mod{n} $, then all solutions
    $ x^\prime $ satisfy
    \[ x\equiv x^\prime\Mod{\varphi(n)}. \]
    \tcblower{}
    \textbf{Proof}: Let $ x $ be a solution to $ a^x\equiv b\Mod{n} $, and let $ k=\ord{a} $.
    By the Division Algorithm, we have
    \[ x=kq+r\text{ where }0\le r<k. \]
    But then,
    \[ a^x\equiv a^{kq+r}\equiv (a^k)^q a^r\equiv 1^q a^r\equiv a^r\Mod{n}. \]
    Thus, WLOG we may assume $ 0\le x<k $.

    Now, suppose there exists some other $ x^\prime $ such that $ a^{x^\prime}\equiv b\Mod{n} $.
    Once again, WLOG, we may assume that $ 0\le x\le x^\prime<k $. But then,
    \[ a^x\equiv b\equiv a^{x^\prime}\Mod{n}\implies a^{x^\prime-x} \equiv 1\Mod{n}, \]
    since $ 0\le x^\prime-x<k $ and $ k $ is the order of $ a $. So, we have $ x^\prime-x=0 $.
    Therefore, all solutions $ x^\prime $ to $ a^x\equiv b\Mod{n} $ satisfy $ x^\prime\equiv x\Mod{\ord{a}} $.
\end{Proposition}
\begin{Example}{}{}
    \begin{enumerate}[(a)]
        \item Compute all the solutions to the exponential equation $ 3^x\equiv 1\Mod{17} $.
        \item Compute all the solutions to the exponential equation $ 9^x\equiv 1\Mod{17} $.
    \end{enumerate}
\end{Example}
\begin{Proposition}{}{}
    If $ a\in\mathbf{Z} $, $ n\in\mathbf{Z} $, and $ (a,n)=1 $, then all the numbers
    \[ a,a^2,a^3,\ldots,a^k=1 \]
    are distinct modulo $ n $.
    \tcblower{}
    \textbf{Proof}: Suppose that we have a repetition $ a^j\equiv a^i\Mod{n} $, where $ 1\le i<j\le k $.
    Thus, $ a^{j-i}\equiv 1\Mod{n} $. Since $ 1\le j-i\le k $, we contradict the minimality of $ k $.
\end{Proposition}
To understand further what Proposition 3 is saying, it is worth looking at some examples.
\begin{Example}{}{}
    For $ n=19 $, and $ a=2 $, we have $ \ord{2}=18 $.
    \[ \begin{array}{cccccccccccccccccc}
            2 & 2^2 & 2^3 & 2^4 & 2^5 & 2^6 & 2^7 & 2^8 & 2^9 & 2^{10} & 2^{11} & 2^{12} & 2^{13} & 2^{14} & 2^{15} & 2^{16} & 2^{17} & 2^{18} \\
            \midrule
            2 & 4   & 8   & 16  & 13  & 7   & 14  & 9   & 18  & 17     & 15     & 11     & 3      & 6      & 12     & 5      & 10     & 1
        \end{array} \]
    Find the order of $ 2^2,2^4,2^5\Mod{19} $.
\end{Example}
\makeheading{Lecture 17}{\printdate{2022-06-10}}%chktex 8
\begin{Theorem}{}{}
    \[ \bigl(a\in\mathbf{Z}\land n\in\mathbf{Z}_{\ge 1}\land (a,n)=1\bigr)
        \implies \biggl(\ord{a^m}=\frac{\ord{a}}{(\ord{a},m)}\biggr) \]
    \tcblower{}
    \textbf{Proof}: Let $ \ord{a^m}=\ell $. We will show that $ \ell=\frac{k}{(k,m)} $,
    where $ k=\ord{a} $. Note that
    \[ a^{\ell m}\equiv (a^m)^\ell\equiv 1\Mod{n}. \]
    Thus, by Proposition 1 (Lecture 16), $ k\mid \ell m $. Hence, there exists $ q\in\mathbf{Z} $ such that $ \ell m=kq $, but then
    \[ \ell \frac{m}{(k,m)}=\frac{k}{(k,m)}q. \]
    Hence, $ \frac{k}{(k,m)}\mid \ell \frac{m}{(k,m)} $, which implies
    that $ \bigl(\frac{m}{(k,m)},\frac{k}{(k,m)}\bigr)=1 $ by Proposition 2 (Lecture 3). It follows from Proposition 2 (Lecture 4)
    that
    \[ \frac{k}{(k,m)}\mid \ell. \]
    On the other hand, we have
    \[ (a^m)^{\frac{k}{(k,m)}} \equiv (a^k)^{\frac{m}{(k,m)}}\equiv 1\Mod{n}. \]
    Thus, by Proposition 1 (Lecture 16), $ \ell\mid \frac{k}{(k,m)} $. Thus, we conclude that $ \ell=\frac{k}{(k,m)} $; that is,
    \[ \ord{a^m}=\frac{\ord{a}}{(\ord{a},m)}. \]
\end{Theorem}
\begin{Corollary}{}{}
    \[ \bigl(a\in\mathbf{Z}\land n\in\mathbf{Z}_{\ge 2}\land k\in\mathbf{Z}_{\ge 1}\land (a,n)=1\bigr)
        \implies \Bigl(\ord{a^k}=\ord{a}\iff \bigl(k,\ord{a}\bigr)=1\Bigr). \]
\end{Corollary}
\begin{Proposition}{}{}
    Define $ \ord{a}=k $ and $ \ord{b}=\ell $, where
    $ a\in\mathbf{Z}\land n\in\mathbf{Z}_{\ge 2}\land (a,n)=1=(b,n)\land k,\ell\in\mathbf{Z}^+ $.
    \[ (k,\ell)=1\implies \ord{ab}=k\ell. \]
\end{Proposition}
\begin{Exercise}{}{}
    Prove Proposition 1.
\end{Exercise}
Lambert was the first to look at primitive roots. In 1769, he conjectured that
for any prime $ p $, there was a number $ g $ such that $ p\mid (g^{p-1}-1) $,
but $ p\nmid (g^e-1) $ for any $ 0<e<p-1 $.

Euler was the first to use the term `primitive root' in 1773 when he tried to
prove Lambert's claim. However, his proof was not correct. Gauss, in 1801, gave
two proofs of the existence of a primitive root for any prime $p$.
\begin{Definition}{}{}
    An element $ [a]\in\mathbf{Z}_n^* $ is called a primitive root if
    $ \ord{a}=\varphi(n) $. In terms of congruences, let $ a\in\mathbf{Z}\land n\in\mathbf{Z}_{\ge 2}\land (a,n)=1 $.
    If $ \ord{a}=\varphi(n) $, then $ a $ is called a \textbf{primitive root} of modulo $ n $.
\end{Definition}
Note that a primitive root modulo $ n $ is an element of $ [a]\in\mathbf{Z}_n^* $ whose
powers generate all $ \mathbf{Z}_n^* $; that is, every element $ [b]\in\mathbf{Z}_n^* $
can be written as $ a^x\Mod{n} $ for some positive $ x\in\mathbf{Z} $.
\begin{Example}{}{}
    If $ n=5 $, then $ \varphi(5)=4 $. We see that $ 2 $ is the primitive root modulo $ 5 $ since
    \begin{align*}
        2^1   & \equiv 2\Mod{5}  \\
        2^2   & \equiv 4\Mod{5}  \\
        2^3   & \equiv 3\Mod{5}  \\
        2^{4} & \equiv 1\Mod{5}.
    \end{align*}
    Thus, $ \ord{a}=4 $. For every integer relatively prime to $ 5 $, there is a power
    of $ 2 $ that is congruent.

    We see that $ 4 $ is \underline{not} a primitive root modulo $ 5 $ since
    \begin{align*}
        4^1 & \equiv 4\Mod{5}  \\
        4^2 & \equiv 1\Mod{5}.
    \end{align*}
    Thus, $ \ord{4}=2 $. Powers of $ 4\Mod{5} $ are only congruent to $ 1 $ or $ 4 $.
    There is no power of $ 4 $ that is congruent to $ 2 $ or $ 3 $.
\end{Example}
\begin{Theorem}{Primitive Root Theorem}{}
    If $ p $ is prime, then there exists a root modulo $ p $.
\end{Theorem}
\begin{Exercise}{}{}
    Prove Theorem 2.
\end{Exercise}
\begin{Exercise}{}{}
    Find a modulo $ n $ where no primitive roots exist.
\end{Exercise}
Let us determine how many primitive roots exists.
\begin{Proposition}{}{}
    If there is a primitive root modulo $ n $, then the total number of
    primitive roots modulo $ n $ is $ \varphi\bigl(\varphi(n)\bigr) $.
    \tcblower{}
    \textbf{Proof}: Let $ a $ be the primitive root modulo $ n $, so that $ \ord{a}=\varphi(n) $. Thus,
    \[ a,a^2,\ldots,a^{\varphi(n)}=1 \]
    are all distinct. So every other integer relatively prime to $ n $ is a power of $ a\Mod{n} $.
    The other primitive roots are those powers $ a^j $ in the list for which
    \[ \ord{a^j}=\varphi(n)=\ord{a}. \]
    According to Corollary 1, these powers $ a^j $, where $ j $ from $ 1 $ to $ \varphi(n) $ is coprime to $ \varphi(n) $,
    and there are precisely $ \varphi\bigl(\varphi(n)\bigr) $.
\end{Proposition}