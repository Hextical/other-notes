\makeheading{Lecture 12}{\printdate{2022-05-30}}%chktex 8
\section{Linear Congruences}
\begin{Definition}{}{}
    An equation of the form
    \[ a_1x_1+a_2x_2+\cdots+a_k x_k\equiv b\Mod{n} \]
    with unknowns $ x_1,x_2,\ldots,x_k $ is a linear congruence equation in $ k $ variables.
\end{Definition}
Observe that by definition of mod, we can rewrite a linear congruence equation as
\[ a_1x_1+a_2x_2+\cdots+a_k x_k-nx_{k+1}=b, \]
which is a Diophantine equation in $ k+1 $ variables.

Observe that a linear congruence equation either has no solution or infinitely many
solutions. Indeed, if $ x_i=s_i $, $ 1\le i\le k $ is solutions of the form
\[ a_1x_1+a_2x_2+\cdots+a_k x_k\equiv b\Mod{n}, \]
then
\[ x_i=s_i+qn,\; 1\le i\le k \]
is also a solution for all $ q\in\mathbf{Z} $. This implies that the corresponding Diophantine equation also either has no
solution or infinitely many solutions.

\underline{Remark}: When writing the solutions of a linear congruence equation $ ax\equiv b\Mod{n} $,
we typically either write all solutions in the form
\[ x\equiv s\Mod{n} \]
or we say that $ s $ is the unique solution modulo $ n $.

\begin{Theorem}{}{}
    Let $ a,b\in\mathbf{Z} $ and $ n\in\mathbf{Z}^+ $. Let $ (a,n)=d $ and consider the linear congruence
    \[ ax\equiv b\Mod{n}. \]
    If $ d\mid b $, then the linear congruence has no solution. If $ d\mid b $, then the linear
    congruence has exactly $ d $ distinct solutions modulo $ n $.
    \tcblower{}
    \textbf{Proof}: Solving the congruence $ ax\equiv b\Mod{n} $ is equivalent to solving the
    linear Diophantine equation $ ax+ny=b $ for some $ y $. If $ d\nmid b $, then the Diophantine
    equation has no solution, so the congruence has no solution either.
    If $ d\mid b $, then by Theorem 1 (Lecture 5), the solution of the Diophantine equation take the form
    \[ x=x_0-\frac{n}{d}t,\; y=y_0+\frac{a}{d}t, \]
    where $ (x_0,y_0) $ is any particular solution (obtained from the Euclidean algorithm,
    for instance).

    We need to show that of these infinitely many solutions, there are exactly $d$
    distinct solutions mod $n$. Suppose two solutions of this form are congruent mod $ n $, that is,
    \[ x_0-\frac{n}{d}t_1\equiv x_0-\frac{n}{d}t_2\Mod{n}. \]
    Then,
    \[ \frac{n}{d}t_1\equiv \frac{n}{d}t_2\Mod{n}. \]
    Now, $ \bigl(\frac{n}{d},n\bigr)=\frac{n}{d} $, so by Proposition 3 (Lecture 10), we can divide this congruence by $ \frac{n}{d} $ to obtain
    \[ t_1\equiv t_2\Mod{d}. \]
    Likewise, suppose $ t_1\equiv t_2\Mod{d} $. This means that $ t_1 $ and $ t_2 $ differ by a multiple of $ d $, that is,
    \[ t_1-t_2=kd. \]
    So,
    \[ \frac{n}{d}t_1-\frac{n}{d}t_2=\frac{n}{d}kd=nk. \]
    This implies that
    \[ \frac{n}{d}t_1\equiv \frac{n}{d}t_2\Mod{n}. \]
    By Corollary 1 (Lecture 10),
    \[ x_0-\frac{n}{d}t_1\equiv x_0-\frac{n}{d}t_2\Mod{n}. \]
    We have proven that two solutions of the above form are equal mod n if and only
    if their parameter values are equal mod d, that is, If we let t range over a complete
    system of residues mod d, then $x_0+\frac{n}{d}t$ ranges over all possible solutions mod n.
    To be very specific, all the solutions mod $n$ are given by
    \[ x_0+\frac{n}{d}t\Mod{n},\; t=0,1,2,\ldots,d-1. \]
\end{Theorem}
\begin{Corollary}{}{}
    Let $ a,b\in\mathbf{Z} $ and $ n\in\mathbf{Z}^+ $. If $ (a,n)=1 $, then the equation
    \[ ax\equiv b\Mod{n} \]
    has a solution. Moreover, the unique solution modulo $ n $ is
    \[ x\equiv a^{-1}b\Mod{n}. \]
\end{Corollary}
\begin{Example}{}{}
    Solve $ 6x\equiv 7\Mod{8} $.
    \tcblower{}
    \textbf{Solution}: Since $ (6,8)=2\mid 7 $, there are no solutions.
\end{Example}
\begin{Example}{}{}
    Solve $ 3x\equiv 7\Mod{4} $.
    \tcblower{}
    \textbf{Solution}: Since $ (3,4)=1\mid 7 $, there is exactly one solution modulo $ 7 $. We have $ 3x+4y=7 $ for some $ y\in\mathbf{Z} $.
    By the EEA, we have
    \[ \begin{array}{lllll}
            r_i & q_{i-1} & s_i & t_i & \text{Check}          \\
            \midrule
            4   &         & 1   & 0                           \\
            3   & 1       & 0   & 1                           \\
            1   & 3       & 1   & -1  & 4\cdot 1+3\cdot(-1)=1
        \end{array} \]
    So, $ 4\cdot 7+3\cdot (-7)=7 $. Thus, $ x_0=-7 $, $ y_0=7 $ is a particular solution. So the general solution is:
    \[ x=-7-4t,\; y=7+3t. \]
    The $ y $ equation is irrelevant, and the $ x $ equation says
    \[ x\equiv 1\Mod{4}. \]
\end{Example}
\begin{Example}{}{}
    Find all solutions of $ 7x\equiv 5\Mod{39} $.
    \tcblower{}
    \textbf{Solution}: Since $ (7,5)=1\mid 39 $, there is exactly one solution modulo $ 39 $. We have
    $ 7x+39y=5 $ for some $ y\in\mathbf{Z} $. By the EEA, we have
    \[ \begin{array}{lllll}
            r_i & q_{i-1} & s_i & t_i & \text{Check}   \\
            \midrule
            39  &         & 1   & 0                    \\
            7   & 5       & 0   & 1                    \\
            4   & 1       & 1   & -5  & 39(1)+7(-5)=4  \\
            3   & 1       & -1  & 6   & 39(-1)+7(6)=3  \\
            1   & 3       & 2   & -11 & 39(2)+7(-11)=1 \\
        \end{array} \]
    So,
    \begin{align*}
        7(-11)+39(2)               & =1        \\
        7(-11\cdot 5)+39(2\cdot 5) & =1\cdot 5 \\
        7(-55)+39(10)=5.
    \end{align*}
    Thus, $ x_0=-55 $, $ y_0=10 $ is a particular solution.
    The general solution is:
    \[ x\equiv x_0+\frac{n}{d}(0)\equiv -55\equiv 23\Mod{39}.  \]
\end{Example}
\section{Linear Equations in \texorpdfstring{$ \mathbf{Z}_n $}{Zn}}
Let $ n $ be a modulus. We will now turn our attention to equations
in $ \mathbf{Z}_n $. Let $ a,b\in\mathbf{Z} $, and consider
\[ [a][x]=[b] \]
in $ \mathbf{Z}_n $ where $ x\in\mathbf{Z} $ is unknown.
\begin{Example}{}{}
    The linear equation $ [2][x]=[3] $ has only one solution in $ \mathbf{Z}_9 $,
    namely $ [x]=[6] $.
\end{Example}
\begin{Example}{}{}
    The equation $ [3][x]=[7] $ has no solution in $ \mathbf{Z}_9 $.
\end{Example}
\begin{Example}{}{}
    The linear equation $ [3][x]=[6] $ has three solutions in $ \mathbf{Z}_9 $, namely
    $ [x]=[2] $, $ [x]=[5] $, and $ [x]=[8] $.
\end{Example}
Note: From Example 6, we see the principal difference between the linear
equations in $ \mathbf{Z}_n $ and the linear equation $cx = d$ in $ \mathbf{Z} $. The only way $cx = d$ can
have more than one solution is if $c = d = 0$.

It turns out that the tools that we have in our hands right now can help us to
solve the linear congruence easily. Observe that
\begin{align*}
    [a][x] & =[b]  \\
    [ax]   & =[b],
\end{align*}
and this is the same as solving the linear congruence
\[ ax\equiv b\Mod{n}. \]
\begin{Proposition}{}{}
    Let $ a,b\in\mathbf{Z} $, $ n\in\mathbf{Z}^+ $ and $ a\ne 0 $. The linear equation
    $ [a][x]=[b] $ in $ \mathbf{Z}_n $ if $ d=(a,n)\mid b $ and total no.\ of residue classes
    satisfying $ [a][x]=[b] $ in $ \mathbf{Z}_n $ is equal to $ d=(a,n) $.
\end{Proposition}
\begin{Example}{}{}
    Solve $ [440][x]=[80] $ in $ \mathbf{Z}_{300} $.
    \tcblower{}
    \textbf{Solution}: By the EEA, the general solution to $ 440x+300y=80 $ is
    \[ x=-8-15t,\; y=12+22t,\; t\in\mathbf{Z}. \]
    By evaluating $ -8-15t $ at $ t=0,1,\ldots,19 $, we obtain $ 20 $ distinct solutions in $ \mathbf{Z}_{300} $.
\end{Example}
\begin{Proposition}{}{}
    Let $ a,b\in\mathbf{Z} $, $ n\in\mathbf{Z}^+ $, and $ (a,n)=1 $. The linear equation
    $ [a][x]=[b] $ has a unique solution in $ \mathbf{Z}_n $.
\end{Proposition}
\section{Chinese Remainder Theorem}
Around the year 300 a solution to the following mathematical problem appeared
in the mathematical manual of Chinese Master, Sun Tzu Suan Ching.

``We have a number of things, but we do not know exactly how many. If we count
them by threes, we have two left over. If we count them by fives, we have three
left over. If we count them by sevens, we have two left over. How many things are
there?''

The master was asking us to solve the three simultaneous congruences:
\begin{align*}
    x & \equiv 2\Mod{3}  \\
    x & \equiv 3\Mod{5}  \\
    x & \equiv 2\Mod{7}.
\end{align*}
The Chinese remainder theorem tells us that $ x\equiv 23\Mod{105} $ is the solution to
above problem.

Before proceeding to its statement, let us prove the following result.

\begin{Proposition}{}{}
    Let $m$ and $n$ be integers greater than $1$ that are coprime (relatively prime). Then the congruence
    \[ a\equiv b\Mod{mn} \]
    is true if and only if both the congruences
    \begin{align*}
        a & \equiv b\Mod{m} \\
        a & \equiv b\Mod{n}
    \end{align*}
    are true.
    \tcblower{}
    \textbf{Proof}: Suppose that $ a\equiv b\Mod{mn} $, then $ mn\mid (a-b) $, so $ m\mid (a-b) $ and $ n\mid (a-n) $.
    Therefore, $ a\equiv b\Mod{m} $ and $ a\equiv b\Mod{n} $. On the other hand, suppose that $ a\equiv b\Mod{m} $ and $ a\equiv b\Mod{n} $,
    Then, $ m\mid (a-b) $ and $ n\mid (a-b) $. Since $ (m,n)=1 $, $ mn\mid (a-b) $ by Proposition 1 (Lecture 4). Thus, $ a\equiv b\Mod{mn} $.
\end{Proposition}
\begin{Theorem}{The Chinese Remainder Theorem (CRT)}{}
    If $ m,n $ are coprime (relatively prime) moduli and $ a,b\in\mathbf{Z} $, then
    \begin{align*}
        x & \equiv a\Mod{m} \\
        x & \equiv b\Mod{n}
    \end{align*}
    have a common solution $ x_0 $. Furthermore, any two solutions $ x_0,y_0 $ to this pair of
    congruences must be such that $ x_0\equiv y_0\Mod{mn} $. The congruences have a unique
    solution.
    \tcblower{}
    \textbf{Proof}: Since $ m,n $ are coprime, by Bezout's Identity there exists $ x,y\in\mathbf{Z} $ such that
    \[ mx+ny=1. \]
    Multiplying both sides by $ (b-a) $, we obtain a solution to
    \[ mx^\prime+ny^\prime=b-a, \]
    where $ x^\prime=(b-a)x $ and $ y^\prime=(b-a)y $. Thus, $ a+mx^\prime=b-ny^\prime=x_0 $, we see that
    \begin{align*}
        x_0 & \equiv a\Mod{m}  \\
        x_0 & \equiv b\Mod{n}.
    \end{align*}
    So $ x_0 $ is a common solution. Now, let $ y_0 $ be any other solution to the system of congruences, then
    \begin{align*}
        x_0 & \equiv y_0\Mod{m}  \\
        x_0 & \equiv y_0\Mod{n}.
    \end{align*}
    So, by Proposition 3, we conclude that
    \[ x_0\equiv y_0\Mod{mn}. \]
\end{Theorem}
\makeheading{Lecture 13}{\printdate{2022-06-01}}%chktex 8
Here is an alternate proof for Chinese Remainder Theorem.

We have that there exists $ s\in\mathbf{Z} $ such that $ x=ms+a $. Substituting this into the other congruence
gives
\begin{align*}
    ms+a & \equiv b\Mod{n}      \\
    ms   & \equiv (b-a)\Mod{n}.
\end{align*}
Since $ (m,n)=1 $, there exists $ c\in\mathbf{Z} $ such that $ ms\equiv 1\Mod{n} $. Multiplying by $ c $
gives
\[ s\equiv c(b-a)\Mod{n}. \]
Thus, there exists $ t\in\mathbf{Z} $ such that $ s=tn+c(b-a) $.

Hence,
\begin{align*}
    x & =m\bigl(tn+c(b-a)\bigr)+a \\
      & =mnt+mc(b-a)+a.
\end{align*}
Thus, the unique solution is
\[ x\equiv mc(b-a)+a\Mod{mn}. \]
We can easily generalize this result to arbitrary number of coprime moduli.
\begin{Theorem}{Generalized Chinese Remainder Theorem}{}
    Suppose $ n_1,\ldots,n_k $ are moduli that are pairwise coprime. If $ a_1,\ldots,a_k\in\mathbf{Z} $
    then there exists $ x\in\mathbf{Z} $ such that
    \begin{align*}
        x & \equiv a_1\Mod{n_1}, \\
          & \vdotswithin{\equiv} \\
        x & \equiv a_k\Mod{n_k}.
    \end{align*}
    Furthermore, if $ x_0 $ is a solution of these congruences, then the complete solution
    to all the equations is given by all
    \[ x\equiv x_0\Mod{n_1\cdots n_k}. \]
\end{Theorem}
\textbf{Process to solve systems of congruences with CRT}:
\begin{itemize}
    \item Begin with the largest modulus $ x\equiv a_k\Mod{n_k} $.
          Rewrite it as $ x=n_k j_k+a_k $ for some $ j_k\in\mathbf{Z} $.
    \item Substitute the expression for $x$ into the congruence with the next largest
          modulus, that is,
          \[ x\equiv a_{k-1}\Mod{n_{k-1}}\implies n_k j_k+a_k\equiv a_{k-1}\Mod{n_{k-1}}. \]
    \item Solve this congruence for $j_k$.
    \item Write the solved congruence as an equation, and then substitute this expression for
          $ j_k $ into the equation for $x$.
    \item Continue substituting and solving the congruences until the equation for
          $x$ implies the solution to the system of congruences.
\end{itemize}
\begin{Example}{}{}
    Solve the system of congruences:
    \begin{align*}
        x & \equiv 3\Mod{6}   \\
        x & \equiv 7\Mod{13}.
    \end{align*}
    \tcblower{}
    \textbf{Solution}: First, $ x\equiv 7\Mod{13}\iff x=13j+7 $ for some $ j\in\mathbf{Z} $. Then,
    \[ x\equiv 3\Mod{6}\implies 13j+7\equiv 3\Mod{6}. \]
    Now, solve for $ j $ to get: $ j\equiv 2\Mod{6}\iff j=6k+2 $ for some $ k\in\mathbf{Z} $. Then,
    \[ x=13(6k+2)+7=78k+33\implies x\equiv 33\Mod{78}. \]
\end{Example}
\begin{Exercise}{}{}
    Solve the system of congruences:
    \begin{align*}
        x & \equiv 6\Mod{11}  \\
        x & \equiv 13\Mod{16} \\
        x & \equiv 9\Mod{21}
    \end{align*}
\end{Exercise}
\begin{Exercise}{}{}
    Calvin Butterball keeps pet meerkats in his backyard. If he divides
    them into $5$ equal groups, $4$ are left over. If he divides them into $8$ equal groups, $6$
    are left over. If he divides them into $9$ equal groups, $8$ are left over. What is the
    smallest number of meerkats that Calvin could have?
\end{Exercise}
Sometimes we can solve a system even if moduli aren't relatively prime.
\begin{Theorem}{}{}
    Consider the system of congruences:
    \begin{align*}
        x & \equiv a\Mod{m}  \\
        x & \equiv b\Mod{n}.
    \end{align*}
    \begin{enumerate}[(a)]
        \item If $ (m,n)\nmid (a-b) $, then there are no solutions.
        \item If $ (m,n)\mid (a-b) $, then there is a unique solution mod $ [m,n] $.
    \end{enumerate}
    \tcblower{}
    \underline{Remark}: If $ (m,n)=1 $ case (b) automatically holds, and $ [m,n]=mn $; that is, we get the
    Chinese Remainder Theorem for two equations.
    \tcblower{}
    \textbf{Proof}: Exercise.
\end{Theorem}
\section{Euler \texorpdfstring{$ \varphi $}{φ} Function and Euler's Theorem}
Firstly, we will study the units for $ \mathbf{Z}_n $.
\begin{Definition}{}{}
    Let $ n\ge 2 $. An element $ [a] $ in $ \mathbf{Z}_n $ is a unit
    provided the equation $ [a][x]=[1] $ has a unique solution. The integer $ a $
    representing $ [a] $ is then called unit modulo $ n $. The unique $ [x] $ is called
    the inverse of $ [a] $ in $ \mathbf{Z}_n $. The set of all units of $ \mathbf{Z}_n $
    is denoted by $ \mathbf{Z}_n^* $.
    \tcblower{}
    \underline{Remark}: Proposition 2 (Lecture 12) tells the ways to think about units. An element
    $ [a]\in\mathbf{Z}_n $ is unit if and only if $a$ is coprime with $n$.
\end{Definition}
\begin{Proposition}{}{}
    If $ p $ is a prime and $ [a]\ne 0 $ in $ \mathbf{Z}_p $, then $ [a] $ is a unit. In other words,
    every non-zero element of $ \mathbf{Z}_p $ is a unit.
\end{Proposition}
The importance of Proposition 1 lies in the fact that $ \mathbf{Z}_p $ behaves just like the
well understood sets of numbers $ \mathbf{Q},\mathbf{R},\mathbf{C} $. Namely, $ \mathbf{Z}_p $ admits addition, subtraction,
multiplication, and division by everything except zero. Indeed, if $ [a]\ne 0 $ in $ \mathbf{Z}_p $,
then $[a][x] = [b]$ always has a solution. So we can divide. Systems that admits all
four of the arithmetic operation are usually called \textbf{fields}.
\begin{Proposition}{}{}
    Let $ n\ge 2 $. Then,
    \begin{enumerate}[i.]
        \item The product of a unit is another unit, that is, if $ [a],[b]\in\mathbf{Z}_n^* $, then $ [a][b]\in\mathbf{Z}_n^* $.
        \item The product of units is associative, that is, $ ([a][b])[c]=[a]([b][c]) $ for all $ [a],[b],[c]\mathbf{Z}_n^* $.
        \item The residue class $ [1] $ is always a unit, that is, $ [1]\in \mathbf{Z}_n^* $.
        \item The inverse of a unit is also a unit, that is, if $ a\in \mathbf{Z}_n^* $ and $ x\in \mathbf{Z}_n^* $ is a unique
              residue class that gives $ [a][x]=[1] $, then $ [x]\in \mathbf{Z}_n^* $.
              The product of units is commutative, that is, $ [a][b]=[b][a] $ in $ \mathbf{Z}_n^* $.
    \end{enumerate}
    Any set that enjoys the five properties of Proposition 2 is called an Abelian
    Group.
\end{Proposition}
\begin{Example}{}{}
    Compute $ \mathbf{Z}_{10}^* $ and construct its multiplication table.
    \tcblower{}
    \textbf{Solution}: The integers that are coprime to $10$ are $ 1 $, $ 3 $, $ 7 $, and $ 9 $. Thus,
    $ \mathbf{Z}_{10}^*=\Set{[1],[3],[7],[9]} $ and its multiplication table is TODO\@.
\end{Example}
\makeheading{Lecture 14}{\printdate{2022-06-03}}%chktex 8
As you may have noticed already, the condition that numbers are relatively
prime is useful and interesting. So, it should not be surprising that there are times
when it is useful to restrict a complete residue system modulo $n$ to just the numbers
which are relatively prime to $n$.
\begin{Definition}{}{}
    Let $ \varphi(n) $ denote the number of integers $ m $ such that $ 0\le m<n $ and $ (m,n)=1 $. The function
    $ \varphi $ is called the Euler's totient function.
\end{Definition}
\begin{Example}{}{}
    Find $ \varphi(18) $.
    \tcblower{}
    \textbf{Solution}: Numbers from $0$ to $17$ that are coprime with $18$ are $ 1,5,7,11,13,17 $.
    So, $ \varphi(18)=6 $.
\end{Example}
\begin{Example}{}{}
    Find $ \varphi(101) $.
    \tcblower{}
    \textbf{Solution}: Since $101$ is itself prime, the full list of numbers that are
    coprime with $101$ are $ 1,2,3,\ldots,100 $. Thus, $ \varphi(101)=100 $.
\end{Example}
\underline{Remark}: For any prime $ p $, the numbers $ 1,2,\ldots,(p-1) $ are coprime
with $ p $. Therefore, $ \varphi(p)=p-1 $.
\begin{Theorem}{Euler's Theorem}{}
    If $ [a]\in\mathbf{Z}_n^* $, then $ [a]^{\varphi(n)}=[1] $.
    \tcblower{}
    \textbf{Proof}: Let $ k=\varphi(n) $. Let $ [u_1],\ldots,[u_k] $
    be the complete list of residues of $ \mathbf{Z}_n^* $. Form a new list
    \[ [a][u_1],\ldots,[a][u_k]. \]
    Since $ \mathbf{Z}_n^* $ is a group, this list of residues is also in $ \mathbf{Z}_n^* $. Furthermore,
    no element appears in this list twice, so if $ [a][u_i]=[a][u_j] $ for some $ i\ne j $, then $ [u_i]=[u_j] $
    by cancelling the unit $ [a] $. Hence, the second list is a permutation of the original list.

    It follows that the product of residues in the first list equals to the product of residues in the second list. After all, the two lists
    contain the same residues, only written in a different order. Thus, we obtain
    \[ [u_1][u_2]\cdots[u_k]=([a][u_1])\cdots([a][u_k]). \]
    Now, we can cancel the unit element $ [u_1]\cdots[u_k] $ and conclude that $ [a]^{\varphi(n)}=[1] $.
\end{Theorem}
In the language of congruences, Euler's Theorem translates to
\[ a^{\varphi(n)}\equiv 1\Mod{n} \]
for any integer that is invertible modulo $ n $, that is, $ (a,n)=1 $. In other words,
if $ a $ is coprime with modulus $ n $, then
\[ \frac{a^{\varphi(n)}-1}{n}\in\mathbf{Z}. \]
\begin{Example}{}{}
    Prove $ 623^{1222}\equiv 1\Mod{1223} $.
    \tcblower{}
    \textbf{Solution}: $ \varphi(1223)=1222 $ and $ (1223,623)=1 $. Hence, by Euler's Theorem
    $ 623^{1222}\equiv 1\Mod{1223} $.
\end{Example}
When the modulus is a prime, say $ p $, we know that $ \varphi(p)=p-1 $.
Thus, Euler's Theorem specializes to another famous result attributed by Fermat.
\begin{Theorem}{Fermat's Little Theorem}{}
    If $ p $ is a prime and $ p\nmid a $ for $ p\in\mathbf{Z}_n^* $, then
    \[ [a]^{p-1}=[1]. \]
    In other words,
    \[ a^{p-1}\equiv 1\Mod{p}. \]
\end{Theorem}
\begin{Corollary}{Fermat Variant}{}
    If $ p $ is a prime and $ a\in\mathbf{Z} $, then
    \[ a^{p}=a\Mod{p}. \]
    \tcblower{}
    \textbf{Proof}: If $ p\nmid a $, then $ a^{p-1}\equiv 1\Mod{p} $ by Fermat's Theorem.
    Multiplying by $ a $ to get $ a^p\equiv a\Mod{p} $. On the other hand,
    if $ p\mid a $, then $ a\equiv 0\Mod{p} $, in which case it is obvious that
    \[ a^p\equiv 0^p\equiv 0\equiv a\Mod{p}. \]
    So the result holds for all $ a\in\mathbf{Z} $.
\end{Corollary}
\section{Pseudoprimes}
The converse of Fermat's Little Theorem is not true in general, that is, if
there exists $ a $ such that $ a^{n-1}\equiv 1\Mod{n} $, then we can not infer that $ n $ is prime.
\begin{Definition}{}{}
    A number $ n $ that is composite and satisfies $ a^{n-1}\equiv 1\Mod{n} $ is called a Fermat pseudoprime to base $ a $.
    In the special case of $ a=2 $, it is sometimes called the \textbf{Poulet number}.
\end{Definition}
\begin{Example}{}{}
    The following table gives us the first Fermat pseudoprime to some small bases $ a $:
    \[ \begin{array}{cl}
            a & \text{Fermat pseudoprime}             \\
            \midrule
            2 & 341,561,645,1105,1387,1729,1905       \\
            3 & 91,121,286,671,703,949,1105,1541,1729 \\
            4 & 15,85,91,341,435,561,645,703          \\
            5 & 4,124,217,561,781,1541,1729,1891
        \end{array} \]
    The first example of even pseudoprime ($n = 161038$) to the base $2$ was given
    by Lehmer in 1950.
\end{Example}
\begin{Example}{}{}
    Prove that $ 341 $ is a Fermat pseudoprime to the base $ 2 $.
    \tcblower{}
    \textbf{Solution}: We want to show that $ 2^{340}\equiv 1\Mod{341} $. Note that $ 341=11\times 31 $.
    By FLT, $ 2^{10}\equiv 1\Mod{11} $. Thus,
    \[ 2^{340}\equiv (2^{10})^{34}\equiv 1^{34}\equiv 1\Mod{11}. \]
    Thus, $ 11\mid (2^{340}-1) $. Also, $ 2^5\equiv 32\equiv 1\Mod{31} $. So,
    \[ 2^{340}\equiv (2^5)^{68}\equiv 1\Mod{31}. \]
    Thus, $ 31\mid (2^{340}-1) $. So, by Proposition 1 (Lecture 4), $ 341\mid (2^{340}-1) $.
\end{Example}
\begin{Exercise}{}{}
    Show that $ 341 $ is not a Fermat pseudoprime.
    \tcblower{}
    \textbf{Solution}: We want to show that $ 3^{340}\not\equiv 1\Mod{341} $. Note that $ 341=11\times 31 $.
    By FLT, $ 3^{10}\equiv 1\Mod{11} $. Thus, $ 3^{340}=(3^{10})^{34}\equiv 1^{34}\equiv 1\Mod{11} $.
    So we need to show that $ 3^{340}\not\equiv 1\Mod{31} $.
    Note that $ 340=30\times 11+10 $ and by FLT $ 3^{30}\equiv 1\Mod{31} $.
    \[ 3^{340}\equiv (3^{11})^{30}3^{10}\equiv 1^{30}3^{10}\equiv 3^{10}\Mod{31}. \]
    So, $ 3^2\equiv 9\Mod{31} $, $ 3^3\equiv 27\equiv -4\Mod{31} $, $ 3^4\equiv 81\equiv 19\Mod{31} $, $ 3^6\equiv (3^3)^2\equiv (-4)^2\equiv 16\Mod{31} $.
    Therefore,
    \[ 3^{10}\equiv 3^6\cdot 3^4\equiv 16\cdot 19\equiv 304\Mod{31}\equiv 25\Mod{31}.  \]
    Note that $ 3^{340}\equiv 25\not\equiv 1\Mod{31} $.
\end{Exercise}
\section{Polynomial Congruence}
We have seen how to solve linear congruences $ ax\equiv b\Mod{n} $. What about
polynomial congruences? These, of course, are also important in number theory.

We first note that there are some immediate differences from what we are used
to with solving polynomials over $ \mathbf{R} $.

For example, we know the polynomial $ f(x)=x^2+1 $ has no roots over $ \mathbf{R} $,
but
\[ x^2+1\equiv 0\Mod{5} \]
has $ x=2 $ and $ x=3 $ as solutions.

Also, we are used to $ d{\textsuperscript{th}} $ degree polynomials having exactly $ d $ roots,
but
\[ x^2+x\equiv 0\Mod{6} \]
has four distinct roots modulo $ 6 $, namely $ x=0,2,3,5 $.

The Chinese Remainder Theorem can also be utilized to solve polynomial congruences.

\begin{Definition}{}{}
    Let $d$ be a positive integer and consider a polynomial
    \[ f(x)=c_d x^d+\cdots + c_1 x+c_0, \]
    where $ c_0,\ldots,c_d\in\mathbf{Z} $ and $ c_d \ne 0 $. Then the congruence of the form
    $ f(x)\equiv 0\Mod{n} $ is called a \textbf{polynomial congruence}.
    \tcblower{}
    \underline{Goal}: Find all $ x\in\mathbf{Z} $ which satisfy the above congruence.
\end{Definition}

Note that if we replace $ c_i $ with $ [c_i] $, then we reduce the polynomial
from $ \mathbf{Z} $ to $ \mathbf{Z}_n $. Solving the above congruence is equivalent to solving
$ f([x])=[0] $ in $ \mathbf{Z}_n $. If such an equation is satisfied by some residue
class $ [x_0] $, we say that $ [x_0] $ is a root of $ f(x) $ in $ \mathbf{Z}_n $.