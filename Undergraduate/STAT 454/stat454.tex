\documentclass[oneside]{book}
\usepackage[svgnames]{xcolor}
\usepackage[british]{babel}
\usepackage[protrusion,expansion,babel,final]{microtype}
\usepackage[margin=1in]{geometry}
\usepackage[pdfversion=1.7]{hyperref}
\usepackage[shortlabels]{enumitem}
\usepackage{graphicx}
\usepackage{mathtools}
\usepackage{cleveref}
\usepackage{booktabs}
\usepackage{nicematrix}
\usepackage{derivative}
\usepackage{etoolbox}
\usepackage{siunitx}
\usepackage{lmodern}
\usepackage[T1]{fontenc}
\usepackage[scaled=.98]{XCharter}
\usepackage[scaled=1.04,varqu,varl]{inconsolata}% inconsolata typewriter
\usepackage{amssymb}
\makeatletter
\@namedef{T1/zi4/m/it}{<->ssub*lmr/m/it}
\makeatother

\usepackage{bm}
\usepackage{tikz}
\usepackage{float}

% Functions
\providecommand\given{} % just to make sure it exists
\let\v\relax%
\DeclarePairedDelimiterXPP{\E}[1]{\operatorname{\textsf{E}}}[]{}{%
    \renewcommand\given{\nonscript\:\delimsize\vert\nonscript\:\mathopen{}}%
    \ifblank{#1}{\:\cdot\:}%
    #1}%
\DeclarePairedDelimiterXPP{\Esp}[2]{\operatorname{\textsf{E}_{#2}}}[]{}{%
    \renewcommand\given{\nonscript\:\delimsize\vert\nonscript\:\mathopen{}}%
    \ifblank{#1}{\:\cdot\:}%
    #1}%
\DeclarePairedDelimiterXPP{\V}[1]{\operatorname{\textsf{V}}}(){}{%
    \renewcommand\given{\nonscript\:\delimsize\vert\nonscript\:\mathopen{}}%
    \ifblank{#1}{\:\cdot\:}%
    #1}%
\DeclarePairedDelimiterXPP{\Vsp}[2]{\operatorname{\textsf{V}_{#2}}}(){}{%
    \renewcommand\given{\nonscript\:\delimsize\vert\nonscript\:\mathopen{}}%
    \ifblank{#1}{\:\cdot\:}%
    #1}%
\DeclarePairedDelimiterXPP{\v}[1]{\operatorname{\textsf{v}}}(){}{%
    \renewcommand\given{\nonscript\:\delimsize\vert\nonscript\:\mathopen{}}%
    \ifblank{#1}{\:\cdot\:}%
    #1}%
\DeclarePairedDelimiterXPP{\Cov}[1]{\operatorname{\textsf{Cov}}}(){}{%
    \renewcommand\given{\nonscript\:\delimsize\vert\nonscript\:\mathopen{}}%
    \ifblank{#1}{\:\cdot\:}%
    #1}%
\DeclarePairedDelimiterXPP\Prob[1]{\operatorname{\textsf{P}}}(){}{%
    \renewcommand\given{\nonscript\:\delimsize\vert\nonscript\:\mathopen{}}%
    \ifblank{#1}{\:\cdot\:}%
    #1}%
\DeclarePairedDelimiterXPP\Ind[1]{\operatorname{\textsf{I}}}\{\}{}{%
    \renewcommand\given{\nonscript\:\delimsize\vert\nonscript\:\mathopen{}}%
    \ifblank{#1}{\:\cdot\:}%
    #1}%
\DeclarePairedDelimiterXPP{\se}[1]{\operatorname{\textsf{se}}}(){}{%
    \ifblank{#1}{\:\cdot\:}%
    #1}%
\DeclarePairedDelimiterXPP{\CV}[1]{\operatorname{\textsf{CV}}}(){}{
    \renewcommand\given{\nonscript\:\delimsize\vert\nonscript\:\mathopen{}}%
    \ifblank{#1}{\:\cdot\:}%
    #1}%
\let\exp\relax%
\let\log\relax%
\let\ln\relax%
\DeclarePairedDelimiterXPP{\exp}[1]{\operatorname{\textsf{exp}}}\{\}{}{#1}%
\DeclarePairedDelimiterXPP{\log}[1]{\operatorname{\textsf{log}}}(){}{#1}%
\DeclarePairedDelimiterXPP{\ln}[1]{\operatorname{\textsf{ln}}}(){}{#1}%
\DeclarePairedDelimiterXPP{\diag}[1]{\operatorname{\textsf{diag}}}(){}{#1}%
\DeclarePairedDelimiterXPP{\sign}[1]{\operatorname{\textsf{sign}}}(){}{#1}%

\DeclarePairedDelimiterXPP{\expit}[1]{\operatorname{\textsf{expit}}}(){}{#1}%
\DeclarePairedDelimiterXPP{\logit}[1]{\operatorname{\textsf{logit}}}(){}{#1}%
\newcommand{\HN}{\textsl{H}_{\textsl{0}}}%
\newcommand{\HA}{\textsl{H}_{\textsl{A}}}%

% Distributions
\DeclarePairedDelimiterXPP{\N}[1]{\mathcal{N}}(){}{#1}%
\DeclarePairedDelimiterXPP{\POI}[1]{\text{POI}}(){}{#1}%
\DeclarePairedDelimiterXPP{\BIN}[1]{\text{BIN}}(){}{#1}%
\DeclarePairedDelimiterXPP{\BERN}[1]{\text{BERN}}(){}{#1}%
\DeclarePairedDelimiterXPP{\MVN}[1]{\text{MVN}}(){}{#1}%
\DeclarePairedDelimiterXPP{\NB}[1]{\text{NB}}(){}{#1}%
\DeclarePairedDelimiterXPP{\GAM}[1]{\text{GAM}}(){}{#1}%
\DeclarePairedDelimiterXPP{\BetaDist}[1]{\text{Beta}}(){}{#1}%

\newcommand{\iid}{\overset{\text{iid}}{\sim}}%
\newcommand{\ind}{\overset{\text{ind}}{\sim}}%
\newcommand{\tod}{\xrightarrow[]{d}}%
\newcommand{\inp}{\xrightarrow[]{p}}%
\newcommand{\OR}{\text{OR}}%
\newcommand{\RR}{\text{RR}}%
\newcommand{\cOR}{\text{cOR}}%
\newcommand{\st}{\text{st}}%
\newcommand{\post}{\text{post}}%
\newcommand{\HT}{\text{HT}}%

\DeclarePairedDelimiter\abs{\lvert}{\rvert}
% can be useful to refer to this outside \Set
\newcommand\SetSymbol[1][]{%
    \nonscript\:#1\vert{}
    \allowbreak\nonscript\:
    \mathopen{}}
\DeclarePairedDelimiterX\Set[1]\{\}{%
    \renewcommand\given{:}
    #1
}
\DeclareMathOperator*{\argmax}{arg\,max}
\DeclareMathOperator*{\argmin}{arg\,min}
\DeclareMathOperator*{\arginf}{arg\,inf}
\DeclareMathOperator*{\argsup}{arg\,sup}

\providecommand{\RandomVector}[1]{\bm{#1}}% general vectors in bold italic
\providecommand{\Vector}[1]{\bm{#1}}% general vectors in bold italic
\providecommand{\Matrix}[1]{\bm{#1}}
\providecommand{\MatrixCal}[1]{\bm{\mathcal{#1}}}
\providecommand{\Field}[1]{\bm{#1}}

\usepackage{stackengine}
\usepackage[british]{isodate}
\newcommand{\makeheading}[2]%
{%
\begin{center}%
    \makebox[\linewidth]{\raisebox{-.5ex}[0cm][0cm]{\stackanchor{\textcolor{Gray}{\textsc{#1}}}{\emph{\scriptsize\printyearoff#2}}\;}\color{Crimson!50}\hrulefill}%
\end{center}%
}%

\usepackage[breakable]{tcolorbox}
\tcbset{
    regular/.style={
        breakable,
        sharp corners
    }
}

\newtcolorbox{Example}[1]{regular,colframe=Green!20!white,colback=Green!10!white,coltitle=Green,title={#1}}%
\newtcolorbox{Regular}[1]{breakable,sharp corners,title={#1}}%
\newtcolorbox{Result}[1]{regular,colframe=Red!15!white,colback=Red!5!white,coltitle=Red,title={#1}}%

\hypersetup{colorlinks=true,%
linkcolor=[rgb]{0,0.5,1},%
pdftitle={Sampling Theory and Practice (STAT 454/854)},%
pdfauthor={Cameron Roopnarine, Changbao Wu},%
pdfsubject={Statistics},%
pdfkeywords={University of Waterloo, Winter 2022 (1221)}}%

\title{%
\LARGE Sampling Theory and Practice\\%
\large STAT 454\thanks{STAT 454$\equiv$ STAT 854}\\%
\normalsize Winter 2022 (1221)\thanks{Online Course until \printdate{2022-02-07}}}%chktex 8
\author{Cameron Roopnarine\thanks{\LaTeX{}er}\and Changbao Wu\thanks{Instructor}}%
\date{\today}%
\usepackage{pgfplots}
\pgfplotsset{compat=1.18}
\usetikzlibrary{petri,decorations.pathreplacing,calc}
\usepackage{minted}
\usepackage[parfill]{parskip}

\begin{document}
\maketitle
\tableofcontents
\makeheading{Lecture 1}{\printdate{2022-09-07}}%chktex 8
\begin{itemize}
    \item Textbook: \textbf{Statistical Inference} by
          \emph{George Casella + Roger L.\ Berger}
    \item Office hours: Monday 1:30--2:30 in HH210.
\end{itemize}
\section*{Set Theory}
\begin{Definition}{Containment}{}
    \[ A\subset B\iff x\in A\implies x\in B. \]
\end{Definition}
\begin{Definition}{Equality}{}
    \[ A=B\iff A\subset B\text{ and }B\subset A. \]
\end{Definition}
\begin{Definition}{Union}{}
    The \textbf{union} of $ A $ and $ B $, written $ A\cup B $, is the set of
    elements that belong to either $ A $ or $ B $ or both:
    \[ A\cup B=\Set{x\given x\in A\text{ or } x\in B}. \]
    \tcblower{}
    For example, if $ A=\Set{0,2,4,6,8} $ and $ B=\Set{0,3,6,9} $, then
    \[ A\cup B=\Set{0,2,3,4,6,8,9}. \]
\end{Definition}
\begin{Definition}{Intersection}{}
    The \textbf{intersection} of $ A $ and $ B $, written $ A\cap B $, is the set of elements
    that belong to both $ A $ and $ B $:
    \[ A\cap B=\Set{x\given x\in A\text{ and } x\in B}. \]
\end{Definition}
\begin{Definition}{Complementation}{}
    The \textbf{complement} of $ A $, written $ A^c $, is the set of all elements that are not in $ A $:
    \[ A^c=\Set{x\given x\notin A}. \]
\end{Definition}
\begin{Definition}{Relative Complement}{}
    The \textbf{relative complement} of $ A $ in $ B $, written $ B\setminus A $, is the set of all elements that are in $ B $ and not in $ A $:
    \[ B\setminus A=\Set{x\given x\in B\text{ and } x\notin A}=B\cap A^c. \]
\end{Definition}
\begin{Theorem}{De Morgan's Laws}{}
    For any events $ A $ and $ B $ defined on a sample space $ S $,
    \begin{enumerate}[(i)]
        \item $ (A\cup B)^c=A^c\cap B^c $.
        \item $ (A\cap B)^c=A^c\cup B^c $.
    \end{enumerate}
    \tcblower{}
    \textbf{Proof}:
    \begin{enumerate}[(i)]
        \item Let $ x\in (A\cup B)^c $. We know that $ x\notin (A\cup B) $, so
              $ x\notin A $ and $ x\notin B $. Hence, $ x\in A^c $ and $ x\in B^c $, which means
              $ x\in (A^c\cap B^c) $. Therefore, $ (A\cup B)^c\subset (A^c\cap B^c) $.

              Let $ y\in (A^c\cap B^c) $. We know that $ y\in A^c $ and $ y\in B^c $, so
              $ y\notin A $ and $ y\notin B $. Hence, $ y\notin (A\cup B) $, which means
              $ y\in (A\cup B)^c $. Therefore, $ (A^c\cap B^c)\subset (A\cap B)^c $.
        \item Let $ x\in (A\cap B)^c $. We know that $ x\notin (A\cap B) $,
              so $ x\notin A $ or $ x\notin B $. Hence, $ x\in A^c $ or $ x\in B^c $, which means
              $ x\in (A^c\cup B^c) $. Therefore, $ (A\cap B)^c\subset (A^c\cup B^c) $.

              Let $ y\in (A^c\cup B^c) $. We know that $ y\in A^c $ or $ y\in B^c $,
              so $ y\notin A $ or $ y\notin B $. Hence, $ y\notin (A\cap B) $,
              which means $ y\in (A\cap B)^c $. Therefore, $ (A^c\cup B^c)\subset (A\cap B)^c $.
    \end{enumerate}
\end{Theorem}
\begin{Theorem}{Distributive Laws}{}
    \begin{enumerate}[(i)]
        \item $ A\cap (B\cup C)=(A\cap B)\cup (A\cap C) $.
        \item $ A\cup (B\cap C)=(A\cup B)\cap (A\cup C) $.
    \end{enumerate}
\end{Theorem}
\begin{Definition}{Injective and Surjective}{}
    Let $ A $ and $ B $ be sets and let $ f\colon A\to B $.
    \begin{itemize}
        \item We say $ f $ is \textbf{injective} (or \textbf{one-to-one}, written as $ 1\colon 1 $)
              when for all $ x,y\in A $, if $ f(x)=f(y) $, then $ x=y $.
        \item We say $ f $ is \textbf{surjective} (or \textbf{onto}) when for every $ y\in B $,
              there exists at least one $ x\in A $ such that $ f(x)=y $.
    \end{itemize}
\end{Definition}
\begin{Definition}{Countability}{}
    A set $ S $ is \textbf{countable} if there exists an injective function
    $ f\colon S\to\mathbf{N} $.
\end{Definition}
\begin{Example}{}{}
    The set $ \mathbf{Z} $ of all integers is countable.
    First, match $ 0 $ with $ 1 $. Then, for $ n>0 $, match $ n $ with
    $ 2n $ and match $ -n $ with $ 2n+1 $.
    \[ \begin{array}{c|c}
            1 & 0  \\
            \hline
            2 & 1  \\
            \hline
            3 & -1 \\
            \hline
            4 & 2  \\
            \hline
            5 & -2 \\
            \hline
            6 & 3  \\
            \hline
            7 & -3
        \end{array} \]
\end{Example}
\begin{Theorem}{}{}
    The unit interval $ [0,1] $ is not countable.
    \tcblower{}
    \textbf{Proof} (Cantor's diagonalization argument):
    Assume for a contradiction that there is some bijection
    $ f\colon \mathbf{N}\to [0,1] $.
    \[ \begin{array}{c|l}
            1 & f(1)=0.5000\cdots            \\
            \hline
            2 & f(2)=0.14152\cdots           \\
            \hline
            3 & f(3)=0.33333\cdots           \\
            \hline
            4 & f(4)=0.110100100010000\cdots \\
            \hline
            5 & f(5)=0.12345\cdots
        \end{array} \]
    Denote
    \begin{align*}
        f(1) & =0.a_{11}a_{12}a_{13}a_{14}\cdots \\
        f(2) & =0.a_{21}a_{22}a_{23}a_{24}\cdots \\
             & \vdotswithin{=}                   \\
        f(n) & =0.a_{n1}a_{n2}a_{n3}a_{n4}\cdots
    \end{align*}
    For example, $ a_{24}=5 $.
    Let
    \begin{align*}
        b_1 & =9-a_{11}\cdots \\
        b_2 & =9-a_{22}\cdots \\
        b_3 & =9-a_{33}\cdots \\
            & \vdotswithin{=} \\
        b_n & =9-a_{nn}\cdots
    \end{align*}
    Then, $ 0.b_1b_2b_3\cdots $ does not
    appear anywhere in my list, since for every $ n\ge 1 $, the
    $ n\textsuperscript{th} $ digit of this number is different
    from the $ n\textsuperscript{th} $ digit of the
    $ n\textsuperscript{th} $ number on my list. This contradicts
    my assumption that $ f $ is a bijection.
\end{Theorem}
\begin{Definition}{}{}
    A \textbf{probability space} is an ordered triple
    $ (\Omega,\mathcal{F},\mathbb{P}) $ where
    \begin{itemize}
        \item $ \Omega $ is a non-empty set, called the \emph{sample space}
              (where elements
              $ \omega\in\Omega $ are called ``events''),
        \item $ \mathcal{F} $ is a collection of subsets of $ \Omega $,
              called the \emph{$ \sigma $-algebra} (where elements
              $ A\in\mathcal{F} $ are called ``events'') with the following properties:
              \begin{enumerate}[S1]
                  \item $ \Omega\in\mathcal{F} $,
                  \item $ \forall A\in \mathcal{F} $, $ (\Omega\setminus A)=A^c\in\mathcal{F} $ (closed under complements),
                  \item For any sequence $ A_1,A_2,A_3,\ldots\in\mathcal{F} $, we get
                        $ \bigcup_i A_i\in \mathcal{F} $ (closed under countable unions),
              \end{enumerate}
        \item $ \mathbb{P}\colon \mathcal{F}\to[0,1] $ with
              \begin{enumerate}[P1]
                  \item $ \Prob{\Omega}=1 $,
                  \item $ \Prob{A}\ge 0 $ for all $ A $, and
                  \item if $ A_1,A_2,\ldots, $ are disjoint elements of $ \mathcal{F} $,
                        then
                        \[ \Prob*{\bigcup_i A_i}=\sum_i \Prob{A_i}\quad\text{(countable additivity)}. \]
              \end{enumerate}
    \end{itemize}
\end{Definition}
\makeheading{Lecture 2}{\printdate{2022-09-09}}%chktex 8
\begin{Example}{}{}
    Flip a fair coin.
    \begin{itemize}
        \item Sample space: $ \Omega=\Set{\text{H},\text{T}} $;
              that is, $ \abs{\Omega}=2 $.
        \item $ \mathcal{F}=\Set{\emptyset,
                      \Set{\text{H}},\Set{\text{T}},\Set{\text{H},\text{T}}} $;
              that is,
              $ \abs{\mathcal{F}}=2^{\abs{\Omega}}=4 $.
    \end{itemize}
    Whenever $ \Omega $ is countable, we define $ \mathcal{F} $ to be
    the set of \underline{all} subsets of $ \Omega $,
    $ \mathcal{F}=2^{\Omega} $ (we can always choose the power set of $ \Omega $
    as our discrete $ \sigma $-algebra).
    \begin{itemize}
        \item $ \text{H} $ is an outcome, $ \text{H}\in\Omega $.
        \item $ \emptyset $ is an event, $ \emptyset\in\mathcal{F} $.
        \item $ \Set{\text{H}} $ is an event,
              but $ \text{H} $ is not an event, and $ \Set{\text{H}} $
              is not an outcome.
        \item $ \Prob{\emptyset}=0 $.
        \item $ \Prob{\Set{\text{H}}}=1/2 $.
        \item $ \Prob{\Set{\text{T}}}=1/2 $.
        \item $ \Prob{\Set{\text{H},\text{T}}}=1 $.
        \item $ \Prob{\text{H}}=\text{undefined} $.
    \end{itemize}
\end{Example}
\begin{Example}{}{}
    Let $ \Omega=\Set{(x,y)\in\mathbf{R}^2\given x^2+y^2\le 25} $
    (disc of radius $5$). Suppose that we have a bullseye of
    radius $ 1 $, the probability of hitting the bullseye
    is $ 1/25 $.
    \begin{align*}
        \text{Bullseye} & =\Set{(x,y)\in\Omega\given x^2+y^2\le 1}.                 \\
        \Prob{\text{Bullseye}}
                        & =\frac{\text{Area}(\text{Bullseye})}{\text{Area}(\Omega)} \\
                        & =\frac{\pi\cdot 1^2}{\pi\cdot 5^2}                        \\
                        & =\frac{1}{25}.
    \end{align*}
    My $ \sigma $-algebra $ \mathcal{F} $ for dart-throwing will be
    the \underline{smallest $ \sigma $-algebra} that includes all sets of the form
    \[ \bigl(\interval[open left]{a}{b}\times \interval[open left]{c}{d}\bigr)\cap \Omega,\; a<b,\; c<d,\; a,b,c,d\in\mathbf{R}. \]
    \begin{itemize}
        \item $ \abs{\mathbf{N}}=\abs{\mathbf{Z}}=\abs{\mathbf{Q}}=\aleph_0 $.
        \item $ \abs{\mathbf{R}}=\abs[\big]{[0,1]}=\abs{\mathbf{R}^n}=2^{\mathbf{N}}
                      =2^{\aleph_0} $.
        \item $ \abs{2^{\mathbf{R}^2}}=2^{2^{\aleph_0}}\gg 2^{\aleph_0} $.
    \end{itemize}
\end{Example}
\begin{Proposition}{}{}
    Given a probability space $ (\Omega,\mathcal{F},\mathbb{P}) $,
    \begin{enumerate}[(i)]
        \item For all $ A\in\mathcal{F} $, $ \Prob{A^c}=1-\Prob{A} $.
        \item $ \Prob{\emptyset}=0 $.
        \item $ \forall A\in\mathcal{F} $, $ \Prob{A}\le 1 $.
        \item $ \forall A,B\in\mathcal{F} $,
              $ \Prob{B\cap A^c}=\Prob{B}-\Prob{A\cap B} $.
    \end{enumerate}
    \tcblower{}
    \textbf{Proof}:
    \begin{enumerate}[(i)]
        \item By (S2), $ A^c\in\mathcal{F} $. Since $ A^c\cap A=\emptyset $,
              \begin{align*}
                  \Prob{A^c}+\Prob{A}
                   & =\Prob{A^c\cup A} &  & \text{by (P3)}  \\
                   & =\Prob{\Omega}                         \\
                   & =1                &  & \text{by (P1)}.
              \end{align*}
        \item $ \Prob{\emptyset}=\Prob{\Omega^c}=1-\Prob{\Omega}=0 $ by (i).
        \item $ \Prob{A}=1-\Prob{A^c}\le 1 $ since $ \Prob{A^c}\ge 0 $.
        \item $ (A\cap B)\subseteq A $, and $ (A^c\cap B)\subseteq A^c $,
              so $ \bigl((A\cap B)\cap (A^c\cap B)\bigr)\subseteq (A\cap A^c)=\emptyset $.
              Thus,
              \begin{align*}
                  \Prob{A\cap B}+\Prob{A^c\cap B}
                   & =\Prob[\big]{(A\cap B)\cup (A^c\cap B)} \\
                   & =\Prob[\big]{B\cap (A\cup A^c)}         \\
                   & =\Prob{B\cap \Omega}                    \\
                   & =\Prob{B}.
              \end{align*}
    \end{enumerate}
\end{Proposition}
\begin{Theorem}{Inclusion-exclusion for two events}{}
    $ \Prob{A\cup B}=\Prob{A}+\Prob{B}-\Prob{A\cap B} $.
    \tcblower{}
    \textbf{Proof}:
    \begin{align*}
        A\cup B
         & =A\cup (B\cap \Omega)              \\
         & =A\cup (B\cap (A\cup A^c))         \\
         & =(A\cup (B\cap A))\cup (B\cap A^c) \\
         & =A\cup (B\cap A^c).
    \end{align*}
    Therefore, $ A $ is disjoint from $ B\cap A^c $. Thus,
    \begin{align*}
        \Prob{A\cup B}
         & =\Prob{A}+\Prob{B\cap A^c}           \\
         & =\Prob{A}+(\Prob{B}-\Prob{A\cap B}).
    \end{align*}
\end{Theorem}
\begin{Theorem}{Inclusion-exclusion principle for probabilities}{}
    For any $ A_1,A_2,\ldots,A_n\in\mathcal{F} $,
    \begin{align*}
        \Prob*{\bigcup_{i=1}^n A_i}
         & =
        \underbrace{\Prob{A_1}+\Prob{A_2}+\cdots+\Prob{A_n}}_{n\text{ terms}}                              \\
         & \quad\underbrace{-\Prob{A_1\cap A_2}-\cdots-\Prob{A_{n-1}\cap A_n}}_{\binom{n}{2}\text{ terms}} \\
         & \quad\underbrace{+\Prob{A_1\cap A_2\cap A_3}+\cdots}_{\binom{n}{3}\text{ terms}}                \\
         & \quad - \Prob{A_1\cap A_2\cap A_3\cap A_4}-\cdots                                               \\
         & \quad\vdots                                                                                     \\
         & =\sum_{J\subseteq \Set{1,2,\ldots,n},J\ne \emptyset}(-1)^{\abs{J}+1}
        \Prob*{\bigcap_{i\in J}A_i}.
    \end{align*}
\end{Theorem}
\begin{Proposition}{Bonferroni's Inequality}{}
    \[ \Prob{A\cap B}\ge \Prob{A}+\Prob{B}-1. \]
    \tcblower{}
    \textbf{Proof}: Using the inclusion-exclusion theorem,
    we have
    \begin{align*}
        \Prob{A\cap B}
         & =\Prob{A}+\Prob{B}-\Prob{A\cup B}                                             \\
         & \ge \Prob{A}+\Prob{B}-1           &  & \text{ since $ \Prob{A\cup B}\le 1 $}.
    \end{align*}
\end{Proposition}

\makeheading{Lecture 4}{\printdate{2022-05-09}}%chktex 8
\begin{Example}{}{}
    Use the Euclidean algorithm to compute $ (124,348) $.
    \tcblower{}
    \textbf{Solution}: Here are the divisions:
    \begin{align*}
        348 & = 124\cdot 2+100 \\
        124 & =100\cdot 1+24,  \\
        100 & =24\cdot 4+4     \\
        24  & =4\cdot 6+0.
    \end{align*}
    Therefore, $ (124,348)=4 $.

    It's easier to remember this visually by arranging the computations in a table.
    Compare the numbers above to the numbers in the following table:
    \[ \begin{array}{cc}
            r_i & q_{i-1} \\
            \midrule
            348           \\
            124 & 2       \\
            100 & 1       \\
            24  & 4       \\
            4   & 6
        \end{array} \]
    The next remainder is 0, so we didn't write it. The successive remainders go in the
    first column. The successive quotients go in the second column.
\end{Example}
To compute the greatest common divisors of three numbers, just compute the
greatest common divisor of two numbers at a time.
\begin{Example}{}{}
    Compute $ (42,105,91) $.
    \tcblower{}
    \textbf{Solution}: Since $ (42,105)=21 $, so $ (42,105,91)=\bigl((42,105),91\bigr)=(21,91)=7 $.
\end{Example}
\section{Bézout's Identity}
The next result is extremely important, and is often used in proving things
about greatest common divisors. First, We will recall a definition from linear algebra.
\begin{Definition}{}{}
    If $ a $ and $ b $ are numbers, a linear combination of $ a $ and $ b $ (with integer coefficients) is a number of the form
    \[ ax+by,\; x,y\in\mathbf{Z}. \]
    \tcblower{}
    For instance, $ 29=2\cdot 10+1\cdot 9 $ shows that $ 29 $ is a linear combination of $10$ and $9$.
    Further, $ 7=(-2)\cdot 10+3\cdot 9 $ shows that $ 7 $ is a linear combination of $ 10 $ and $ 9 $ as well.
\end{Definition}
\begin{Example}{}{}
    Find the smallest positive integer $ c $ that has the form $ 12x+8y=c $, where $ x,y\in\mathbf{Z} $.
    \tcblower{}
    \textbf{Solution}: We can see that $ 12(1)+8(-1)=4 $. The question is ``can we get a smaller positive integer?''

    Can we find $ x,y\in\mathbf{Z} $ such that $ 12x+8y=3 $? If we could, we would have $ 4(3x+2y)=3 $. Since $ 3x+2y\in\mathbf{Z} $,
    this would imply that $ 4\mid 3 $ which is a contradiction. Using the same argument on $ 12x+8y=2 $ and $ 12x+8y=1 $, we
    see that none of these are possible. Hence, the smallest positive integer is $4$. So, in this case, the smallest positive integer of the form
    $ 12x+8y=c $ is equal to $ (12,8) $.
\end{Example}
\begin{Example}{}{}
    Find the smallest positive integer $ c $ that has the form $ 28x+105y=c $, where $ x,y\in\mathbf{Z} $.
\end{Example}
\begin{Theorem}{Bézout's Identity}{}
    Let $ a,b\in\mathbf{Z} $ (not both zero). If $d$ is
    the least positive integer combination of $a$ and $b$, then $d$ divides every combination
    of $a$ and $b$. Furthermore, $d = (a, b)$.
    \tcblower{}
    \textbf{Proof}: We know that $ ax+by=d>0 $. Now consider some integer combination
    \[ c=as+bt,\; s,t\in\mathbf{Z}. \]
    We want to show that $ d\mid c $. By DA, there exists $ q,r\in\mathbf{Z} $ such that
    \[ c=dq+r,\; 0\le r<d. \]
    Thus,
    \begin{align*}
        0 & \le r            \\
          & =c-dq            \\
          & =as+bt-(ax+by)q  \\
          & =a(s-q)+b(t-yq).
    \end{align*}
    We see that r is an integer combination of $a$ and $b$, which is less than $d$, and
    non-negative. Because $d$ is the least positive integer combination of $a$ and $b$, the
    only option is that $r = 0$. Hence, $ d\mid c $. In particular, $ d\mid a $ and $ d\mid b $.
    So $ d $ is a common divisor of $ a $ and $ b $. We will now show that $ d=(a,b) $. Let $ d^\prime $
    be a common divisor of $ a $ and $ b $. Then, $ d^\prime \mid a $ and $ d^\prime \mid b $. Hence,
    \[ d^\prime \mid ax+by \]
    by property 2 of~\Cref{prop:LEC2_PROP1}. Thus, we have $ d^\prime \mid d $, and by definition of GCD we have
    $ d=(a,b) $.
\end{Theorem}
\begin{Corollary}{}{}
    The set of all linear combinations of integers $a$ and $b$ is the set
    of all multiples of $(a, b)$.
    \tcblower{}
    \textbf{Proof}: On one hand,
    \[ (a,b)\mid ax+by,\, x,y\in\mathbf{Z}. \]
    So every linear combination of $a$ and $b$ is a multiple of $(a, b)$.
    On the other hand,
    \[ (a,b)=ax+by\text{ so }k(a,b)=a(kx)+b(ky), \]
    that is, every multiple of $ (a,b) $ is a linear combination of $ a $ and $ b $.
\end{Corollary}
\begin{Corollary}{}{}
    Two integers $ a $ and $ b $ are relatively prime if and only if $ ax+by=1 $ for some $ x,y\in\mathbf{Z} $.
    \tcblower{}
    \textbf{Proof}: Suppose $ a $ and $ b $ are relatively prime; that is, $ (a,b)=1 $. By Theorem 1 (Bézout's Identity),
    \[ ax+by=(a,b)=1,\;\text{for some }x,y\in\mathbf{Z}. \]
    On the other hand, suppose $ ax+by=1 $ for some $ x,y\in\mathbf{Z} $. Since $ (a,b)\mid a $ and $ (a,b)\mid b $, we have
    \[ (a,b)\mid ax+by=1. \]
    The only positive integer that divides $1$ is $1$. Therefore, $(a, b) = 1$
\end{Corollary}
\begin{Exercise}{}{}
    Prove that if $ n\in\mathbf{Z} $, then $ (3n+17,2n+11)=1 $.
    \tcblower{}
    \textbf{Solution}: $2(3n+17)-3(2n+11)=1$, and $ (2,3)=1 $. Therefore, $(3n+17,2n+11)=1$.
\end{Exercise}
\begin{Proposition}{}{l4_prop1}
    Let $ a,b,c\in\mathbf{Z} $. If $ (a,b)=1 $, $ a\mid c $, and $ b\mid c $,
    then $ ab\mid c $.
    \tcblower{}
    \textbf{Proof}: Since $ a\mid c $ and $ b\mid c $, there exists $ d,f\in\mathbf{Z} $ such that
    \begin{align*}
        c                    & =ad\text{ and }c           =bf \\
        \implies \frac{c}{a} & =d \text{ and }\frac{c}{b} =f.
    \end{align*}
    Further, since $a$ and $b$ are coprime, by Theorem 1 (Bézout's Identity) there exist
    integers $x$ and $y$ such that
    \[ ax+by=1. \]
    Thus,
    \begin{align*}
        acx+byc                   & =c            \\
        \frac{c}{b}x+\frac{c}{a}y & =\frac{c}{ab} \\
        fx+dy                     & =\frac{c}{ab} \\
        ab(fx+dy)                 & =c,
    \end{align*}
    which implies $ ab\mid c $.
\end{Proposition}
\begin{Proposition}{}{l4_prop2}
    Let $ a,b,n\in\mathbf{Z} $. If $ (n,a)=1 $ and $ n\mid ab $, then $ n\mid b $.
    \tcblower{}
    \textbf{Proof}: By Bézout's Identity, there exists $ x,y\in\mathbf{Z} $ such that
    \[ nx+ay=(n,a)=1. \]
    Multiplying by $b$ gives
    \[ nxb+ayb=b. \]
    Since $ n\mid n $ and $ n\mid ab $, we get by property 2 of~\Cref{prop:LEC2_PROP1} that
    \[ n\mid nxb+ayb. \]
    Therefore, $ n\mid b $.
\end{Proposition}
\section{The Extended Euclidean Algorithm}
We will start by reviewing the Euclidean algorithm, in which the extended
Euclidean algorithm is used.
\begin{Example}{}{}
    Find $ (1914,899) $. Further, find $ x,y\in\mathbf{Z} $ such that $ 1914x+899y=29 $.
    \tcblower{}
    \textbf{Solution}: We first follow the Euclidean Algorithm,
    \begin{equation}\label{EEAeqA}
        \begin{aligned}
            1914 & =2\cdot 899+116 \\
            899  & =7\cdot 116+87  \\
            116  & =1\cdot 87+29   \\
            87   & =3\cdot 29+0.
        \end{aligned}
    \end{equation}
    We usually write this in tabular form:
    \[ \begin{array}{cc}
            r_i  & q_{i-1} \\
            \midrule
            1914 & 899     \\
            899  & 2       \\
            116  & 7       \\
            87   & 1       \\
            29   & 3
        \end{array} \]
    So, $ (1914,899)=29 $. We can rewrite the first two equations as:
    \begin{equation}\label{EEAeq1}
        1914-2\cdot 899=116.
    \end{equation}
    \begin{equation}\label{EEAeq2}
        899-7\cdot 116=87.
    \end{equation}
    Substitute~(\ref{EEAeq1}) into~(\ref{EEAeq2}) to get
    \[ 899-7\cdot(1914-2\cdot 899)=87. \]
    \begin{equation}\label{EEAeq3}
        -7\cdot 1914+15\cdot 899=87.
    \end{equation}
    We can now rewrite the third equation of~(\ref{EEAeqA}) as:
    \begin{equation}\label{EEAeq4}
        116-1\cdot 87=29.
    \end{equation}
    Substituting~(\ref{EEAeq1}) and~(\ref{EEAeq3}) into~(\ref{EEAeq4}) gives
    \begin{align*}
        (1914-2\cdot 899)-1\cdot(-7\cdot 1914+15\cdot 899) & =29  \\
        8\cdot 1914 -17\cdot 899                           & =29.
    \end{align*}
    Thus, $x = 8$ and $y = 17$.
\end{Example}
The above procedure is painful to carry out by hand, or even with a basic
calculator. Let's explore a method of calculations, i.e., an algorithm, for solving
the equation
\[ ax+by=(a,b),\;x,y\in\mathbf{Z}. \]
It is called a backward recurrence, and is due to S. P. Glasby. It will look a
little complicated, but you'll see that it's really easy to use in practice.
\begin{Theorem}{}{}
    Let $ a,b\in\mathbf{Z}^+ $ with $ b>a $. Define
    \[ \begin{array}{cc}
            r_1=b & r_2=a \\
            s_1=1 & s_2=0 \\
            t_1=0 & t_2=1
        \end{array} \]
    and sequences as:
    \begin{align*}
        r_{i+1} & =r_{i-1}-q_{i-1}r_i  \\
        s_{i+1} & =s_{i-1}-q_{i-1}s_i  \\
        t_{i+1} & =t_{i-1}-q_{i-1}t_i.
    \end{align*}
    Then, for $ i\in\mathbf{Z}^+ $, we have
    \[ bs_i+at_i=r_i. \]
    In particular, if $ r_n=(a,b) $, then
    \[ bs_n+at_n=(a,b). \]
    \tcblower{}
    \textbf{Proof}: Use induction on $ n $.
\end{Theorem}
\begin{Example}{}{}
    Find $ x,y\in\mathbf{Z} $ such that $ 1914x+899y=(1914,899) $.
    \tcblower{}
    \textbf{Solution}:
    \[ \begin{array}{lllll}
            r_i  & q_{i-1} & s_i & t_i & \text{Check}          \\
            \midrule
            1914 &         & 1   & 0                           \\
            899  & 2       & 0   & 1                           \\
            116  & 7       & 1   & -2  & 1(1914)+(-2)(899)=116 \\
            87   & 1       & -7  & 15  & (-7)(1914)+15(899)=87 \\
            29   & 3       & 8   & -17 & 8(1914)+(-17)(899)=29
        \end{array} \]
    You can fill the columns of $ s_i $ and $ t_i $ for $ i\ge 3 $ with
    \begin{align*}
        \text{next }s & =\text{previous to last }s-(\text{last }q)(\text{last }s), \\
        \text{next }t & =\text{previous to last }t-(\text{last }q)(\text{last }t).
    \end{align*}
\end{Example}
\begin{Exercise}{}{}
    Compute $ (187,102) $ and express it as a linear combination of $ 187 $ and $ 102 $.
    \tcblower{}
    \textbf{Solution}: We first follow the Euclidean Algorithm,
    \[ \begin{array}{lllll}
            r_i & q_{i-1} & s_i & t_i & \text{Check}          \\
            \midrule
            187 &         & 1   & 0                           \\
            102 & 1       & 0   & 1                           \\
            85  & 1       & 1   & -1  & 1(187)+(-1)(102)=85   \\
            17  & 5       & -1  & 2   & (-1)(187)+(2)(102)=17
        \end{array} \]
    Therefore,
    \[ 187\cdot (-1)+102\cdot 2=(187,102)=17. \]
\end{Exercise}
\begin{Exercise}{}{}
    Find the smallest $ c $, and $ x,y $ such that $ c=246x+194y $.
    \tcblower{}
    \textbf{Solution}:
    \[ \begin{array}{lllll}
            r_i & q_{i-1} & s_i & t_i & \text{Check}            \\
            \midrule
            246 &         & 1   & 0                             \\
            194 & 1       & 0   & 1                             \\
            52  & 3       & 1   & -1  & 1(246)+(-1)(194)=52     \\
            38  & 1       & -3  & 4   & (-3)(246)+(4)(194)=38   \\
            14  & 2       & 4   & -5  & (4)(246)+(-5)(194)=14   \\
            10  & 1       & -11 & 14  & (-11)(246)+(14)(194)=10 \\
            4   & 2       & 15  & -19 & (15)(246)+(-19)(194)=4  \\
            2   & 2       & -41 & 52  & (-41)(246)+(52)(194)=2.
        \end{array} \]
    Thus, $ x=-41 $, $ y=52 $, and $ c=2 $.
\end{Exercise}
\begin{Exercise}{}{}
    Find $ x,y\in\mathbf{Z} $ such that $ 126x+91y=(126,91) $.
\end{Exercise}
\makeheading{Lecture 5}{\printdate{2022-05-11}}%chktex 8
\section{Diophantine Equations}
A polynomial equation in several variables in which we are only interested
in integer solutions is called a Diophantine equation. Diophantine equations are
named after the 3rd century mathematician Diophantus of Alexandria who wrote
a series of books called Arithmetica wherein he raised the matter of solving the
equations now named in his honour.
\begin{Example}{}{}
    Let $ a,b,c,n\in\mathbf{Z} $. Some famous Diophantine equations are:
    \begin{itemize}
        \item $ ax+by+c $: Linear Diophantine equation in two variables.
        \item $ x^2+y^2=z^2 $: Pythagorean Triple.
        \item $ x^2-dy^2\pm 1 $, where $ d\in\mathbf{Z}^+ $ is non-square: Pell's Equation.
        \item $ ax^n+by^n=cz^n $, where $ n\in\mathbf{Z} $, $ n\ge 3 $: Fermat type Equation.
    \end{itemize}
\end{Example}
For now, we will just look at linear Diophantine equation in two variables,
\[ ax+by=c, \]
where $ a,b,c $ are fixed integers and $ x,y $ are integer variables. When analyzing equations,
we would like to answer the following questions.
\begin{enumerate}[(1)]
    \item Does a solution exist?
    \item If solutions exist, how many of them exist? (finite, infinite, countably, or uncountably many)
    \item What are the solutions?
    \item Are there any algorithms which generates the solution(s)?
\end{enumerate}
We address the same questions when analyzing Diophantine equations.
\begin{Theorem}{}{}
    Let $ a,b,c\in\mathbf{Z} $. Let $ (x,y) $ be a pair of integers satisfying the Diophantine equation
    \[ ax+by=c. \]
    \begin{enumerate}[(a)]
        \item If $ (a,b)\nmid c $, then no solutions exist for $ ax+by=c $.
        \item If $ (a,b)=d\mid c $, then there are infinitely many solutions of the form
              \begin{align*}
                  x^\prime & =x_0-\frac{b}{d}t, \\
                  y^\prime & =y_0+\frac{a}{d}t,
              \end{align*}
              where the pair $ (x_0,y_0) $ is a particular solution to the equation $ ax+by=c $,
              and $ t\in\mathbf{Z} $.
    \end{enumerate}
    \tcblower{}
    \textbf{Proof}:
    \begin{enumerate}[(a)]
        \item Suppose $ (a,b)\nmid c $. Let the pair $ (x^\prime,y^\prime) $ be solutions of the equation
              $ ax+by=c $; that is, $ ax^\prime+by^\prime=c $. Since $ (a,b)\mid a $ and $ (a,b)\mid b $,
              \[ (a,b)\mid ax+by=c \]
              by property 2 of~\Cref{prop:LEC2_PROP1}, which is a contradiction. Hence, no solution exists.
        \item Suppose $ (a,b)=d\mid c $, then $ c=dk $ for some $ k\in\mathbf{Z} $. By Bézout's Identity,
              there are integers $ m,n $ such that
              \[ am+bn=d=(a,b). \]
              Then,
              \[ amk+bnk=dk=c. \]
              Hence, the pair $(mk, nk)$ is a solution.

              Suppose the pair $ (x_0,y_0) $ is a particular solution. Then,
              \[ a\biggl(x_0-\frac{b}{d}t\biggr)+b\biggl(y_0+\frac{a}{d}t\biggr)=\frac{ab}{d}t-\frac{ab}{d}t+(ax_0+by_0)=0+c=c, \]
              which proves that the pair $ (x_0-\frac{b}{d}t,y_0+\frac{a}{d}t) $ is a solution for every $ t\in\mathbf{Z} $.

              Let $ (x^\prime,y^\prime) $ and $ (x_0,y_0) $ be the pairs such that $ ax^\prime+by^\prime=c $ and $ ax_0+by_0=c $. Hence,
              \begin{align*}
                  a(x_0-x^\prime)                    & =b(y^\prime-y_0)            \\
                  \implies \frac{a(x_0-x^\prime)}{d} & =\frac{b(y^\prime-y_0)}{d}.
              \end{align*}
              Now, $ \frac{b}{d}\mid \frac{a}{d}(x_0-x^\prime) $. However, $ (\frac{a}{d},\frac{b}{d})=1 $ by~\Cref{prop:l3_prop2}. Therefore,
              \[ \frac{b}{d}\mid x_0-x^\prime, \]
              (using~\Cref{prop:l4_prop2}) would imply
              \[ x_0-x^\prime=t \frac{b}{d},\; \text{for some $ t\in\mathbf{Z} $}. \]
              Thus,
              \[ x^\prime=x_0-\frac{b}{d}t. \]
              Substituting $ x_0-x^\prime=t \frac{b}{d} $ into the equation $ a(x_0-x)=b(y-y_0) $, we see that
              \[ y^\prime=y_0+\frac{a}{d}t. \]
    \end{enumerate}
\end{Theorem}
\makeheading{Lecture 6}{\printdate{2022-05-13}}%chktex 8
\begin{Example}{}{}
    Solve the Diophantine equation $ 6x+9y=5 $.
    \tcblower{}
    \textbf{Solution}: Since $ (9,6)=3\nmid 5 $, the equation has no solution.
\end{Example}
\begin{Example}{}{}
    Find all the solutions $ (x,y) $ to the Diophantine equation
    \[ 11x+13y=369. \]
    \tcblower{}
    \textbf{Solution}: Since $ (11,13)=1\mid 369 $, there are infinitely many solutions. It is hard to guess
    the particular solution, so we will use the EEA\@:
    \[ \begin{array}{lllll}
            r_i & q_{i-1} & s_i & t_i & \text{Check}       \\
            \midrule
            13  &         & 1   & 0                        \\
            11  & 1       & 0   & 1                        \\
            2   & 5       & 1   & -1  & (1)(13)+(-1)(11)=2 \\
            1   & 2       & -5  & 6   & (-5)(13)+(6)(11)=1
        \end{array} \]
    \begin{align*}
        (11)(6)+(13)(-5)       & =1    \\
        (11)(2214)+(13)(-1845) & =369.
    \end{align*}
    So, $ (2214,-1845) $ is a particular solution. The general solution is
    \[ x=2214-13t,\quad y=-1845+11t,\; t\in\mathbf{Z}. \]
\end{Example}
\begin{Exercise}{}{}
    Find all solutions of the linear Diophantine equation
    \[ 132x+84y=144. \]
    \tcblower{}
    \textbf{Solution}:
    \[ \begin{array}{lllll}
            r_i & q_{i-1} & s_i & t_i & \text{Check}     \\
            \midrule
            132 &         & 1   & 0                      \\
            84  & 1       & 0   & 1                      \\
            48  & 1       & 1   & -1  & 132(1)+84(-1)=48 \\
            36  & 1       & -1  & 2   & 132(-1)+84(2)=36 \\
            12  & 1       & 2   & -3  & 132(2)+84(-3)=12
        \end{array} \]
    Hence,
    \begin{align*}
        132(2)+84(-3)                 & =12         \\
        132(2\cdot 12)+84(-3\cdot 12) & =12\cdot 12 \\
        132\cdot 24+84\cdot (-36)     & =144.
    \end{align*}
    So, $ (x_0,y_0)=(24,-36) $ is a particular solution. The general solution is:
    \begin{align*}
        x & =x_0-\frac{b}{d}t=24-\frac{84}{12}t=24-7t,                      \\
        y & =y_0-\frac{a}{d}t=-36+\frac{132}{12}t=-36+11t,\; t\in\mathbf{Z}
    \end{align*}
\end{Exercise}
Consider a 3-variable equation
\[ ax+by+cz=d. \]
The equation has solutions if $ (a,b,c)\mid d $. If it has a solution, there will be infinitely many,
determined by two integer parameters.
\begin{Exercise}{}{}
    Find the general solution to the Diophantine equation
    \[ 8x+14y+5z=11. \]
    \tcblower{}
    \textbf{Solution}:
    \[ 2(4x+7y)+5z=11. \]
    Let $ w=4x+7y $, so
    \[ 2w+5z=11. \]
    Now, $ w=-22 $ and $ z=11 $ is a particular solution, so
    \[ w=-22+5s,\quad z=11-2s,\; s\in\mathbf{Z}. \]
    Then,
    \[ 4x+7y=w=-22+5s. \]
    $ x=-44+10s $ and $ y=22-5s $ is a particular solution. The general solution is
    \begin{align*}
        x & =-44+10s+7t \\
        y & =22-5s-4t   \\
        z & =11-2s.
    \end{align*}
\end{Exercise}
\section{Prime Numbers}
\begin{Definition}{}{}
    An integer $ p $ is called prime when $ p\ne 0 $, $ p\ne \pm 1 $, and the only factors that $ p $
    have are $ \pm 1 $ and $ \pm p $.
\end{Definition}
Clearly, $p$ is prime if and only if $p$ is prime. To avoid this double counting
of primes we shall work only with positive primes and to be brief we shall usually
omit the word ``positive.''
\begin{Lemma}{}{}
    Every integer greater than 1 is divisible by at least one prime.
    \tcblower{}
    \textbf{Proof}: Let's use induction. To begin with, the result is true for $ n=2 $ since $ 2 $ is prime.

    Suppose $ 2,3,4,\ldots,k-1 $ is divisible by at least one prime. If $ k $ is prime, it is divisible
    by a prime --- namely itself! If $ k $ is composite, then $ k=ab $, where $ 1<a<k $ and $ 1<b<k $. Since
    $ a $ and $ b $ are among the integers $ 2,3,4,\ldots,k-1 $, each of them is divisible by at least one prime;
    that is, there exists $ p $ such that $ p\mid a $ and $ a\mid k $ implies $ p\mid k $, so $ k $ has a prime factor
    as well. This shows that the result is true for all $ n>1 $ by induction.
\end{Lemma}
And now comes a classic discovery and its proof occur as Proposition 20 in
Book 9 of Euclid's Elements.

\begin{Proposition}{Euclid's Theorem}{}
    There are infinitely many prime numbers.
    \tcblower{}
    \textbf{Proof}: Suppose on the contrary that there are only finitely many primes $ p_1,\ldots,p_n $. Look at
    \[ p_1p_2\cdots p_n+1. \]
    This number is not divisible by any of the primes $ p_1,\ldots,p_n $ because it leaves the remainder of $1$
    when divided by any of them. According to Lemma 1, there exists a new prime number $ q $ such that
    $ q\mid p_1p_2\cdots p_n+1 $, a contradiction. This contradiction implies that there cannot be finitely many primes;
    that is, there are infinitely many.
\end{Proposition}
The special thing about primes is that there is only one way to write an integer
into primes. Ambiguous factoring such as
\[ 30=(6)(5)=(15)(2) \]
do not occur when only primes are involved in the factors. To prove the Unique
factorization, we need what we can only be called the signature property of primes.
\begin{Proposition}{Euclid's Lemma}{euclid_lemma}
    Let $ a,b\in\mathbf{Z} $. If $ p $ is a prime number and $ p\mid ab $,
    then $ p\mid a $ or $ p\mid b $.
    \tcblower{}
    \textbf{Proof}: Assume $ p\mid ab $, then there exists $ n\in\mathbf{Z} $ such that $ pn=ab $.
    Further, assume $ p\nmid a $. Since $ a $ is not a multiple of $ p $ and the only factors of $ p $
    are $ 1 $ and $ p $, we must have $ (a,p)=1 $. So by Bézout's Identity, there exists $ x,y\in\mathbf{Z} $ such that
    \begin{align*}
        ax+py    & =1. \\
        abx+pby  & =b. \\
        pnx+pby  & =b. \\
        p(nx+by) & =b.
    \end{align*}
    Since $ nx+by\in\mathbf{Z} $, $ p\mid b $.
\end{Proposition}
Note that~\Cref{prop:euclid_lemma} fails when p is not a prime. For instance
$ 6\mid 12=(3)(4) $, but $ 6\nmid 3 $ and $ 6\nmid 4 $.
\makeheading{Lecture 5}{\printdate{2022-09-21}}%chktex 8
\begin{Definition}{}{}
    Fix some event $ B\in\mathcal{F} $ with $ \Prob{B}>0 $.
    Define $ \mu\colon \mathcal{F}\to\mathbf{R} $ by
    \[ \mu(A)=\Prob{A\given B}. \]
\end{Definition}
\begin{Theorem}{}{}
    $ \mu $ is a probability measure on $ (\Omega,\mathcal{F}) $.
    Conditional probabilities are a probability measure.
    \tcblower{}
    \textbf{Proof}: We need to check properties (i)--(iii)
    for $ \mu $.
    \begin{enumerate}[(i)]
        \item \begin{align*}
                  \Prob{\Omega}
                   & =\Prob{\Omega\given B}                \\
                   & =\frac{\Prob{\Omega\cap B}}{\Prob{B}} \\
                   & =\frac{\Prob{B}}{\Prob{B}}            \\
                   & =1.
              \end{align*}
        \item $ \forall A\in\mathcal{F} $,
              $ \mu(A)=\dfrac{\Prob{A\cap B}}{\Prob{B}}\ge 0 $.
        \item Suppose $ A_1,A_2,\ldots $ are disjoint events.
              Then,
              \begin{align*}
                  \mu(A_1\cup A_2\cup\cdots)
                   & =\Prob{A_1\cup A_2\cdots\given B}                                 \\
                   & =\frac{\Prob{(A_1\cup A_2\cup\cdots)\cap B}}{\Prob{B}}            \\
                   & =\frac{\Prob{(A_1\cap B)\cup (A_2\cap B)\cup \cdots }}{\Prob{B}}.
              \end{align*}
              Note that for all $ 1\le i<j $, $ (A_i\cap B)\cap (A_j\cap B)=
                  A_i\cap A_j\cap B=\emptyset\cap B=\emptyset $,
              so the events $ (A_1\cap B),(A_2\cap B),\ldots $ are pairwise disjoint.
              Thus, by the countable additivity of $ \mathbb{P} $,
              \begin{align*}
                  \mu(A_1\cup A_2\cup\cdots)
                   & =\frac{\Prob{A_1\cap B}+\Prob{A_2\cap B}+\cdots}{\Prob{B}} \\
                   & =\mu(A_1)+\mu(A_2)+\cdots,
              \end{align*}
              as desired.
    \end{enumerate}
\end{Theorem}
\begin{Remark}{Expected value of Geometric Series}{}
    \begin{align*}
        \frac{\E{X}-(1-p)\E{X}}{p}
         & =\sum_{k=1}^{\infty}k(1-p)^{k-1}-\sum_{j=1}^{\infty}j(1-p)^j                             \\
         & =\sum_{j=0}^{\infty}(j+1)(1-p)^j-\sum_{j=1}^{\infty}j(1-p)^j &  & \text{sum index }j=k+1 \\
         & =1\cdot(1-p)^0+\sum_{j=1}^{\infty}(1-p)^j[(j+1)-j]                                       \\
         & =1+\frac{(1-p)^1}{1-(1-p)}                                                               \\
         & =1+\frac{1-p}{p}                                                                         \\
         & =1+\frac{1}{p}-\frac{p}{p}                                                               \\
         & =\frac{1}{p}.
    \end{align*}
    Therefore, we have
    \begin{align*}
        \E{X}\frac{1-(1-p)}{p} & =\frac{1}{p}  \\
        \E{X}\frac{p}{p}       & =\frac{p}{p}  \\
        \E{X}                  & =\frac{1}{p}.
    \end{align*}
\end{Remark}
\begin{Example}{}{}
    Roll a 6-sided die until we get a 6. Let $ X= $ the number of rolls.
    Let $ B=\Set{\text{all rolls are even numbers}} $.
    What is $ \E{X\given B} $?
    \tcblower{}
    \textbf{Solution}:
    \[ \Prob{B\given X=k}=\biggl(\frac{2}{5}\biggr)^{\!k-1}. \]
    Hence,
    \begin{align*}
        \Prob{B}
         & =\sum_{k=1}^{\infty}\Prob{\Set{X=k}}\Prob{B\given X=k}                                            \\
         & =\sum_{k=1}^{\infty}\biggl(\frac{5}{6}\biggr)^{\!k-1}\frac{1}{6}\biggl(\frac{2}{5}\biggr)^{\!k-1} \\
         & =\frac{1}{6}\sum_{k=1}^{\infty}\biggl(\frac{2}{6}\biggr)^{\!k-1}                                  \\
         & =\frac{1}{6}\cdot\frac{1}{1-1/3}                                                                  \\
         & =\frac{1}{6}\cdot\frac{3}{2}                                                                      \\
         & =\frac{1}{4}.
    \end{align*}
    Hence,
    \begin{align*}
        \E{X\given B}
         & =\sum_{k=1}^{\infty}k\Prob{X=k\given B}                                                                                                               \\
         & =\sum_{k=1}^{\infty}k \frac{\Prob{\Set{X=k}\cap B}}{\Prob{B}}                                                                                         \\
         & =\frac{1}{\Prob{B}}\sum_{k=1}^{\infty}k \Prob{\Set{X=k}}\Prob{B\given X=k}                                                                            \\
         & =\frac{1}{1/4}\sum_{k=1}^{\infty}k \frac{1}{6}\biggl(\frac{1}{3}\biggr)^{\!k-1}                                                                       \\
         & =4\cdot \frac{1}{6}\cdot \frac{3}{2}\underbrace{\sum_{k=1}^{\infty}k\biggl(\frac{1}{3}\biggr)^{\!k-1}\frac{2}{3}}_{\text{EV of $\GEO*{\frac{2}{3}}$}} \\
         & =4\cdot \frac{1}{6}\cdot \frac{3}{2}\cdot \frac{3}{2}                                                                                                 \\
         & =\frac{3}{2}.
    \end{align*}
\end{Example}
\begin{Theorem}{}{}
    $ p\colon S\to\mathbf{R} $ is a PMF for some RV if and only if
    \begin{enumerate}[(i)]
        \item $ p(v)\ge 0 $ for all $ v\in S $, and
        \item $ \sum_{v\in S}p(v)=1 $.
    \end{enumerate}
    \underline{Remark}: This implies that $ \Set{v\in S\given p(s)>0\text{ is countable}} $.
\end{Theorem}
\begin{Exercise}{}{}
    The sum of uncountably infinitely many positive numbers always
    diverges to infinity.
\end{Exercise}
\begin{Theorem}{}{}
    $ f\colon \mathbf{R}\to\mathbf{R} $ is a PDF for some RV if and only if
    \begin{enumerate}[(i)]
        \item $ f(x)\ge 0 $ for all $ x\in\mathbf{R} $, and
        \item $ \int_{-\infty}^{\infty}f(x)\odif{x}=1 $.
    \end{enumerate}
\end{Theorem}
\begin{Example}{}{}
    Let $ U\sim \text{Uniform}[0,1] $. The PDF is
    \[ f_U(t)=\begin{cases}
            1, & 0\le t\le 1,      \\
            0, & \text{otherwise}.
        \end{cases}. \]
    Technically, the derivative doesn't exist at $0$ since there's a
    change of direction, but it doesn't matter since we only integrate
    PDFs.
\end{Example}
\begin{Example}{}{}
    Let $ U \sim \text{Uniform}[0,1/2] $. The PDF is
    \[ f_U(t)=\begin{cases}
            2, & 0\le t\le 1/2,    \\
            0, & \text{otherwise}.
        \end{cases} \]
\end{Example}
\begin{Definition}{}{}
    A standard logistic distribution is defined by the CDF
    \[ F(t)=\frac{1}{1+e^{-t}}. \]
    The PDF is
    \[ f_X(t)=\odv{F}{x}=(-1)\frac{1}{(1+e^{-t})^2}(-e^{-t})=\frac{e^{-t}}{(1+e^{-t})^2}. \]
    The PDF looks like a bell curve, but with heavier tails.
\end{Definition}
\begin{Example}{}{}
    Calculate $ \Prob{\Set{-1\le X\le 1}} $ for the standard logistic distribution.
    \tcblower{}
    \textbf{Solution}:
    \begin{itemize}
        \item \underline{Method 1}:
              \[ \Prob{\Set{-1\le X\le 1}}=F(1)-F(-1). \]
        \item \underline{Method 2}:
              \[  \Prob{\Set{-1\le X\le 1}}=\int_{-1}^{1}f(t)\odif{t}. \]
    \end{itemize}
\end{Example}
\begin{Example}{}{}
    Calculate $ \E{X} $ for the standard logistic distribution.
    \tcblower{}
    \textbf{Solution}:
    \begin{align*}
        \E{X}
         & =\int_{-\infty}^{\infty}t f(t)\odif{t}                        \\
         & =\int_{-\infty}^{\infty}\frac{t e^{-t}}{(1+e^{-t})^2}\odif{t} \\
         & =\text{IBP}.
    \end{align*}
\end{Example}
\makeheading{Lecture 6}{\printdate{2022-09-23}}%chktex 8
\begin{Definition}{}{}
    If $ f $ is a function $ f\colon A\to B $, then
    the \textbf{pre-image} of a set
    $ C\subseteq B $ under $ f $ is
    \[ f^{-1}(C)=\Set{x\in A\given f(x)\in C}. \]
    The \textbf{image} of a set $ D\subseteq A $
    under $ f $ is
    \[ f(D)=\Set{f(x)\given x\in D}. \]
\end{Definition}
\begin{Example}{}{preim}
    If $ f\colon \mathbf{R}\to\mathbf{R} $ is the function
    $ f(x)=x^2 $, then the pre-image
    \begin{align*}
        f^{-1}\bigl([0,4]\bigr)   & =[-2,2].            \\
        f^{-1}\bigl([1,9]\bigr)   & =[-3,-1]\cup [1,3]. \\
        f^{-1}\bigl([-5,-2]\bigr) & =\emptyset.
    \end{align*}
\end{Example}
\begin{Remark}{}{}
    $\forall C,D\subseteq A $,
    \[ f(C\cup D)=f(C)\cup f(D). \]
    It is \underline{not} always the case (for non-injective functions)
    that
    \[ f(C\cap D)=f(C)\cap f(D). \]
    In~\ref{ex:preim}, if we consider $ C=[-2,-1] $ and $ D=[1,2] $, then
    $ C\cap D=\emptyset $, $ f(C\cap D)=\emptyset $, and
    $ f(C)\cap f(D)=[1,4]\cap [1,4]=[1,4] $.
\end{Remark}
\begin{Proposition}{}{}
    \begin{align*}
        f^{-1}(C\cup D)=f^{-1}(C)\cup f^{-1}(D). \\
        f^{-1}(C\cap D)=f^{-1}(C)\cap f^{-1}(D).
    \end{align*}
    \tcblower{}
    \textbf{Proof}: Exercise.
\end{Proposition}
Suppose $ X\colon \Omega\to\mathbf{R} $ is a discrete
random variable and $ Y=g(X) $ for some
$ g\colon\mathbf{R}\to\mathbf{R} $.
How would we find the PMF of $ Y $ using the PMF $ p_X $ of $ X $?
\begin{align*}
    p_Y(9)
     & =\Prob{\Set{Y=9}}                   \\
     & =\Prob{\Set{X\in g^{-1}(\Set{9})}}  \\
     & =\sum_{j\in g^{-1}(\Set{9})}p_X(j).
\end{align*}
In general,
\[ p_Y(k)=\sum_{j\in g^{-1}(\Set{k})}p_X(j). \]
\begin{Theorem}{}{}
    If $ Y=g(X) $ for some random variable $ X $
    and some (measurable) function $ g\colon\mathbf{R}\to\mathbf{R} $,
    then for any set of $ A\subseteq \mathbf{R} $,
    \[ \Prob{\Set{Y\in A}}=\Prob{\Set{X\in g^{-1}(A)}}. \]
\end{Theorem}
\begin{Example}{}{}
    Suppose $ g(x)=\sqrt{x} $, $ X $ is a non-negative random variable,
    and $ Y=\sqrt{X} $. Find the CDF of $ Y $.
    \tcblower{}
    \textbf{Solution}:
    \begin{align*}
        F_Y(v)
         & =\Prob{\Set{Y\le v}}        &  & v\ge 0 \\
         & =\Prob{\Set{\sqrt{X}\le v}}             \\
         & =\Prob{\Set{X\le v^2}}                  \\
         & =F_X(v^2).
    \end{align*}
    $ \sqrt{x} $ was a monotone increasing function, so it
    preserved the inequality.
\end{Example}
\begin{Theorem}{}{}
    Let $ Y=g(X) $.
    \begin{itemize}
        \item If $ g(X) $ is a strictly increasing function, then
              \[ F_Y(v)=F_X(g^{-1}(v)). \]
        \item If $ g(X) $ is a strictly decreasing function, then
              \[ F_Y(v)=1-F_X(g^{-1}(v)). \]
    \end{itemize}
\end{Theorem}
\begin{Example}{}{}
    Let $ X\sim \text{Uniform}[0,1] $.
    \[ f_X(t)=\begin{cases}
            1, & t\in[0,1],    \\
            0, & t\notin[0,1],
        \end{cases}\qquad
        F_X(t)=\begin{cases}
            0, & t<0,       \\
            t, & t\in[0,1], \\
            1, & t>1.
        \end{cases} \]
    If $ Y=-\log{X} $. Note that $ \log{1}=0 $ and $ \lim\limits_{{t} \to {0}}\log{t}=-\infty $.
    Also,
    \[ g(x)=-\log{x}\iff -g(x)=\log{x}\iff x=e^{-g(x)}, \]
    so $ g^{-1}(v)=e^{-v} $.
    For $ v\ge 0 $,
    \begin{align*}
        F_Y(v)
         & =1-F_X(g^{-1}(v)) \\
         & =1-F_X(e^{-v})    \\
         & =1-e^{-v}.
    \end{align*}
    Hence, $ Y $ is a continuous RV with PDF
    \[ f_Y(v)=\odv*{
            \begin{cases}
                0,        & v<0,   \\
                1-e^{-v}, & v\ge 0
            \end{cases}
        }{v}=\begin{cases}
            0,      & v<0,    \\
            e^{-v}, & v\ge 0.
        \end{cases} \]
    Thus, $ Y\sim \EXP{1} $.
\end{Example}
\begin{Definition}{}{}
    The \textbf{quantile function} of a random variable $ X $
    is the right-continuous (almost) left-inverse of the CDF of $ X $,
    \[ Q_X(v)=\inf\Set{t\in\mathbf{R}:F_X(t)> v}. \]
    Hence, if $ F_X $ is strictly increasing at $ t $, then
    \[ Q_X(F_X(t))=t. \]
    $ Q_X(90\%) $ is the $ 90\textsuperscript{th} $ percentile
    of the value of $ X $ --- the value that $ X $ is less than 90\% of the time.
\end{Definition}
\begin{Theorem}{}{}
    If $ U\sim\text{Uniform}[0,1] $ and $ F $ is a continuous CDF that is
    strictly increasing, then
    $ F^{-1}(U) $ is a random variable whose CDF is $ F $.
\end{Theorem}
\begin{Remark}{}{}
    Suppose $ X $ is a continuous random variable, $ g $ is a differentiable and
    strictly increasing. Before, we had $ Y=g(X) $,
    $ F_Y(v)=F_X\circ g^{-1}(v) $, so the PDF of $ Y $ is
    \begin{align*}
        f_Y(v)
         & =\odv*{F_X(g^{-1}(v))}{v}                      \\
         & =f_X(g^{-1}(v))(g^{-1})' (v).                  \\
         & =f_X(g^{-1}(v))\frac{1}{g'(g^{-1}(v))}         \\
         & =\frac{f_X\circ g^{-1}(v)}{g'\circ g^{-1}(v)}.
    \end{align*}
    You can think of it as taking
    the reflection along the line $ y=x $ for $ g $.

    If $ g $ is differentiable and strictly decreasing, then
    \[ f_Y(v)=-\frac{f_X\circ g^{-1}(v)}{g'\circ g^{-1}(v)}. \]

    We can simplify these formulas for \underline{any} differentiable
    function $ g $ (strictly increasing or decreasing) as
    \[ f_Y(v)=\frac{f_X\circ g^{-1}(v)}{\abs{g'\circ g^{-1}(v)}}. \]
\end{Remark}
\begin{Example}{}{}
    What if our function is neither strictly increasing nor decreasing?
    In general,
    \[ f_Y(v)=\sum_{f\in g^{-1}(\Set{v})}\frac{f_X(t)}{\abs{g'(t)}}, \]
    for all
    $ t $ such that $ g(t)=v $.
    We require that $ g $ is differentiable, and $ g $ is
    not constant on any interval.

    Let $ X \sim \text{Uniform}\interval[open right]{0}{2\pi}$, and $ Y=\sin^2(X) $.
    Find $ \Prob{\Set{Y\le t}} $.
\end{Example}
\begin{Definition}{Expectation}{}
    If $ X $ is discrete, then
    \[ \E{X}=\sum_v v\Prob{\Set{X=v}}=\sum_v p_X(v). \]
    If $ X $ is continuous, then
    \[ \E{X}=\int_{-\infty}^{\infty}x f_X(x)\odif{x}. \]
    Furthermore, if $ X $ is discrete then
    \[ \E{g(X)}=\sum_v g(v)p_X(v), \]
    or if $ X $ is continuous then
    \[ \E{g(X)}=\int_{-\infty}^{\infty}g(x)f_X(x)\odif{x}. \]
\end{Definition}
\begin{Definition}{Variance}{}
    The variance (or $ 2\textsuperscript{nd} $
    central moment) of a random variable $ X $ is
    \[ \Var{X}=\E{X^2}=\E{X}^2=\E*{(X-\E{X})^2}. \]
\end{Definition}
\begin{Definition}{Moments}{}
    For an integer $ p\ge 1 $, the $ p\textsuperscript{th} $
    moment of $ X $ is $ \E{X^p} $. The $ p\textsuperscript{th} $
    central moment of $ X $ is
    $ \E*{(X-\E{X})^p} $.
\end{Definition}
\begin{Definition}{Moment Generating Function (MGF)}{}
    The \textbf{moment generating function} (MGF) of a random variable
    $ X $ is the function
    \[ M_X(t)=\E{e^{tX}}. \]
\end{Definition}
\begin{Remark}{}{}
    For each value of $ t $ that we plug in, we're calculating a different
    expected value. Why?
    \begin{enumerate}[(1)]
        \item The MGF uniquely specifies the probability distribution.
        \item Grants easy access to all moments.
        \item Easy to handle sums of independent random variables.
    \end{enumerate}
\end{Remark}
\begin{Theorem}{}{}
    Suppose $ X $ and $ Y $ are random variables and
    their MGFs are both defined (integrals exist)
    in some interval $ (-\delta,\delta) $ for some $ \delta>0 $.
    If $ M_X(t)=M_Y(t) $ for all $ -\delta<t<\delta $, then
    $ X\dist Y $.
\end{Theorem}
\begin{Theorem}{}{}
    Suppose $ X,X_1,X_2\ldots $ all have MGFs that are defined
    on $ (-\delta,\delta) $ for some $ \delta>0 $. If
    $ M_{X_n}(t)\to M_X(t) $ as $ n\to\infty $
    for all $ -\delta<t<\delta $, then
    $ F_{X_n}(x)\to F_X(x) $ as $ n\to\infty $
    for all $ x\in\mathbf{R} $.
\end{Theorem}
\begin{Theorem}{}{mgf_0}
    For $ p\ge 1 $, if the MGF of $ X $ is
    differentiable $ p $ times at $ t=0 $, then
    \[ \E{X^p}=M_X^{(p)}(0). \]
    \tcblower{}
    \textbf{(Rough) Proof}:
    \[ \biggl(\frac{d}{dt}\biggr)^p M_X(t)
        =\biggl(\frac{d}{dt}\biggr)^p \E{e^{tX}}
        \underbrace{=}_{\text{next lecture}}
        \E*{\biggl(\frac{d}{dt}\biggr)^p e^{tX}}=
        \E{X^p e^{tX}}. \]
    At $ t=0 $, this is $ \E{X^p\cdot 1}=\E{X^p} $.
\end{Theorem}
\begin{Example}{}{}
    Let $ G\sim \GEO{p} $. Find the MGF
    of $ G $, and then calculate the first moment.
    \tcblower{}
    \textbf{Solution}: The MGF is given by
    \begin{align*}
        M_G(t)
         & =\E{e^{tG}}                                \\
         & =\sum_{k=1}^{\infty}e^{tk}(1-p)^{k-1}p     \\
         & =\sum_{k=1}^{\infty}(e^{t})^k(1-p)^{k-1}p  \\
         & =p e^t \sum_{k=1}^{\infty}((1-p)e^t)^{k-1} \\
         & =p e^t \frac{1}{1-(1-p)e^t}                \\
         & =\frac{pe^t}{1-(1-p)e^t}                   \\
         & =\frac{p}{e^{-t}-1+p}.
    \end{align*}
    We can calculate the first moment (expected value) as follows:
    \[ M_G'(t)=(-1)\frac{p}{(e^{-t}-1+p)^2}(-e^{-t})
        \implies M_G'(0)=\frac{p}{(1-1+p)^2}(1)=\frac{p}{p^2}=\frac{1}{p}. \]
\end{Example}
\begin{Theorem}{}{}
    If $ X_1,\ldots,X_n $ are jointly independent random variables,
    $ S=X_1+X_2+\cdots+X_n $, and these random variables' MGFs are
    all defined at some value $ t $, then
    \[ M_S(t)=M_{X_1}(t) M_{X_2}(t)\cdots M_{X_n}(t). \]
    \tcblower{}
    \textbf{Proof}: Since $ X_1,\ldots,X_n $ are jointly independent,
    we have
    \[ \E{e^{tS}}=
        \E{e^{t(X_1+\cdots+X_n)}}=\E{e^{tX_1+\cdots+tX_n}}=
        \E{e^{tX_1}\cdots e^{tX_n}}=
        \E{e^{tX_1}}\cdots \E{e^{tX_n}}=
        \prod_{i=1}^n M_{X_i}(t). \]
\end{Theorem}
\begin{Example}{}{}
    Suppose $ I_1,I_2,\ldots $ are a sequence of independent
    and identically distributed (IID) $ \BERN{p} $
    trials $ p_{I_j}(0)=1-p $, $ p_{I_j}(1)=p $. Find the
    MGF of $ I_j $, and then find the MGF of
    $ \BIN{n,p} $.
    \tcblower{}
    \textbf{Solution}: For a single Bernoulli RV,
    \[ M_{I_j}(t)=(1-p)\cdot 1+p\cdot e^{1t}=(1-p)+pe^t. \]
    Now, note that the Binomial RV is the sum of $ n $ IID
    Bernoulli trials, so
    \[ M_S(t)=(1-p+pe^t)^n. \]
\end{Example}
\begin{Example}{}{}
    Suppose $ N\sim \POI{\lambda} $; that is,
    \[ p_N(k)=e^{-\lambda}\frac{\lambda^k}{k!},\; k\ge 0. \]
    Find the MGF of $ N $ and calculate $ \E{N} $ using the MGF\@.
    \tcblower{}
    \textbf{Solution}: The MGF of $ N $ is
    \begin{align*}
        M_N(t)
         & =\sum_{k=0}^{\infty}e^{tk}e^{-\lambda}\frac{\lambda^k}{k!} \\
         & =e^{-\lambda}\sum_{k=0}^{\infty}\frac{(e^t \lambda)^k}{k!} \\
         & =e^{-\lambda}e^{e^{t}\lambda}                              \\
         & =e^{\lambda(e^t-1)}.
    \end{align*}
    Therefore, the expected value is
    \[ M_N'(t)=\lambda e^{\lambda(e^t-1)}e^t
        \implies M_N'(0)=\E{N}=\lambda e^{\lambda(1-1)}e^0=\lambda. \]
\end{Example}
\begin{Proposition}{}{bin_to_poi}
    If $ S_k \sim\BIN{k,\lambda/k} $
    and $ N\sim\POI{\lambda} $, then
    $ M_{S_k}(t)\to M_N(t) $ for all $ t\in\mathbf{R} $.
    \tcblower{}
    \textbf{Proof}: Note that
    \[ \biggl(1+\frac{a}{n}\biggr)^{\!bn}\xrightarrow{n\to\infty} e^{ab}. \]
    Hence,
    \begin{align*}
        M_{S_k}(t)
         & =\biggl(1-\frac{\lambda}{k}+\frac{\lambda}{k}e^t\biggr)^{\!k} \\
         & =\biggl(1-\frac{\lambda(e^t-1)}{k}\biggr)^{\!k}               \\
         & \xrightarrow{k\to\infty}
        e^{\lambda(e^t-1)}=M_N(t),
    \end{align*}
    which is the MGF for $ N $, as desired.
\end{Proposition}
\begin{Proposition}{}{}
    In the same setup as~\Cref{prop:bin_to_poi},
    $ p_{S_k}(j)\to p_N(j) $ as $ k\to\infty $
    for all $ j\ge 0 $.
    \tcblower{}
    \textbf{Proof}:
    \begin{align*}
        \binom{k}{j}\biggl(\frac{\lambda}{k}\biggr)^j
        \biggl(1-\frac{\lambda}{k}\biggr)^{k-j}
         & =\frac{k(k-1)\cdots (k-j+1)}{j!}\frac{\lambda^j}{k^j}
        \underbrace{\biggl(1-\frac{\lambda}{k}\biggr)^k}_{\xrightarrow{k\to\infty} e^{-\lambda}} \underbrace{\biggl(1-\frac{\lambda}{k}\biggr)^{-j}}_{\xrightarrow{k\to\infty} 1} \\
         & \xrightarrow{k\to\infty}\frac{k(k-1)\cdots (k-j+1)}{k^j}\frac{\lambda^j}{j!}e^{-\lambda}(1)                                                                            \\
         & \xrightarrow{k\to\infty}\underbrace{\frac{k}{k^j}\cdot \frac{k-1}{k^j}\cdots
        \frac{k-j+1}{k^j}}_{\xrightarrow{k\to\infty} 1} \frac{\lambda^j}{j!}e^{-\lambda}                                                                                          \\
         & \xrightarrow{k\to\infty}\frac{\lambda^j}{j!}e^{-\lambda}                                                                                                               \\
    \end{align*}
\end{Proposition}
\chapter{Stratified Sampling and Cluster Sampling}
\makeheading{Lecture 6}{\printdate{2022-01-24}}%chktex 8
\section{Stratified Simple Random Sampling}
The survey population is divided into $H$ non-overlapping strata:
\[ U=U_1\cup \cdots \cup U_H, \]
with corresponding break-down of population size as
\[ N=\sum_{h=1}^{H}N_h, \]
where $ N_h $ is the size of stratum $ h $.

For any \emph{stratified sampling} designs, there are two basic features:
\begin{itemize}
      \item A sample $ S_h $ of size $ n_h $ is taken from stratum $ h $ using
            a chosen sampling design, and this is done for every stratum.
      \item The $ H $ stratum samples $ S_h $, $ h=1,2,\ldots,H $ are selected
            independent of each other.
\end{itemize}

The stratum sample sizes $ (n_1,n_2,\ldots,n_H) $ are pre-determined
at the design stage. The total sample size is
\[ n=\sum_{h=1}^{H}n_h. \]
\textbf{Stratified Simple Random Sampling}:

The stratum sample $ S_h $ is selected by SRSWOR, for every stratum $ h=1,2,\ldots,H $.

\textbf{The required sampling frames}:

Complete list of $ N_h $ units in stratum $ h $, for every stratum $ h=1,2,\ldots,H $.

\textbf{Notes}:
\begin{itemize}
      \item The population size $ N $ and the stratum sizes $ N_h $ are
            all known under stratified sampling (as part of the frame information).
      \item Even if a complete list of all $N$ units is available, it does not
            imply that stratified sampling frames are automatically available.
\end{itemize}

\textbf{The stratum weights}:
\[ W_h=\frac{N_h}{N},\; h=1,2,\ldots,H. \]
\[ \sum_{h=1}^{H}N_h=N,\qquad \sum_{h=1}^{H}W_h=1. \]
\textbf{The variables}:

$ (y_{hi},\Vector{x}_{hi}) $: the value of $ (y,\Vector{x}) $ for unit $ i $
in stratum $ h $, $ i=1,2,\ldots,N_h $, $ h=1,2,\ldots,H $.

\textbf{The population (i.e., the census) ``data file''}:
\[ \Set[\big]{(y_{hi},\Vector{x}_{hi})\given i=1,2,\ldots,N_h,\, h=1,2,\ldots,H}. \]
\textbf{The sample data set}:
\[ \Set[\big]{(y_{hi},\Vector{x}_{hi})\given i\in S_h,\, h=1,2,\ldots,H}. \]
\subsection{Population parameters}

\textbf{The stratum population mean and population total}:
\[ \begin{matrix}
            \displaystyle \mu_{yh}=\frac{1}{N_h}\sum_{i=1}^{N_h}y_{hi}, & \text{and} & \displaystyle  T_{yh}=\sum_{i=1}^{N_h}y_{hi}. \\\\
            \displaystyle T_{yh}=N_h\mu_{yh},                           & \text{and} & \displaystyle \mu_{yh}=\frac{T_{yh}}{N_h}.
      \end{matrix} \]
\textbf{The overall population mean and population total}:
\[ \begin{matrix}
            \displaystyle \mu_y=\frac{1}{N}\sum_{h=1}^{H}\sum_{i=1}^{N_h}y_{hi}, & \text{and} & \displaystyle  T_y=\sum_{h=1}^{H}\sum_{i=1}^{N_h}y_{hi}.
      \end{matrix} \]
\textbf{The relations between $ \mu_y $, $ T_y $ and $ \mu_{yh} $, $ T_{yh} $}:
\[ T_y=\sum_{h=1}^{H}T_{yh}=\sum_{h=1}^{H}N_h\mu_{yh}. \]
\[ \mu_y=\sum_{h=1}^{H}W_h\mu_{yh}. \]
\textbf{The stratum population variances}:
\[ \sigma_{yh}^2=\frac{1}{N_h-1}\sum_{i=1}^{N_h}(y_{hi}-\mu_{yh})^2,\; h=1,2,\ldots,H. \]
\textbf{The overall population variance}:
\[ \sigma_y^2=\frac{1}{N-1}\sum_{h=1}^{H}\sum_{i=1}^{N_h}(y_{hi}-\mu_y)^2. \]
\textbf{The relation between $ \sigma_y^2 $ and $ \sigma_{yh}^2 $}:
\begin{align*}
      \sigma_y^2             & \approx \sum_{h=1}^{H}W_h \sigma_{yh}^2+\sum_{h=1}^{H}W_h(\mu_{yh}-\mu_y)^2. \\
      \text{Total variation} & =\text{Variation within stratum} + \text{Variation between strata}.
\end{align*}
\begin{align*}
      (N-1)\sigma_y^2
                 & =\sum_{h=1}^{H}\sum_{i=1}^{N_h}(y_{hi}-\mu_y)^2                                                      \\
                 & =\sum_{h=1}^{H}\sum_{i=1}^{N_h}\bigl((y_{hi}-\mu_{yh})+(\mu_{yh}-\mu_y)\bigr)^{\!2}                  \\
                 & =\sum_{h=1}^{H}\sum_{i=1}^{N_h}(y_{hi}-\mu_{yh})^2+\sum_{h=1}^{H}\sum_{i=1}^{N_h}(\mu_{yh}-\mu_y)^2+
      2\sum_{h=1}^{H}\sum_{i=1}^{N_h}(y_{hi}-\mu_{yh})(\mu_{yh}-\mu_y)                                                  \\
                 & =\sum_{h=1}^{H}(N_h-1)\sigma_{yh}^2+\sum_{h=1}^{H}N_h(\mu_{yh}-\mu_y)^2                              \\
      \sigma_y^2 & =\sum_{h=1}^{H}\frac{N_h-1}{N-1}\sigma_{yh}^2+\sum_{h=1}^{H}\frac{N_h}{N-1}(\mu_{yh}-\mu_y)^2,
\end{align*}
where
\[ W_h=\frac{N_h}{N},\qquad \frac{N_h-1}{N-1}\approx W_h,\qquad \frac{N_h}{N-1}\approx W_h. \]
\subsection{Sample data and summary statistics}

Let's focus on the study variable $y$. The sample data under stratified
sampling are given by
\[ \Set{y_{hi}, i\in S_h,\, h=1,2,\ldots,H}. \]
The \textbf{stratum sample mean} and the \textbf{stratum sample variance} are
defined as
\[ \bar{y}_h=\frac{1}{n_h}\sum_{i\in S_h}y_{hi},\qquad s_{yh}^2=\frac{1}{n_h-1}\sum_{i\in S_h}(y_{hi}-\bar{y}_h)^2, \]
where $ n_h $ is the stratum sample size.

The overall sample mean
\[ \bar{y}=\frac{1}{n}\sum_{h=1}^{H}\sum_{i\in S_h}y_{hi} \]
is not a useful statistic (generally a biased estimator for $ \mu_y $).

\subsection{Estimation of the overall population mean \texorpdfstring{$\mu_y$}{μy}}

In general, the overall population mean $ \mu_y=\sum_{h=1}^{H}W_h \mu_{yh} $ can be estimated by
\[ \hat{\mu}_y=\sum_{h=1}^{H}W_h\hat{\mu}_{yh}, \]
where $ \hat{\mu}_{yh} $ is an estimator of $ \mu_{yh} $ using the data from the $ h\textsuperscript{th} $
stratum.

\textbf{Three general properties of $ \hat{\mu}_{y} $ under any stratified sampling
      designs}:
\begin{enumerate}
      \item $ \E{\hat{\mu}_y}=\sum_{h=1}^{H}W_h \E{\hat{\mu}_{yh}} $.
      \item $ \V{\hat{\mu}_y}=\sum_{h=1}^{H}W_h^2\V{\hat{\mu}_{yh}} $.
      \item $ \v{\hat{\mu}_y}=\sum_{h=1}^{H}W_h^2\v{\hat{\mu}_{yh}} $.
\end{enumerate}
\textbf{Estimation of $ \mu_y $ under stratified simple random sampling}:
\[ \bar{y}_{\st}=\sum_{h=1}^{H}W_h\bar{y}_h. \]
This is called the stratified sample mean estimator.

Under \textbf{stratified simple random sampling},
\begin{itemize}
      \item The stratum weights $ W_h $, $ h=1,\ldots,H $ are known constants.
      \item $ \bar{y}_h $, $ h=1,\ldots,H $ are independent.
      \item $ \E{\bar{y}_h}=\mu_{yh} $.
      \item $ \E{s_{yh}^2}=\sigma_{yh}^2 $.
      \item $ \V{\bar{y}_h}=\bigl(1-\frac{n_h}{N_h}\bigr)\frac{\sigma_{yh}^2}{n_h} $.
      \item $ \v{\bar{y}_h}=\bigl(1-\frac{n_h}{N_h}\bigr)\frac{s_{yh}^2}{n_h} $.
\end{itemize}
\textbf{Three main properties of $ \bar{y}_{\st} $ under stratified simple random
      sampling}:
\begin{enumerate}[(a)]
      \item $ \E{\bar{y}_{\st}}=\sum_{h=1}^{H}W_h\E{\bar{y}_h}=\sum_{h=1}^{H}W_h \mu_{yh}=\mu_y $.
      \item $ \V{\bar{y}_{\st}}=\sum_{h=1}^{H}W_h^2\V{\bar{y}_h}=\sum_{h=1}^{H}W_h^2\bigl(1-\frac{n_h}{N_h}\bigr)\frac{\sigma_{yh}^2}{n_h} $.
      \item $ \v{\bar{y}_{\st}}=\sum_{h=1}^{H}W_h^2\bigl(1-\frac{n_h}{N_h}\bigr)\frac{s_{yh}^2}{n_h} $.
\end{enumerate}
\textbf{Homework}: Show that the overall sample mean
\[ \bar{y}=\frac{1}{n}\sum_{h=1}^{H}\sum_{i\in S_h}y_{hi} \]
is not an unbiased estimator of $ \mu_y $ under stratified simple random sampling
unless
\[ \frac{n_h}{n}=W_h,\; h=1,\ldots,H. \]
(This is called the so-called proportional sample size allocation)
\[ \bar{y}=\frac{1}{n}\sum_{h=1}^{H}n_h\bar{y}_h. \]
\subsection{Justifications for using stratified sampling}
\begin{itemize}
      \item \emph{Administrative convenience}. A survey at the national level can
            be organized more conveniently if each province surveys the
            allocated portion of the sample independently. In this case the
            provinces would be a natural choice for stratification.
      \item \emph{Estimation of subpopulation parameters}. Large surveys often
            have multiple objectives. In addition to estimates for the entire
            population, estimates for certain subpopulations could also be
            required.
      \item \emph{Efficiency considerations}. With suitable stratification and
            reasonable sample size allocation, stratified sampling leads to
            more efficient statistical inference.
      \item \emph{More balanced or controlled samples}. Stratified sampling can
            protect from possible disproportionate samples under probability
            sampling among subpopulations
\end{itemize}
\makeheading{Lecture 7}{\printdate{2022-01-26}}%chktex 8
\section{Sample Size Allocation Under Stratified Sampling}

Sample size allocations need to be addressed at the survey design
stage. There are practical constraints on sample size allocations.

We consider three theoretical questions on sample size allocations:

\begin{itemize}
      \item For a given overall sample size $n$, how to find the ``optimal
            allocation'' $ (n_1,n_2,\ldots,n_H) $?
      \item For a total cost $C$ and cost per unit, how to find the ``optimal
            allocation?''
      \item For a pre-specified requirement on variance of the estimators,
            how to find the ``optimal allocation?''
\end{itemize}
Sample size allocations can be complicated by the use of complex
survey sampling methods within each of the strata and more advanced
inferential problems.

We focus on stratified simple random sampling and the estimation of
the population mean $ \mu_y $.

\subsection{Proportional allocation}

The overall sample size $n$ has already been decided. The question is
how to choose $ n_h $ such that $ \sum_{h=1}^{H}n_h=n $.

The \textbf{proportional allocation} method chooses $ n_h\propto N_h $
under the constraint $ \sum_{h=1}^{H}n_h=n $.
\[ n_h=c N_h,\; h=1,\ldots,H. \]
\[ n=\sum_{h=1}^{H}c N_h=c N. \]
\[ c=\frac{n}{N},\qquad n_h=\frac{n}{N}N_h. \]
The allocation methods leads to
\[ n_h=\frac{n}{N}N_h=n W_h,\; h=1,\ldots,H. \]
Under stratified simple random sampling with proportional allocation:
\begin{itemize}
      \item The point estimator $ \bar{y}_{\st} $ remains unbiased for $ \mu_y $.
      \item The theoretical variance formula $ \V{\bar{y}_{\st}} $ reduces to
            \[ \Vsp{\bar{y}_{\st}}{\text{prop}}=\biggl(1-\frac{n}{N}\biggr)\frac{1}{n}\sum_{h=1}^{H}W_h \sigma_{yh}^2. \]
            \[ \V{\bar{y}_{\st}}=\sum_{h=1}^{H}W_h^2\biggl(1-\frac{n_h}{N_h}\biggr)\frac{\sigma_{yh}^2}{n_h}. \]
            \[ n_h=n W_h=n \frac{N_h}{N}\implies \frac{n_h}{N_h}=\frac{n}{N}. \]
            \[ W_h\cdot \frac{1}{n_h}=\frac{1}{n}. \]
            \[ W_h\biggl(1-\frac{n_h}{N_h}\biggr)\frac{1}{n_h}=\biggl(1-\frac{n}{N}\biggr)\frac{1}{n}. \]
\end{itemize}
\textbf{A comparison between $ \bar{y} $ under SRSWOR and $ \bar{y}_{\st} $ under stratified
      simple random sampling with proportional allocation, with the
      same overall sample size $n$}:
\begin{itemize}
      \item Point estimators: Both $ \bar{y} $ and $ \bar{y}_{\st} $ are unbiased for $ \mu_y $ under
            the respective sampling design.
      \item The two variances satisfy
            \[ \V{\bar{y}}-\Vsp{\bar{y}_{\st}}{\text{prop}}\approx \biggl(1-\frac{n}{N}\biggr)\frac{1}{n}\sum_{h=1}^{H}W_h(\mu_{yh}-\mu_y)^2. \]
            \[ \V{\bar{y}}=\biggl(1-\frac{n}{N}\biggr)\frac{1}{n}\sigma_y^2. \]
            \[ \Vsp{\bar{y}_{\st}}{\text{prop}}=\biggl(1-\frac{n}{N}\biggr)\frac{1}{n}\sum_{h=1}^{H}W_h \sigma_{yh}^2. \]
            \[ \sigma_y^2\approx \sum_{h=1}^{H}W_h \sigma_{yh}^2+\sum_{h=1}^{H}W_h(\mu_{yh}-\mu_y)^2. \]
\end{itemize}
\textbf{Two important implications}:
\begin{itemize}
      \item The stratified simple random sampling design under proportional
            sample size allocation always provides more efficient estimate of
            the population mean than SRSWOR\@.
      \item The gain of efficiency under stratified sampling is larger when
            units within each stratum are more homogeneous, or
            equivalently, the units from different strata are more
            heterogeneous so that the between strata variation is large.
\end{itemize}
\textbf{Example}. $ N=10 $; $ \Set{y_1,\ldots,y_{10}}=\Set{0,1,0,0,1,1,1,0,0,0} $.
\[ \mu_y=\frac{4}{10},\qquad \sigma_y^2=\frac{N}{N-1}P(1-P)=\frac{10}{9}\frac{4}{10}\frac{6}{10}. \]
Take a sample with $ n=4 $:
\begin{enumerate}[(a)]
      \item SRSWOR\@: $ \bar{y} $, $ \E{\bar{y}}=\mu_y $, $ \V{\bar{y}}=\cdots $.
      \item Stratified sampling: $ N_1=6 $, $ N_2=4 $.
            \[ \underbrace{\Set{0,0,0,0,0,0}}_{n_1=2},\qquad \underbrace{\Set{1,1,1,1}}_{n_2=2}. \]
            \[ \bar{y}_{\st}=W_1\bar{y}_1+W_2\bar{y}_2=\frac{6}{10}\times 0+\frac{4}{10}\times 1=\frac{4}{10}=\mu_y. \]
\end{enumerate}
\subsection{Neyman allocation}

The overall sample size $n$ is fixed. Find the optimal allocation
$ (n_1,\ldots,n_H) $ such that $ \V{\bar{y}_{\st}} $ is minimized subject to the constraint $ \sum_{h=1}^{H}n_h=n $.

This is called the \emph{Neyman allocation} (Neyman, 1934). The solution is given by
\[ n_h\propto W_h\sigma_{yh},\; h=1,2,\ldots,H. \]
The constraint $ \sum_{h=1}^{H}n_h=n $ leads to the allocation formula
\[ n_h=n \frac{W_h \sigma_{yh}}{\sum_{k=1}^{H}W_k \sigma_{yk}}=n \frac{N_h \sigma_{yh}}{\sum_{k=1}^{H}N_k \sigma_{yk}},\; h=1,2,\ldots,H. \]
\[ n_h=c W_h \sigma_{yh},\; h=1,\ldots,H. \]
\[ n=\sum_{h=1}^{H}n_h=c \sum_{h=1}^{H}W_h \sigma_{yh}. \]
\[ c=\frac{n}{\sum_{h=1}^{H}W_h \sigma_{yh}}. \]
\begin{align*}
      \V{\bar{y}_{\st}}
       & =\sum_{h=1}^{H}W_h^2\biggl(1-\frac{n_h}{N_h}\biggr)\frac{\sigma_{yh}^2}{n_h} \\
       & =\sum_{h=1}^{H}W_h^2\biggl(\frac{1}{n_h}-\frac{1}{N_h}\biggr)\sigma_{yh}^2.
\end{align*}
\[ L(n_1,\ldots,n_H)=\V{\bar{y}_{\st}}+\lambda\biggl(\sum_{h=1}^{H}n_h-n\biggr). \]
\[ 0=\pdv{L}{n_h}=-\frac{1}{n_h^2}W_h^2\sigma_{yh}^2+\lambda. \]
\[ n_h^2=\frac{1}{\lambda}W_h^2\sigma_{yh}^2, \]
that is, $ n_h\propto W_h \sigma_{yh} $.

\textbf{The theoretical variance $ \V{\bar{y}_{\st}} $ under Neyman allocation reduces
      to}
\[ \Vsp{\bar{y}_{\st}}{\text{neym}}=\frac{1}{n}\biggl(\sum_{h=1}^{H}W_h \sigma_{yh}\biggr)^{\!2}-\frac{1}{N}\sum_{h=1}^{H}W_h \sigma_{yh}^2. \]
(Details can be skipped)

Two major implications of Neyman allocation:
\[ n_h\propto W_h\sigma_{yh},\; h=1,2,\ldots,H. \]
\begin{itemize}
      \item Under Neyman allocation, population strata with bigger size $ N_h $
            or bigger variation (i.e., bigger $ \sigma_{yh}^2 $) or both should be assigned to
            a bigger sample size $ n_h $.
      \item If all strata have similar variation, i.e., similar values of $ \sigma_{yh}^2 $,
            Neyman allocation reduces to $ n_h\propto W_h $, which is proportional allocation.
\end{itemize}

\subsection{Optimal allocation with pre-specified cost or variance}
The total direct cost for the overall sample is $ C_1 $. Cost for sampling
one unit in stratum $h$ is $ c_h $. The cost constraint for allocation
$ (n_1,\ldots,n_H) $:
\[ C_1=\sum_{h=1}^{H}c_h n_h. \]
The variance formula $ \V{\bar{y}_{\st}} $ can be re-written as
\[ \V{\bar{y}_{\st}}=\sum_{h=1}^{H}W_h^2 \frac{\sigma_{yh}^2}{n_h}-\sum_{h=1}^{H}W_h^2 \frac{\sigma_{yh}^2}{N_h}. \]
The variance constraint for allocation $ (n_1,\ldots,n_H) $:
\[ V_1=\sum_{h=1}^{H}W_h^2 \frac{\sigma_{yh}^2}{n_h}. \]
Under both allocation constraints, the overall sample size $n$ depends
on $C_1$ and $V_1$.

\textbf{Two optimal allocation methods}: Find $ (n_1,\ldots,n_H) $ to
\begin{itemize}
      \item Minimize $ V_1 $ with a pre-specified $ C_1 $.
      \item Minimize $ C_1 $ with a pre-specified $ V_1 $.
\end{itemize}
\textbf{The solution to (either) the optimal allocation}:
\[ n_h\propto W_h \sigma_{yh}/\sqrt{c_h},\; h=1,2,\ldots,H. \]
The formulas for calculating the $ n_h $ are given by
\[ n_h=n \frac{W_h\sigma_{yh}/\sqrt{c_h}}{\sum_{k=1}^{H}W_k \sigma_{yk}/\sqrt{c_k}},\; h=1,2,\ldots,H, \]
where the overall sample size $ n $ is determined by the pre-specified $ C_1 $ or $ V_1 $.

The Cauchy-Schwarz inequality:
\begin{align*}
      \bigl(\E{XY}\bigr)^{\!2}                  & \le \E{X^2}\E{Y^2}.                               \\
      \biggl(\sum_{i=1}^{n}x_i y_i\biggr)^{\!2} & \le \sum_{i=1}^{n}x_i^2\cdot \sum_{i=1}^{n}y_i^2.
\end{align*}
The equality holds iff $ y_i=a x_i $ for all $ i $.

Consider
\begin{align*}
      V_1 C_1 & =\biggl(\sum_{h=1}^{H}W_h^2 \sigma_{yh}^2 \frac{1}{n_h}\biggr)\cdot \biggl(\sum_{h=1}^{H}c_h n_h\biggr) \\
              & \ge \biggl(\sum_{h=1}^{H}W_h \sigma_{yh}\frac{1}{\sqrt{n}_h}\cdot \sqrt{c_h}\sqrt{n_h}\biggr)^{\!2},
\end{align*}
where we note that the RHS does not involve $ n_h $.

The minimum of $ V_1 C_1 $ is achieved when
\[ W_h \sigma_{yh}\frac{1}{\sqrt{n_h}}\propto \sqrt{c_h}\sqrt{n_h}
      \implies n_h\propto W_h\sigma_{yh}/\sqrt{c_h}. \]

\textbf{Implications of the two optimal allocation methods}:
\[ n_h\propto W_h\sigma_{yh}/\sqrt{c_h},\; h=1,2,\ldots,H. \]
\begin{enumerate}
      \item With unequal costs for different strata, the more expensive
            stratum should be assigned a smaller sample size.
      \item With equal cost for all strata, the two versions of optimal
            allocation both reduce to Neyman allocation, and hence the
            stratum sample size $ n_h $ is decided by the stratum population size
            $ N_h $ and the stratum variance $ \sigma_{yh}^2 $.
      \item With equal or nearly equal cost and no information on stratum
            variations, proportional allocation would be the natural choice
            for sample size allocations.
\end{enumerate}

\section{Post-stratification}
 (Problem 3.7 in the textbook)
\begin{itemize}
      \item Stratified sampling cannot be implemented if the sampling
            frames are not available, such as stratification by the gender and
            age groups for a large human population.
      \item Stratum membership can be determined relatively easily for all
            units in the sample once the sample is selected.
      \item Post-stratification: Divide a ``non-stratified'' sample into
            subsamples by the stratum membership, and construct a stratified
            estimator from the non-stratified sample data set, assuming the
            stratum weights $ W_h $, $ h=1,\ldots,H $ are known.
\end{itemize}
Let $ \Set{y_i,i\in S} $ be the survey data set and $ S $ is a sample of size $ n $
selected by SRSWOR\@. The sample $ S $ can be post-stratified as
\[ S=S_1\cup \cdots \cup S_H, \]
with corresponding breakdown of $ n $ as $ n=n_1+\cdots+n_H $.

The post-stratified estimator of $ \mu_y $ is computed as
\[ \bar{y}_{\post}=\sum_{h=1}^{H}W_h\bar{y}_h. \]
The key differences between $ \bar{y}_{\post} $ and $ \bar{y}_{\st} $:
\begin{itemize}
      \item Under stratified sampling, the stratum sample sizes $ n_h $ are
            decided at the survey design stage and are fixed.
      \item Under post-stratification, the stratum sample sizes $ n_h $ are random
            numbers. The technical arguments for $ \bar{y}_{\st} $ cannot be used directly
            for $ \bar{y}_{\post} $.
\end{itemize}
\textbf{Homework for STAT 854}: Argue that the post-stratified estimator
$ \bar{y}_{\post} $ is usually more efficient than $ \bar{y} $ under SRSWOR\@.
(Hint: Need to go through Problem 3.7)
\makeheading{Lecture 12}{\printdate{2022-05-30}}%chktex 8
\section{Linear Congruences}
\begin{Definition}{}{}
    An equation of the form
    \[ a_1x_1+a_2x_2+\cdots+a_k x_k\equiv b\Mod{n} \]
    with unknowns $ x_1,x_2,\ldots,x_k $ is a linear congruence equation in $ k $ variables.
\end{Definition}
Observe that by definition of mod, we can rewrite a linear congruence equation as
\[ a_1x_1+a_2x_2+\cdots+a_k x_k-nx_{k+1}=b, \]
which is a Diophantine equation in $ k+1 $ variables.

Observe that a linear congruence equation either has no solution or infinitely many
solutions. Indeed, if $ x_i=s_i $, $ 1\le i\le k $ is solutions of the form
\[ a_1x_1+a_2x_2+\cdots+a_k x_k\equiv b\Mod{n}, \]
then
\[ x_i=s_i+qn,\; 1\le i\le k \]
is also a solution for all $ q\in\mathbf{Z} $. This implies that the corresponding Diophantine equation also either has no
solution or infinitely many solutions.

\underline{Remark}: When writing the solutions of a linear congruence equation $ ax\equiv b\Mod{n} $,
we typically either write all solutions in the form
\[ x\equiv s\Mod{n} \]
or we say that $ s $ is the unique solution modulo $ n $.

\begin{Theorem}{}{}
    Let $ a,b\in\mathbf{Z} $ and $ n\in\mathbf{Z}^+ $. Let $ (a,n)=d $ and consider the linear congruence
    \[ ax\equiv b\Mod{n}. \]
    If $ d\mid b $, then the linear congruence has no solution. If $ d\mid b $, then the linear
    congruence has exactly $ d $ distinct solutions modulo $ n $.
    \tcblower{}
    \textbf{Proof}: Solving the congruence $ ax\equiv b\Mod{n} $ is equivalent to solving the
    linear Diophantine equation $ ax+ny=b $ for some $ y $. If $ d\nmid b $, then the Diophantine
    equation has no solution, so the congruence has no solution either.
    If $ d\mid b $, then by Theorem 1 (Lecture 5), the solution of the Diophantine equation take the form
    \[ x=x_0-\frac{n}{d}t,\; y=y_0+\frac{a}{d}t, \]
    where $ (x_0,y_0) $ is any particular solution (obtained from the Euclidean algorithm,
    for instance).

    We need to show that of these infinitely many solutions, there are exactly $d$
    distinct solutions mod $n$. Suppose two solutions of this form are congruent mod $ n $, that is,
    \[ x_0-\frac{n}{d}t_1\equiv x_0-\frac{n}{d}t_2\Mod{n}. \]
    Then,
    \[ \frac{n}{d}t_1\equiv \frac{n}{d}t_2\Mod{n}. \]
    Now, $ \bigl(\frac{n}{d},n\bigr)=\frac{n}{d} $, so by Proposition 3 (Lecture 10), we can divide this congruence by $ \frac{n}{d} $ to obtain
    \[ t_1\equiv t_2\Mod{d}. \]
    Likewise, suppose $ t_1\equiv t_2\Mod{d} $. This means that $ t_1 $ and $ t_2 $ differ by a multiple of $ d $, that is,
    \[ t_1-t_2=kd. \]
    So,
    \[ \frac{n}{d}t_1-\frac{n}{d}t_2=\frac{n}{d}kd=nk. \]
    This implies that
    \[ \frac{n}{d}t_1\equiv \frac{n}{d}t_2\Mod{n}. \]
    By Corollary 1 (Lecture 10),
    \[ x_0-\frac{n}{d}t_1\equiv x_0-\frac{n}{d}t_2\Mod{n}. \]
    We have proven that two solutions of the above form are equal mod n if and only
    if their parameter values are equal mod d, that is, If we let t range over a complete
    system of residues mod d, then $x_0+\frac{n}{d}t$ ranges over all possible solutions mod n.
    To be very specific, all the solutions mod $n$ are given by
    \[ x_0+\frac{n}{d}t\Mod{n},\; t=0,1,2,\ldots,d-1. \]
\end{Theorem}
\begin{Corollary}{}{}
    Let $ a,b\in\mathbf{Z} $ and $ n\in\mathbf{Z}^+ $. If $ (a,n)=1 $, then the equation
    \[ ax\equiv b\Mod{n} \]
    has a solution. Moreover, the unique solution modulo $ n $ is
    \[ x\equiv a^{-1}b\Mod{n}. \]
\end{Corollary}
\begin{Example}{}{}
    Solve $ 6x\equiv 7\Mod{8} $.
    \tcblower{}
    \textbf{Solution}: Since $ (6,8)=2\mid 7 $, there are no solutions.
\end{Example}
\begin{Example}{}{}
    Solve $ 3x\equiv 7\Mod{4} $.
    \tcblower{}
    \textbf{Solution}: Since $ (3,4)=1\mid 7 $, there is exactly one solution modulo $ 7 $. We have $ 3x+4y=7 $ for some $ y\in\mathbf{Z} $.
    By the EEA, we have
    \[ \begin{array}{lllll}
            r_i & q_{i-1} & s_i & t_i & \text{Check}          \\
            \midrule
            4   &         & 1   & 0                           \\
            3   & 1       & 0   & 1                           \\
            1   & 3       & 1   & -1  & 4\cdot 1+3\cdot(-1)=1
        \end{array} \]
    So, $ 4\cdot 7+3\cdot (-7)=7 $. Thus, $ x_0=-7 $, $ y_0=7 $ is a particular solution. So the general solution is:
    \[ x=-7-4t,\; y=7+3t. \]
    The $ y $ equation is irrelevant, and the $ x $ equation says
    \[ x\equiv 1\Mod{4}. \]
\end{Example}
\begin{Example}{}{}
    Find all solutions of $ 7x\equiv 5\Mod{39} $.
    \tcblower{}
    \textbf{Solution}: Since $ (7,5)=1\mid 39 $, there is exactly one solution modulo $ 39 $. We have
    $ 7x+39y=5 $ for some $ y\in\mathbf{Z} $. By the EEA, we have
    \[ \begin{array}{lllll}
            r_i & q_{i-1} & s_i & t_i & \text{Check}   \\
            \midrule
            39  &         & 1   & 0                    \\
            7   & 5       & 0   & 1                    \\
            4   & 1       & 1   & -5  & 39(1)+7(-5)=4  \\
            3   & 1       & -1  & 6   & 39(-1)+7(6)=3  \\
            1   & 3       & 2   & -11 & 39(2)+7(-11)=1 \\
        \end{array} \]
    So,
    \begin{align*}
        7(-11)+39(2)               & =1        \\
        7(-11\cdot 5)+39(2\cdot 5) & =1\cdot 5 \\
        7(-55)+39(10)=5.
    \end{align*}
    Thus, $ x_0=-55 $, $ y_0=10 $ is a particular solution.
    The general solution is:
    \[ x\equiv x_0+\frac{n}{d}(0)\equiv -55\equiv 23\Mod{39}.  \]
\end{Example}
\section{Linear Equations in \texorpdfstring{$ \mathbf{Z}_n $}{Zn}}
Let $ n $ be a modulus. We will now turn our attention to equations
in $ \mathbf{Z}_n $. Let $ a,b\in\mathbf{Z} $, and consider
\[ [a][x]=[b] \]
in $ \mathbf{Z}_n $ where $ x\in\mathbf{Z} $ is unknown.
\begin{Example}{}{}
    The linear equation $ [2][x]=[3] $ has only one solution in $ \mathbf{Z}_9 $,
    namely $ [x]=[6] $.
\end{Example}
\begin{Example}{}{}
    The equation $ [3][x]=[7] $ has no solution in $ \mathbf{Z}_9 $.
\end{Example}
\begin{Example}{}{}
    The linear equation $ [3][x]=[6] $ has three solutions in $ \mathbf{Z}_9 $, namely
    $ [x]=[2] $, $ [x]=[5] $, and $ [x]=[8] $.
\end{Example}
Note: From Example 6, we see the principal difference between the linear
equations in $ \mathbf{Z}_n $ and the linear equation $cx = d$ in $ \mathbf{Z} $. The only way $cx = d$ can
have more than one solution is if $c = d = 0$.

It turns out that the tools that we have in our hands right now can help us to
solve the linear congruence easily. Observe that
\begin{align*}
    [a][x] & =[b]  \\
    [ax]   & =[b],
\end{align*}
and this is the same as solving the linear congruence
\[ ax\equiv b\Mod{n}. \]
\begin{Proposition}{}{}
    Let $ a,b\in\mathbf{Z} $, $ n\in\mathbf{Z}^+ $ and $ a\ne 0 $. The linear equation
    $ [a][x]=[b] $ in $ \mathbf{Z}_n $ if $ d=(a,n)\mid b $ and total no.\ of residue classes
    satisfying $ [a][x]=[b] $ in $ \mathbf{Z}_n $ is equal to $ d=(a,n) $.
\end{Proposition}
\begin{Example}{}{}
    Solve $ [440][x]=[80] $ in $ \mathbf{Z}_{300} $.
    \tcblower{}
    \textbf{Solution}: By the EEA, the general solution to $ 440x+300y=80 $ is
    \[ x=-8-15t,\; y=12+22t,\; t\in\mathbf{Z}. \]
    By evaluating $ -8-15t $ at $ t=0,1,\ldots,19 $, we obtain $ 20 $ distinct solutions in $ \mathbf{Z}_{300} $.
\end{Example}
\begin{Proposition}{}{}
    Let $ a,b\in\mathbf{Z} $, $ n\in\mathbf{Z}^+ $, and $ (a,n)=1 $. The linear equation
    $ [a][x]=[b] $ has a unique solution in $ \mathbf{Z}_n $.
\end{Proposition}
\section{Chinese Remainder Theorem}
Around the year 300 a solution to the following mathematical problem appeared
in the mathematical manual of Chinese Master, Sun Tzu Suan Ching.

``We have a number of things, but we do not know exactly how many. If we count
them by threes, we have two left over. If we count them by fives, we have three
left over. If we count them by sevens, we have two left over. How many things are
there?''

The master was asking us to solve the three simultaneous congruences:
\begin{align*}
    x & \equiv 2\Mod{3}  \\
    x & \equiv 3\Mod{5}  \\
    x & \equiv 2\Mod{7}.
\end{align*}
The Chinese remainder theorem tells us that $ x\equiv 23\Mod{105} $ is the solution to
above problem.

Before proceeding to its statement, let us prove the following result.

\begin{Proposition}{}{}
    Let $m$ and $n$ be integers greater than $1$ that are coprime (relatively prime). Then the congruence
    \[ a\equiv b\Mod{mn} \]
    is true if and only if both the congruences
    \begin{align*}
        a & \equiv b\Mod{m} \\
        a & \equiv b\Mod{n}
    \end{align*}
    are true.
    \tcblower{}
    \textbf{Proof}: Suppose that $ a\equiv b\Mod{mn} $, then $ mn\mid (a-b) $, so $ m\mid (a-b) $ and $ n\mid (a-n) $.
    Therefore, $ a\equiv b\Mod{m} $ and $ a\equiv b\Mod{n} $. On the other hand, suppose that $ a\equiv b\Mod{m} $ and $ a\equiv b\Mod{n} $,
    Then, $ m\mid (a-b) $ and $ n\mid (a-b) $. Since $ (m,n)=1 $, $ mn\mid (a-b) $ by Proposition 1 (Lecture 4). Thus, $ a\equiv b\Mod{mn} $.
\end{Proposition}
\begin{Theorem}{The Chinese Remainder Theorem (CRT)}{}
    If $ m,n $ are coprime (relatively prime) moduli and $ a,b\in\mathbf{Z} $, then
    \begin{align*}
        x & \equiv a\Mod{m} \\
        x & \equiv b\Mod{n}
    \end{align*}
    have a common solution $ x_0 $. Furthermore, any two solutions $ x_0,y_0 $ to this pair of
    congruences must be such that $ x_0\equiv y_0\Mod{mn} $. The congruences have a unique
    solution.
    \tcblower{}
    \textbf{Proof}: Since $ m,n $ are coprime, by Bezout's Identity there exists $ x,y\in\mathbf{Z} $ such that
    \[ mx+ny=1. \]
    Multiplying both sides by $ (b-a) $, we obtain a solution to
    \[ mx^\prime+ny^\prime=b-a, \]
    where $ x^\prime=(b-a)x $ and $ y^\prime=(b-a)y $. Thus, $ a+mx^\prime=b-ny^\prime=x_0 $, we see that
    \begin{align*}
        x_0 & \equiv a\Mod{m}  \\
        x_0 & \equiv b\Mod{n}.
    \end{align*}
    So $ x_0 $ is a common solution. Now, let $ y_0 $ be any other solution to the system of congruences, then
    \begin{align*}
        x_0 & \equiv y_0\Mod{m}  \\
        x_0 & \equiv y_0\Mod{n}.
    \end{align*}
    So, by Proposition 3, we conclude that
    \[ x_0\equiv y_0\Mod{mn}. \]
\end{Theorem}
\makeheading{Lecture 13}{\printdate{2022-06-01}}%chktex 8
Here is an alternate proof for Chinese Remainder Theorem.

We have that there exists $ s\in\mathbf{Z} $ such that $ x=ms+a $. Substituting this into the other congruence
gives
\begin{align*}
    ms+a & \equiv b\Mod{n}      \\
    ms   & \equiv (b-a)\Mod{n}.
\end{align*}
Since $ (m,n)=1 $, there exists $ c\in\mathbf{Z} $ such that $ ms\equiv 1\Mod{n} $. Multiplying by $ c $
gives
\[ s\equiv c(b-a)\Mod{n}. \]
Thus, there exists $ t\in\mathbf{Z} $ such that $ s=tn+c(b-a) $.

Hence,
\begin{align*}
    x & =m\bigl(tn+c(b-a)\bigr)+a \\
      & =mnt+mc(b-a)+a.
\end{align*}
Thus, the unique solution is
\[ x\equiv mc(b-a)+a\Mod{mn}. \]
We can easily generalize this result to arbitrary number of coprime moduli.
\begin{Theorem}{Generalized Chinese Remainder Theorem}{}
    Suppose $ n_1,\ldots,n_k $ are moduli that are pairwise coprime. If $ a_1,\ldots,a_k\in\mathbf{Z} $
    then there exists $ x\in\mathbf{Z} $ such that
    \begin{align*}
        x & \equiv a_1\Mod{n_1}, \\
          & \vdotswithin{\equiv} \\
        x & \equiv a_k\Mod{n_k}.
    \end{align*}
    Furthermore, if $ x_0 $ is a solution of these congruences, then the complete solution
    to all the equations is given by all
    \[ x\equiv x_0\Mod{n_1\cdots n_k}. \]
\end{Theorem}
\textbf{Process to solve systems of congruences with CRT}:
\begin{itemize}
    \item Begin with the largest modulus $ x\equiv a_k\Mod{n_k} $.
          Rewrite it as $ x=n_k j_k+a_k $ for some $ j_k\in\mathbf{Z} $.
    \item Substitute the expression for $x$ into the congruence with the next largest
          modulus, that is,
          \[ x\equiv a_{k-1}\Mod{n_{k-1}}\implies n_k j_k+a_k\equiv a_{k-1}\Mod{n_{k-1}}. \]
    \item Solve this congruence for $j_k$.
    \item Write the solved congruence as an equation, and then substitute this expression for
          $ j_k $ into the equation for $x$.
    \item Continue substituting and solving the congruences until the equation for
          $x$ implies the solution to the system of congruences.
\end{itemize}
\begin{Example}{}{}
    Solve the system of congruences:
    \begin{align*}
        x & \equiv 3\Mod{6}   \\
        x & \equiv 7\Mod{13}.
    \end{align*}
    \tcblower{}
    \textbf{Solution}: First, $ x\equiv 7\Mod{13}\iff x=13j+7 $ for some $ j\in\mathbf{Z} $. Then,
    \[ x\equiv 3\Mod{6}\implies 13j+7\equiv 3\Mod{6}. \]
    Now, solve for $ j $ to get: $ j\equiv 2\Mod{6}\iff j=6k+2 $ for some $ k\in\mathbf{Z} $. Then,
    \[ x=13(6k+2)+7=78k+33\implies x\equiv 33\Mod{78}. \]
\end{Example}
\begin{Exercise}{}{}
    Solve the system of congruences:
    \begin{align*}
        x & \equiv 6\Mod{11}  \\
        x & \equiv 13\Mod{16} \\
        x & \equiv 9\Mod{21}
    \end{align*}
\end{Exercise}
\begin{Exercise}{}{}
    Calvin Butterball keeps pet meerkats in his backyard. If he divides
    them into $5$ equal groups, $4$ are left over. If he divides them into $8$ equal groups, $6$
    are left over. If he divides them into $9$ equal groups, $8$ are left over. What is the
    smallest number of meerkats that Calvin could have?
\end{Exercise}
Sometimes we can solve a system even if moduli aren't relatively prime.
\begin{Theorem}{}{}
    Consider the system of congruences:
    \begin{align*}
        x & \equiv a\Mod{m}  \\
        x & \equiv b\Mod{n}.
    \end{align*}
    \begin{enumerate}[(a)]
        \item If $ (m,n)\nmid (a-b) $, then there are no solutions.
        \item If $ (m,n)\mid (a-b) $, then there is a unique solution mod $ [m,n] $.
    \end{enumerate}
    \tcblower{}
    \underline{Remark}: If $ (m,n)=1 $ case (b) automatically holds, and $ [m,n]=mn $; that is, we get the
    Chinese Remainder Theorem for two equations.
    \tcblower{}
    \textbf{Proof}: Exercise.
\end{Theorem}
\section{Euler \texorpdfstring{$ \varphi $}{φ} Function and Euler's Theorem}
Firstly, we will study the units for $ \mathbf{Z}_n $.
\begin{Definition}{}{}
    Let $ n\ge 2 $. An element $ [a] $ in $ \mathbf{Z}_n $ is a unit
    provided the equation $ [a][x]=[1] $ has a unique solution. The integer $ a $
    representing $ [a] $ is then called unit modulo $ n $. The unique $ [x] $ is called
    the inverse of $ [a] $ in $ \mathbf{Z}_n $. The set of all units of $ \mathbf{Z}_n $
    is denoted by $ \mathbf{Z}_n^* $.
    \tcblower{}
    \underline{Remark}: Proposition 2 (Lecture 12) tells the ways to think about units. An element
    $ [a]\in\mathbf{Z}_n $ is unit if and only if $a$ is coprime with $n$.
\end{Definition}
\begin{Proposition}{}{}
    If $ p $ is a prime and $ [a]\ne 0 $ in $ \mathbf{Z}_p $, then $ [a] $ is a unit. In other words,
    every non-zero element of $ \mathbf{Z}_p $ is a unit.
\end{Proposition}
The importance of Proposition 1 lies in the fact that $ \mathbf{Z}_p $ behaves just like the
well understood sets of numbers $ \mathbf{Q},\mathbf{R},\mathbf{C} $. Namely, $ \mathbf{Z}_p $ admits addition, subtraction,
multiplication, and division by everything except zero. Indeed, if $ [a]\ne 0 $ in $ \mathbf{Z}_p $,
then $[a][x] = [b]$ always has a solution. So we can divide. Systems that admits all
four of the arithmetic operation are usually called \textbf{fields}.
\begin{Proposition}{}{}
    Let $ n\ge 2 $. Then,
    \begin{enumerate}[i.]
        \item The product of a unit is another unit, that is, if $ [a],[b]\in\mathbf{Z}_n^* $, then $ [a][b]\in\mathbf{Z}_n^* $.
        \item The product of units is associative, that is, $ ([a][b])[c]=[a]([b][c]) $ for all $ [a],[b],[c]\mathbf{Z}_n^* $.
        \item The residue class $ [1] $ is always a unit, that is, $ [1]\in \mathbf{Z}_n^* $.
        \item The inverse of a unit is also a unit, that is, if $ a\in \mathbf{Z}_n^* $ and $ x\in \mathbf{Z}_n^* $ is a unique
              residue class that gives $ [a][x]=[1] $, then $ [x]\in \mathbf{Z}_n^* $.
              The product of units is commutative, that is, $ [a][b]=[b][a] $ in $ \mathbf{Z}_n^* $.
    \end{enumerate}
    Any set that enjoys the five properties of Proposition 2 is called an Abelian
    Group.
\end{Proposition}
\begin{Example}{}{}
    Compute $ \mathbf{Z}_{10}^* $ and construct its multiplication table.
    \tcblower{}
    \textbf{Solution}: The integers that are coprime to $10$ are $ 1 $, $ 3 $, $ 7 $, and $ 9 $. Thus,
    $ \mathbf{Z}_{10}^*=\Set{[1],[3],[7],[9]} $ and its multiplication table is TODO\@.
\end{Example}
\makeheading{Lecture 14}{\printdate{2022-06-03}}%chktex 8
As you may have noticed already, the condition that numbers are relatively
prime is useful and interesting. So, it should not be surprising that there are times
when it is useful to restrict a complete residue system modulo $n$ to just the numbers
which are relatively prime to $n$.
\begin{Definition}{}{}
    Let $ \varphi(n) $ denote the number of integers $ m $ such that $ 0\le m<n $ and $ (m,n)=1 $. The function
    $ \varphi $ is called the Euler's totient function.
\end{Definition}
\begin{Example}{}{}
    Find $ \varphi(18) $.
    \tcblower{}
    \textbf{Solution}: Numbers from $0$ to $17$ that are coprime with $18$ are $ 1,5,7,11,13,17 $.
    So, $ \varphi(18)=6 $.
\end{Example}
\begin{Example}{}{}
    Find $ \varphi(101) $.
    \tcblower{}
    \textbf{Solution}: Since $101$ is itself prime, the full list of numbers that are
    coprime with $101$ are $ 1,2,3,\ldots,100 $. Thus, $ \varphi(101)=100 $.
\end{Example}
\underline{Remark}: For any prime $ p $, the numbers $ 1,2,\ldots,(p-1) $ are coprime
with $ p $. Therefore, $ \varphi(p)=p-1 $.
\begin{Theorem}{Euler's Theorem}{}
    If $ [a]\in\mathbf{Z}_n^* $, then $ [a]^{\varphi(n)}=[1] $.
    \tcblower{}
    \textbf{Proof}: Let $ k=\varphi(n) $. Let $ [u_1],\ldots,[u_k] $
    be the complete list of residues of $ \mathbf{Z}_n^* $. Form a new list
    \[ [a][u_1],\ldots,[a][u_k]. \]
    Since $ \mathbf{Z}_n^* $ is a group, this list of residues is also in $ \mathbf{Z}_n^* $. Furthermore,
    no element appears in this list twice, so if $ [a][u_i]=[a][u_j] $ for some $ i\ne j $, then $ [u_i]=[u_j] $
    by cancelling the unit $ [a] $. Hence, the second list is a permutation of the original list.

    It follows that the product of residues in the first list equals to the product of residues in the second list. After all, the two lists
    contain the same residues, only written in a different order. Thus, we obtain
    \[ [u_1][u_2]\cdots[u_k]=([a][u_1])\cdots([a][u_k]). \]
    Now, we can cancel the unit element $ [u_1]\cdots[u_k] $ and conclude that $ [a]^{\varphi(n)}=[1] $.
\end{Theorem}
In the language of congruences, Euler's Theorem translates to
\[ a^{\varphi(n)}\equiv 1\Mod{n} \]
for any integer that is invertible modulo $ n $, that is, $ (a,n)=1 $. In other words,
if $ a $ is coprime with modulus $ n $, then
\[ \frac{a^{\varphi(n)}-1}{n}\in\mathbf{Z}. \]
\begin{Example}{}{}
    Prove $ 623^{1222}\equiv 1\Mod{1223} $.
    \tcblower{}
    \textbf{Solution}: $ \varphi(1223)=1222 $ and $ (1223,623)=1 $. Hence, by Euler's Theorem
    $ 623^{1222}\equiv 1\Mod{1223} $.
\end{Example}
When the modulus is a prime, say $ p $, we know that $ \varphi(p)=p-1 $.
Thus, Euler's Theorem specializes to another famous result attributed by Fermat.
\begin{Theorem}{Fermat's Little Theorem}{}
    If $ p $ is a prime and $ p\nmid a $ for $ p\in\mathbf{Z}_n^* $, then
    \[ [a]^{p-1}=[1]. \]
    In other words,
    \[ a^{p-1}\equiv 1\Mod{p}. \]
\end{Theorem}
\begin{Corollary}{Fermat Variant}{}
    If $ p $ is a prime and $ a\in\mathbf{Z} $, then
    \[ a^{p}=a\Mod{p}. \]
    \tcblower{}
    \textbf{Proof}: If $ p\nmid a $, then $ a^{p-1}\equiv 1\Mod{p} $ by Fermat's Theorem.
    Multiplying by $ a $ to get $ a^p\equiv a\Mod{p} $. On the other hand,
    if $ p\mid a $, then $ a\equiv 0\Mod{p} $, in which case it is obvious that
    \[ a^p\equiv 0^p\equiv 0\equiv a\Mod{p}. \]
    So the result holds for all $ a\in\mathbf{Z} $.
\end{Corollary}
\section{Pseudoprimes}
The converse of Fermat's Little Theorem is not true in general, that is, if
there exists $ a $ such that $ a^{n-1}\equiv 1\Mod{n} $, then we can not infer that $ n $ is prime.
\begin{Definition}{}{}
    A number $ n $ that is composite and satisfies $ a^{n-1}\equiv 1\Mod{n} $ is called a Fermat pseudoprime to base $ a $.
    In the special case of $ a=2 $, it is sometimes called the \textbf{Poulet number}.
\end{Definition}
\begin{Example}{}{}
    The following table gives us the first Fermat pseudoprime to some small bases $ a $:
    \[ \begin{array}{cl}
            a & \text{Fermat pseudoprime}             \\
            \midrule
            2 & 341,561,645,1105,1387,1729,1905       \\
            3 & 91,121,286,671,703,949,1105,1541,1729 \\
            4 & 15,85,91,341,435,561,645,703          \\
            5 & 4,124,217,561,781,1541,1729,1891
        \end{array} \]
    The first example of even pseudoprime ($n = 161038$) to the base $2$ was given
    by Lehmer in 1950.
\end{Example}
\begin{Example}{}{}
    Prove that $ 341 $ is a Fermat pseudoprime to the base $ 2 $.
    \tcblower{}
    \textbf{Solution}: We want to show that $ 2^{340}\equiv 1\Mod{341} $. Note that $ 341=11\times 31 $.
    By FLT, $ 2^{10}\equiv 1\Mod{11} $. Thus,
    \[ 2^{340}\equiv (2^{10})^{34}\equiv 1^{34}\equiv 1\Mod{11}. \]
    Thus, $ 11\mid (2^{340}-1) $. Also, $ 2^5\equiv 32\equiv 1\Mod{31} $. So,
    \[ 2^{340}\equiv (2^5)^{68}\equiv 1\Mod{31}. \]
    Thus, $ 31\mid (2^{340}-1) $. So, by Proposition 1 (Lecture 4), $ 341\mid (2^{340}-1) $.
\end{Example}
\begin{Exercise}{}{}
    Show that $ 341 $ is not a Fermat pseudoprime.
    \tcblower{}
    \textbf{Solution}: We want to show that $ 3^{340}\not\equiv 1\Mod{341} $. Note that $ 341=11\times 31 $.
    By FLT, $ 3^{10}\equiv 1\Mod{11} $. Thus, $ 3^{340}=(3^{10})^{34}\equiv 1^{34}\equiv 1\Mod{11} $.
    So we need to show that $ 3^{340}\not\equiv 1\Mod{31} $.
    Note that $ 340=30\times 11+10 $ and by FLT $ 3^{30}\equiv 1\Mod{31} $.
    \[ 3^{340}\equiv (3^{11})^{30}3^{10}\equiv 1^{30}3^{10}\equiv 3^{10}\Mod{31}. \]
    So, $ 3^2\equiv 9\Mod{31} $, $ 3^3\equiv 27\equiv -4\Mod{31} $, $ 3^4\equiv 81\equiv 19\Mod{31} $, $ 3^6\equiv (3^3)^2\equiv (-4)^2\equiv 16\Mod{31} $.
    Therefore,
    \[ 3^{10}\equiv 3^6\cdot 3^4\equiv 16\cdot 19\equiv 304\Mod{31}\equiv 25\Mod{31}.  \]
    Note that $ 3^{340}\equiv 25\not\equiv 1\Mod{31} $.
\end{Exercise}
\section{Polynomial Congruence}
We have seen how to solve linear congruences $ ax\equiv b\Mod{n} $. What about
polynomial congruences? These, of course, are also important in number theory.

We first note that there are some immediate differences from what we are used
to with solving polynomials over $ \mathbf{R} $.

For example, we know the polynomial $ f(x)=x^2+1 $ has no roots over $ \mathbf{R} $,
but
\[ x^2+1\equiv 0\Mod{5} \]
has $ x=2 $ and $ x=3 $ as solutions.

Also, we are used to $ d{\textsuperscript{th}} $ degree polynomials having exactly $ d $ roots,
but
\[ x^2+x\equiv 0\Mod{6} \]
has four distinct roots modulo $ 6 $, namely $ x=0,2,3,5 $.

The Chinese Remainder Theorem can also be utilized to solve polynomial congruences.

\begin{Definition}{}{}
    Let $d$ be a positive integer and consider a polynomial
    \[ f(x)=c_d x^d+\cdots + c_1 x+c_0, \]
    where $ c_0,\ldots,c_d\in\mathbf{Z} $ and $ c_d \ne 0 $. Then the congruence of the form
    $ f(x)\equiv 0\Mod{n} $ is called a \textbf{polynomial congruence}.
    \tcblower{}
    \underline{Goal}: Find all $ x\in\mathbf{Z} $ which satisfy the above congruence.
\end{Definition}

Note that if we replace $ c_i $ with $ [c_i] $, then we reduce the polynomial
from $ \mathbf{Z} $ to $ \mathbf{Z}_n $. Solving the above congruence is equivalent to solving
$ f([x])=[0] $ in $ \mathbf{Z}_n $. If such an equation is satisfied by some residue
class $ [x_0] $, we say that $ [x_0] $ is a root of $ f(x) $ in $ \mathbf{Z}_n $.
\makeheading{Lecture 11}{\printdate{2022-10-26}}%chktex 8
\begin{Definition}{}{}
    The \textbf{joint probability mass function}
    (joint pmf) of a sequence $ X_1,\ldots,X_n $
    of discrete random variables is a function
    $ p\colon\mathbf{R}^n\to[0,1] $
    with
    \[ p(a_1,\ldots,a_n)=\Prob[\big]{\Set{X_1=a_1}\cap\cdots\cap \Set{X_n=a_n}}. \]
\end{Definition}
\begin{Example}{}{}
    Suppose we are rolling two 4-sided die independently. The joint pmf is
    \[ p(a,b)=\begin{cases}
            \frac{1}{16}, & a,b\in\Set{1,2,3,4}, \\
            \text{0},     & \text{otherwise}.
        \end{cases} \]
\end{Example}
\begin{Example}{}{}
    Suppose we roll a die and flip a coin. Let $ X $ be a die roll and
    \[ Y=\begin{cases}
            X,   & \text{if H}, \\
            5-X, & \text{if T}.
        \end{cases} \]
    \[ \begin{array}{cc|c|c|c|c}
            \multicolumn{2}{c}{} & \multicolumn{4}{c}{a}                         \\
                                 &                       & 1   & 2   & 3   & 4   \\
            \cline{2-6}
            \multirow{4}{*}{$b$} & 1                     & 1/8 & 0   & 0   & 1/8 \\
            \cline{2-6}
                                 & 2                     & 0   & 1/8 & 1/8 & 0   \\
            \cline{2-6}
                                 & 3                     & 0   & 1/8 & 1/8 & 0   \\
            \cline{2-6}
                                 & 4                     & 1/8 & 0   & 0   & 1/8
        \end{array} \]
    Note that
    \[ \Prob{\Set{Y=3}}=\frac{1}{8}+\frac{1}{8}=\frac{1}{4}. \]
\end{Example}
\begin{Remark}{}{}
    If $ p $ is the joint pmf of $ (X,Y) $, then
    \begin{align*}
        \Prob{\Set{X=k}} & =\sum_{j}\underbrace{p(k,j)}_{\Prob{\Set{X=k,Y=j}}}. \\
        \Prob{\Set{Y=k}} & =\sum_{j}p(j,k).
    \end{align*}
    In this context of starting with a joint distribution,
    distribution of components are called ``marginal distributions.''
\end{Remark}
\begin{Definition}{}{}
    If $ p $ is the joint pmf of
    $ X_1,\ldots,X_n $, then the marginal distribution
    of $ X_k $ for any $ k\in\Set{1,2,\ldots,n} $ is
    \[ \Prob{\Set{X_k=a}}=\sum_{b_1,\ldots,b_{k-1},b_{k+1},\ldots,b_n}
        p(b_1,\ldots,b_{k-1},a,b_{k+1},\ldots,b_n). \]
\end{Definition}
\begin{Theorem}{}{}
    $ X_1,\ldots,X_n $ (discrete) are jointly independent
    if and only if their joint pmf is the product of their individual pmfs;
    that is,
    \[ p_{X_1,\ldots,X_n}(b_1,\ldots,b_n)=
        p_{X_1}(b_1)\cdots p_{X_n}(b_n). \]
\end{Theorem}
\begin{Example}{}{}
    Let $ X $ and $ Y $ be independent with pmfs
    \begin{align*}
        p_X(-1) & =\frac{1}{2}, \\
        p_X(0)  & =\frac{1}{4}, \\
        p_X(1)  & =\frac{1}{4}, \\
        p_Y(0)  & =\frac{1}{3}, \\
        p_Y(1)  & =\frac{2}{3}.
    \end{align*}
    They have joint pmf
    \[ \begin{array}{cc|c|c|c}
            \multicolumn{2}{c}{} & \multicolumn{3}{c}{X}                     \\
                                 &                       & -1  & 0    & 1    \\
            \cline{2-5}
            \multirow{2}{*}{$Y$} & 0                     & 1/6 & 1/12 & 1/12 \\
            \cline{2-5}
                                 & 1                     & 1/3 & 1/6  & 1/6
        \end{array} \]
\end{Example}
\begin{Definition}{}{}
    If $ X_1,\ldots,X_n $ are continuous random variables and
    $ f\colon\mathbf{R}^n\to\interval[open right]{0}{\infty} $ ($ A\subseteq\mathbf{R}^n $)
    that satisfies
    \[ \int\cdots\int\limits_{A}f(x_1,\ldots,x_n)\odif{x_1}\cdots\odif{x_n} \]
    then $ f $ is a joint pdf for these variables, and they are said to be
    jointly continuous.
\end{Definition}
\begin{Example}{}{}
    Suppose we have two continuous random variables $ X $ and $ Y $.
    \[ \Prob{\Set{(X,Y)\in A}}=\iint\limits_{A}f(x,y)\odif{x}\odif{y}. \]
    If $ A  $ is a rectangle, then $ A=[a,b]\times[c,d] $,
    which implies
    \[ \Prob{\Set{a\le X\le b, c\le Y\le d}}
        =\int_{c}^{d}\int_{a}^{b}f(x,y)\odif{x}\odif{y}. \]
\end{Example}
\begin{Theorem}{}{}
    $ X_1,\ldots,X_n $ (continuous) are jointly independent
    if and only if they are jointly continuous with joint pdf
    \[ f_{X_1,\ldots,X_n}(a_1,\ldots,a_n)=f_{X_1}(a_1)\cdots f_{X_n}(a_n). \]
\end{Theorem}
\begin{Example}{}{}
    \[ f(x,y)=\begin{cases}
            2x^2, & x\in[0,1],\; \abs{y}\le x, \\
            0,    & \text{otherwise}.
        \end{cases} \]
    \underline{Verifying we have a probability density function}:
    \begin{align*}
        \int_{0}^{1}\int_{-x}^{x}2x^2\odif{y}\odif{x}
         & =\int_{0}^{1}\biggl[2x^2y\biggr]_{y=-x}^{y=x}\odif{x} \\
         & =\int_{0}^{1}2x^2(x-(-x))\odif{x}                     \\
         & =\int_{0}^{1}4x^3\odif{x}                             \\
         & =\biggl[x^4\biggr]_{x=0}^{x=1}                        \\
         & =1.
    \end{align*}
    \underline{Calculating Probabilities}:
    To calculate $ \Prob{\Set{Y\ge 1/2}} $,
    we could work out
    the system of inequalities: $ 0\le x\le 1 $, $ -x\le y\le x $, and $ 1/2\le y $ yields
    \[ 1/2\le y\le x\le 1. \]
    Or we can work it out graphically.
    \begin{align*}
        \Prob*{\Set*{Y\ge \frac{1}{2}}}
         & =\int_{1/2}^{1}\int_{1/2}^{x}2x^2\odif{y}\odif{x}        \\
         & =\int_{1/2}^{1}\biggl[2x^2y\biggr]_{y=1/2}^{y=x}\odif{x} \\
         & =\int_{1/2}^{1}(2x^3-x^2)\odif{x}                        \\
         & =\biggl[\frac{x^4}{2}-\frac{x^3}{3}\biggr]_{x=1/2}^{x=1} \\
         & =\frac{1}{2}-\frac{1}{32}-\frac{1}{3}+\frac{1}{24}.
    \end{align*}
\end{Example}
\begin{Definition}{}{}
    The marginal density of $ X $ is
    \[ f_X(t)=\int_{-\infty}^{\infty}f_{X,Y}(t,u)\odif{u}. \]
\end{Definition}
\begin{Definition}{Expectation (Continuous)}{}
    \[ \E[\big]{g(X,Y)}
        =\int_{-\infty}^{\infty}\int_{-\infty}^{\infty}g(x,y)f_{X,Y}(x,y)\odif{x}\odif{y}. \]
    \tcblower{}
    For example, to calculate $ \E{XY} $ we use $ g(x,y)=xy $.
\end{Definition}
\begin{Example}{Polya Urn}{}
    \[ \Prob{\Set{3\textsuperscript{rd}\text{ pick R}}\given \Set{\text{BB}}}
        =\frac{1}{4}. \]
    If $ Y $ is the limiting percentage of blue, then
    \begin{align*}
        \Prob*{\Set*{Y\le \frac{1}{2}}\given \Set{\text{BB}}}
         & =\Prob*{X_1,X_2,Y\le \frac{1}{2}}                                          \\
         & =\int_{0}^{1/2}\int_{0}^{1/2}\int_{0}^{1/2}1\odif{x_1}\odif{x_2}\odif{x_3} \\
         & =\frac{1}{2}\cdot \frac{1}{2}\cdot \frac{1}{2}                             \\
         & =\frac{1}{8}.
    \end{align*}
    \[ \Prob{\Set{Y\le t}}=t^3. \]
    \[ f_Y(t)=\begin{cases}
            3t^2, & t\in[0,1]         \\
            0,    & \text{otherwise},
        \end{cases} \]
    which is a $ \BetaDist{3,1} $ distribution.
    \[ \frac{1}{B(\alpha,\beta)}x^{\alpha-1}(1-x)^{\beta-1},\; x\in[0,1]. \]
\end{Example}
\makeheading{Lecture 12}{\printdate{2022-10-28}}%chktex 8
Discussion on gamma function when $ \alpha=0 $.
\begin{Example}{}{}
    Suppose $ X \sim \GAM{\alpha,\lambda} $. Find
    $ M_X(t) $.
    \tcblower{}
    \textbf{Solution}:
    \begin{align*}
        M_X(t)
         & =\E{e^{tX}}                                                                                                                                                                               \\
         & =\int_{0}^{\infty}e^{tx}\frac{\lambda^\alpha}{\Gamma(\alpha)}x^{\alpha-1}e^{-\lambda x}\odif{x}                                                                                           \\
         & =\frac{\lambda^\alpha}{\Gamma(\alpha)}\int_{0}^{\infty}x^{\alpha-1}e^{-\lambda x(1-t/\lambda)}\odif{x}                                                                                    \\
         & =\frac{\lambda^\alpha}{\Gamma(\alpha)}\int_{0}^{\infty}\frac{u^{\alpha -1}}{(\lambda-t)^{\alpha-1}}e^{-u}\frac{1}{\lambda-t}\odif{u} &  & u=x(\lambda-t)\iff \odif{u}=(\lambda-t)\odif{x} \\
         & =\frac{\lambda^\alpha}{\Gamma(\alpha)(\lambda-t)^{\alpha}}\int_{0}^{\infty}u^{\alpha-1}e^{-u}\odif{u}                                                                                     \\
         & =\frac{\lambda^\alpha}{\Gamma(\alpha)(\lambda-t)^{\alpha}}\Gamma(\alpha)                                                                                                                  \\
         & =\biggl(\frac{\lambda}{\lambda-t}\biggr)^{\! \alpha}.
    \end{align*}
\end{Example}
\begin{Example}{}{}
    Suppose $ X_1 \sim \GAM{1/2,2} $ and $ X_2 \sim \GAM{3,2} $ are independent.
    Find $ M_Y(t) $ where $ Y=X_1+X_2 $.
    \tcblower{}
    \textbf{Solution}: Since $ X_1 $ and $ X_2 $ are independent,
    \begin{align*}
        M_Y(t)
         & =M_{X_1}(t)M_{X_2}(t)                                             \\
         & =\biggl(\frac{2}{2-t}\biggr)^{1/2}\biggl(\frac{2}{2-t}\biggr)^{3} \\
         & =\biggl(\frac{2}{2-t}\biggr)^{7/2}.
    \end{align*}
    Therefore, $ Y \sim \GAM{3.5,2} $.
\end{Example}
\begin{Example}{}{}
    The pdf for $ \GAM{1,\lambda} $ is
    \[ f_X(t)=\frac{\lambda^1}{\Gamma(1)}t^0 e^{-\lambda t}=\lambda e^{-\lambda t}, \]
    which is $ \EXP{1} $.
\end{Example}
\begin{Remark}{}{}
    \[ \BIN*{n,\frac{\lambda}{n}}\xrightarrow{n\to\infty} \POI{\lambda}. \]
\end{Remark}
\begin{Example}{}{}
    Suppose Chocolat gets 1 customer every 10 minutes, on average (discrete time).
    \begin{enumerate}[(i)]
        \item Model level 1:
              \begin{itemize}
                  \item Every minute there is an independent $ 1/10 $ chance for a customer to enter
                        ($0$ chance for multiple customers in the same minute).
                  \item Let $ T_1 $ be the waiting time for the first customer in minutes,
                        \[ T_1=\text{waiting time for the first customer in minutes}\sim \GEO*{\frac{1}{10}}, \]
                        and $ \E{T_1}=10 $.
                        \[ N_{60}=\text{number of customers in the first hour}\sim \BIN*{60,\frac{1}{10}}. \]
              \end{itemize}
        \item Model level 2:
              \begin{itemize}
                  \item Every second there is a $ 1/600 $ chance for a customer to enter, independently.
                        \[ T_1=\text{waiting time in minutes}=\frac{\tilde{T}_1}{60},\; \text{where }\tilde{T}_1 \sim \GEO*{\frac{1}{600}}, \]
                        and $ \E{T_1}=600/60=10 $.
                        \[ N_{60}=\text{number of customers in the first hour}\sim \BIN*{3600,\frac{1}{600}}. \]
              \end{itemize}
    \end{enumerate}
    As we approach continuity,
    \[ N_{60}\tod\POI*{\frac{60}{10}},\;
        T_1\tod\EXP*{\frac{1}{10}}. \]
    For $ t\ge 0 $,
    \[ N(t)=\text{number of arrivals in the first $t$ minutes}\sim \POI*{\frac{1}{10}t}. \]
\end{Example}
\begin{Definition}{}{}
    A \textbf{Poisson process} ($ N(t) $ for $ t\ge 0 $) with rate $ \lambda $ is a stochastic process
    with the properties:
    \begin{enumerate}[(1)]
        \item For $ 0\le t_1<t_2 $,
              \[ (N(t_2)-N(t_1))\sim \POI{\lambda(t_2-t_1)}. \]
        \item For $ 0\le t_1<t_2<\cdots<t_n $, the variables
              \[ (N(t_2)-N(t_1)),(N(t_3)-N(t_2)),\ldots (N(t_n)-N(t_{n-1})) \]
              are jointly independent.
    \end{enumerate}
\end{Definition}
\begin{Definition}{}{}
    \[ T_n=\inf{\Set{t\ge 0\given N(t)\ge n}} \]
    is the arrival time of the $ n\textsuperscript{th} $ customer.
\end{Definition}
\begin{Theorem}{Interarrival Times}{}
    $ \Delta_1=T_1 $, and $ \Delta_n=T_n-T_{n-1} $ for $ n\ge 2 $ are known as
    \textbf{interarrival times}. Then,
    $ \Delta_1,\ldots,\Delta_n \iid \EXP{\lambda} $ variables.
\end{Theorem}
\begin{Corollary}{}{}
    For $ 0\le n_1<n_2<\cdots<n_k $,
    \[ T_{n_2}-T_{n_1},T_{n_3}-T_{n_2},\ldots,T_{n_k}-T_{n_{k-1}} \]
    are jointly independent with respective probability
    distributions
    \[ (T_{n_{j+1}}-T_{n_j})\sim \GAM{n_{j+1}-n_j,\lambda}. \]
\end{Corollary}
\begin{Example}{}{}
    $ T_3 \sim \GAM{3,\lambda} $ and
    $ T_5-T_3 \sim \GAM{2,\lambda} $ are independent.
\end{Example}
\begin{Example}{}{}
    Suppose $ X \sim \GAM{\alpha,1} $ and
    $ Y \sim \GAM{\beta,1} $
    are independent (rate doesn't matter, set it equal to $1$
    for simplicity).
    \begin{Example}{}{}
        $ \alpha=3 $, $ \beta=5 $, $ X=T_3 $, $ Y=T_8-T_3 $.
        What is the distribution of $ T_3/T_8 $?
        If it took two hours for 8 people to arrive, what is the
        conditional distribution of how long it took for three people to arrive?
    \end{Example}
    That is, find the distribution of
    \[ Z=\frac{X}{X+Y},\; 0\le Z\le 1. \]
    For $ t\in[0,1] $,
    \[ \frac{x}{x+y}\le t\implies x\le \frac{ty}{1-t}. \]
    Thus, noting that $ X $ and $ Y $ are independent,
    \begin{align*}
        \Prob{\Set{Z\le t}}
         & =\int_{0}^{\infty}\int_{0}^{ty/(1-t)}
        \frac{1}{\Gamma(\alpha)}x^{\alpha-1}e^{-x}\frac{1}{\Gamma(\beta)} y^{\beta-1}e^{-y}\odif{x}\odif{y} \\
         & =\frac{1}{\Gamma(\alpha)\Gamma(\beta)}\int_{0}^{\infty}\int_{0}^{ty/(1-t)}
        x^{\alpha-1}y^{\beta-1}e^{-(x+y)}\odif{x}\odif{y}.
    \end{align*}
    Multivariable substitution:
    \[ u=\frac{x}{x+y},\; v=x+y\implies x=uv,\; y=v-uv=v(1-u). \]
    \[ J=\pdv{(x,y)}{(u{,}v)}=\begin{vmatrix}
            \pdv{x}{u} & \pdv{x}{v} \\
            \pdv{y}{u} & \pdv{y}{v}
        \end{vmatrix}=\begin{vmatrix}
            v  & u   \\
            -v & 1-u
        \end{vmatrix}=\abs{(v)(1-u)-(u)(-v)}=v. \]
    Note that $ u\le t $ and $ 0\le v<\infty $, which implies
    \begin{align*}
         & =\frac{1}{\Gamma(\alpha)\Gamma(\beta)}\int_{0}^{\infty}\int_{0}^{1}
        (uv)^{\alpha-1}\bigl(v(1-u)\bigr)^{\beta-1}e^{-v}v\odif{u}\odif{v}                                                                       \\
         & =\frac{1}{\Gamma(\alpha)\Gamma(\beta)}
        \int_{0}^{\infty}\underbrace{v^{\alpha-1}v^{\beta-1}}_{v^{\alpha+\beta-1}} e^{-v}\odif{v}\int_{0}^{1}u^{\alpha-1}(1-u)^{\beta-1}\odif{u} \\
         & =\frac{\Gamma(\alpha+\beta)}{\Gamma(\alpha)\Gamma(\beta)}\int_{0}^{1}u^{\alpha-1}(1-u)^{\beta-1}\odif{u}.
    \end{align*}
\end{Example}
\begin{Definition}{Beta Distribution}{}
    We say $ X \sim \BetaDist{\alpha,\beta} $ with shape parameters $ 0<\alpha\in\mathbf{R} $ and $ 0<\beta\in\mathbf{R} $ if it has pdf
    \[ f_X(t\mid \alpha,\beta)=\frac{1}{B(\alpha,\beta)}t^{\alpha-1}(1-t)^{\beta-1},\; t\in[0,1] \]
    where $ B(\alpha,\beta) $ denotes the beta function,
    \[ B(\alpha,\beta)=\int_{0}^{1}x^{\alpha-1}(1-x)^{\beta-1}\odif{x}=
        \frac{\Gamma(\alpha)\Gamma(\beta)}{\Gamma(\alpha+\beta)}. \]
\end{Definition}
\begin{Theorem}{}{}
    If $ X \sim \GAM{\alpha,1} $ and $ Y \sim \GAM{\beta,1} $, then
    \[ Z=\frac{X}{X+Y}\sim \BetaDist{\alpha,\beta} \]
    and is independent of
    \[ X+Y \sim \GAM{\alpha+\beta,1}. \]
\end{Theorem}
\section{PPS Sampling Procedures}
\makeheading{Lecture 12}{\printdate{2022-02-14}}%chktex 8

\makeheading{Lecture 13}{\printdate{2022-02-16}}%chktex 8
\end{document}