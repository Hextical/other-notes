\chapter{Review of Basic Concepts in Survey Sampling}
\makeheading{Lecture 1}{\printdate{2022-01-05}}%chktex 8

\textbf{Survey sampling as a scientific discipline}:
\begin{itemize}
      \item Started from Jerzy Neyman's 1934 paper (1894--1981).
      \item Fast development since the 1940s and 1950s.
      \item Became an important area of statistics and social science.
      \item (Used to be) the primary tool of data collection for official
            statistics and researchers in social science and health studies.
      \item Face challenges in the big data and internet era.
\end{itemize}

\textbf{Some ongoing well-known surveys}:
\begin{itemize}
      \item The Current Population Survey of the US (CPS).
      \item The National Health and Nutrition Examination Survey of the
            US (NHANES).
      \item The General Social Survey of Canada (GSS).
      \item The Canadian Community Health Survey (CCHS).
      \item The International Tobacco Control Policy Evaluation Surveys
            (The ITC Surveys, headquartered at UWaterloo).
      \item The Canadian Longitudinal Study of Aging
            (CLSA, McMaster, McGill, and Dalhousie).
\end{itemize}

\textbf{Statistics Canada}:\\
One of the most respected survey organizations in the world.

\textbf{Some Canadian survey statisticians}:
\begin{itemize}
      \item J.N.K. Rao (Carleton University, retired).
      \item David Bellhouse (University of Western Ontario, retired).
      \item Jiahua Chen (University of British Columbia).
      \item David Haziza (University of Ottawa).
      \item Carl E. Särndal (University of Montreal, retired).
      \item Louis-Paul Rivest (Laval University).
      \item David Binder (Statistics Canada, 1949--2012).
      \item Carl Schwarz (Simon Fraser University, retired).
      \item Steve Thompson (Simon Fraser University).
      \item Randy Sitter (Simon Fraser University, 1961--2007).
      \item V. P. Godambe (University of Waterloo, 1926--2016).
      \item Mary E. Thompson (University of Waterloo, retired).
      \item Matthias Schonlau (University of Waterloo).
      \item Changbao Wu (University of Waterloo).
\end{itemize}

\textbf{Example 1.1}. The Math Faculty plans to conduct a survey to study
the well-being of recent graduates from the faculty.
\begin{itemize}
      \item What is exactly the group to be studied?\\
            (The target population)
      \item What information is to be collected?\\
            (Variables to be measured; sample data)
      \item From what can we select individuals to be surveyed?\\
            (Sampling frame(s))
      \item How to select individuals to be surveyed?\\
            (Sampling methods; sampling procedures)
      \item What method to use to collect data?\\
            (The mode of data collection: Face-to-face? Telephone? Mailed
            questionnaire?)
      \item How to use the data to draw conclusions?\\
            (Statistical analysis)
\end{itemize}

\textbf{Three versions of survey populations (with reference to Example 1.1)}:
\begin{itemize}
      \item \emph{The target population}: The set of all units covered by the main
            objective of the study.\\
            (All students who received a formal degree from Waterloo
            between 2016 and 2019)
      \item \emph{The frame population}: The set of all units covered by the
            sampling frame(s).\\
            (Sampling frame: The list of personal email addresses of
            students who graduated between 2016 and 2019)
      \item \emph{The sampled population} (\emph{the study population}):
            The population represented by the sample. Under probability sampling, the
            sampled population is the set of all units which have a non-zero
            probability to be selected in the sample.
            \begin{itemize}
                  \item The sampled population is not the set of sampled units!
                  \item Units which cannot be reached or do not respond to surveys
                        (non-response) are not part of the sampled population.
            \end{itemize}
\end{itemize}

\textbf{Population structures and sampling frames}:
\[ U=\Set{1,2,\ldots,N}, \]
where $ N $ is the population size, and the labels $ 1,2,\ldots,N $ represent
the $ N $ units.
\begin{itemize}
      \item \textbf{Unstructured population}:
            There exists a single complete list of all $N$ units, which can be
            used as the sampling frame.
      \item \textbf{Stratified population}:
            The population $U$ has a stratified structure if it is divided into $H$
            non-overlapping subpopulations:
            \[ U=U_1\cup U_2\cup\cdots\cup U_H, \]
            where the subpopulation $ U_h $ is called stratum $ h $, with stratum
            population size $ N_h $, $ h=1,2,\ldots,H $. It follows that
            \[ N=\sum_{h=1}^{H}N_h. \]
            Sampling frames for stratified sampling:
            $H$ separate lists, each list consists of all units in one stratum.
      \item \textbf{Clustered population}:
            If the survey population can be divided into groups, called
            \emph{clusters}, such that every unit in the population belongs to one
            and only one group, we say the population is clustered.

            First stage sampling frame for cluster sampling: A complete list of clusters (but not all the units within each
            cluster).
      \item Stratified sampling versus cluster sampling:
            \begin{itemize}
                  \item Under stratified sampling, sample data are collected from every
                        stratum.
                  \item Under cluster sampling, only a portion of the clusters has
                        members in the final sample.
            \end{itemize}
\end{itemize}

\textbf{Example 1.2}. Survey of the population of high school students in the
Waterloo region. There are a total of $15$ high schools. Take a sample
of $300$ students from the population.
\begin{itemize}
      \item \textbf{Plan A}. Randomly select $20$ students from each high school.\\
            (Stratified sampling)
      \item \textbf{Plan B}. Randomly select $5$ high schools from the list of $15$ schools,
            and then randomly select $60$ students from each of the $5$ selected
            schools.\\
            (Two-stage cluster sampling)
      \item \textbf{Plan C}. The Waterloo region can be divided into KW area ($8$ high
            schools) and non-KW area ($7$ high schools). First, randomly select $3$
            schools from the KW area and 2 schools from the non-KW area, then
            randomly select $60$ students from each of the $5$ selected schools.\\
            (Stratified two-stage cluster sampling)
\end{itemize}

\textbf{Sampling units and observational units}:
\begin{itemize}
      \item \emph{Sampling units}: Units used to select the survey sample.
            \begin{itemize}
                  \item Under clustering sampling, sampling units are the clusters.
                  \item Under non-clustering sampling, sampling units are the individual
                        units.
            \end{itemize}
      \item \emph{PSU and SSU}: Under two-stage cluster sampling, the first stage
            sampling units are clusters, called the \emph{primary sampling unit}
            (PSU); the second stage sampling units are individual units,
            called the \emph{secondary sampling unit} (SSU).
      \item \emph{Observational units}: Observational units are always the individual units from which
            measurements are taken.
\end{itemize}

\textbf{Example 1.3}. An educational worker wanted to find out the average
number of hours each week (of a certain month and year) spent on
watching television by four and five-year-old children in the Waterloo
Region. She conducted a survey using the list of $123$ pre-school
kindergartens administered by the Waterloo Region District School
Board. She first randomly selected $10$ kindergartens from the list.
Within each selected kindergarten, she was able to obtain a complete
list of all four and five-year-old children, with contact information for
their parents/guardians. She then randomly selected $50$ children from
the list and mailed the survey questionnaire to their parents/guardians.
The planned sample size is $ 10\times 50=500 $ and the sample data were
compiled from those who completed and returned the questionnaires.

\begin{itemize}
      \item \emph{The target population}:
            All four and five-year-old children in the Region of Waterloo at
            the time of the survey. This is defined by the overall objective of
            the study.
      \item \emph{Sampling frames}: Two-stage cluster sampling methods were used (further details to
            follow). The first stage sampling frame is the list of $123$
            kindergartens administered by the school board. The second
            stage sampling frames are the complete lists of all four and five-year-old
            children for the $10$ selected kindergartens.
      \item \emph{Sampling units and observational units}: The first stage sampling units are the kindergartens; the second
            stage sampling units are the individual children (or equivalently,
            their parents); observational units are individual children.
      \item \emph{The frame population}: All four and five-year-old children who attend one of the 123
            kindergartens in the Region of Waterloo. It is apparent that
            children who are homeschooled are not covered by the frame
            population. Thus, as is frequently the case, the frame population
            is not the same as the target population.
      \item \emph{The sampled population}: All four and
            five-year-old children who attend one of the $123$
            kindergartens in the Region of Waterloo and whose
            parents/guardians would complete and return the survey
            questionnaire if the child was selected for the survey.
\end{itemize}
