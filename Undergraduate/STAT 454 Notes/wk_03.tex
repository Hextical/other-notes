\makeheading{Lecture 4}{\printdate{2022-01-17}}%chktex 8
\textbf{Summary of theoretical results under SRSWOR}:
\begin{enumerate}[1]
      \item $ \E{\bar{y}}=\mu_y $.
      \item $ \V{\bar{y}}=(1-\frac{n}{N})\frac{\sigma_y^2}{n}=(\frac{1}{n}-\frac{1}{N})\sigma_y^2 $.
      \item $ \E{s_y^2}=\sigma_y^2 $.
      \item $ \E{\v{\bar{y}}}=\V{\bar{y}} $, where $ \v{\bar{y}}=(1-\frac{n}{N})\frac{s_y^2}{n} $.
\end{enumerate}
\textbf{Special cases where the response variable $y$ is binary}:
\begin{itemize}
      \item $ \mu_y=\frac{M}{N}=P $; $ \sigma_y^2=\frac{N}{N-1}P(1-P)\approx P(1-P) $ if $ N $ is large.
      \item $ \bar{y}=\frac{1}{n}\sum_{i\in S}y_i=\frac{m}{n}=p $.
            \begin{itemize}
                  \item $ m= $ \# units in $ S $ with attribute $ A $.
                  \item $ p=\frac{m}{n}= $ sample proportion.
            \end{itemize}
      \item $ s_y^2=\frac{n}{n-1}p(1-p)\approx p(1-p) $ if $ n $ is large.
      \item $ \E{p}=P $; $ \v{p}=(1-\frac{n}{N})\frac{1}{n}\frac{n}{n-1}p(1-p)=(1-\frac{n}{N})\frac{1}{n-1}p(1-p) $.
\end{itemize}

\section{Simple Random Sampling With Replacement (SRSWR)}
\textbf{The required sampling frame}:

A complete list of all $ N $ units in the population.

\textbf{The sampling procedure}:
\begin{enumerate}
      \item Select the first unit from the $N$ units on the sampling frame with
            equal probabilities $1/N$; denote the selected unit as $ i_1 $;
      \item Select the second unit from the $N$ units on the sampling frame
            with equal probabilities $1/N$; denote the selected unit as $ i_2 $;
      \item Continue the process and select the $n$th unit from the $N$ units on
            the sampling frame with equal probabilities $1/N$; denote the
            selected unit as $ i_n $.
\end{enumerate}
\textbf{Note}: SRSWR is not very useful in survey practice but has theoretical
values due to its connection to iid samples.

\textbf{Two possible treatments for SRSWR}:

\textbf{(1) Keep duplicated units}

Let $ S^*=\Set{i_1,i_2,\ldots,i_n} $. Under SRSWR, certain units might be included in
$ S^* $ more than once ($ S^* $ may include duplicated units).

Let $ Z_k=y_{ik} $ be the $ y $ value from the $ k $th selection, $ k=1,2,\ldots,n $. Let
the sample mean be computed as
\[ \bar{Z}=\frac{1}{n}\sum_{k=1}^{n}Z_k. \]
We have
\[ \E{\bar{Z}}=\mu_y\quad\text{ and }\quad \V{\bar{Z}}=\biggl(1-\frac{1}{N}\biggr)\frac{\sigma_y^2}{n}. \]
\begin{enumerate}[(i)]
      \item The $ Z_1,Z_2,\ldots,Z_n $ are iid random variables.
      \item The common probability function for $ Z_1,Z_2,\ldots,Z_n $:
            \[ \begin{NiceArray}{c|cccc}
                        Z_k          & y_1 & y_2 & \cdots & y_N \\
                        \midrule
                        f(\:\cdot\:) & 1/N & 1/N & \cdots & 1/N
                  \end{NiceArray}\iid \text{Discrete Uniform} \]
      \item The mean and variance of $ Z_k $:
            \begin{align*}
                  \E{Z_k}
                   & =\sum_{i=1}^{N}y_i\times \frac{1}{N}=\mu_y. \\
                  \V{Z_k}
                   & =\E*{(Z_k-\E{Z_k})^2}                       \\
                   & =\frac{1}{N}\sum_{i=1}^{N}(y_i-\mu_y)^2     \\
                   & =\frac{N-1}{N}\sigma_y^2                    \\
                   & =\biggl(1-\frac{1}{N}\biggr)\sigma_y^2.
            \end{align*}
            \[ \E{\bar{Z}}=\mu_y,\qquad \V{\bar{Z}}=\frac{1}{n}\V{Z_1}=\biggl(1-\frac{1}{N}\biggr)\frac{\sigma_y^2}{n}. \]
\end{enumerate}
\textbf{(2) Remove duplicated units}

Let $ S $ be the set of distinct units from SRSWR\@; let $ m=\abs{S} $ be the number of distinct units.

\textbf{Note}: $ m $ is a random number under SRSWR\@.

The sample mean based on the $ m $ distinct units is computed as
\[ \bar{y}_m=\frac{1}{m}\sum_{i\in S}y_i. \]
It can be shown that (Problem 2.2 of Chapter 2)
\[ \E{\bar{y}_m}=\mu_y\quad\text{ and }\quad \V{\bar{y}_m}=\biggl[\E*{\frac{1}{m}}-\frac{1}{N}\biggr]\sigma_y^2. \]
(Proof is required for Stat 854).

\textbf{Efficiency comparisons between SRSWOR and SRSWR}:

Three estimators of the population mean $ \mu_y $ (assume $ n\ge 2 $):
\begin{enumerate}
      \item $ \bar{y} $ under SRSWOR\@.
      \item $ \bar{Z} $ under SRSWR\@.
      \item $ \bar{y}_m $ under SRSWR\@.
\end{enumerate}
\begin{itemize}
      \item All three estimators are unbiased (first-order equivalence).
      \item $ \bar{y} $ is more efficient than the other two in terms of variance:
            \begin{align*}
                  \V{\bar{y}}=\biggl(1-\frac{n}{N}\biggr)\frac{\sigma_y^2}{n} & <\biggl(1-\frac{1}{N}\biggr)\frac{\sigma_y^2}{n}=\V{\bar{Z}}.        \\
                  \V{\bar{y}}=\biggl(\frac{1}{n}-\frac{1}{N}\biggr)\sigma_y^2 & <\biggl[\E*{\frac{1}{m}}-\frac{1}{N}\biggr]\sigma_y^2=\V{\bar{y}_m}.
            \end{align*}
\end{itemize}

\textbf{Efficiency comparisons through Monte Carlo simulation studies}:
\begin{enumerate}
      \item Generate a finite population of size $ N $, $ \Set{y_1,y_2,\ldots,y_N} $ (from any distribution),
            and compute $ \mu_y $: This is the ``unknown'' population mean.
      \item Take a sample $ S $ of size $ n $, and obtain the sample data $ \Set{y_i,i\in S} $;
            compute the estimate $ \hat{\mu}_1 $ for the estimator $ \hat{\mu}_y $.
      \item Repeat (2) a large number $ K $ ($ \ge 1000 $) times, independently, to obtain $ \hat{\mu}_1,\hat{\mu}_2,\ldots,\hat{\mu}_K $.
      \item Evaluate the performance of the estimator $ \hat{\mu}_y $ using the relative bias (RB, in \%) and the mean squared
            error (MSE) from the simulation:
            \begin{itemize}
                  \item RB (in \%):
                        \[ RB=\frac{1}{K}\sum_{k=1}^{K}\frac{\hat{\mu}_k-\mu_y}{\mu_y}\times 100, \]
                        if $ \mu_y=0 $ then we use the regular bias.
                  \item MSE\@:
                        \[ MSE=\frac{1}{K}\sum_{k=1}^{K}(\hat{\mu}_k-\mu_y)^2=\text{Var}+\text{Bias}^2. \]
            \end{itemize}
            (The MSE $ \approx $ the variance if the RB is very small (i.e., $ <1\% $)).
\end{enumerate}

\textbf{A simulation example in R comparing $ \bar{y} $ and $ \bar{Z} $}:

\begin{minted}{R}
set.seed(1234567,kind=NULL) #Results duplicable!
N=1000
n=200
Y=rexp(N)
muy=mean(Y)
RB=c(0,0)
MSE=c(0,0)
K=1000
for(k in 1:K){
sam1=sample(N,n)
ysam1=Y[sam1]
sam2=sample(N,n,replace=T)
ysam2=Y[sam2]
mu1=mean(ysam1)
mu2=mean(ysam2)
RB[1]=RB[1]+mu1-muy
RB[2]=RB[2]+mu2-muy
MSE[1]=MSE[1]+(mu1-muy)^2
MSE[2]=MSE[2]+(mu2-muy)^2
}
RB=(RB/(K*muy))*100
MSE=MSE/K
> RB
[1] 0.09909944 -0.45677068
> MSE
[1] 0.003621282 0.004583952
    \end{minted}
\textbf{Re-do the simulation with $N = 20000$ and $n = 200$}:
\begin{minted}{R}
> RB
[1] 0.2398581 0.1099942
> MSE
[1] 0.005152047 0.005227150
\end{minted}

\textbf{Note}: The rule of thumb on how many decimal points to be reported
\begin{itemize}
      \item For RB in percentages, two decimal points, i.e., $0.10\%$ and
            $-0.46\%$ from the 1st example.
      \item For MSE, use two or three nearest decimal points to reflect the
            difference, i.e., $0.0036$ and $0.0046$ from the 1st example.
\end{itemize}
\textbf{Homework}:
\begin{itemize}
      \item Install the R package on your laptop \href{https://www.r-project.org}{https://www.r-project.org}.
      \item Re-run the simulation study with a different seed for the random
            number generator and compare the results.
      \item Re-run the simulation study with fixed $n = 200$ and different
            sampling fractions $n/N = 1\%, 2\%, 5\%, 10\%$ and compare the
            results.
      \item \textbf{Challenge part}: Include $ \bar{y}_m $ in the simulation and compare the
            results.
\end{itemize}

\makeheading{Lecture 5}{\printdate{2022-01-19}}%chktex 8
\section{Central Limit Theorem and Confidence Intervals}
\textbf{Asymptotic framework for finite populations}\\
(A frame to allow $ n\to\infty $):

We assume there is a sequence of finite populations (indexed by $ \nu $)
and an associated sequence of survey samples. Both the population size $ N_\nu $ and the sample
size $ n_\nu $ go to infinite as $ \nu\to\infty $. The particular finite population and the survey
sample are part of the sequence.

We use $ n\to\infty $ or $ N\to\infty $, but the limiting process is under $ \nu\to\infty $.

For stratified populations, there are two versions of the asymptotic framework:
\begin{itemize}
      \item The total number of strata is bounded, but the stratum population sizes grow to infinity
            for the sequence of populations.
      \item The stratum population sizes are bounded, but the total number of
            strata goes to infinity for the sequence of populations.
\end{itemize}

\textbf{The Hájek Theorem (1960)}

Suppose that the sampling fraction $ n/N\to f\in(0,1) $ as $ n\to\infty $.

Suppose also that the population values of the response variable $ y $ satisfy
\[ \lim\limits_{{N} \to {\infty}}\frac{\max_{1\le i\le N}{(y_i-\mu_y)^2}}{\sum_{i=1}^{N}(y_i-\mu_y)^2}=0. \]
Then under SRSWOR, the Wald-type statistic
\[ \frac{\bar{y}-\mu_y}{\sqrt{\v{\bar{y}}}}\tod \N{0,1}, \]
as $ n\to\infty $, where $ \v{\bar{y}}=\bigl(1-\frac{n}{N}\bigr)\frac{s_y^2}{n} $ is the estimated variance of $ \bar{y} $.

We also have
\[ \frac{\bar{y}-\mu_y}{\sqrt{\V{\bar{y}}}}\tod \N{0,1}, \]
as $ n\to\infty $, where $ \V{\bar{y}}=\bigl(1-\frac{n}{N}\bigr)\frac{\sigma_y^2}{n} $ is the theoretical variance of $ \bar{y} $.

Note that
\[ \frac{\bar{y}-\mu_y}{\sqrt{\V{\bar{y}}}}=\frac{\bar{y}-\mu_y}{\sqrt{\v{\bar{y}}}}\cdot \frac{\sqrt{\v{\bar{y}}}}{\sqrt{\V{\bar{y}}}}\quad\text{and}\quad \frac{\sqrt{\v{\bar{y}}}}{\sqrt{\V{\bar{y}}}}\inp 1. \]

\section{Sample Size Calculation}

One of the major questions for survey design and planning: How large
should the sample size $n$ be? The answer depends on three factors:
\begin{itemize}
      \item The total budget for the survey.
      \item The cost for surveying one unit and taking all required
            measurements.
      \item The accuracy required for the main statistical inference problem
            from the survey data.
\end{itemize}
The answer also depends on the sampling methods: More efficient
sampling methods require a smaller sample size to achieve the same
goal. We discuss sample size calculation under the simple scenario
where
\begin{itemize}
      \item The sampling method is SRSWOR\@.
      \item The accuracy requirements are for estimating the population
            mean.
\end{itemize}


\textbf{(1) Accuracy specified by the absolute tolerable error}

We want the estimator $ \bar{y} $ for estimating the parameter $ \mu_y $ to satisfy
\[ \Prob[\big]{\abs{\bar{y}-\mu_y}\ge e}\le \alpha, \]
or equivalently,
\[ \Prob[\big]{\abs{\bar{y}-\mu_y}< e}\le 1-\alpha, \]
for a given $ \alpha\in(0,1) $ and a pre-specified error margin $ e $. What is the required $ n $?

We assume that
\[ \frac{\bar{y}-\mu_y}{\sqrt{\V{\bar{y}}}}\tod \N{0,1} \]
can be used as an approximation to derive the required sample size.

We compare
\[ \Prob*{\frac{\abs{\bar{y}-\mu_y}}{\sqrt{\V{\bar{y}}}}<\frac{e}{\V{\bar{y}}}}\ge 1-\alpha \]
with
\[ \Prob[\big]{\abs{Z}<Z_{\alpha/2}}=1-\alpha, \]
where $ Z \sim \N{0,1} $ and $ Z_{\alpha/2} $ is the upper $ \alpha/2 $ quantile of $ \N{0,1} $.

\begin{align*}
      \frac{e}{\sqrt{\V{\bar{y}}}}=Z_{\alpha/2}\implies\V{\bar{y}} & =\frac{e^2}{Z_{\alpha/2}^2}  \\
      \biggl(\frac{1}{n}-\frac{1}{N}\biggr)\sigma_y^2              & =\frac{e^2}{Z_{\alpha/2}^2}.
\end{align*}
Doing some algebra,
\begin{align*}
      n   & =\frac{Z_{\alpha/2}^2\sigma_y^2/e^2}{1+(Z_{\alpha/2}^2\sigma_y^2/e^2)/N}=\frac{n_0}{1+n_0/N}<n_0 \\
      n_0 & =Z_{\alpha/2}^2\sigma_y^2/e^2,
\end{align*}
where $n\approx n_0 $ for large $ N $ ($ N=+\infty $).

\textbf{(2) Accuracy specified by the relative tolerable error}

Suppose that $ \mu_y\ne 0 $. We want the estimator $ \bar{y} $ satisfies
\[ \Prob*{\frac{\abs{\bar{y}-\mu_y}}{\abs{\mu_y}}\ge e}\le \alpha. \]
What is the required $ n $?

Why is sometimes relative tolerable error preferred?

The absolute tolerable error $ e $ specified in $ \abs{\bar{y}-\mu_y}<e $ is scale-dependent.
The choice of $ e $ in the relative tolerable error is scale-free, and can easily be decided as, for instance,
0.01--0.03 (that is, 1\%--3\%).

The accuracy requirement can be re-written as
\[ \Prob[\big]{\abs{\bar{y}-\mu_y}\ge e^*}\le \alpha, \]
where $ e^*=e\abs{\mu_y} $. The required sample size $ n $ is given by
\[ n=\frac{n_0}{1+n_0/N}. \]
\begin{align*}
      n_0
       & =Z_{\alpha/2}^2\sigma_y^2/(e^*)^2                           \\
       & =Z_{\alpha/2}^2\biggl(\frac{\sigma_y^2}{\mu_y^2}\biggr)/e^2 \\
       & =Z_{\alpha/2}^2\bigl[\CV{y}\bigr]^2/e^2,
\end{align*}
where $ \CV{y}=\frac{\sigma_y}{\mu_y} $.

A useful result: if $ y_i=a x_i $ for all $ i $, then $ \CV{y}=\CV{x} $.

\textbf{Notes on sample size calculations}:
\begin{itemize}
      \item The question of sample size calculation or sample size
            determination is part of the survey planning; the actual survey
            sample data are not available at this stage.
      \item Formulas for sample size calculations typically involve unknown
            population quantities such as $ \mu_y $ and $ \sigma_y^2 $.
      \item How to obtain the required population information to calculate
            $n$?
            \begin{itemize}
                  \item Existing data sources:
                        \begin{itemize}
                              \item Other similar surveys.
                              \item Census data.
                        \end{itemize}
                  \item Pilot surveys
                        \begin{itemize}
                              \item Do a small survey first ($ n=50 $?)
                        \end{itemize}
            \end{itemize}
      \item The population information for sample size calculations does not
            need to be very accurate, because the calculated n is used for
            survey planning, which needs to be further adjusted by cost and
            other factors.
\end{itemize}

\textbf{Example 2.1. Sample size calculation for estimating a population proportion}

Suppose that the goal is to estimate the population proportion $ P=M/N $
using a survey sample to be selected by SRSWOR\@. Using the sample proportion $ p=m/n $
to estimate $ P $, a common absolute tolerable error is 3\% and the $ \alpha $ is set to $0.05$. In other words,
the estimation accuracy is specified as
\[ \Prob[\big]{\abs{p-P}\le 0.03}\ge 0.95. \]
Noting that $ 0.95=19/20 $, the probability statement is often quoted in media reports as
``\emph{The result is accurate within three percentage points, 19 times out of 20}.''

What is the required sample size $ n $?

\[ n=\frac{n_0}{1+n_0/N}<n_0. \]
$ n_0 $ is a conservative choice for any $ N $.
\[ \sigma_y^2\approx P(1-P)\le \frac{1}{4}. \]
\begin{align*}
      n_0
       & =Z_{\alpha/2}^2\sigma_y^2/e^2       \\
       & =1.96^2\sigma_y^2/0.03^2            \\
       & \le 1.96^2\times \frac{1}{4}/0.03^2 \\
       & \approx 1067.
\end{align*}
