
\makeheading{Lecture 10}{\printdate{2022-10-21}}%chktex 8
\underline{Author's Note}:
Missed lecture due to convocation. The following lecture is
typed \underline{after} the lecture, and the notes were
sourced from \textbf{Zhang Yiran} (main source), Baek Inwook, and Zhang Zhiyue.

\begin{Definition}{Polya's Urn}{}
    Start with one \textcolor{red}{red} ($ \R $) and one
    \textcolor{blue}{blue} ($ \B $) ball.
    At each step, select a
    ball at random, then put it back into the urn along with an
    additional ball of the same colour.
    \tcblower{}
    \textbf{Question}: Does the percentage of $ \B $ converge?
    If so, to what number?
\end{Definition}
\begin{Example}{Order in Polya's Urn is Irrelevant}{}
    \begin{align*}
        \ProbB{\R\B\B\B\R\B\R}
         & =\frac{1}{2}\frac{1}{3}\frac{2}{4}\frac{3}{5}\frac{2}{6}\frac{4}{7}\frac{3}{8} \\
         & =\frac{4!3!}{8!}.                                                              \\
        \ProbB{\R\R\R\B\B\B\B}
         & =\frac{1}{2}\frac{2}{3}\frac{3}{4}\frac{1}{5}\frac{2}{6}\frac{3}{7}\frac{4}{8} \\
         & =\frac{4!3!}{8!}.
    \end{align*}
    Therefore, the two sequences are exchangeable; that is,
    order of $ \R $ and $ \B $ is irrelevant.
\end{Example}
\begin{Example}{}{}
    \begin{align*}
        \ProbB{3\R+4\B\text{ in the first $7$ picks}}
         & =\frac{4!3!}{8!}\binom{7}{3}=\frac{1}{8},
    \end{align*}
    where we multiplied by $ \binom{7}{3} $ because this is the
    number of ways to make a sequence of $ 3\R4\B $. Also,
    \[ \ProbB{1\R+6\B}=\frac{1!6!}{8!}\binom{7}{1}=\frac{1}{8}. \]
\end{Example}
\begin{Example}{Random Spinner Game}{}
    Suppose $ X_i \iid\text{Uniform}[0,1] $
    is independent of $ Y $ for $ i\ge 1 $, and define
    \[ C_i=\begin{cases}
            \B, & X_i<Y,    \\
            \R, & X_i\ge Y.
        \end{cases} \]
    \underline{Remarks}:
    \begin{itemize}
        \item $ \ProbB{\text{$1\textsuperscript{st}$ is }\B}=
                  \ProbB{C_1=\B}=\ProbB{X_1<Y} $.
        \item Since $ X_1 $ and $ Y $ are iid,
              \[ \ProbB{X_1<Y}=\ProbB{Y<X_1}. \]
        \item Since $ X_1 $ and $ Y $ are continuous and independent,
              \[ \ProbB{X_1=Y}=0. \]
    \end{itemize}
    Using these facts, we have
    \[ 2\ProbB{X_1<Y}=\ProbB{X_1<Y}+\ProbB{Y<X_1}=1-\ProbB{X_1=Y}=1. \]
    Therefore, $ \ProbB{X_1<Y}=1/2 $, that is $ \Prob{C_1=\B}=1/2 $.
    \begin{align*}
        \Prob{\Set{C_2=\B}\given \Set{C_1=\B}}
         & =\frac{\ProbB{C_1=\B,C_2=\B}}{\ProbB{C_1=\B}} \\
         & =\frac{\ProbB{C_1=C_2=\B}}{\ProbB{C_1=\B}}    \\
         & =\frac{1/3}{1/2}                              \\
         & =\frac{2}{3},
    \end{align*}
    where the numerator was calculated via
    \begin{align*}
        \ProbB{C_1=C_2=\B}
         & =\ProbB{X_1<Y,X_2<Y}                                               \\
         & =\int_{0}^{1}\int_{0}^{y}\int_{0}^{y}1\odif{x_1}\odif{x_2}\odif{y} \\
         & =\frac{1}{3}.
    \end{align*}
    Therefore, $ \Prob{\Set{C_2=\B}\given \Set{C_1=\B}}=\frac{1/3}{1/2}=\frac{2}{3} $.

    For the same reason as before,
    \[
        \ProbB{X_1<X_2<Y}=\ProbB{X_2<X_1<Y}=\cdots=\ProbB{Y<X_1<X_2}=\frac{1}{3!}=\frac{1}{6}.
    \]
    \begin{align*}
        \Prob{\Set{C_9=\B}\given \Set{\B\B\R\B\R\R\B\B}}
         & =\ProbB{X_1,X_2,X_4,X_7,X_8<Y,\; X_3,X_5,X_6>Y} \\
         & =\frac{5!3!}{9!}.
    \end{align*}
    The random spinner game is the same process as Polya's urn.

    \begin{itemize}
        \item Conditionally given $ Y $, the $ C_i $'s are
              independent each with probability $ Y $ of being $ \B $.
        \item By the law of large numbers, the percentage of
              $ \B $ picks converges to $ Y $.
    \end{itemize}
\end{Example}
\begin{Theorem}{De Finetti's Theorem for Polya's Urn}{}
    The percentage of $ \B $ picks converges almost surely
    (100\% probability to converge). Let $ Y $ denote the
    limit,
    \begin{itemize}
        \item $ Y \sim \text{Uniform}[0,1] $.
        \item Given $ Y $, the picks are conditionally
              independent each with probability $ Y $ of being
              $ \B $.
    \end{itemize}
\end{Theorem}
\begin{Example}{}{}
    The conditional cdf of $ Y $ given $ C_1=\B $ is
    \begin{align*}
        F_{Y\mid C_1=\B}
         & =\Prob{\Set{Y\le t}\given \Set{C_1=\B}}                      \\
         & =\frac{\Prob{\Set{Y\le t}\cap \Set{C_1=\B}}}{\ProbB{C_1=\B}} \\
         & =\frac{t^2/2}{1/2}                                           \\
         & =t^2,
    \end{align*}
    where numerator is calculated via
    \[ \Prob{\Set{Y\le t}\cap\Set{C_1=\B}}
        =\int_{0}^{t}\int_{0}^{y}1\odif{x_1}\odif{y}=\frac{t^2}{2}.
    \]
    The conditional pdf of $ Y $ given $ C_1=\B $ is
    \[ f_{Y\mid C_1=\B}(t)
        =\begin{cases} 2t, & t\in[0,1],        \\
              0,  & \text{otherwise}.\end{cases} \]
\end{Example}
\begin{Theorem}{Law of Total Probability (Continuous)}{}
    If $ Y $ is a continuous random variable with pdf $
        f_Y $, then for any event $ A $,
    \[ \Prob{A}=\int_{-\infty}^{\infty}\Prob{A\given
            \Set{Y=y}}f_Y(y)\odif{y}, \]
    given we can make sense of the conditional probability.
\end{Theorem}
\begin{Example}{}{}
    Using the Law of Total Probability, the conditional cdf of
    $ Y $ given $ \B\B\R\B\R\R\B\B $
    \begin{align*}
        F_{Y\mid \B\B\R\B\R\R\B\B}(t)
         & =\Prob{\Set{Y\le t}\cap\Set{\B\B\R\B\R\R\B\B}}      \\
         & =\int_{0}^{t}\Prob{\Set{\B\B\R\B\R\R\B\B}\given
        \Set{Y=y}}\odif{y}                                     \\
         & =\int_{0}^{t}\frac{y^5(1-y)^3}{(5!3!)/(9!)}\odif{y} \\
         & =\frac{9!}{5!3!}\int_{0}^{t}y^5(1-y)^3\odif{y}.
    \end{align*}
    The conditional pdf of $ Y $ given $ \B\B\R\B\R\R\B\B $ is
    \[
        f_{Y\mid \B\B\R\B\R\R\B\B}(t)
        =\begin{dcases} \frac{9!}{5!3!}t^5(1-t)^3, & t\in[0,1],        \\
               0,                         & \text{otherwise},\end{dcases} \]
    which is the
    $ \BetaDist{6,4} $ distribution.
\end{Example}