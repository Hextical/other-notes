\makeheading{Lecture 11}{\printdate{2023-02-13}}%chktex 8
\section{Lecture 11: \texorpdfstring{$ g $}{g}-inverse}
\subsection*{Algorithm for Finding a $ g $-inverse}
Let $ \Matrix{A}\in\R^{n\times k} $ with $ \rank{\Matrix{A}}=r<\min{n,k} $.
\begin{itemize}
    \item Step 1: Find an invertible sub-matrix $ \Matrix{M}\in\R^{r\times r} $.
    \item Step 2: Compute $ (\Matrix{M}^{-1})' $.
    \item Step 3: Replace $ \Matrix{M} $ with $ (\Matrix{M}^{-1})' $ in $ \Matrix{A} $.
    \item Step 4: Set all other elements in $ \Matrix{A} $ to be $ 0 $.
    \item Step 5: Transpose the resulting matrix to $ \Matrix{G}\in\R^{k\times n} $.
\end{itemize}
\begin{Example}{}{}
    Compute a $ g $-inverse of
    $ \Matrix{A}=\begin{pmatrix}
            4 & 1 & 2 & 0  \\
            1 & 1 & 5 & 15 \\
            3 & 1 & 3 & 5
        \end{pmatrix} $.
    \tcblower{}
    \textbf{Solution}: Note that $ n=3 $ and $ k=4 $. Let
    \begin{align*}
        \Vector{v}_1 & =\begin{pmatrix}
                            4 & 1 & 2 & 0
                        \end{pmatrix}, \\
        \Vector{v}_2 & =\begin{pmatrix}
                            1 & 1 & 5 & 15
                        \end{pmatrix}, \\
        \Vector{v}_3 & =\begin{pmatrix}
                            3 & 1 & 3 & 5
                        \end{pmatrix}.
    \end{align*}
    $ \Vector{v}_1 $ and $ \Vector{v}_2 $
    are linearly independent since
    \[ a \Vector{v}_1+b \Vector{v}_2=\Vector{0}\implies 15b=0\text{ and }4a=0\implies a=b=0. \]
    Also, $ 3 \Vector{v}_3=2 \Vector{v}_1+\Vector{v}_2 $.
    Therefore, $ \rank{\Matrix{A}}=2 $. Now,
    \begin{itemize}
        \item Step 1:
              \[ \Matrix{M}=\begin{pmatrix}
                      4 & 0 \\
                      3 & 5
                  \end{pmatrix}. \]
        \item Step 2:
              \[ (\Matrix{M}^{-1})'=\frac{1}{20}\begin{pmatrix}
                      5 & -3 \\
                      0 & 4
                  \end{pmatrix}=
                  \begin{pmatrix}
                      5/20 & -3/20 \\
                      0    & 4/20
                  \end{pmatrix}. \]
        \item Step 3:
              \[ \begin{pmatrix}
                      5/20 & 1 & 2 & -3/20 \\
                      1    & 1 & 5 & 15    \\
                      0    & 1 & 3 & 4/20
                  \end{pmatrix}. \]
        \item Step 4:
              \[ \begin{pmatrix}
                      5/20 & 0 & 0 & -3/20 \\
                      0    & 0 & 0 & 0     \\
                      0    & 0 & 0 & 4/20
                  \end{pmatrix}. \]
        \item Step 5:
              \[ \begin{pmatrix}
                      5/20  & 0 & 0    \\
                      0     & 0 & 0    \\
                      0     & 0 & 0    \\
                      -3/20 & 0 & 4/20
                  \end{pmatrix}. \]
    \end{itemize}
    Verify that $ \Matrix{AGA}=\Matrix{A} $.
\end{Example}
\begin{Theorem}{}{}
    Let $ \Matrix{A}\in\R^{n\times k} $ with $ \rank{\Matrix{A}}=r<\min{n,k} $ and $ \Matrix{G} $
    be a $ g $-inverse of $ \Matrix{A} $. Let $ \Matrix{F} $
    be a $ g $-inverse of $ \Matrix{A}'\Matrix{A} $. Then,
    \begin{enumerate}[(1)]
        \item $ \Matrix{G}' $ is a $ g $-inverse of $ \Matrix{A}' $.
        \item $ \rank{\Matrix{GA}}=\rank{\Matrix{AG}}=\rank{\Matrix{A}}=r $.
        \item $ \Matrix{A}=\Matrix{A}\Matrix{F}\Matrix{A}'\Matrix{A} $ and $ \Matrix{A}'=\Matrix{A}'\Matrix{A}\Matrix{F}\Matrix{A}' $. This means
              that $ \Matrix{FA}' $ is a $ g $-inverse of $ \Matrix{A} $.
    \end{enumerate}
    \tcblower{}
    \textbf{Proof}:
    \begin{enumerate}[(1)]
        \item Since $ \Matrix{AGA}=\Matrix{A} $, we have that $ (\Matrix{AGA})'=\Matrix{A}'\Matrix{G}'\Matrix{A}'=\Matrix{A}' $.
              Therefore, $ \Matrix{G}' $ is a $ g $-inverse of $ \Matrix{A}' $.
        \item Since $ \Matrix{AGA}=\Matrix{A} $, $ \rank{\Matrix{A}}\le \rank{\Matrix{GA}}\le \rank{\Matrix{A}} $. Similarly,
              $ \rank{\Matrix{A}}\le \rank{\Matrix{AG}}\le \rank{\Matrix{A}} $. Therefore,
              \[ \rank{\Matrix{AG}}=\rank{\Matrix{GA}}=\rank{\Matrix{A}}. \]
        \item Since $ \Matrix{F} $ is a $ g $-inverse of $ \Matrix{A}'\Matrix{A} $, we have
              $ \Matrix{A}'\Matrix{A}\Matrix{F}\Matrix{A}'\Matrix{A}=\Matrix{A}'\Matrix{A} $. Rearranging,
              \begin{align*}
                  \Matrix{A}'\Matrix{A}\Matrix{F}\Matrix{A}'\Matrix{A}-\Matrix{A}'\Matrix{A} & =\Matrix{O}  \\
                  (\Matrix{A}'\Matrix{A}\Matrix{F}\Matrix{A}'-\Matrix{A}')\Matrix{A}         & =\Matrix{O}  \\
                  \Matrix{A}'(\Matrix{A}\Matrix{F}\Matrix{A}'\Matrix{A}-\Matrix{A})          & =\Matrix{O}.
              \end{align*}
              Note that $ (\Matrix{A}\Matrix{F}\Matrix{A}'\Matrix{A})'=\Matrix{A}'\Matrix{A}\Matrix{F}'\Matrix{A} $ and
              \begin{align*}
                  \Matrix{A}'\Matrix{A}\Matrix{F}'\Matrix{A}'(\Matrix{A}\Matrix{F}\Matrix{A}'\Matrix{A}-\Matrix{A})
                   & =\Matrix{A}'\Matrix{A}\Matrix{F}'\underbrace{\Matrix{A}'\Matrix{A}\Matrix{F}\Matrix{A}'\Matrix{A}}_{\Matrix{A}'\Matrix{A}}-\Matrix{A}'\Matrix{A}\Matrix{F}'\Matrix{A}'\Matrix{A} \\
                   & =\Matrix{A}'\Matrix{A}\Matrix{F}'\Matrix{A}'\Matrix{A}-\Matrix{A}'\Matrix{A}\Matrix{F}'\Matrix{A}'\Matrix{A}                                                                     \\
                   & =\Matrix{O}.
              \end{align*}
              Therefore,
              \[ (\Matrix{A}\Matrix{F}\Matrix{A}'\Matrix{A}-\Matrix{A})'(\Matrix{A}\Matrix{F}\Matrix{A}'\Matrix{A}-\Matrix{A})=\Matrix{O}. \]
              Hence,
              \[ \Matrix{A}\Matrix{F}\Matrix{A}'\Matrix{A}-\Matrix{A}=\Matrix{O}\implies \Matrix{A}\Matrix{F}\Matrix{A}'\Matrix{A}=\Matrix{A}. \]
              Similarly, $ \Matrix{A}'\Matrix{A}\Matrix{F}\Matrix{A}'\Matrix{A}=\Matrix{A}'\Matrix{A} $, which implies
              \[ (\Matrix{A}'\Matrix{A}\Matrix{F}\Matrix{A}'-\Matrix{A}')\Matrix{A}=\Matrix{O}. \]
              By direct calculation,
              \[
                  (\Matrix{A}'\Matrix{A}\Matrix{F}\Matrix{A}'-\Matrix{A}')\Matrix{A}\Matrix{F}'\Matrix{A}'\Matrix{A}
                  =\Matrix{A}'\Matrix{A}\Matrix{F}'\Matrix{A}'\Matrix{A}-\Matrix{A}'\Matrix{A}\Matrix{F}'\Matrix{A}'\Matrix{A}
                  =\Matrix{O}. \]
              Therefore,
              \[ (\Matrix{A}'\Matrix{A}\Matrix{F}\Matrix{A}'-\Matrix{A}')(\Matrix{A}'\Matrix{A}\Matrix{F}\Matrix{A}'-\Matrix{A}')'=\Matrix{O}. \]
              Hence,
              \[ \Matrix{A}'\Matrix{A}\Matrix{F}\Matrix{A}'=\Matrix{A}'. \]
              $ \Matrix{A}\Matrix{F} $ is a $ g $-inverse of $ \Matrix{A}' $ and $ \Matrix{F}\Matrix{A}' $ is a $ g $-inverse of $ \Matrix{A}. $
    \end{enumerate}
\end{Theorem}
\begin{Theorem}{}{}
    Let $ \Matrix{F} $ be a $ g $-inverse of $ \Matrix{A}'\Matrix{A} $.
    \begin{enumerate}[(1)]
        \item $ \Matrix{F}' $ is a $ g $-inverse of $ \Matrix{A}'\Matrix{A} $.
        \item $ \rank{\Matrix{A}\Matrix{F}\Matrix{A}'}=\rank{\Matrix{A}} $.
        \item Let $ \tilde{\Matrix{F}} $ be any $ g $-inverse of $ \Matrix{A}'\Matrix{A} $,
              then
              $ \Matrix{A}'\Matrix{FA}=\Matrix{A}'\tilde{\Matrix{F}}\Matrix{A} $.
        \item $ \Matrix{A}\Matrix{F}\Matrix{A}' $ is symmetric.
    \end{enumerate}
    \tcblower{}
    \textbf{Proof}:
    \begin{enumerate}[(1)]
        \item Using Theorem 11.1, $ \Matrix{A}'\Matrix{A}\Matrix{F}\Matrix{A}'\Matrix{A}=\Matrix{A}'\Matrix{A} $, so
              \[ \Matrix{A}'\Matrix{A}=(\Matrix{A}'\Matrix{A})'=\Matrix{A}'\Matrix{A}\Matrix{F}'\Matrix{A}'\Matrix{A}. \]
        \item By Theorem 11.1, we have $ \Matrix{A}=\Matrix{A}\Matrix{F}\Matrix{A}'\Matrix{A} $. It follows that
              \[ \rank{\Matrix{A}}\le \rank{\Matrix{A}\Matrix{F}\Matrix{A}'}\le \rank{\Matrix{A}}. \]
              Therefore, $ \rank{\Matrix{AFA}'}=\rank{\Matrix{A}} $.
        \item Let $ \tilde{\Matrix{F}} $  be any $ g $-inverse of $ \Matrix{A}'\Matrix{A} $. Then,
              $ \Matrix{A}=\Matrix{A}\Matrix{F}\Matrix{A}'\Matrix{A}=\Matrix{A}\tilde{\Matrix{F}}\Matrix{A}'\Matrix{A} $ by Theorem 11.1, so
              \[ (\Matrix{A}\Matrix{F}\Matrix{A}'-\Matrix{A}\tilde{\Matrix{F}}\Matrix{A}')\Matrix{A}=\Matrix{O}. \]
              Therefore,
              \begin{align*}
                  (\Matrix{A}\Matrix{F}\Matrix{A}'-\Matrix{A}\tilde{\Matrix{F}}\Matrix{A}')(\Matrix{A}\Matrix{F}\Matrix{A}'-\Matrix{A}\tilde{\Matrix{F}}\Matrix{A}')'
                   & =(\Matrix{A}\Matrix{F}\Matrix{A}'-\Matrix{A}\tilde{\Matrix{F}}\Matrix{A}')(\Matrix{A}\Matrix{F}'\Matrix{A}'-\Matrix{A}\tilde{\Matrix{F}}'\Matrix{A}')'                                                    \\
                   & =(\Matrix{A}\Matrix{F}\Matrix{A}'-\Matrix{A}\tilde{\Matrix{F}}\Matrix{A}')\Matrix{A}(\Matrix{F}'\Matrix{A}'-\tilde{\Matrix{F}}'\Matrix{A}')                                                               \\
                   & =(\underbrace{\Matrix{A}\Matrix{F}\Matrix{A}'\Matrix{A}}_{\Matrix{A}}-\underbrace{\Matrix{A}\tilde{\Matrix{F}}\Matrix{A}'\Matrix{A}}_{\Matrix{A}})(\Matrix{F}'\Matrix{A}'-\tilde{\Matrix{F}}'\Matrix{A}') \\
                   & =\Matrix{O}.
              \end{align*}
              Hence, $ \Matrix{A}\Matrix{F}\Matrix{A}'=\Matrix{A}\tilde{\Matrix{F}}\Matrix{A}' $.
        \item By (1), $ \Matrix{F}' $ is a $ g $-inverse of $ \Matrix{A}'\Matrix{A} $. Hence,
              $ \Matrix{A}\Matrix{F}\Matrix{A}'=\Matrix{A}\Matrix{F}'\Matrix{A}'=(\Matrix{A}\Matrix{F}\Matrix{A}')' $.
              Therefore, $ \Matrix{A}\Matrix{F}\Matrix{A}' $ is symmetric.
    \end{enumerate}
\end{Theorem}
\begin{Theorem}{}{}
    Let $ \Matrix{A}\in\R^{n\times k} $. Consider the system of equations
    \[ \Matrix{A}\Vector{x}=\Vector{y}. \]
    \begin{enumerate}[(1)]
        \item If $ \Vector{x}_0 $ is a solution of the system of equations, then
              $ \Matrix{G}\Matrix{A}\Vector{x}_0 $ is also a solution of the system of equations for any $ g $-inverse
              $ \Matrix{G} $ of $ \Matrix{A} $.
        \item Let $ \Matrix{G} $ be a $ g $-inverse of $\Matrix{A}$, then for any $ \Vector{z}\in\R^{k} $,
              \[ \Matrix{G}\Vector{y}+(\Matrix{G}\Matrix{A}-\Matrix{I})\Vector{z} \]
              is a solution of the system of equations.
        \item Every solution can be written in the form of (2).
    \end{enumerate}
    \tcblower{}
    \textbf{Proof}:
    \begin{enumerate}[(1)]
        \item $ \Vector{x}_0 $ is a solution implies that $ \Matrix{A}\Vector{x}_0=\Vector{y} $. However,
              \[ \Matrix{A}(\Matrix{G}\Matrix{A}\Vector{x}_0)=(\Matrix{AGA})\Vector{x}_0 =\Matrix{A}\Vector{x}_0 =\Vector{y}. \]
        \item Note that
              \[ \Set{ \Matrix{G}\Vector{y}+(\Matrix{G}\Matrix{A}-\Matrix{I})\Vector{z}\given \Vector{z}\in\R^k}. \]
              So,
              \begin{align*}
                  \Matrix{A}(\Matrix{G}\Vector{y}+(\Matrix{G}\Matrix{A}-\Matrix{I})\Vector{z})
                   & =\Matrix{AG}\Vector{y}+(\Matrix{AGA}\Vector{z}-\Matrix{A}\Vector{z}) \\
                   & =\Matrix{AG}\Vector{y}                                               \\
                   & =\Matrix{AG}(\Matrix{A}\Vector{x})                                   \\
                   & =\Matrix{AGA}\Vector{x}                                              \\
                   & =\Matrix{A}\Vector{x}                                                \\
                   & =\Vector{y}.
              \end{align*}
        \item Let $ \Vector{x}_0 $ be any solution, so $ \Matrix{A}\Vector{x}_0=\Vector{y} $.
              Choose $ \Vector{z}=(\Matrix{GA}-\Matrix{I})\Vector{x}_0 $.
              \begin{align*}
                  \Matrix{G}\Vector{y}+(\Matrix{GA}-\Matrix{I})\Vector{z}
                   & =\Matrix{G}\Vector{y}+(\Matrix{GAGA}-2 \Matrix{GA}+\Matrix{I})\Vector{x}_0      \\
                   & =\Matrix{G}\Vector{y}+(-\Matrix{GA}+\Matrix{I})\Vector{x}_0                     \\
                   & =\Matrix{G}\Vector{y}-\Matrix{GA}\Vector{x}_0+\Vector{x}_0                      \\
                   & =\Matrix{G}\Matrix{A}\Vector{x}_0-\Matrix{G}\Matrix{A}\Vector{x}_0+\Vector{x}_0 \\
                   & =\Vector{x}_0.
              \end{align*}
    \end{enumerate}
\end{Theorem}