\makeheading{Lecture 22}{\printdate{2022-06-27}}%chktex 8
\begin{Corollary}{}{}
    If $ p $ is an odd prime, then
    \[ \legendre{-1}{p}=\begin{cases}
            1,  & p\equiv 1\Mod{4}, \\
            -1, & p\equiv 3\Mod{4}.
        \end{cases} \]
    \tcblower{}
    \textbf{Proof}:
\end{Corollary}
\begin{Proposition}{}{}
    There are infinitely many primes congruent to $ 1 $ modulo $ 4 $.
\end{Proposition}
\begin{Proposition}{}{}
    If $ p $ is an odd prime, then
    \[ \legendre{2}{p}=\begin{cases}
            1,  & p\equiv 1,7\Mod{8}, \\
            -1, & p\equiv 3,5\Mod{8}.
        \end{cases} \]
    \tcblower{}
    \textbf{Proof}:
\end{Proposition}
\begin{Exercise}{}{}
    Using the technique in the proof of Proposition 2, show that $2$ does not have a
    quadratic residue modulo $19$.
    \tcblower{}
    \textbf{Solution}: Note that $ 19\equiv 3\Mod{8} $, so
    $ \legendre{2}{19}=-1 $
    by Proposition 2.
\end{Exercise}

\section{The Law of Quadratic Reciprocity}
We have solved the problem of finding $ \legendre{a}{p} $ for $ a=-1 $ and $ a=2 $.
Unfortunately, our method does for doing that does not exist for large values of $ a $.
We need a fast algorithm for calculating $ \legendre{a}{p} $ for any integer $ a $
and odd prime $ p $ with $ p\nmid a $.

The Law of Quadratic Reciprocity was conjectured by Euler and Legendre in
1744. Gauss, in 1796, was the first to prove the Law of Quadratic Reciprocity, at
the age of 19 and subsequently found at least five other proofs. He referred it as
``The Golden Theorem.'' There are now over 240 published proofs (people are still
publishing new proofs).

\begin{Theorem}{Law of Quadratic Reciprocity}{}
    If $ p $ and $ q $ are distinct odd prime numbers, then
    \[ \legendre{p}{q}=\begin{cases}
            \legendre{q}{p},  & p\equiv 1\Mod{4}\lor q\equiv 1\Mod{4},  \\
            -\legendre{q}{p}, & p\equiv 3\Mod{4}\land q\equiv 3\Mod{4}.
        \end{cases} \]
\end{Theorem}
The Law of Quadratic Reciprocity can be stated as:
\[ \legendre{p}{q}\legendre{q}{p}=(-1)^{\frac{p-1}{2}\frac{q-1}{2}}. \]
The proof is quite non-trivial and due to limitations of time we will not present
it in class or in lecture notes as well. For the proof you can see Section 6.4 of ``A taste
of Number Theory'' by Frank Zorzitto.
\begin{Example}{}{}
    Calculate $ \legendre{7}{109} $.
    \tcblower{}
    \textbf{Solution}:
    \begin{align*}
        \legendre{7}{109} & =\legendre{109}{7}              &  & 109\equiv 1\Mod{4}                       \\
                          & =\legendre{4}{7}                &  & 109\equiv 4\Mod{7}                       \\
                          & =\legendre{2}{7}\legendre{2}{7} &  & \text{Legendre symbol is multiplicative} \\
                          & =\legendre{2}{7}^{\!2}                                                        \\
                          & =1.
    \end{align*}
    Thus, $ 7 $ is a quadratic residue modulo $ 109 $.
\end{Example}
\begin{Example}{}{}
    Calculate $ \legendre{191}{839} $.
    \tcblower{}
    \textbf{Solution}:
    \begin{align*}
        \legendre{191}{839}
         & =-\legendre{839}{191}                      &  & 191\equiv 3\Mod{4}\land 839\equiv 3\Mod{4} \\
         & =-\legendre{75}{191}                       &  & 839\equiv 75\Mod{191}                      \\
         & =-\legendre{5}{191}^2 \legendre{3}{191}                                                    \\
         & =-\legendre{5}{191}^{\!2}\legendre{3}{191}                                                 \\
         & =-(1)\legendre{3}{191}                                                                     \\
         & =\legendre{191}{3}                         &  & 191\equiv 3\Mod{4}\land 3 \equiv 3\Mod{4}  \\
         & =\legendre{2}{3}                           &  & 191\equiv 2\Mod{4}                         \\
         & =-1.
    \end{align*}
    Thus, $191$ is quadratic non-residue modulo $839$.
\end{Example}
\begin{Exercise}{}{}
    Calculate $ \legendre{37603}{48611} $ given that $ 37603=31\cdot 1213 $.
    \tcblower{}
    \textbf{Solution}:
    \begin{align*}
        \legendre{37603}{48611}
         & =\legendre{31}{48611}\legendre{1213}{48611}            &  & \text{Legendre symbol is $ \times $}                               \\
         & =-\legendre{48611}{31}\legendre{48611}{1213}           &  & 31\equiv 3\Mod{4}\land 48611\equiv 3\Mod{4},\;1213\equiv 1\Mod{4}  \\
         & =-\legendre{3}{31}\legendre{91}{1213}                  &  & 48611\equiv 3\Mod{31},\; 48611\equiv 91\Mod{1213}                  \\
         & =\legendre{31}{3}\legendre{7}{1213}\legendre{13}{1213} &  & 3\equiv 3\Mod{4}\land 31\equiv 3\Mod{4}                            \\
         & =\legendre{1}{3}\legendre{1213}{7}\legendre{1213}{13}  &  & 1213\equiv 1\Mod{4}                                                \\
         & =\legendre{2}{7}\legendre{4}{13}                       &  & 1213\equiv 2\Mod{7},\; 1213\equiv 4\Mod{13}                        \\
         & =\legendre{2}{13}^{\!2}                                &  & 2^{\frac{7-1}{2}}\equiv 8\equiv 1\Mod{7}\text{ using Euler's Test} \\
         & =(-1)^2                                                &  & 13\equiv 5\Mod{8}\text{ using Proposition 25.15}                   \\
         & =1.
    \end{align*}
\end{Exercise}
\begin{Example}{}{}
    Let $ p $ and $ q $ be distinct primes such that $ p\equiv 3\Mod{4} $ and $ q\equiv 3\Mod{4} $.
    Prove that the equation $ x^2-qy^2=p $ has no integer solution.
\end{Example}

\makeheading{Lecture 23}{\printdate{2022-06-29}}%chktex 8
\section{Sum of Squares}
Recall that way back in Exercise 5 (Lecture 8) you were asked to make a conjecture
to which primes in $ \mathbf{Z} $ are also primes in $ \mathbf{Z}[i] $.
The conjecture is that a prime $ p\in\mathbf{Z} $ is a prime in $ \mathbf{Z}[i] $
if and only if $ p $ cannot be written as a sum of two squares.
We are now finally in a position to figure out which primes can be written as a sum
of two squares.

Look at a few numbers:
\begin{align*}
    1 & =1^2+0^2  \\
    2 & =1^2+1^2  \\
    4 & =2^2+0^2  \\
    5 & =2^2+1^2  \\
    8 & =2^2+2^2.
\end{align*}
Note that $ 3 $, $ 6 $, and $ 7 $ are not expressible in this way.

\begin{Proposition}{}{}
    If $ p $ is a Gaussian prime and $ p\mid zw $ for some $ z,w\in\mathbf{Z}[i] $, then
    $ p\mid z $ or $ p\mid w $.
\end{Proposition}
\begin{Theorem}{}{}
    If $ n\equiv 3\Mod{4} $, then $ n $ is \underline{not} a sum of two squares.
    \tcblower{}
    \textbf{Proof}:
\end{Theorem}
Albert Girard was the first to make the observation, describing all positive
integer numbers (not necessarily primes) expressible as the sum of two squares of
positive integers; this was published in 1625. The statement that every prime $p$ of
the form $4n+1$ is the sum of two squares is sometimes called Girard's theorem. For
his part, Fermat wrote an elaborate version of the statement (in which he also gave
the number of possible expressions of the powers of $p$ as a sum of two squares) in a
letter to Marin Mersenne dated December 25, 1640: for this reason this version of
the theorem is sometimes called Fermat's Christmas theorem:
\begin{Theorem}{Fermat's Christmas Theorem}{}
    If $ p $ is a prime such that $ p\equiv 1\Mod{4} $, then there exists $ a,b\in\mathbf{Z} $ such that
    \[ p=a^2+b^2. \]
    \tcblower{}
    \textbf{Proof}:
\end{Theorem}
Now, we know that when $ p $ is an odd prime, the $ p=x^2+y^2 $ has a solution in the positive integers $ x $ and $ y $ if
and only if $ p\equiv 1\Mod{4} $. Notice this also has a solution when $ p=2 $
since $ 2=1^2+1^2 $. We would like to generalize this result to all $ n\in\mathbf{Z}^+ $.
\begin{Proposition}{}{}
    If $ m,n\in\mathbf{Z} $ are expressible as a sum of two squares, then so is $ mn $.
    \tcblower{}
    \textbf{Proof}:
\end{Proposition}
\begin{Theorem}{}{}
    $ n\in\mathbf{Z}^+ $ is expressible as sum of two squares
    if and only if every prime factor which is congruent to $ 3\Mod{4} $
    appears with an even power.
\end{Theorem}
Due to time limitations we will not present the proof in class or in these lectures
notes. If you would like to see the proof, see Section 7 in ``A Taste of Number
Theory'' by Frank Zorzitto.

\begin{Example}{}{}
    Can we express $ 490 $ as a sum of two squares? If yes, then find $ x,y\in\mathbf{Z} $ such that
    $ 490=x^2+y^2 $.
    \tcblower{}
    \textbf{Solution}:
    Note that $ 490=2\cdot 5\cdot 7^2 $. Since $ 5\equiv 1\Mod{4} $ and $ 7\equiv 3\Mod{4} $ (appears with an even power), by Fermat's Christmas Theorem
    and Theorem 3, $ 490 $ is expressible as a sum of squares. Further,
    \begin{align*}
        490 & =7^2\cdot 5\cdot 2                 \\
            & =7^2\cdot (2^2+1^2)\cdot (1^2+1^2) \\
            & =(7^2+0^2)\cdot (3^2+1^2)          \\
            & =21^2+7^2.
    \end{align*}
\end{Example}
\begin{Exercise}{}{}
    Can we express $ 584820=2^2\cdot 3^4\cdot 5\cdot 19^2 $ as sum of two squares?
    \tcblower{}
    \textbf{Solution}: Since $ 2^2=4=2^2+0^2 $, $ 3\equiv 3\Mod{4} $ (appears with an even power; Fermat's Christmas Theorem),
    $ 5\equiv 1\Mod{4} $ (Theorem 3), and $ 19\equiv 3\Mod{4} $ (appears with an even power; Fermat's Christmas Theorem),
    then $ 584820 $ is expressible as a sum of squares. Hence,
    \begin{align*}
        584820
         & =2^2\cdot 3^4\cdot 5\cdot 19^2                                                    \\
         & =(2^2+0^2)\cdot (9^2+0^2)\cdot (2^2+1^2)\cdot (19^2+0^2)                          \\
         & =2^2\cdot 9^2\cdot 19^2\cdot (2^2+1^2)                                            \\
         & =2^2\cdot 9^2\cdot 19^2\cdot 2^2+2^2\cdot 9^2\cdot 19^2\cdot 1^2                  \\
         & =4^2\cdot 9^2\cdot 19^2+2^2\cdot 9^2\cdot 19^2                                    \\
         & =(\underbrace{4\cdot 9\cdot 19}_{684})^2+(\underbrace{2\cdot 9\cdot 19}_{342})^2.
    \end{align*}
    $ x^2=342\land y^2=684 \iff 584820=x^2+y^2 $ for $ x,y\in\mathbf{Z}^+ $.
\end{Exercise}


\makeheading{Week 9 | Friday}{\printdate{2022-07-01}}%chktex 8
Holiday (Canada Day).