\makeheading{Lecture 4}{\printdate{2022-05-09}}%chktex 8
\begin{Example}{}{}
    Use the Euclidean algorithm to compute $ (124,348) $.
    \tcblower{}
    \textbf{Solution}: Here are the divisions:
    \begin{align*}
        348 & = 124\cdot 2+100 \\
        124 & =100\cdot 1+24,  \\
        100 & =24\cdot 4+4     \\
        24  & =4\cdot 6+0.
    \end{align*}
    Therefore, $ (124,348)=4 $.

    It's easier to remember this visually by arranging the computations in a table.
    Compare the numbers above to the numbers in the following table:
    \[ \begin{array}{cc}
            r_i & q_{i-1} \\
            \midrule
            348           \\
            124 & 2       \\
            100 & 1       \\
            24  & 4       \\
            4   & 6
        \end{array} \]
    The next remainder is 0, so we didn't write it. The successive remainders go in the
    first column. The successive quotients go in the second column.
\end{Example}
To compute the greatest common divisors of three numbers, just compute the
greatest common divisor of two numbers at a time.
\begin{Example}{}{}
    Compute $ (42,105,91) $.
    \tcblower{}
    \textbf{Solution}: Since $ (42,105)=21 $, so $ (42,105,91)=\bigl((42,105),91\bigr)=(21,91)=7 $.
\end{Example}
\section{Bézout's Identity}
The next result is extremely important, and is often used in proving things
about greatest common divisors. First, We will recall a definition from linear algebra.
\begin{Definition}{}{}
    If $ a $ and $ b $ are numbers, a linear combination of $ a $ and $ b $ (with integer coefficients) is a number of the form
    \[ ax+by,\; x,y\in\mathbf{Z}. \]
    \tcblower{}
    For instance, $ 29=2\cdot 10+1\cdot 9 $ shows that $ 29 $ is a linear combination of $10$ and $9$.
    Further, $ 7=(-2)\cdot 10+3\cdot 9 $ shows that $ 7 $ is a linear combination of $ 10 $ and $ 9 $ as well.
\end{Definition}
\begin{Example}{}{}
    Find the smallest positive integer $ c $ that has the form $ 12x+8y=c $, where $ x,y\in\mathbf{Z} $.
    \tcblower{}
    \textbf{Solution}: We can see that $ 12(1)+8(-1)=4 $. The question is ``can we get a smaller positive integer?''

    Can we find $ x,y\in\mathbf{Z} $ such that $ 12x+8y=3 $? If we could, we would have $ 4(3x+2y)=3 $. Since $ 3x+2y\in\mathbf{Z} $,
    this would imply that $ 4\mid 3 $ which is a contradiction. Using the same argument on $ 12x+8y=2 $ and $ 12x+8y=1 $, we
    see that none of these are possible. Hence, the smallest positive integer is $4$. So, in this case, the smallest positive integer of the form
    $ 12x+8y=c $ is equal to $ (12,8) $.
\end{Example}
\begin{Example}{}{}
    Find the smallest positive integer $ c $ that has the form $ 28x+105y=c $, where $ x,y\in\mathbf{Z} $.
\end{Example}
\begin{Theorem}{Bézout's Identity}{}
    Let $ a,b\in\mathbf{Z} $ (not both zero). If $d$ is
    the least positive integer combination of $a$ and $b$, then $d$ divides every combination
    of $a$ and $b$. Furthermore, $d = (a, b)$.
    \tcblower{}
    \textbf{Proof}: We know that $ ax+by=d>0 $. Now consider some integer combination
    \[ c=as+bt,\; s,t\in\mathbf{Z}. \]
    We want to show that $ d\mid c $. By DA, there exists $ q,r\in\mathbf{Z} $ such that
    \[ c=dq+r,\; 0\le r<d. \]
    Thus,
    \begin{align*}
        0 & \le r            \\
          & =c-dq            \\
          & =as+bt-(ax+by)q  \\
          & =a(s-q)+b(t-yq).
    \end{align*}
    We see that r is an integer combination of $a$ and $b$, which is less than $d$, and
    non-negative. Because $d$ is the least positive integer combination of $a$ and $b$, the
    only option is that $r = 0$. Hence, $ d\mid c $. In particular, $ d\mid a $ and $ d\mid b $.
    So $ d $ is a common divisor of $ a $ and $ b $. We will now show that $ d=(a,b) $. Let $ d^\prime $
    be a common divisor of $ a $ and $ b $. Then, $ d^\prime \mid a $ and $ d^\prime \mid b $. Hence,
    \[ d^\prime \mid ax+by \]
    by property 2 of~\Cref{prop:LEC2_PROP1}. Thus, we have $ d^\prime \mid d $, and by definition of GCD we have
    $ d=(a,b) $.
\end{Theorem}
\begin{Corollary}{}{}
    The set of all linear combinations of integers $a$ and $b$ is the set
    of all multiples of $(a, b)$.
    \tcblower{}
    \textbf{Proof}: On one hand,
    \[ (a,b)\mid ax+by,\, x,y\in\mathbf{Z}. \]
    So every linear combination of $a$ and $b$ is a multiple of $(a, b)$.
    On the other hand,
    \[ (a,b)=ax+by\text{ so }k(a,b)=a(kx)+b(ky), \]
    that is, every multiple of $ (a,b) $ is a linear combination of $ a $ and $ b $.
\end{Corollary}
\begin{Corollary}{}{}
    Two integers $ a $ and $ b $ are relatively prime if and only if $ ax+by=1 $ for some $ x,y\in\mathbf{Z} $.
    \tcblower{}
    \textbf{Proof}: Suppose $ a $ and $ b $ are relatively prime; that is, $ (a,b)=1 $. By Theorem 1 (Bézout's Identity),
    \[ ax+by=(a,b)=1,\;\text{for some }x,y\in\mathbf{Z}. \]
    On the other hand, suppose $ ax+by=1 $ for some $ x,y\in\mathbf{Z} $. Since $ (a,b)\mid a $ and $ (a,b)\mid b $, we have
    \[ (a,b)\mid ax+by=1. \]
    The only positive integer that divides $1$ is $1$. Therefore, $(a, b) = 1$
\end{Corollary}
\begin{Exercise}{}{}
    Prove that if $ n\in\mathbf{Z} $, then $ (3n+17,2n+11)=1 $.
    \tcblower{}
    \textbf{Solution}: $2(3n+17)-3(2n+11)=1$, and $ (2,3)=1 $. Therefore, $(3n+17,2n+11)=1$.
\end{Exercise}
\begin{Proposition}{}{l4_prop1}
    Let $ a,b,c\in\mathbf{Z} $. If $ (a,b)=1 $, $ a\mid c $, and $ b\mid c $,
    then $ ab\mid c $.
    \tcblower{}
    \textbf{Proof}: Since $ a\mid c $ and $ b\mid c $, there exists $ d,f\in\mathbf{Z} $ such that
    \begin{align*}
        c                    & =ad\text{ and }c           =bf \\
        \implies \frac{c}{a} & =d \text{ and }\frac{c}{b} =f.
    \end{align*}
    Further, since $a$ and $b$ are coprime, by Theorem 1 (Bézout's Identity) there exist
    integers $x$ and $y$ such that
    \[ ax+by=1. \]
    Thus,
    \begin{align*}
        acx+byc                   & =c            \\
        \frac{c}{b}x+\frac{c}{a}y & =\frac{c}{ab} \\
        fx+dy                     & =\frac{c}{ab} \\
        ab(fx+dy)                 & =c,
    \end{align*}
    which implies $ ab\mid c $.
\end{Proposition}
\begin{Proposition}{}{l4_prop2}
    Let $ a,b,n\in\mathbf{Z} $. If $ (n,a)=1 $ and $ n\mid ab $, then $ n\mid b $.
    \tcblower{}
    \textbf{Proof}: By Bézout's Identity, there exists $ x,y\in\mathbf{Z} $ such that
    \[ nx+ay=(n,a)=1. \]
    Multiplying by $b$ gives
    \[ nxb+ayb=b. \]
    Since $ n\mid n $ and $ n\mid ab $, we get by property 2 of~\Cref{prop:LEC2_PROP1} that
    \[ n\mid nxb+ayb. \]
    Therefore, $ n\mid b $.
\end{Proposition}
\section{The Extended Euclidean Algorithm}
We will start by reviewing the Euclidean algorithm, in which the extended
Euclidean algorithm is used.
\begin{Example}{}{}
    Find $ (1914,899) $. Further, find $ x,y\in\mathbf{Z} $ such that $ 1914x+899y=29 $.
    \tcblower{}
    \textbf{Solution}: We first follow the Euclidean Algorithm,
    \begin{equation}\label{EEAeqA}
        \begin{aligned}
            1914 & =2\cdot 899+116 \\
            899  & =7\cdot 116+87  \\
            116  & =1\cdot 87+29   \\
            87   & =3\cdot 29+0.
        \end{aligned}
    \end{equation}
    We usually write this in tabular form:
    \[ \begin{array}{cc}
            r_i  & q_{i-1} \\
            \midrule
            1914 & 899     \\
            899  & 2       \\
            116  & 7       \\
            87   & 1       \\
            29   & 3
        \end{array} \]
    So, $ (1914,899)=29 $. We can rewrite the first two equations as:
    \begin{equation}\label{EEAeq1}
        1914-2\cdot 899=116.
    \end{equation}
    \begin{equation}\label{EEAeq2}
        899-7\cdot 116=87.
    \end{equation}
    Substitute~(\ref{EEAeq1}) into~(\ref{EEAeq2}) to get
    \[ 899-7\cdot(1914-2\cdot 899)=87. \]
    \begin{equation}\label{EEAeq3}
        -7\cdot 1914+15\cdot 899=87.
    \end{equation}
    We can now rewrite the third equation of~(\ref{EEAeqA}) as:
    \begin{equation}\label{EEAeq4}
        116-1\cdot 87=29.
    \end{equation}
    Substituting~(\ref{EEAeq1}) and~(\ref{EEAeq3}) into~(\ref{EEAeq4}) gives
    \begin{align*}
        (1914-2\cdot 899)-1\cdot(-7\cdot 1914+15\cdot 899) & =29  \\
        8\cdot 1914 -17\cdot 899                           & =29.
    \end{align*}
    Thus, $x = 8$ and $y = 17$.
\end{Example}
The above procedure is painful to carry out by hand, or even with a basic
calculator. Let's explore a method of calculations, i.e., an algorithm, for solving
the equation
\[ ax+by=(a,b),\;x,y\in\mathbf{Z}. \]
It is called a backward recurrence, and is due to S. P. Glasby. It will look a
little complicated, but you'll see that it's really easy to use in practice.
\begin{Theorem}{}{}
    Let $ a,b\in\mathbf{Z}^+ $ with $ b>a $. Define
    \[ \begin{array}{cc}
            r_1=b & r_2=a \\
            s_1=1 & s_2=0 \\
            t_1=0 & t_2=1
        \end{array} \]
    and sequences as:
    \begin{align*}
        r_{i+1} & =r_{i-1}-q_{i-1}r_i  \\
        s_{i+1} & =s_{i-1}-q_{i-1}s_i  \\
        t_{i+1} & =t_{i-1}-q_{i-1}t_i.
    \end{align*}
    Then, for $ i\in\mathbf{Z}^+ $, we have
    \[ bs_i+at_i=r_i. \]
    In particular, if $ r_n=(a,b) $, then
    \[ bs_n+at_n=(a,b). \]
    \tcblower{}
    \textbf{Proof}: Use induction on $ n $.
\end{Theorem}
\begin{Example}{}{}
    Find $ x,y\in\mathbf{Z} $ such that $ 1914x+899y=(1914,899) $.
    \tcblower{}
    \textbf{Solution}:
    \[ \begin{array}{lllll}
            r_i  & q_{i-1} & s_i & t_i & \text{Check}          \\
            \midrule
            1914 &         & 1   & 0                           \\
            899  & 2       & 0   & 1                           \\
            116  & 7       & 1   & -2  & 1(1914)+(-2)(899)=116 \\
            87   & 1       & -7  & 15  & (-7)(1914)+15(899)=87 \\
            29   & 3       & 8   & -17 & 8(1914)+(-17)(899)=29
        \end{array} \]
    You can fill the columns of $ s_i $ and $ t_i $ for $ i\ge 3 $ with
    \begin{align*}
        \text{next }s & =\text{previous to last }s-(\text{last }q)(\text{last }s), \\
        \text{next }t & =\text{previous to last }t-(\text{last }q)(\text{last }t).
    \end{align*}
\end{Example}
\begin{Exercise}{}{}
    Compute $ (187,102) $ and express it as a linear combination of $ 187 $ and $ 102 $.
    \tcblower{}
    \textbf{Solution}: We first follow the Euclidean Algorithm,
    \[ \begin{array}{lllll}
            r_i & q_{i-1} & s_i & t_i & \text{Check}          \\
            \midrule
            187 &         & 1   & 0                           \\
            102 & 1       & 0   & 1                           \\
            85  & 1       & 1   & -1  & 1(187)+(-1)(102)=85   \\
            17  & 5       & -1  & 2   & (-1)(187)+(2)(102)=17
        \end{array} \]
    Therefore,
    \[ 187\cdot (-1)+102\cdot 2=(187,102)=17. \]
\end{Exercise}
\begin{Exercise}{}{}
    Find the smallest $ c $, and $ x,y $ such that $ c=246x+194y $.
    \tcblower{}
    \textbf{Solution}:
    \[ \begin{array}{lllll}
            r_i & q_{i-1} & s_i & t_i & \text{Check}            \\
            \midrule
            246 &         & 1   & 0                             \\
            194 & 1       & 0   & 1                             \\
            52  & 3       & 1   & -1  & 1(246)+(-1)(194)=52     \\
            38  & 1       & -3  & 4   & (-3)(246)+(4)(194)=38   \\
            14  & 2       & 4   & -5  & (4)(246)+(-5)(194)=14   \\
            10  & 1       & -11 & 14  & (-11)(246)+(14)(194)=10 \\
            4   & 2       & 15  & -19 & (15)(246)+(-19)(194)=4  \\
            2   & 2       & -41 & 52  & (-41)(246)+(52)(194)=2.
        \end{array} \]
    Thus, $ x=-41 $, $ y=52 $, and $ c=2 $.
\end{Exercise}
\begin{Exercise}{}{}
    Find $ x,y\in\mathbf{Z} $ such that $ 126x+91y=(126,91) $.
\end{Exercise}
\makeheading{Lecture 5}{\printdate{2022-05-11}}%chktex 8
\section{Diophantine Equations}
A polynomial equation in several variables in which we are only interested
in integer solutions is called a Diophantine equation. Diophantine equations are
named after the 3rd century mathematician Diophantus of Alexandria who wrote
a series of books called Arithmetica wherein he raised the matter of solving the
equations now named in his honour.
\begin{Example}{}{}
    Let $ a,b,c,n\in\mathbf{Z} $. Some famous Diophantine equations are:
    \begin{itemize}
        \item $ ax+by+c $: Linear Diophantine equation in two variables.
        \item $ x^2+y^2=z^2 $: Pythagorean Triple.
        \item $ x^2-dy^2\pm 1 $, where $ d\in\mathbf{Z}^+ $ is non-square: Pell's Equation.
        \item $ ax^n+by^n=cz^n $, where $ n\in\mathbf{Z} $, $ n\ge 3 $: Fermat type Equation.
    \end{itemize}
\end{Example}
For now, we will just look at linear Diophantine equation in two variables,
\[ ax+by=c, \]
where $ a,b,c $ are fixed integers and $ x,y $ are integer variables. When analyzing equations,
we would like to answer the following questions.
\begin{enumerate}[(1)]
    \item Does a solution exist?
    \item If solutions exist, how many of them exist? (finite, infinite, countably, or uncountably many)
    \item What are the solutions?
    \item Are there any algorithms which generates the solution(s)?
\end{enumerate}
We address the same questions when analyzing Diophantine equations.
\begin{Theorem}{}{}
    Let $ a,b,c\in\mathbf{Z} $. Let $ (x,y) $ be a pair of integers satisfying the Diophantine equation
    \[ ax+by=c. \]
    \begin{enumerate}[(a)]
        \item If $ (a,b)\nmid c $, then no solutions exist for $ ax+by=c $.
        \item If $ (a,b)=d\mid c $, then there are infinitely many solutions of the form
              \begin{align*}
                  x^\prime & =x_0-\frac{b}{d}t, \\
                  y^\prime & =y_0+\frac{a}{d}t,
              \end{align*}
              where the pair $ (x_0,y_0) $ is a particular solution to the equation $ ax+by=c $,
              and $ t\in\mathbf{Z} $.
    \end{enumerate}
    \tcblower{}
    \textbf{Proof}:
    \begin{enumerate}[(a)]
        \item Suppose $ (a,b)\nmid c $. Let the pair $ (x^\prime,y^\prime) $ be solutions of the equation
              $ ax+by=c $; that is, $ ax^\prime+by^\prime=c $. Since $ (a,b)\mid a $ and $ (a,b)\mid b $,
              \[ (a,b)\mid ax+by=c \]
              by property 2 of~\Cref{prop:LEC2_PROP1}, which is a contradiction. Hence, no solution exists.
        \item Suppose $ (a,b)=d\mid c $, then $ c=dk $ for some $ k\in\mathbf{Z} $. By Bézout's Identity,
              there are integers $ m,n $ such that
              \[ am+bn=d=(a,b). \]
              Then,
              \[ amk+bnk=dk=c. \]
              Hence, the pair $(mk, nk)$ is a solution.

              Suppose the pair $ (x_0,y_0) $ is a particular solution. Then,
              \[ a\biggl(x_0-\frac{b}{d}t\biggr)+b\biggl(y_0+\frac{a}{d}t\biggr)=\frac{ab}{d}t-\frac{ab}{d}t+(ax_0+by_0)=0+c=c, \]
              which proves that the pair $ (x_0-\frac{b}{d}t,y_0+\frac{a}{d}t) $ is a solution for every $ t\in\mathbf{Z} $.

              Let $ (x^\prime,y^\prime) $ and $ (x_0,y_0) $ be the pairs such that $ ax^\prime+by^\prime=c $ and $ ax_0+by_0=c $. Hence,
              \begin{align*}
                  a(x_0-x^\prime)                    & =b(y^\prime-y_0)            \\
                  \implies \frac{a(x_0-x^\prime)}{d} & =\frac{b(y^\prime-y_0)}{d}.
              \end{align*}
              Now, $ \frac{b}{d}\mid \frac{a}{d}(x_0-x^\prime) $. However, $ (\frac{a}{d},\frac{b}{d})=1 $ by~\Cref{prop:l3_prop2}. Therefore,
              \[ \frac{b}{d}\mid x_0-x^\prime, \]
              (using~\Cref{prop:l4_prop2}) would imply
              \[ x_0-x^\prime=t \frac{b}{d},\; \text{for some $ t\in\mathbf{Z} $}. \]
              Thus,
              \[ x^\prime=x_0-\frac{b}{d}t. \]
              Substituting $ x_0-x^\prime=t \frac{b}{d} $ into the equation $ a(x_0-x)=b(y-y_0) $, we see that
              \[ y^\prime=y_0+\frac{a}{d}t. \]
    \end{enumerate}
\end{Theorem}
\makeheading{Lecture 6}{\printdate{2022-05-13}}%chktex 8
\begin{Example}{}{}
    Solve the Diophantine equation $ 6x+9y=5 $.
    \tcblower{}
    \textbf{Solution}: Since $ (9,6)=3\nmid 5 $, the equation has no solution.
\end{Example}
\begin{Example}{}{}
    Find all the solutions $ (x,y) $ to the Diophantine equation
    \[ 11x+13y=369. \]
    \tcblower{}
    \textbf{Solution}: Since $ (11,13)=1\mid 369 $, there are infinitely many solutions. It is hard to guess
    the particular solution, so we will use the EEA\@:
    \[ \begin{array}{lllll}
            r_i & q_{i-1} & s_i & t_i & \text{Check}       \\
            \midrule
            13  &         & 1   & 0                        \\
            11  & 1       & 0   & 1                        \\
            2   & 5       & 1   & -1  & (1)(13)+(-1)(11)=2 \\
            1   & 2       & -5  & 6   & (-5)(13)+(6)(11)=1
        \end{array} \]
    \begin{align*}
        (11)(6)+(13)(-5)       & =1    \\
        (11)(2214)+(13)(-1845) & =369.
    \end{align*}
    So, $ (2214,-1845) $ is a particular solution. The general solution is
    \[ x=2214-13t,\quad y=-1845+11t,\; t\in\mathbf{Z}. \]
\end{Example}
\begin{Exercise}{}{}
    Find all solutions of the linear Diophantine equation
    \[ 132x+84y=144. \]
    \tcblower{}
    \textbf{Solution}:
    \[ \begin{array}{lllll}
            r_i & q_{i-1} & s_i & t_i & \text{Check}     \\
            \midrule
            132 &         & 1   & 0                      \\
            84  & 1       & 0   & 1                      \\
            48  & 1       & 1   & -1  & 132(1)+84(-1)=48 \\
            36  & 1       & -1  & 2   & 132(-1)+84(2)=36 \\
            12  & 1       & 2   & -3  & 132(2)+84(-3)=12
        \end{array} \]
    Hence,
    \begin{align*}
        132(2)+84(-3)                 & =12         \\
        132(2\cdot 12)+84(-3\cdot 12) & =12\cdot 12 \\
        132\cdot 24+84\cdot (-36)     & =144.
    \end{align*}
    So, $ (x_0,y_0)=(24,-36) $ is a particular solution. The general solution is:
    \begin{align*}
        x & =x_0-\frac{b}{d}t=24-\frac{84}{12}t=24-7t,                      \\
        y & =y_0-\frac{a}{d}t=-36+\frac{132}{12}t=-36+11t,\; t\in\mathbf{Z}
    \end{align*}
\end{Exercise}
Consider a 3-variable equation
\[ ax+by+cz=d. \]
The equation has solutions if $ (a,b,c)\mid d $. If it has a solution, there will be infinitely many,
determined by two integer parameters.
\begin{Exercise}{}{}
    Find the general solution to the Diophantine equation
    \[ 8x+14y+5z=11. \]
    \tcblower{}
    \textbf{Solution}:
    \[ 2(4x+7y)+5z=11. \]
    Let $ w=4x+7y $, so
    \[ 2w+5z=11. \]
    Now, $ w=-22 $ and $ z=11 $ is a particular solution, so
    \[ w=-22+5s,\quad z=11-2s,\; s\in\mathbf{Z}. \]
    Then,
    \[ 4x+7y=w=-22+5s. \]
    $ x=-44+10s $ and $ y=22-5s $ is a particular solution. The general solution is
    \begin{align*}
        x & =-44+10s+7t \\
        y & =22-5s-4t   \\
        z & =11-2s.
    \end{align*}
\end{Exercise}
\section{Prime Numbers}
\begin{Definition}{}{}
    An integer $ p $ is called prime when $ p\ne 0 $, $ p\ne \pm 1 $, and the only factors that $ p $
    have are $ \pm 1 $ and $ \pm p $.
\end{Definition}
Clearly, $p$ is prime if and only if $p$ is prime. To avoid this double counting
of primes we shall work only with positive primes and to be brief we shall usually
omit the word ``positive.''
\begin{Lemma}{}{}
    Every integer greater than 1 is divisible by at least one prime.
    \tcblower{}
    \textbf{Proof}: Let's use induction. To begin with, the result is true for $ n=2 $ since $ 2 $ is prime.

    Suppose $ 2,3,4,\ldots,k-1 $ is divisible by at least one prime. If $ k $ is prime, it is divisible
    by a prime --- namely itself! If $ k $ is composite, then $ k=ab $, where $ 1<a<k $ and $ 1<b<k $. Since
    $ a $ and $ b $ are among the integers $ 2,3,4,\ldots,k-1 $, each of them is divisible by at least one prime;
    that is, there exists $ p $ such that $ p\mid a $ and $ a\mid k $ implies $ p\mid k $, so $ k $ has a prime factor
    as well. This shows that the result is true for all $ n>1 $ by induction.
\end{Lemma}
And now comes a classic discovery and its proof occur as Proposition 20 in
Book 9 of Euclid's Elements.

\begin{Proposition}{Euclid's Theorem}{}
    There are infinitely many prime numbers.
    \tcblower{}
    \textbf{Proof}: Suppose on the contrary that there are only finitely many primes $ p_1,\ldots,p_n $. Look at
    \[ p_1p_2\cdots p_n+1. \]
    This number is not divisible by any of the primes $ p_1,\ldots,p_n $ because it leaves the remainder of $1$
    when divided by any of them. According to Lemma 1, there exists a new prime number $ q $ such that
    $ q\mid p_1p_2\cdots p_n+1 $, a contradiction. This contradiction implies that there cannot be finitely many primes;
    that is, there are infinitely many.
\end{Proposition}
The special thing about primes is that there is only one way to write an integer
into primes. Ambiguous factoring such as
\[ 30=(6)(5)=(15)(2) \]
do not occur when only primes are involved in the factors. To prove the Unique
factorization, we need what we can only be called the signature property of primes.
\begin{Proposition}{Euclid's Lemma}{euclid_lemma}
    Let $ a,b\in\mathbf{Z} $. If $ p $ is a prime number and $ p\mid ab $,
    then $ p\mid a $ or $ p\mid b $.
    \tcblower{}
    \textbf{Proof}: Assume $ p\mid ab $, then there exists $ n\in\mathbf{Z} $ such that $ pn=ab $.
    Further, assume $ p\nmid a $. Since $ a $ is not a multiple of $ p $ and the only factors of $ p $
    are $ 1 $ and $ p $, we must have $ (a,p)=1 $. So by Bézout's Identity, there exists $ x,y\in\mathbf{Z} $ such that
    \begin{align*}
        ax+py    & =1. \\
        abx+pby  & =b. \\
        pnx+pby  & =b. \\
        p(nx+by) & =b.
    \end{align*}
    Since $ nx+by\in\mathbf{Z} $, $ p\mid b $.
\end{Proposition}
Note that~\Cref{prop:euclid_lemma} fails when p is not a prime. For instance
$ 6\mid 12=(3)(4) $, but $ 6\nmid 3 $ and $ 6\nmid 4 $.