\makeheading{Lecture 27}{\printdate{2022-07-11}}%chktex 8
\section{Continued Fraction}
\subsection*{Finite Continued Fractions}
In particular, we begin by look at how we can use the Euclidean algorithm
to write rational numbers in an alternate way.
\begin{Example}{}{}
    Using the Euclidean Algorithm to find $ (87,32) $
    gives
    \begin{align*}
        87 & =2\cdot 32+23 \\
        32 & =1\cdot 23+9  \\
        23 & =2\cdot 9+5   \\
        9  & =1\cdot 5+4   \\
        5  & =1\cdot 4+1.
    \end{align*}
    We can rewrite each of these equations by dividing through
    the divisor as:
    \begin{align*}
        \frac{87}{23} & =2+\frac{23}{32} \\
        \frac{32}{23} & =1+\frac{9}{23}  \\
        \frac{23}{9}  & =2+\frac{5}{9}   \\
        \frac{9}{5}   & =1+\frac{4}{5}   \\
        \frac{5}{4}   & =1+\frac{1}{4}.
    \end{align*}
    Observe that the last number in each line is the
    reciprocal of the first number in the line below it.
    This allows us to write $ \frac{87}{32} $ as:
    \begin{align*}
        \frac{87}{23}
         & =2+\frac{1}{\frac{32}{23}}                                     \\
         & =2+\frac{1}{1+\frac{9}{23}}                                    \\
         & =2+\frac{1}{1+\frac{1}{\frac{23}{9}}}                          \\
         & =2+\frac{1}{1+\frac{1}{2+\frac{5}{9}}}                         \\
         & =2+\frac{1}{1+\frac{1}{2+\frac{1}{\frac{9}{5}}}}               \\
         & =2+\frac{1}{1+\frac{1}{2+\frac{1}{1+\frac{4}{5}}}}             \\
         & =2+\frac{1}{1+\frac{1}{2+\frac{1}{1+\frac{1}{\frac{5}{4}}}}}   \\
         & =2+\frac{1}{1+\frac{1}{2+\frac{1}{1+\frac{1}{1+\frac{1}{4}}}}} \\
    \end{align*}
\end{Example}
\begin{Definition}{}{}
    An expression of the form
    \[ a_0+\frac{1}{a_1+\frac{1}{a_2+\frac{1}{a_3+\cdots}}}, \]
    where $ a_0\in\mathbf{Z} $, and $ a_i\in\mathbf{Z}^+ $ for all $ i\ge 1 $
    is called a \textbf{continued fraction}.
    The number $ a_i $'s are called the partial quotient of the continued fraction.
\end{Definition}
\begin{Remark}{}{}
    In some cases, people have considered continued fractions where the
    numerators don't have to be 1. For example,
    \[ \frac{4}{\pi}=1+\frac{1^2}{2+\frac{3^2}{2+\frac{5^2}{2+\cdots}}}. \]
    In this case, we refer to it as a \textbf{simple continued fraction}.
\end{Remark}
Throughout this section, we will only
be considering continued fractions with numerator $ 1 $
and use the notation $ \conf{a_0;a_1,a_2,\ldots} $
for continued fractions
introduced by Dirichlet in 1854.

The useful formulas are:
\[ \conf{a_0;a_1,a_2,\ldots,a_n}=a_0+\frac{1}{\conf{a_1;a_2,a_3,\ldots,a_n}}
    =\conf*{a_0;a_1,a_2,\ldots,a_{n-1}+\frac{1}{a_n}}. \]
\begin{Example}{}{}
    Find a continued fraction expansion for $ x=\frac{47}{17} $.
    \tcblower{}
    \textbf{Solution}:
    We have:
    \begin{align*}
        \frac{47}{17}
         & =2+\frac{13}{17}                        \\
         & =2+\frac{1}{\frac{17}{13}}              \\
         & =2+\frac{1}{1+\frac{4}{13}}             \\
         & =2+\frac{1}{1+\frac{1}{\frac{13}{4}}}   \\
         & =2+\frac{1}{1+\frac{1}{3+\frac{1}{4}}}.
    \end{align*}
    Thus,
    \[ \frac{47}{17}=\conf{2;1,3,4}. \]
\end{Example}
\underline{Remark}: The continued fraction
of a rational number is not unique.
In particular, for Example 2, we could have written
\[ \frac{47}{17}=2+\frac{1}{1+\frac{1}{3+\frac{1}{4}}}=2+\frac{1}{1+\frac{1}{3+\frac{1}{3+\frac{1}{1}}}}. \]
And in general,
\[ \conf{a_0;a_1,a_2,\ldots,a_{n-1},a_n}=
    \conf{a_0;a_1,a_2,\ldots,a_n-1,1}. \]
\begin{Exercise}{}{}
    Calculate a continued fraction expansion
    for $ x=\frac{354}{49} $.
    \tcblower{}
    \textbf{Answer}: $ \conf{7;4,2,5} $ or $ \conf{7;4,2,4,1} $.
\end{Exercise}
\begin{Exercise}{}{}
    Calculate a continued fraction expansion
    for $ x=\frac{72}{19} $.
    \tcblower{}
    \textbf{Answer}: $ \conf{3;1,3,1,3} $.
\end{Exercise}
We can also evaluate a given continued fraction.
\begin{Example}{}{}
    Find the rational number $ x=[-2;5,2,4] $.
    \tcblower{}
    \textbf{Answer}: $ -\frac{89}{49} $.
\end{Example}
\begin{Exercise}{}{}
    Find the rational number $ x=[0;1,8,2] $.
    \tcblower{}
    \textbf{Answer}: $ \frac{17}{19} $.
\end{Exercise}
\begin{Proposition}{}{}
    Every finute continued fraction with integer terms represent a rational
    number and vice versa; that is,
    every rational number can be expressed as a simple continued fraction.
\end{Proposition}
\subsection*{Infinite Continued Fractions}
Our goal is to approximate
irrational numbers. So, we now look at how to find
a simple continued fraction expansion
of an irrational number. Continued fractions can be traced back to Euclid's time.
However, until the 1500s, they were just used to solve linear equations.
In the 1500s, Cataldi and Bombelli used continued fractions to approximate square roots.

\begin{Example}{}{}
    Calculate a continued fraction expansion for $ \sqrt{2} $.
    \tcblower{}
    \textbf{Solution 1}: First, we know the integer part of $ \sqrt{2} $
    is $ 1 $, so we can write
    \[ \sqrt{2}=1+x. \]
    Rather than trying to work with decimals, we can simply take
    $ x=\sqrt{2}-1 $ and the difference of squares formula tells us that
    \begin{align*}
        (\sqrt{2}-1)(\sqrt{2}+1) & =2-1=1 \\
        \sqrt{2}-1=\frac{1}{1+\sqrt{2}}.
    \end{align*}
    Thus, we have
    \[ \sqrt{2}=1+\frac{1}{1+\sqrt{2}}. \]
    We can simply substitute what we have in for $ \sqrt{2} $ to get
    \begin{align*}
        \sqrt{2}
         & =1+\frac{1}{1+1+\frac{1}{1+\sqrt{2}}}            \\
         & =1+\frac{1}{2+\frac{1}{1+\sqrt{2}}}              \\
         & =1+\frac{1}{2+\frac{1}{2+\frac{1}{1+\sqrt{2}}}}.
    \end{align*}
    Thus, $ \sqrt{2}=\conf{1,\overline{2}} $.

    \textbf{Solution 2}: We can also verify that this answer by a similar idea used in Example 3.
\end{Example}
\begin{Exercise}{}{}
    Calculate the continued fraction expansion for $ \sqrt{3} $
    and $ \sqrt{6} $ and verify your answer.
    \tcblower{}
    \textbf{Answer}: $ \sqrt{3}=\conf{1;1,2,1,2,\ldots} $
    and $ \sqrt{6}=\conf{2;2,4,2,4,\ldots} $.
\end{Exercise}
\makeheading{Lecture 28}{\printdate{2022-07-13}}%chktex 8